
%!TEX program=xelatex
\documentclass{oblivoir}
%\usepackage{amsmath}
\usepackage{ikps}
\setlength\parindent{0pt}
\linespread{1.5} 

\begin{document}


<<<A>>>①

<<<S>>>

[[c1001]]

$\frac{1}{\sqrt[4]{3}} \times 3^{-\frac{7}{4}}$

$=3^{-\frac{1}{4}} \times 3^{-\frac{7}{4}}$

[[c1003]]

$=3^{-\frac{1}{4}-\frac{7}{4}}$

$=3^{-2}=\frac{1}{9}$

<<<P>>>2022학년도 대학수학능력시험 9월 모의평가 시작

1. $\frac{1}{\sqrt[4]{3}} \times 3^{-\frac{7}{4}}$ 의 값은? [2점]

① $\frac{1}{9}$
② $\frac{1}{3}$
③ $1$
④ $3$
⑤ $9$


<<<A>>>⑤

<<<S>>>

[[d2003]]

$ f^{\prime}(x)=6 x^{2}+4$에서

[[d2004]]

$f^{\prime}(1)=6+4=10$


<<<P>>>2. 함수 $f(x)=2 x^{3}+4 x+5$ 에 대하여 $f^{\prime}(1)$ 의 값은? [2점]

① $6$
② $7$
③ $8$
④ $9$
⑤ $10$

<<<A>>>⑤

<<<S>>>

[[c3004]]

등비수열 $\left\{a_{n}\right\}$ 의 공비를 $r$ 라 하자.

[[c3006]]

$a_{2} a_{4}=a_{1} r \times a_{1} r^{3}=a_{1}^{2} r^{4}=4 r^{4}=36$에서 $r^{4}=9$이다.

[[m0001]]

따라서 $\frac{a_{7}}{a_{3}}=\frac{a_{1} r^{6}}{a_{1} r^{2}}=r^{4}=9$


<<<P>>>3. 등비수열 $\left\{a_{n}\right\}$ 에 대하여 $a_{1}=2, \quad a_{2} a_{4}=36$ 일 때, $\frac{a_{7}}{a_{3}}$ 의 값은? [3점]

① $1$
② $\sqrt{3}$
③ $3$
④ $3 \sqrt{3}$
⑤ $9$

<<<A>>>④

<<<S>>>

[[d1004]]

$ f(-1)=\lim _{x \rightarrow-1-} f(x)=-2+a$
$\lim _{x \rightarrow-1+} f(x)=6-a$

[[d1008]]

$f(x)$ 가 $x=1$에서 연속이므로 $-2+a=6-a$
따라서 $a=4$


<<<P>>>4. 함수 $f(x)= \begin{cases}2 x+a & (x \leq-1) \\ x^{2}-5 x-a & (x>-1)\end{cases}$이 실수 전체의 집합에서 연속일 때, 상수 $a$ 의 값은? [3점]

① $1$
② $2$
③ $3$
④ $4$
⑤ $5$


<<<A>>>③

<<<S>>>

[[d2003]]

$f(x)=2 x^{3}+3 x^{2}-12 x+1$ 에서

$f^{\prime}(x)=6 x^{2}+6 x-12=6(x+2)(x-1)$

[[d2010]]

$f(x)$의 증가, 감소를 조사하면 다음과 같다.

[IMG]

따라서 $f(x)$는 $x=-2$, $x=1$에서 각각 극대, 극소가 된다.

[[d2013]]

극댓값은 $M=f(-2)=-16+12+24+1=21$이고
극솟값은 $m=f(1)=2+3-12+1=-6$

따라서 $M+m=21-6=15$


<<<P>>>5. 함수 $f(x)=2 x^{3}+3 x^{2}-12 x+1$ 의 극댓값과 극솟값을 각각 $M, m$ 이라 할 때, $M+m$ 의 값은? [3점]

① $13$
② $14$
③ $15$
④ $16$
⑤ $17$

<<<A>>>①

<<<S>>>

[[m0008]]

$ \frac{\sin \theta}{1-\sin \theta}-\frac{\sin \theta}{1+\sin \theta}=\frac{2 \sin ^{2} \theta}{1-\sin ^{2} \theta}=4$

[[c2006]]

$2 \sin ^{2} \theta=4-4 \sin ^{2} \theta$

$\sin ^{2} \theta=\frac{2}{3}$

따라서 $\sin \theta=\frac{\sqrt{6}}{3}\;\left(\because \frac{\pi}{2}<\theta<\pi\right)$

[[c2008]]

$\cos \theta<0$ 이므로 $\cos \theta=-\frac{\sqrt{3}}{3}$


<<<P>>>6. $\frac{\pi}{2}<\theta<\pi$ 인 $\theta$ 에 대하여 $\frac{\sin \theta}{1-\sin \theta}-\frac{\sin \theta}{1+\sin \theta}=4$ 일 때, $\cos \theta$ 의 값은? [3점]

① $-\frac{\sqrt{3}}{3}$
② $-\frac{1}{3}$
③ $0$
④ $\frac{1}{3}$
⑤ $\frac{\sqrt{3}}{3}$

<<<A>>>④

<<<S>>>

[[c3011]]

$ \sum_{k=1}^{n} \frac{a_{k+1}-a_{k}}{a_{k} a_{k+1}} =\sum_{k=1}^{n}\left(\frac{1}{a_{k}}-\frac{1}{a_{k+1}}\right)$

[[c3012]]

$=\frac{1}{a_{1}}-\frac{1}{a_{n+1}}=\frac{1}{n}$

따라서 $\frac{1}{a_{n+1}}=\frac{1}{a_{1}}-\frac{1}{n}$ 이다.

[[c3013]]

$n=12$ 를 대입하면

$\frac{1}{a_{13}}=\frac{1}{a_{1}}-\frac{1}{12}=-\frac{1}{4}-\frac{1}{12}=-\frac{1}{3}$

따라서 $a_{13}=-3$


<<<P>>>7. 수열 $\left\{a_{n}\right\}$ 은 $a_{1}=-4$ 이고, 모든 자연수 $n$ 에 대하여
$$
\sum_{k=1}^{n} \frac{a_{k+1}-a_{k}}{a_{k} a_{k+1}}=\frac{1}{n}
$$
을 만족시킨다. $a_{13}$ 의 값은? [3점]

① $-9$
② $-7$
③ $-5$
④ $-3$
⑤ $-1$

<<<A>>>②

<<<S>>>

[[d1009]]

$\lim _{x \rightarrow 0} \frac{f(x)}{x}=1$에서 $f(0)=0$, $f^{\prime}(0)=1$

$\lim _{x \rightarrow 1} \frac{f(x)}{x-1}=1$에서 $f(1)=0$, $f^{\prime}(1)=1$

[[d2011]]

$f(x)=x(x-1)(a x+b)$(단, $a, b$는 상수)이라 하면

[[d2003]]

$f^{\prime}(x)=(x-1)(a x+b)+x(a x+b)+a x(x-1)$

[[d2004]]

$f^{\prime}(0)=-b=1$

$f^{\prime}(1)=a+b=1$

$\therefore a=2, b=-1$

[[m0001]]

$f(x)=x(x-1)(2 x-1)$
$f(2)=2 \times 1 \times 3=6$


<<<P>>>8. 삼차함수 $f(x)$ 가
$$
\lim _{x \rightarrow 0} \frac{f(x)}{x}=\lim _{x \rightarrow 1} \frac{f(x)}{x-1}=1
$$
을 만족시킬 때, $f(2)$ 의 값은? [3점]

① $4$
② $6$
③ $8$
④ $10$
⑤ $12$

<<<A>>>③

<<<S>>>

[[d3016]]

점 $\mathrm{P}$ 의 시각 $t$ 에서의 가속도를 $a(t)$ 라 하자.

$v(t)=-4 t^{3}+12 t^{2}$에서 $a(t)=-12 t^{2}+24 t $ 이다.
$a(k)=-12 k^{2}+24 k=12$에서 $k=1$이다.

[[d3017]]

$t=3$ 에서 $t=4$ 까지 점 $\mathrm{P}$ 가 움직인 거리를 $l$ 이라 하자.

$l =\int_{3}^{4}\left|-4 t^{3}+12 t^{2}\right| d t$

$=\int_{3}^{4}\left(4 t^{3}-12 t^{2}\right) d t$

$=\left[t^{4}-4 t^{3}\right]_{3}^{4}$

$=(256-256)-(81-108)=27$


<<<P>>>9. 수직선 위를 움직이는 점 $\mathrm{P}$ 의 시각 $t(t>0)$ 에서의 속도 $v(t)$ 가
$$
v(t)=-4 t^{3}+12 t^{2}
$$
이다. 시각 $t=k$ 에서 점 $\mathrm{P}$ 의 가속도가 12 일 때, 시각 $t=3 k$ 에서 $t=4 k$ 까지 점 $\mathrm{P}$ 가 움직인 거리는? (단, $k$ 는 상수이다.) [4점]

① $23$
② $25$
③ $27$
④ $29$
⑤ $31$

<<<A>>>③

<<<S>>>

[[c2010]]

$y=a \sin b \pi x$ 의 주기는 $\frac{2 \pi}{b \pi}=\frac{2}{b}$ 이다.

[[c2011]]

따라서 점 $\mathrm{A}\left(\frac{1}{2 b}, a\right)$ 이고, 점 $\mathrm{B}\left(\frac{5}{2 b}, a\right)$ 이다.

[[m0012]]

(삼각형 $\mathrm{OAB}$ 의 넓이) $=\frac{1}{2} \times \frac{4}{2 b} \times a=\frac{a}{b}=5\cdots\cdots$㉠

(직선 $\mathrm{OA}$ 의 기울기 $)=\frac{a}{\frac{1}{2 b}}=2 a b$

(직선 $\mathrm{OB}$ 의 기울기 $)=\frac{a}{\frac{5}{2 b}}=\frac{2 a b}{5}$

$\therefore 2 a b \times \frac{2 a b}{5}=\frac{5}{4}$

[[m0011]]

㉠에서 $a=5 b$ 이므로 $\frac{4}{5} \times(5 b)^{2} \times b^{2}=\frac{5}{4}$,

$b^{4}=\frac{1}{16}$

$b=\frac{1}{2}(\because b>0)$

$\therefore a=\frac{5}{2}$

따라서 $a+b=3$ 이다.


<<<P>>>10. 두 양수 $a, b$ 에 대하여 곡선 $y=a \sin b \pi x\left(0 \leq x \leq \frac{3}{b}\right)$ 이 직선 $y=a$ 와 만나는 서로 다른 두 점을 $\mathrm{A}, \mathrm{B}$ 라 하자.

삼각형 $\mathrm{OAB}$ 의 넓이가 5 이고 직선 $\mathrm{OA}$ 의 기울기와 직선 $\mathrm{OB}$ 의 기울기의 곱이 $\frac{5}{4}$ 일 때, $a+b$ 의 값은? (단, $\mathrm{O}$ 는 원점이다.) [4점]

① $1$
② $2$
③ $3$
④ $4$
⑤ $5$

[IMG]

<<<A>>>④

<<<S>>>

[[d3005]]

$x f(x)=2 x^{3}+a x^{2}+3 a+\int_{1}^{x} f(t) d t \cdots \cdots$ ㉠

㉠에 $x=1$ 을 대입하면 $f(1)=4 a+2 \quad \cdots \cdots$ ㉡

[[d3006]]

㉠의 양변을 $x$ 에 대하여 미분하면

$f(x)+x f^{\prime}(x)=6 x^{2}+2 a x+f(x) $ 이므로

$f^{\prime}(x)=6 x+2 a$

[[d3010]]

따라서 $f(x)=3 x^{2}+2 a x+C$ (단, $C$ 는 상수 ) $\cdots \cdots$  ㉢

[[m0009]]

[[함수 $f(x)$로 구한 $f(1)$과 항등식에서 구한 $f(1)$로 방정식을 세웁니다.]] 

㉢에 $x=1$ 을 대입하면 $f(1)=3+2 a+C$

㉡에 의해 $3+2 a+C=4 a+2$ 이다.

$C=2 a-1 $ 이므로 $f(x)=3 x^{2}+2 a x+2 a-1$

[[정적분 $\int_{0}^{1} f(t) d t$을 계산해서 주어진 방정식을 풉니다.]]

$\int_{0}^{1} f(t) d t $

$=\int_{0}^{1}\left(3 t^{2}+2 a t+2 a-1\right) d t$

$=\left[t^{3}+a t^{2}+(2 a-1) t\right]_{0}^{1}$

$=1+a+2 a-1=3 a$

$f(1)=\int_{0}^{1} f(t) d t$에서

$3+2 a+2 a-1=3 a$ 이므로 $a=-2$ 이다.

[[m0001]]

$f(3)=27+6 a+2 a-1=8 a+26=10$ 이므로 $a+f(3)=-2+10=8$


<<<P>>>11. 다항함수 $f(x)$ 가 모든 실수 $x$ 에 대하여
$$
x f(x)=2 x^{3}+a x^{2}+3 a+\int_{1}^{x} f(t) d t
$$
를 만족시킨다. $f(1)=\int_{0}^{1} f(t) d t$ 일 때, $a+f(3)$ 의 값은? (단, $a$ 는 상수이다.) [4점]

① $5$
② $6$
③ $7$
④ $8$
⑤ $9$

<<<A>>>②

<<<S>>>

[[c2012]]

삼각형 $\mathrm{ABC}$와 $\mathrm{BCD}$에 사인법칙을 적용하면 

$\frac{\overline{\mathrm{BC}}}{\sin A}=2R$, $\frac{\overline{\mathrm{BD}}}{\sin (\angle \mathrm{BCD})} =2R$ 이므로

$\overline{\mathrm{BC}}=2 \times 2 \sqrt{7} \times \sin \frac{\pi}{3}=2 \sqrt{21}$

$\overline{\mathrm{BD}}=2 \times 2 \sqrt{7} \times \frac{2 \sqrt{7}}{7}=8$

[[c2007]]

한편, $\angle \mathrm{A}=\frac{\pi}{3}$에서 $ \angle \mathrm{BDC} = \dfrac{2\pi}{3}$이고

$\overline{\mathrm{CD}}=x$ 로 놓고, 삼각형 $\mathrm{BCD}$ 에서 코사인법칙을 적용하면

$ \left( 2 \sqrt{21} \right)^{2}=x^{2}+8^{2}-2 \times x \times 8 \times \left( -\frac{1}{2} \right)$

$x^{2}+8x-20=0$에서 $x=2$

$\therefore \overline{\mathrm{BD}}+\overline{\mathrm{CD}}=8+2=10$


<<<P>>>12. 반지름의 길이가 $2 \sqrt{7}$ 인 원에 내접하고 $\angle \mathrm{A}=\frac{\pi}{3}$ 인 삼각형 $\mathrm{ABC}$ 가 있다. 점 $\mathrm{A}$ 를 포함하지 않는 호 $\mathrm{BC}$ 위의 점 $\mathrm{D}$ 에 대하여 $\sin (\angle \mathrm{BCD})=\frac{2 \sqrt{7}}{7}$ 일 때, $\overline{\mathrm{BD}}+\overline{\mathrm{CD}}$ 의 값은? [4점]

① $\frac{19}{2}$
② $10$
③ $\frac{21}{2}$
④ $11$
⑤ $\frac{23}{2}$

[IMG]

<<<A>>>②

<<<S>>>

[[m0016]]

[[$d > 0$ 을 이용해서 절댓값 기호을 없애서 자연수 $m$과 $d$의 식을 구합니다.]] 

$\left|a_{m}\right|=\left|a_{m+3}\right|$ 에서 $a_{m+3}=\pm a_{m}$ 이다.

$d$는 자연수이므로 $a_{m+3} \neq a_{m}$ 이다.

따라서 $a_{m+3}=-a_{m}$ 이므로

$-45+(m+2) d=45-(m-1) d$에서 $(2 m+1) d=90 $ 이다.

[[m0017]]

$m, d$ 는 자연수이므로 가능한 $(m, d)$ 의 순서쌍은 $(1,30),(2,18),(4,10),(7,6),(22,2)$ 이다.

[[c3015]]

[[각각의 $m,d$에 대하여 그 합을 구해서 조건 (나)를 만족시키는 $m,d$를 찾습니다.]]

모든 자연수 $n$ 에 대하여 $\sum_{k=1}^{n} a_{k}>-100$ 이려면 

$\sum_{k=1}^{n} a_{k}$의 최솟값이 $-100$보다 커야 한다.

앞에서 구한 각각의 $m,d$에 대하여 수열의 항들을 열거하여 $\sum_{k=1}^{n} a_{k}$의 최솟값을 구해보면

조건 (나)를 만족하는 경우는  $(1,30),(2,18)$ 뿐이다.

따라서 구하는 $d$의 값의 합은 $30+18=48$ 이다.

[다른 풀이]

[[m0016]]

[[$d > 0$ 을 이용해서 절댓값 기호을 없애서 자연수 $m$과 $d$의 식을 구합니다.]]

$\left|a_{m}\right|=\left|a_{m+3}\right|$ 에서 $a_{m+3}=\pm a_{m}$ 이다.

$d$는 자연수이므로 $a_{m+3} \neq a_{m}$ 이다.

따라서 $a_{m+3}=-a_{m}$ 이므로

$-45+(m+2) d=45-(m-1) d$에서 $(2 m+1) d=90 $ 이다.

$m, d$ 는 자연수이므로 가능한 $(m, d)$ 의 순서쌍은 $(1,30),(2,18),(4,10),(7,6),(22,2)$ 이다.

[[c3015]]

[[$n=m+1$일 때의 (나)를 만족시키면 충분하므로 그 때의 부등식 (나)를 풉니다.]]

$a_{1}=-45$ 이고 $a_{m}+a_{m+3}=a_{m+1}+a_{m+2}=0$ 이므로 $a_{m+1}<0$ 이고 $a_{m+2}>0$ 이다.

따라서 $\sum_{k=1}^{n} a_{k}$ 가 최소가 되려면 $n=m+1$ 일 때이다.

모든 자연수 $n$ 에 대하여 $\sum_{k=1}^{n} a_{k}>-100$ 이려면

$\sum_{k=1}^{m+1} a_{k}>-100 $이면 된다.

$\sum_{k=1}^{m+1} a_{k}=\frac{(m+1)(-90+m d)}{2}>-100 \cdots \cdots $ ㉠

가능한 $(m, d)$ 의 순서쌍 중에서  ㉠을 만족시키는 것은 $(1,30),(2,18)$ 뿐이다.

따라서 $30+18=48$ 이다.

<<<P>>>13. 첫째항이 $-45$ 이고 공차가 $d$ 인 등차수열 $\left\{a_{n}\right\}$ 이 다음 조건을 만족시키도록 하는 모든 자연수 $d$ 의 값의 합은? [4점]

(가) $\left|a_{m}\right|=\left|a_{m+3}\right|$ 인 자연수 $m$ 이 존재한다.

(나) 모든 자연수 $n$ 에 대하여 $\sum_{k=1}^{n} a_{k}>-100$ 이다.

① $44$
② $48$
③ $52$
④ $56$
⑤ $60$

<<<A>>>⑤

<<<S>>>

[[d2013]]

최고차항의 계수가 1 이고 $f^{\prime}(0)=f^{\prime}(2)=0$이므로 $f^{\prime}(x)=3 x(x-2)$ 이다.

그러므로 $f(x)=x^{3}-3 x^{2}+C$ (단, $C$ 는 상수 $)$

ㄱ.

[[d2004]]

[[$p=1$ 일 때 $x>0$ 에서 $g'(x)$를 구하여 $x=1$을 대입합니다.]]

$p=1$ 일 때 $x>0$ 에서 $g(x)=f(x+1)-f(1)$ 이므로 $g^{\prime}(1)=f^{\prime}(2)=0$ (참)

ㄴ.

[[d2024]]

[[$\lim _{x \rightarrow 0-} \dfrac{g(x)-g(0)}{x} = \lim _{x \rightarrow 0+} \dfrac{g(x)-g(0)}{x}$이 성립하는 $p$의 값을 구합니다.]]

$\lim _{x \rightarrow 0-} g(x)=\lim _{x \rightarrow 0-}\{f(x)-f(0)\}$

$=f(0)-f(0)=0$

$\lim _{x \rightarrow 0+} g(x)=\lim _{x \rightarrow 0+}\{f(x+p)-f(p)\}$

$=f(p)-f(p)=0$이고 $g(0)=0$ 이므로 함수 $g(x)$ 는 실수 전체의 집합에서 연속이다.

$g(x)$ 가 실수 전체의 집합에서 미분가능하려면

$\lim _{x \rightarrow 0-} g^{\prime}(x)=\lim _{x \rightarrow 0+} g^{\prime}(x)$

$\therefore f^{\prime}(0)=f^{\prime}(p)$

따라서 $p=0, p=2$ 이고 양수인 $p$ 는 $p=2$ $1$개 뿐이다. (참)

ㄷ.

[[d3008]]

[[구간별로 $g(x)$의 식을 구체적으로 구한 후 정적분을 계산해서 부등식을 확인합니다.]]

$g(x)= \begin{cases}x^{3}-3 x^{2} & (x \leq 0)\\ (x+p)^{3}-3(x+p)^{2}+C-\left(p^{3}-3 p^{2}+C\right) &(x>0)\end{cases}$

정리하면

$g(x)=\begin{cases}
x^{3}-3 x^{2} & (x \leq 0)\\
x^{3}+(3 p-3) x^{2}+\left(3 p^{2}-6 p\right) x & (x>0)
\end{cases}$

$\int_{-1}^{1} g(x) d x$

$=\int_{-1}^{0}\left(x^{3}-3 x^{2}\right)dx+\int_{0}^{1}\left\{x^{3}+(3 p-3) x^{2}+\left(3 p^{2}-6 p\right) x\right\} d x$

$=-\frac{5}{4}+\left\{\frac{1}{4}+(p-1)+\left(\frac{3}{2} p^{2}-3 p\right)\right\}$

$=\frac{3}{2} p^{2}-2 p-2$

$=\frac{3}{2}\left(p+\frac{2}{3}\right)(p-2)$

$p \geq 2 $ 일 때 $\int_{-1}^{1} g(x) d x \geq 0 $ 이다. (참)

[[m0018]]

따라서 옳은 것은 ㄱ, ㄴ, ㄷ이다.


<<<P>>>14. 최고차항의 계수가 $1$ 이고 $f^{\prime}(0)=f^{\prime}(2)=0$ 인 삼차함수 $f(x)$ 와 양수 $p$ 에 대하여 함수 $g(x)$ 를
$g(x)= \begin{cases}f(x)-f(0) & (x \leq 0) \\ f(x+p)-f(p) & (x>0)\end{cases}$
이라 하자. <보기>에서 옳은 것만을 있는 대로 고른 것은? [4점]

ㄱ. $p=1$ 일 때, $g^{\prime}(1)=0$ 이다.

ㄴ. $g(x)$ 가 실수 전체의 집합에서 미분가능하도록 하는 양수 $p$ 의 개수는 $1$ 이다.

ㄷ. $p \geq 2$ 일 때, $\int_{-1}^{1} g(x) d x \geq 0$ 이다.

① ㄱ,
② ㄱ, ㄴ
③ ㄱ, ㄷ
④ ㄴ, ㄷ
⑤ ㄱ, ㄴ, ㄷ


<<<A>>>①

<<<S>>>

[[주어진 수열의 관계식이 $a_{n+1}=f\left(a_{n}\right)$이 되는 함수 $f(x)$를 도입합니다.]]

$f(x)=\left\{\begin{array}{cl}-2 x-2 & \left(-1 \leq x<-\frac{1}{2}\right) \\ 2 x & \left(-\frac{1}{2} \leq x \leq \frac{1}{2}\right) \\ -2 x+2 & \left(\frac{1}{2} < x \leq 1\right)\end{array}\right.$

이라 하면 $a_{n+1}=f\left(a_{n}\right)$ 이다.

함수 $y=f(x)$ 의 그래프를 그려보면 다음과 같다.

[IMG]

[[주어진 조건을 이용하혀 $a_{5}$의 값을 구합니다.]]

$a_{5}+a_{6}=a_{5}+f\left(a_{5}\right)=0$

$a_{5}>0 $ 이면 그래프에서 $ f\left(a_{5}\right) \geq 0 $ 이므로 $a_{5}+f\left(a_{5}\right)>0 $ 이다.

$a_{5}<0 $ 이면 그래프에서 $f\left(a_{5}\right) \leq 0 $ 이므로 $a_{5}+f\left(a_{5}\right)<0 $ 이다.

$a_{5}=0 $ 이면 그래프에서 $f\left(a_{5}\right)=0 $ 이므로 $a_{5}+f\left(a_{5}\right)=0 $ 이다.

따라서 $a_{5}=0$ 이다.

[[$a_{5}=0$에서 출발하여 차례대로 $a_{4},a_{3},\cdots$를 구합니다.]]

$a_{1}<0$ 이면 $a_{n} \leq 0$ 이므로 $\sum_{k=1}^{5} a_{k} \leq 0$ 이 되어 모순이다.

$a_{1} \geq 0$ 이므로 $y=f(x)$ 의 그래프에서 $x \geq 0$ 인 부분만 살펴보면 된다.

$a_{n+1}=0$ 이면 $f\left(a_{n}\right)=0$ 이므로 $a_{n}=0,1$

$a_{n+1}=1$ 이면 $f\left(a_{n}\right)=1$ 이므로 $a_{n}=\frac{1}{2}$

$a_{n+1}=\frac{1}{2}$ 이면 $f\left(a_{n}\right)=\frac{1}{2}$ 이므로 $a_{n}=\frac{1}{4}, \frac{3}{4}$

$a_{n+1}=\frac{1}{4}$ 이면 $f\left(a_{n}\right)=\frac{1}{4} $ 이므로 $a_{n}=\frac{1}{8}, \frac{7}{8}$

$a_{n+1}=\frac{3}{4}$ 이면 $f\left(a_{n}\right)=\frac{3}{4} $ 이므로 $a_{n}=\frac{3}{8}, \frac{5}{8}$

[[여러 가지 가능한 수열을 조사합니다.]]

따라서 가능한 수열 $\left\{a_{n}\right\}$ 을 표로 나타내면 다음과 같다.

[IMG]

따라서 가능한 $a_{1}$ 의 합은 $1+\frac{1}{2}+\frac{1}{4}+\frac{3}{4}+\frac{1}{8}+\frac{7}{8}+\frac{3}{8}+\frac{5}{8}=\frac{9}{2}$


<<<P>>>15. 수열 $\left\{a_{n}\right\}$ 은 $\left|a_{1}\right| \leq 1$ 이고, 모든 자연수 $n$ 에 대하여

$a_{n+1}=\left\{\begin{array}{cl}
-2 a_{n}-2 & \left(-1 \leq a_{n}<-\frac{1}{2}\right)\\
2 a_{n} & \left(-\frac{1}{2} \leq a_{n} \leq \frac{1}{2}\right)\\
-2 a_{n}+2 & \left(\frac{1}{2}<a_{n} \leq 1\right)
\end{array}\right.$

을 만족시킨다. $a_{5}+a_{6}=0$ 이고 $\sum_{k=1}^{5} a_{k}>0$ 이 되도록 하는 모든 $a_{1}$ 의 값의 합은? [4점]

① $\frac{9}{2}$
② $5$
③ $\frac{11}{2}$
④ $6$
⑤ $\frac{13}{2}$

<<<A>>>$2$

<<<S>>>

[[c1006]]

$ \log _{2} 100-2 \log _{2} 5$ $=\log _{2} 100-\log _{2} 5^{2}$
$=\log _{2} \frac{100}{25}=\log _{2} 4=2 $


<<<P>>>16. $\log _{2} 100-2 \log _{2} 5$ 의 값을 구하시오. [3점]


<<<A>>>$8$

<<<S>>>

[[d3001]]

$f^{\prime}(x)=8 x^{3}-12 x^{2}+7$에서
$f(x)=2 x^{4}-4 x^{3}+7 x+C $ (단, $C$ 는 상수)

[[d3009]]

$ f(0)=3 $ 이므로 $C=3 $ 이고 $ f(x)=2 x^{4}-4 x^{3}+7 x+3$ 

[[m0004]]

따라서 $f(1)=2-4+7+3=8$


<<<P>>>17. 함수 $f(x)$ 에 대하여 $f^{\prime}(x)=8 x^{3}-12 x^{2}+7$ 이고 $f(0)=3$ 일 때, $f(1)$ 의 값을 구하시오. [3점]

<<<A>>>$9$

<<<S>>>

[[c3010]]

$ \sum_{k=1}^{10}\left(a_{k}+2 b_{k}\right)=45 \quad \cdots \cdots$ ㉠

$\sum_{k=1}^{10}\left(a_{k}-b_{k}\right)=3 \cdots \cdots$ ㉡

㉠, ㉡을 변변끼리 빼서 정리하면

$\sum_{k=1}^{10} 3 b_{k}=42$ 에서 $\sum_{k=1}^{10} b_{k}=14$

[[c3009]]

$ \sum_{k=1}^{10}\left(b_{k}-\frac{1}{2}\right) $

$=\sum_{k=1}^{10} b_{k}-10 \times \frac{1}{2}=14-5=9 $


<<<P>>>18. 두 수열 $\left\{a_{n}\right\},\left\{b_{n}\right\}$ 에 대하여 $\sum_{k=1}^{10}\left(a_{k}+2 b_{k}\right)=45, \quad \sum_{k=1}^{10}\left(a_{k}-b_{k}\right)=3$일 때,

$\sum_{k=1}^{10}\left(b_{k}-\frac{1}{2}\right)$ 의 값을 구하시오. [3점]

<<<A>>>$11$

<<<S>>>

[[d2002]]

$ \frac{f(4)-f(0)}{4-0}=3 a^{2}-12 a+5$에서

$\frac{64-96+20}{4}=3 a^{2}-12 a+5 $이고 정리하면

$3 a^{2}-12 a+8=0$

[[m0025]]

근의 공식에 대입하면 $a=\frac{6 \pm \sqrt{12}}{3}$ 이므로 두 근은 모두 $0<a<4$ 이다.

따라서 모든 실수 $a$의 값의 곱은 $\frac{8}{3}$이므로 $p+q=11$ 이다.


<<<P>>>19. 함수 $f(x)=x^{3}-6 x^{2}+5 x$ 에서 $x$ 의 값이 0 에서 4 까지 변할 때의 평균변화율과 $f^{\prime}(a)$ 의 값이 같게 되도록 하는 $0<a<4$ 인 모든 실수 $a$ 의 값의 곱은 $\frac{q}{p}$ 이다. $p+q$ 의 값을 구하시오. (단, $p$ 와 $q$ 는 서로소인 자연수이다.) [3점]

<<<A>>>$21$

<<<S>>>

[[d2025]]

주어진 방정식을 정리하면 $f(x)+ \left| f(x)+x \right| -6x =k$이고,

$g(x)=f(x)+ \left| f(x)+x \right| -6x $라 하자.

[[d2016]]

[[$\left| f(x)+x \right|$에서 구간을 나누어 절댓값이 없는 식으로 변형합니다.]]

$f(x)+x=\frac{1}{2} x^{3}-\frac{9}{2} x^{2}+11 x$

$=\frac{1}{2} x\left(x^{2}-9 x+22\right)$

$x^{2}-9 x+22=\left(x-\frac{9}{2}\right)^{2}+\frac{7}{4}>0 $ 이므로

$|f(x)+x|= \begin{cases}f(x)+x & (x \geq 0) \\
-f(x)-x & (x<0)\end{cases}$

[[$y=g(x)$의 그래프를 그리고 $y=k$의 그래프와의 교점의 개수를 구합니다.]]

따라서, 
$g(x)=\begin{cases} 2f(x)-5x & (x \geq 0) \\
    -7x & (x<0)\end{cases} = \begin{cases}
x^{3}-9 x^{2}+15 x & (x \geq 0) \\
-7 x & (x<0)
\end{cases}$이고 그래프를 그리면 다음과 같다.

[IMG]

[[서로 다른 $4$개의 교점을 갖게 하는 $k$의 값의 범위를 구합니다.]]

$y=g(x)$ 의 그래프와 직선 $y=k$ 가 서로 다른 $4$개의 교점을 갖게 하는 정수 $k$ 의 값의 범위는 $0 < k < 7$이고,

[[m0026]]

정수 $k$ 의 합은 $\frac{6 \times 7}{2}=21$ 이다.


<<<P>>>20. 함수 $f(x)=\frac{1}{2} x^{3}-\frac{9}{2} x^{2}+10 x$ 에 대하여 $x$ 에 대한 방정식
$$
f(x)+|f(x)+x|=6 x+k
$$
의 서로 다른 실근의 개수가 $4$가 되도록 하는 모든 정수 $k$의 값의 합을 구하시오. [4점]

<<<A>>>$192$

<<<S>>>

[[c1007]]

$y=\log _{a}(x-1)$ 의 역함수를 구해보면 $y=a^{x}+1$

다음 그림과 같이 $y=a^{x}+1$ 과 $y=-x+4$ 의 그래프의 교점을 $\mathrm{P}$ 라 하자.

[IMG]

$y=a^{x}+1$ 의 그래프는 $y=a^{x-1}$ 의 그래프를 $x$축 방향으로 $-1$ 만큼, $y$축 방향으로 1 만큼 평행이동한 그래프이다.

따라서 점 $\mathrm{A}$ 를 $x$ 축의 방향으로 $-1$ 만큼, $y$ 축의 방향으로 1 만큼 평행이동한 점을 $\mathrm{A}^{\prime}$ 이라 하면 점 $\mathrm{A}^{\prime}$ 은 $y=a^{x}+1$ 의 그래프 위에 있다.

또한 $y=-x+4$ 의 기울기가 $-1$ 이므로 점 $\mathrm{A}^{\prime}$ 은 직선 $y=-x+4$ 위에 있다.

[[m0022]]

[[조건을 이용하여 점 $\mathrm{P}$의 좌표를 구합니다.]]

따라서 점 $\mathrm{A}^{\prime}$ 은 $y=a^{x}+1$ 과 $y=-x+4$ 의 그래프의 교점인 점 $\mathrm{P}$ 이고 $\overline{\mathrm{AP}}=\sqrt{2}$ 이다.

두 점 $\mathrm{B}, \mathrm{P}$ 는 직선 $y=x$ 에 대하여 대칭이고 $\overline{\mathrm{BP}}$ 의 길이는 $3 \sqrt{2}$ 이므로

점 $\mathrm{P}$ 의 좌표는 $\left(\frac{1}{2}, \frac{7}{2}\right)$ 이다.

[[앞의 결과와 조건을 이용하여 $a$의 값을 구합니다.]]

따라서 $a^{\frac{1}{2}}+1=\frac{7}{2}$ 에서 $a=\frac{25}{4}$ 이므로 점 $\mathrm{C}$ 의 좌표는 $\left(0, \frac{4}{25}\right)$ 이다.

[[점 $\mathrm{C}$에서 직선까지의 거리를 높이로 하는 $\triangle \mathrm{ABC}$의 넓이를 구합니다.]]

점 $\mathrm{C}$ 에서 $y=-x+4$ 까지의 거리는
$$
\frac{\left|\frac{4}{25}-4\right|}{\sqrt{1+1}}=\frac{96}{25 \sqrt{2}}
$$

따라서 $\triangle \mathrm{ABC}$ 의 넓이는 $\frac{1}{2} \times 2 \sqrt{2} \times \frac{96}{25 \sqrt{2}}=\frac{96}{25}$ $\therefore 50 S=50 \times \frac{96}{25}=192$


<<<P>>>21. $a>1$ 인 실수 $a$ 에 대하여 직선 $y=-x+4$ 가 두 곡선
$$
y=a^{x-1}, \quad y=\log _{a}(x-1)
$$
과 만나는 점을 각각 $\mathrm{A}, \mathrm{B}$ 라 하고, 곡선 $y=a^{x-1}$ 이 $y$ 축과 만나는 점을 $\mathrm{C}$ 라 하자. $\overline{\mathrm{AB}}=2 \sqrt{2}$ 일 때, 삼각형 $\mathrm{ABC}$ 의 넓이는 $S$ 이다. $50 \times S$ 의 값을 구하시오. [4점]

[IMG]

<<<A>>>$108$

<<<S>>>

[[$f(x)>0, f(x)<0, f(x)=0$일 때로 경우를 나누어 $g(x)$를 정리합니다.]]

$ \lim _{h \rightarrow 0+} \frac{|f(x+h)|-|f(x-h)|}{h}=h(x)$ 라 하자.

$f(x)>0$ 이면

$h(x)=\lim _{h \rightarrow 0+} \frac{f(x+h)-f(x-h)}{h}=2 f^{\prime}(x)$

$f(x)<0 $ 이면

$h(x)=\lim _{h \rightarrow 0+} \frac{-f(x+h)+f(x-h)}{h}=-2 f^{\prime}(x)$

$f(x)=0 $ 일 때, $\left|f^{\prime}(x)\right|=p $ 라 하자.

$\lim _{h \rightarrow 0+} \frac{|f(x+h)|-|f(x-h)|}{h}$

$=\lim _{h \rightarrow 0+}\left\{\frac{|f(x+h)|-|f(x)|}{h} + \frac{|f(x-h)|-|f(x)|}{-h}\right\}$

$=p+(-p)=0$

$\therefore f(x)=0$일 때, $h(x)=0$

따라서 $g(x)=\begin{cases}
2 f(x-3) f^{\prime}(x) & (f(x)>0) \\
-2 f(x-3) f^{\prime}(x) & (f(x)<0) \\
0 & (f(x)=0)
\end{cases}$

[[$f(x)$의 극값의 유무와 $f(x)=0$의 실근의 개수에 따라 $g(x)$의 연속성을 조사합니다.]]

$g(x)$ 가 실수 전체의 집합에서 연속이므로
$f(x)=0$ 일 때, $f(x-3)=0$ 또는 $f^{\prime}(x)=0$

따라서 최고차항의 계수가 1 인 삼차함수 $f(x)$ 의 그래프의 개형은 다음과 같다.

[IMG]

위의 그래프에서 $g(x)=0$의 서로 다른 네 실근은 $\alpha, \alpha+2, \alpha+3, \alpha+6$이므로

$\alpha_{1}+\alpha_{2}+\alpha_{3}+\alpha_{4}=7$에서 $\alpha=-1$이다.

[[$\alpha_{1}$ 값을 이용해서 $f(x)$를 구하고 $5$에서의 함숫값을 알아냅니다.]]

따라서 $f(x)=(x+1)^2 (x-2)$이고, $f(5)=108$이다.


<<<P>>>22. 최고차항의 계수가 1 인 삼차함수 $f(x)$ 에 대하여 함수

$$g(x)=f(x-3) \times \lim _{h \rightarrow 0+} \frac{|f(x+h)|-|f(x-h)|}{h}$$

가 다음 조건을 만족시킬 때, $f(5)$ 의 값을 구하시오. [4점]

(가) 함수 $g(x)$ 는 실수 전체의 집합에서 연속이다.

(나) 방정식 $g(x)=0$ 은 서로 다른 네 실근 $\alpha_{1}, \alpha_{2}, \alpha_{3}, \alpha_{4}$ 를 갖고 $\alpha_{1}+\alpha_{2}+\alpha_{3}+\alpha_{4}=7$ 이다.

<<<A>>>③

<<<S>>>

[[x3010]]

$\mathrm{E}(X)=60 \times \frac{1}{4}=15$


<<<P>>>23. 확률변수 $X$ 가 이항분포 $\mathrm{B}\left(60, \frac{1}{4}\right)$ 을 따를 때, $\mathrm{E}(X)$ 의 값은? [2점]

① $5$
② $10$
③ $15$
④ $20$
⑤ $25$

<<<A>>>③

<<<S>>>

[[x2005]]

$(a, b)$ 의 모든 순서쌍의 개수는 $4 \times 4=16$

[[x2006]]

$a \times b>31$ 인 $(a, b)$ 의 순서쌍은 $(5,8),(7,6)$,
$(7,8)$ 이므로

[[x2007]]

$a \times b>31$ 일 확률을 $p$ 라 하면
$p=\frac{3}{16}$


<<<P>>>24. 네 개의 수 $1,3,5,7$ 중에서 임의로 선택한 한 개의 수를 $a$ 라 하고, 네 개의 수 $2,4,6,8$ 중에서 임의로 선택한 한 개의 수를 $b$ 라 하자. $a \times b>31$ 일 확률은? [3점]

① $\frac{1}{16}$
② $\frac{1}{8}$
③ $\frac{3}{16}$
④ $\frac{1}{4}$
⑤ $\frac{5}{16}$

<<<A>>>②

<<<S>>>

[[x1001]]

$\left(x^{2}+\frac{a}{x}\right)^{5}$ 의 전개식의 일반항은


${ }_{5} \mathrm{C}_{r} \times\left(x^{2}\right)^{5-r} \times\left(\frac{a}{x}\right)^{r}$ (단, $r=0,1, \cdots, 5)$
$={ }_{5} \mathrm{C}_{r} \times a^{r} \times x^{10-3 r}$

[[x1002]]

(i) $\frac{1}{x^{2}}$ 의 계수
$10-3 r=-2$ 에서 $r=4$ 이므로
계수는 ${ }_{5} \mathrm{C}_{4} \times a^{4}=5 a^{4}$ 이다.

[[x1002]]

(ii) $x$ 의 계수
$10-3 r=1$ 에서 $r=3$ 이므로
계수는 ${ }_{5} \mathrm{C}_{3} \times a^{3}=10 a^{3}$ 이다.

[[m0005]]

따라서 $5 a^{4}=10 a^{3}$ 이므로 $a=2$ 이다.


<<<P>>>25. $\left(x^{2}+\frac{a}{x}\right)^{5}$ 의 전개식에서 $\frac{1}{x^{2}}$ 의 계수와 $x$ 의 계수가 같을 때, 양수 $a$ 의 값은? [3점]

① $1$
② $2$
③ $3$
④ $4$
⑤ $5$

<<<A>>>①

<<<S>>>

[[x2003]]

[[x2004]]

주사위의 눈이 5 이상인 사건을 $X$, 꺼낸 공이 모 두 흰색인 사건을 $Y$ 라 하자.

$\mathrm{P}(X \mid Y) $

$=\frac{\mathrm{P}(X \cap Y)}{\mathrm{P}(Y)}$

$=\frac{\mathrm{P}(X \cap Y)}{\mathrm{P}(X \cap Y)+\mathrm{P}\left(X^{c} \cap Y\right)}$

$=\frac{\frac{2}{6} \times \frac{{ }_{2} \mathrm{C}_{2}}{{ }_{6} \mathrm{C}_{2}}}{\frac{2}{6} \times \frac{{ }_{2} \mathrm{C}_{2}}{{ }_{6} \mathrm{C}_{2}}+\frac{4}{6} \times \frac{{ }_{3} \mathrm{C}_{2}}{{ }_{6} \mathrm{C}_{2}}}=\frac{1}{7}$


<<<P>>>26. 주머니 $\mathrm{A}$ 에는 흰 공 2 개, 검은 공 4 개가 들어 있고, 주머니 B에는 흰 공 3 개, 검은 공 3 개가 들어 있다.
두 주머니 $\mathrm{A}, \mathrm{B}$ 와 한 개의 주사위를 사용하여 다음 시행을 한다.
주사위를 한 번 던져
나온 눈의 수가 5 이상이면
주머니 $\mathrm{A}$ 에서 임의로 2 개의 공을 동시에 꺼내고,
나온 눈의 수가 4 이하이면
주머니 B에서 임의로 2 개의 공을 동시에 꺼낸다.
이 시행을 한 번 하여 주머니에서 꺼낸 2 개의 공이 모두 흰색일 때, 나온 눈의 수가 5 이상일 확률은? [3점]

① $\frac{1}{7}$
② $\frac{3}{14}$
③ $\frac{2}{7}$
④ $\frac{5}{14}$
⑤ $\frac{3}{7}$

[IMG]

<<<A>>>⑤

<<<S>>>



확률변수 $X$ 의 표준편차를 $\sigma$ 라고 하면
확률변수 $X$ 는 정규분포 $\mathrm{N}\left(220, \sigma^{2}\right)$ 을 따르고,
확률변수 $\overline{X}$ 는 정규분포 $\mathrm{N}\left(220, \frac{\sigma^{2}}{n}\right)$ 을 따른다.
표준정규분포를 따르는 확률변수 $Z$ 에 대하여

$Z=\frac{\overline{X}-220}{\frac{\sigma}{\sqrt{n}}}$이므로

$\mathrm{P}(\overline{X} \leq 215) $

$=\mathrm{P}\left(Z \leq \frac{215-220}{\frac{\sigma}{\sqrt{n}}}\right)$

$=\mathrm{P}\left(Z \leq-\frac{5 \sqrt{n}}{\sigma}\right)=0.1587$

$\mathrm{P}(Z \leq-1)=0.5-\mathrm{P}(0 \leq Z \leq 1)$

$= 0.5-0.3413=0.1587$

이므로 $-\frac{5 \sqrt{n}}{\sigma}=-1 \quad \therefore \frac{\sigma}{\sqrt{n}}=5$

확률변수 $Y$ 의 표준편차는 $1.5 \sigma=\frac{3}{2} \sigma$ 이므로

확률변수 $Y$ 는 정규분포 $\mathrm{N}\left(240, \frac{9}{4} \sigma^{2}\right)$ 을 따르고,

확률변수 $\overline{Y}$ 는 정규분포 $\mathrm{N}\left(240, \frac{\sigma^{2}}{4 n}\right)$ 을 따른다.

표준정규분포를 따르는 확률변수 $Z$ 에 대하여

$Z=\frac{\overline{Y}-240}{\frac{\sigma}{2 \sqrt{n}}}$이므로

$\mathrm{P}(\overline{Y} \geq 235) $

$=\mathrm{P}\left(Z \geq \frac{235-240}{\frac{\sigma}{2 \sqrt{n}}}\right)$

$=\mathrm{P}(Z \geq-2)$

$=0.5+(0 \leq Z \leq 2)$

$=0.5+0.4772$

$=0.9772$


<<<P>>>27. 지역 $\mathrm{A}$ 에 살고 있는 성인들의 $1$ 인 하루 물 사용량을 확률변수 $X$, 지역 $\mathrm{B}$ 에 살고 있는 성인들의 $1$ 인 하루 물 사용량을 확률변수 $Y$ 라 하자. 두 확률변수 $X, Y$ 는 정규분포를 따르고 다음 조건을 만족시킨다.

(가) 두 확률변수 $X, Y$ 의 평균은 각각 $220$ 과 $240$ 이다.

(나) 확률변수 $Y$ 의 표준편차는 확률변수 $X$ 의 표준편차의 $1.5$ 배이다.

지역 $\mathrm{A}$ 에 살고 있는 성인 중 임의추출한 $n$ 명의 $1$ 인 하루 물 사용량의 표본평균을 $\overline{X}$, 지역 $\mathrm{B}$ 에 살고 있는 성인 중 임의추출한 $9 n$ 명의 $1$ 인 하루 물 사용량의 표본평균을 $\overline{Y}$ 라 하자. $\mathrm{P}(\overline{X} \leq 215)=0.1587$ 일 때, $\mathrm{P}(\overline{Y} \geq 235)$ 의 값을 오른쪽 표준정규분포표를 이용하여 구한 것은? (단, 물 사용량의 단위는 $\mathrm{L}$이다.) [3점]

[IMG]

① $0.6915$
② $0.7745$
③ $0.8185$
④ $0.8413$
⑤ $0.9772$



<<<A>>>④

<<<S>>>



조건 (가)에서 $f(3)+f(4)=5$ 또는 $10$ 이므로

$ f(3)+f(4)=1+4=2+3=3+2=4+1$ 또는 $f(3)+f(4)=4+6=5+5=6+4 $

두 조건 (나), (다)에서 $f(3) \geq 2, f(4) \leq 5 $이므로

$f(3)+f(4)=2+3=3+2=4+1$ 또는 $f(3)+f(4)=5+5=6+4$

(i) $f(3)=2, f(4)=3$ 인 경우

$f(1), f(2)<2$ 이고 $f(5), f(6)>3$ 이므로

(경우의 수 )$=1^{2} \times 3^{2}=9$

(ii) $f(3)=3, f(4)=2$ 인 경우

$f(1), f(2)<3$이고 $f(5), f(6)>2$ 이므로 $2^{2} \times 4^{2}=64$


(iii) $f(3)=4, f(4)=1$

$f(1), f(2)<4$이고 $f(5), f(6)>1 $이므로 $3^{2} \times 5^{2}=225$

(iv) $f(3)=5, f(4)=5$

$f(1), f(2)<5$이고 $f(5), f(6)>5 $이므로

$4^{2} \times 1^{2}=16$

(v) $f(3)=6, f(4)=4$

$f(1), f(2)<6$이고 $f(5), f(6)>4 $이므로 

$5^{2} \times 2^{2}=100$

(i), (ii), (iii), (iv), (v)에서 $ 9+64+225+16+100=414$

<<<P>>>28. 집합 $X=\{1,2,3,4,5,6\}$ 에 대하여 다음 조건을 만족시키는 함수 $f: X \rightarrow X$ 의 개수는? [4점]

(가) $f(3)+f(4)$ 는 5 의 배수이다.

(나) $f(1)<f(3)$ 이고 $f(2)<f(3)$ 이다.

(다) $f(4)<f(5)$ 이고 $f(4)<f(6)$ 이다.

① $384$
② $394$
③ $404$
④ $414$
⑤ $424$


<<<A>>>$78$

<<<S>>>



$\mathrm{E}(X)=1 \times a+3 \times b+5 \times c+7 \times b+9 \times a$
$=10 a+10 b+5 c$

$\mathrm{E}(Y)=1 \times\left(a+\frac{1}{20}\right)+3 \times b+5 \times\left(c-\frac{1}{10}\right)$
$+7 \times b+9 \times\left(a+\frac{1}{20}\right)$
$=10 a+10 b+5 c+\frac{1}{20}-\frac{10}{20}+\frac{9}{20}$
$=\mathrm{E}(X)$

$\mathrm{E}\left(X^{2}\right)=1^{2} \times a+3^{2} \times b+5^{2} \times c+7^{2} \times b$
$+9^{2} \times a$

$=82 a+58 b+25 c$

$ \mathrm{E}\left(Y^{2}\right)$

$=1^{2} \times\left(a+\frac{1}{20}\right)+3^{2} \times b+5^{2} \times\left(c-\frac{1}{10}\right) +7^{2} \times b+9^{2} \times\left(a+\frac{1}{20}\right)$

$\therefore$ $=82 a+58 b+25 c+\frac{1}{20}-\frac{50}{20}+\frac{81}{20}$

$= \mathrm{E}\left(X^{2}\right)+\frac{8}{5}$

$\therefore$ $\mathrm{V}(Y)$

$=\mathrm{E}\left(Y^{2}\right)-\{\mathrm{E}(Y)\}^{2}$

$=\mathrm{E}\left(X^{2}\right)+\frac{8}{5}-\{\mathrm{E}(X)\}^{2}$

$=\mathrm{V}(X)+\frac{8}{5}$

$=\frac{39}{5}$

$\therefore$ $10 \times \mathrm{V}(Y)=10 \times \frac{39}{5}=78 $


<<<P>>>29. 두 이산확률변수 $X, Y$ 의 확률분포를 표로 나타내면 각각 다음과 같다.

[IMG]

[IMG]

$\mathrm{V}(X)=\frac{31}{5}$ 일 때, $10 \times \mathrm{V}(Y)$ 의 값을 구하시오. [4점]

<<<A>>>$218$

<<<S>>>



$\mathrm{A}, \mathrm{B}, \mathrm{C}, \mathrm{D}$ 학생이 받는 사인펜의 개수를 각각 $a, b, c, d$ 라 하면 $a+b+c+d=14$ 이고
두 조건 (가), (나)에서 $1 \leq a, b, c, d \leq 9$ 이고

$a=a^{\prime}+1, b=b^{\prime}+1, c=c^{\prime}+1$

$d=d^{\prime}+1$이라 놓으면

$0 \leq a^{\prime}, b^{\prime}, c^{\prime}, d^{\prime} \leq 8 $이다.

따라서 $a^{\prime}+b^{\prime}+c^{\prime}+d^{\prime}=10$ 의 음이 아닌 정수해 중 조건 (나)에 따라 $9$ 또는 $10$ 인 해가 있는 경우와 조건 (다)에 따라 $a^{\prime}, b^{\prime}, c^{\prime}, d^{\prime}$ 이 모두 짝수인 경우 를 제외한다.

$a^{\prime}+b^{\prime}+c^{\prime}+d^{\prime}=10$ 의 음이 아닌 정수해의 개수는
${ }_{4} \mathrm{H}_{10}=286 \cdots \cdots $㉠

$a^{\prime}, b^{\prime}, c^{\prime}, d^{\prime}$ 중 9 가 있는 경우의 수
${ }_{4} \mathrm{C}_{1} \times{ }_{3} \mathrm{H}_{1}=4 \times 3=12 \cdots \cdots $㉡

$a^{\prime}, b^{\prime}, c^{\prime}, d^{\prime}$ 중 10 이 있는 경우의 수
${ }_{4} \mathrm{C}_{1} \times{ }_{3} \mathrm{H}_{0}=4 \cdots \cdots $㉢

$a^{\prime}, b^{\prime}, c^{\prime}, d^{\prime}$ 이 모두 짝수인 경우는

$a^{\prime}=2 a^{\prime \prime}, b^{\prime}=2 b^{\prime \prime}, c^{\prime}=2 c^{\prime \prime}, d^{\prime}=2 d^{\prime \prime}$ 이라 하면

$a^{\prime \prime}+b^{\prime \prime}+c^{\prime \prime}+d^{\prime \prime}=5 $ 이므로
${ }_{4} \mathrm{H}_{5}={ }_{8} \mathrm{C}_{5}=56 \cdots \cdots $㉣

이때, $a^{\prime}, b^{\prime}, c^{\prime}, d^{\prime}$ 중 10 이 있는 경우는 $a^{\prime}, b^{\prime}, c^{\prime}, d^{\prime}$ 이 모두 짝수인 경우에 포함되므로 문제의 경우의 수는 ㉠$-$㉡$-$㉢$-$㉣$+$㉢$=218$


<<<P>>>30. 네 명의 학생 $\mathrm{A}, \mathrm{B}, \mathrm{C}, \mathrm{D}$ 에게 같은 종류의 사인펜 14 개를 다음 규칙에 따라 남김없이 나누어 주는 경우의 수를 구하시오. [4점]

(가) 각 학생은 1 개 이상의 사인펜을 받는다.

(나) 각 학생이 받는 사인펜의 개수는 9 이하이다.

(다) 적어도 한 학생은 짝수 개의 사인펜을 받는다.


<<<A>>>③

<<<S>>>





$\lim _{n \rightarrow \infty} \frac{2 \times 3^{n+1}+5}{3^{n}+2^{n+1}}$

$=\lim _{n \rightarrow \infty} \frac{6+\frac{5}{3^{n}}}{1+2\left(\frac{2}{3}\right)^{n}}$

$=\frac{6+0}{1+0}$
$=6$


<<<P>>>23. $\lim _{n \rightarrow \infty} \frac{2 \times 3^{n+1}+5}{3^{n}+2^{n+1}}$ 의 값은? [2점]

① $2$
② $4$
③ $6$
④ $8$
⑤ $10$


<<<A>>>②

<<<S>>>



$ 2 \cos \alpha =3 \sin \alpha$에서 

$\frac{\sin \alpha}{\cos \alpha}=\tan \alpha=\frac{2}{3}$이다.

$\therefore \tan \beta$

$=\tan ((\alpha+\beta)-\alpha)$

$=\frac{\tan (\alpha+\beta)-\tan \alpha}{1+\tan (\alpha+\beta) \tan \alpha}$

$= \frac{1-\frac{2}{3}}{1+1 \times \frac{2}{3}}$

$=\frac{3-2}{3+2}$

$=\frac{1}{5} $


<<<P>>>24. $2 \cos \alpha=3 \sin \alpha$ 이고 $\tan (\alpha+\beta)=1$ 일 때, $\tan \beta$ 의 값은?
[3점]

① $\frac{1}{6}$
② $\frac{1}{5}$
③ $\frac{1}{4}$
④ $\frac{1}{3}$
⑤ $\frac{1}{2}$


<<<A>>>④

<<<S>>>



$ \frac{d x}{d t}=e^{t}+4 e^{-t}, \frac{d y}{d t}=1$ 이므로

$\frac{d y}{d x}=\frac{\frac{d y}{d t}}{\frac{d x}{d t}}=\frac{1}{e^{t}+4 e^{-t}}$이다.

따라서 $t=\ln 2$ 일 때 $\frac{d y}{d x}$ 의 값은

$\frac{1}{e^{\ln 2}+4 e^{-\ln 2}}=\frac{1}{2+4 \times \frac{1}{2}}=\frac{1}{4}
$


<<<P>>>25. 매개변수 $t$ 로 나타내어진 곡선
$ x=e^{t}-4 e^{-t}, \quad y=t+1 $
에서 $t=\ln 2$ 일 때, $\frac{d y}{d x}$ 의 값은? [3점]

① $1$
② $\frac{1}{2}$
③ $\frac{1}{3}$
④ $\frac{1}{4}$
⑤ $\frac{1}{5}$



<<<A>>>②

<<<S>>>



$x=t$ 에서 단면의 넓이는
$\left(\sqrt{\frac{3 t+1}{t^{2}}}\right)^{2}=\frac{3 t+1}{t^{2}}=\frac{3}{t}+\frac{1}{t^{2}}$이므로

부피는

$\int_{1}^{2}\left(\frac{3}{t}+\frac{1}{t^{2}}\right) d t $

$=\left[3 \ln |t|-\frac{1}{t}\right]_{1}^{2}$

$=\left(3 \ln 2-\frac{1}{2}\right)-\left(3 \ln 1-\frac{1}{1}\right)$

$=3 \ln 2-\frac{1}{2}+1$

$=\frac{1}{2}+3 \ln 2$


<<<P>>>26. 그림과 같이 곡선 $y=\sqrt{\frac{3 x+1}{x^{2}}}(x>0)$ 과 $x$ 축 및 두 직선 $x=1, x=2$ 로 둘러싸인 부분을 밑면으로 하고 $x$ 축에 수직인 평면으로 자른 단면이 모두 정사각형인 입체도형의 부피는? [3점]

[IMG]

① $3 \ln 2$
② $\frac{1}{2}+3 \ln 2$
③ $1+3 \ln 2$
④ $\frac{1}{2}+4 \ln 2$
⑤ $1+4 \ln 2$



<<<A>>>③

<<<S>>>



$\overline{\mathrm{C}_{1} \mathrm{E}_{1}}=\sqrt{3}, \overline{\mathrm{C}_{1} \mathrm{F}_{1}}=\frac{\sqrt{3}}{3}$ 이므로

$\overline{\mathrm{E}_{1} \mathrm{F}_{1}}=\frac{2}{3} \sqrt{3}$이고 $\overline{\mathrm{F}_{1} \mathrm{H}_{1}}=\frac{2}{3}$이다.

$\therefore \overline{\mathrm{G}_{1} \mathrm{H}_{1}}=\frac{2}{3} \sqrt{3}-\frac{2}{3}=\frac{2}{3}(\sqrt{3}-1)$

$\therefore \triangle \mathrm{G}_{1} \mathrm{E}_{1} \mathrm{H}_{1}=\frac{1}{2} \times \frac{2}{3}(\sqrt{3}-1) \times \frac{2}{3} \sqrt{3}$

$=\frac{2 \sqrt{3}(\sqrt{3}-1)}{9}$

$\triangle \mathrm{H}_{1} \mathrm{F}_{1} \mathrm{D}_{1}=\frac{1}{2} \times \frac{2}{3} \times \frac{\sqrt{3}}{3}=\frac{\sqrt{3}}{9}$

$\therefore S_{1}=\frac{2 \sqrt{3}(\sqrt{3}-1)}{9}+\frac{\sqrt{3}}{9}=\frac{6-\sqrt{3}}{9}$

$\overline{\mathrm{AB}_{2}}=a$라고 하면 $\overline{\mathrm{B}_{2} \mathrm{C}_{2}}=2 a$이다.


점 $\mathrm{E}_{1}$ 에서 $\overline{\mathrm{B}_{2} \mathrm{C}_{2}}$ 에 내린 수선의 발을 $\mathrm{T}$ 라 하자.

$\overline{\mathrm{B}_{1} \mathrm{B}_{2}}=\overline{\mathrm{E}_{1} \mathrm{T}}=1-a$ 이고, $\angle \mathrm{E}_{1} \mathrm{C}_{2} \mathrm{T}=\frac{\pi}{4}$ 이므로 $\overline{\mathrm{C}_{2} \mathrm{T}}=1-a$ 이다. 또, $\overline{\mathrm{B}_{1} \mathrm{E}_{1}}=\overline{\mathrm{B}_{2} \mathrm{T}}=2-\sqrt{3}$ 이므로 $\overline{\mathrm{B}_{2} \mathrm{T}}+\overline{\mathrm{C}_{2} \mathrm{T}}=\overline{\mathrm{B}_{2} \mathrm{C}_{2}}$ 에서 $2-\sqrt{3}+1-a=2 a, a=\frac{3-\sqrt{3}}{3}=\frac{\sqrt{3}-1}{\sqrt{3}}$ 즉, 두 사각형 $\mathrm{AB}_{1} \mathrm{C}_{1} \mathrm{D}_{1}$ 과 $\mathrm{AB}_{2} \mathrm{C}_{2} \mathrm{D}_{2}$ 의 닮음비가 $\frac{\sqrt{3}-1}{\sqrt{3}}$ 이므로 $\lim _{n \rightarrow \infty} S_{n}$ 은 첫째항이 $\frac{6-\sqrt{3}}{9}$ 이고 공비가 $\left(\frac{\sqrt{3}-1}{\sqrt{3}}\right)^{2}=\frac{4-2 \sqrt{3}}{3}$ 인 등비급수이다.

$\lim _{n \rightarrow \infty} S_{n}= \frac{\frac{6-\sqrt{3}}{9}}{1-\frac{4-2 \sqrt{3}}{3}} $

$=\frac{\frac{1}{3}(6-\sqrt{3})}{3-4+2 \sqrt{3}}$

$=\frac{\frac{\sqrt{3}}{3}(2 \sqrt{3}-1)}{2 \sqrt{3}-1}=\frac{\sqrt{3}}{3}$

<<<P>>>27. 그림과 같이 $\overline{\mathrm{AB}_{1}}=1, \overline{\mathrm{B}_{1} \mathrm{C}_{1}}=2$ 인 직사각형 $\mathrm{AB}_{1} \mathrm{C}_{1} \mathrm{D}_{1}$ 이 있다.

$\angle \mathrm{AD}_{1} \mathrm{C}_{1}$ 을 삼등분하는 두 직선이 선분 $\mathrm{B}_{1} \mathrm{C}_{1}$ 과 만나는 점 중 점 $\mathrm{B}_{1}$ 에 가까운 점을 $\mathrm{E}_{1}$,
점 $\mathrm{C}_{1}$ 에 가까운 점을 $\mathrm{F}_{1}$ 이라 하자.

$\overline{\mathrm{E}_{1} \mathrm{F}_{1}}=\overline{\mathrm{F}_{1} \mathrm{G}_{1}}, \angle \mathrm{E}_{1} \mathrm{F}_{1} \mathrm{G}_{1}=\frac{\pi}{2}$ 이고
선분 $\mathrm{AD}_{1}$ 과 선분 $\mathrm{F}_{1} \mathrm{G}_{1}$ 이 만나도록 점 $\mathrm{G}_{1}$ 을 잡아 삼각형 $\mathrm{E}_{1} \mathrm{F}_{1} \mathrm{G}_{1}$ 을 그린다.

선분 $\mathrm{E}_{1} \mathrm{D}_{1}$ 과 선분 $\mathrm{F}_{1} \mathrm{G}_{1}$ 이 만나는 점을 $\mathrm{H}_{1}$ 이라 할 때,

두 삼각형 $\mathrm{G}_{1} \mathrm{E}_{1} \mathrm{H}_{1}, \mathrm{H}_{1} \mathrm{F}_{1} \mathrm{D}_{1}$ 로 만들어진 [IMG] 모양의 도형에 색칠하여 얻은 그림을 $R_{1}$ 이라 하자.

그림 $R_{1}$ 에 선분 $\mathrm{AB}_{1}$ 위의 점 $\mathrm{B}_{2}$, 선분 $\mathrm{E}_{1} \mathrm{G}_{1}$ 위의 점 $\mathrm{C}_{2}$, 선분 $\mathrm{AD}_{1}$ 위의 점 $\mathrm{D}_{2}$ 와 점 $\mathrm{A}$ 를 꼭짓점으로 하고 

$\overline{\mathrm{AB}_{2}}: \overline{\mathrm{B}_{2} \mathrm{C}_{2}}=1: 2$ 인 직사각형 $\mathrm{AB}_{2} \mathrm{C}_{2} \mathrm{D}_{2}$ 를 그린다.

직사각형 $\mathrm{AB}_{2} \mathrm{C}_{2} \mathrm{D}_{2}$ 에 그림 $R_{1}$ 을 얻은 것과 같은 방법으로 [IMG] 모양의 도형을 그리고 색칠하여 얻은 그림을 $R_{2}$ 라 하자.

이와 같은 과정을 계속하여 $n$ 번째 얻은 그림 $R_{n}$ 에 색칠되어 있는 부분의 넓이를 $S_{n}$ 이라 할 때, $\lim _{n \rightarrow \infty} S_{n}$ 의 값은? [3점]


[IMG]

① $\frac{2 \sqrt{3}}{9}$
② $\frac{5 \sqrt{3}}{18}$
③ $\frac{\sqrt{3}}{3}$
④ $\frac{7 \sqrt{3}}{18}$
⑤ $\frac{4 \sqrt{3}}{9}$


<<<A>>>①

<<<S>>>



$\mathrm{A}^{\prime}(0,2)$ 라고 하자. $\frac{\pi}{6} \leq \theta \leq \frac{\pi}{3}$ 일 때 $1 \leq 2 \cos \theta \leq \sqrt{3}$ 이므로 점 Q는 선분 $\mathrm{OA}^{\prime}$ 위에 있고, 점 $\mathrm{R}$ 는 선분 $\mathrm{PB}$ 위 에 있다.

$\overline{\mathrm{A}^{\prime} \mathrm{B}}$ 는 지름이므로 $\angle \mathrm{A}^{\prime} \mathrm{PB}=\frac{\pi}{2}$ 이고 원주각 성 질에 의해 $\angle \mathrm{PA}^{\prime} \mathrm{B}=\angle \mathrm{P} \mathrm{AB}=\theta$ 이다.
점 $\mathrm{Q}$ 에서 $\overline{\mathrm{A}^{\prime} \mathrm{P}}$ 에 내린 수선의 발을 $\mathrm{H}$ 라고 하면 사각형 $\mathrm{QHPR}$ 는 직사각형이므로 $\overline{\mathrm{PR}}=\overline{\mathrm{QH}}$ 이다.

$\overline{\mathrm{A}^{\prime} \mathrm{Q}}=2-2 \cos \theta$ 이므로
직각삼각형 $\mathrm{A}^{\prime} \mathrm{QH}$ 에서
$\overline{\mathrm{QH}}=(2-2 \cos \theta) \sin \theta$ 이다.
$\therefore f(\theta)=(2-2 \cos \theta) \sin \theta$

$\therefore \quad \int_{\frac{\pi}{6}}^{\frac{\pi}{3}} f(\theta) d \theta$

$=\int_{\frac{\pi}{6}}^{\frac{\pi}{3}}(2-2 \cos \theta) \sin \theta d \theta$

$=\int_{\frac{\sqrt{3}}{2}}^{\frac{1}{2}}(2-2 t)(-d t)(\cos \theta=t$ 로 치환 $)$

$=\int_{\frac{1}{2}}^{\frac{\sqrt{3}}{2}}(2-2 t) d t$

$=\left[2 t-t^{2}\right]_{\frac{1}{2}}^{\frac{\sqrt{3}}{2}}=\left(\sqrt{3}-\frac{3}{4}\right)-\left(1-\frac{1}{4}\right)$

$=\sqrt{3}-\frac{3}{2}=\frac{2 \sqrt{3}-3}{2}$


<<<P>>>28. 좌표평면에서 원점을 중심으로 하고 반지름의 길이가 2 인 원 $C$ 와 두 점 $\mathrm{A}(2,0), \mathrm{B}(0,-2)$ 가 있다. 원 $C$ 위에 있고 $x$ 좌표가 음수인 점 $\mathrm{P}$ 에 대하여 $\angle \mathrm{PAB}=\theta$ 라 하자.
점 $\mathrm{Q}(0,2 \cos \theta)$ 에서 직선 $\mathrm{BP}$ 에 내린 수선의 발을 $\mathrm{R}$ 라 하고, 두 점 $\mathrm{P}$ 와 $\mathrm{R}$ 사이의 거리를 $f(\theta)$ 라 할 때, $\int_{\frac{\pi}{6}}^{\frac{\pi}{3}} f(\theta) d \theta$ 의 값은? [4점]

① $\frac{2 \sqrt{3}-3}{2}$
② $\sqrt{3}-1$
③ $\frac{3 \sqrt{3}-3}{2}$
④ $\frac{2 \sqrt{3}-1}{2}$
⑤ $\frac{4 \sqrt{3}-3}{2}$


[IMG]


<<<A>>>$24$

<<<S>>>



$f(x)$ 의 최고차항의 계수가 양수이면 $x \rightarrow \pm \infty$ 일 때 $f(x) \rightarrow \infty$ 이므로 함수 $g(x)$ 는 최댓값을 갖지 않는다.
따라서 $f(x)$ 의 최고차항의 계수는 음수이다. $x \rightarrow \pm \infty$ 일 때 $f(x) \rightarrow-\infty$ 이므로 $g(x)<0$ 이며 $g(x) \rightarrow 0$ 이다.
또, 이차함수의 그래프는 대칭축에 대하여 대칭이므 로 $y=g(x)$ 의 그래프도 $y=f(x)$ 의 그래프의 대 칭축에 대하여 대칭이다.
따라서 $y=g(x)$ 의 그래프의 개형은 다음과 같다.

[IMG]

이때 $b$와 $b+6$이 $a$에 대하여 대칭이므로 $a=b+3$ 

즉, 방정식 $g^{\prime}(x)=0$ 의 서로 다른 세 실근이 $b, b+3, b+6$ 이다.

$g^{\prime}(x)=f^{\prime}(x) \times e^{f(x)}+\{f(x)+2\} e^{f(x)} \times f^{\prime}(x)$

$=e^{f(x)} \times f^{\prime}(x) \times\{f(x)+3\}=0$ 에서 방정식 $f(x)+3=0$ 의 서로 다른 두 실근이 $b, b+6$ 이고

방정식 $f^{\prime}(x)=0$ 의 실근이 $b+3$ 이다.

음수 $k$ 에 대하여 $f(x)+3=k(x-b)(x-b-6)$ 이라 하자.

$x=a=b+3$ 을 대입하면 $6+3=k \times 3 \times(-3)$ 에서 $k=-1$ 이다.

$\therefore f(x)=-(x-b)(x-b-6)-3$

방정식 $f(x)=0$ 의 서로 다른 두 실근이 $\alpha, \beta$ 이므로 방정식 $f(x+b)=0$ 의 서로 다른 두 실근은 $\alpha-b, \beta-b$ 이다.

이때 $\alpha-\beta=(\alpha-b)-(\beta-b)$ 이므로 $(\alpha-\beta)^{2}$ 의 값은 방정식 $f(x+b)=0$ 에서 구할 수 있다.

$f(x+b)=-x(x-6)-3$이므로 $x^{2}-6 x+3=0$에서 $x=3 \pm \sqrt{6}$이다.

$\therefore(\alpha-\beta)^{2}=(2 \sqrt{6})^{2}=24$



<<<P>>>29. 이차함수 $f(x)$ 에 대하여 함수 $g(x)=\{f(x)+2\} e^{f(x)}$ 이 다음 조건을 만족시킨다.

(가) $f(a)=6$ 인 $a$ 에 대하여 $g(x)$ 는 $x=a$ 에서 최댓값을 갖는다.

(나) $g(x)$ 는 $x=b, x=b+6$ 에서 최솟값을 갖는다.

방정식 $f(x)=0$ 의 서로 다른 두 실근을 $\alpha, \beta$ 라 할 때, $(\alpha-\beta)^{2}$ 의 값을 구하시오. (단, $a, b$ 는 실수이다.) [4점]



<<<A>>>$115$

<<<S>>>



조건 (가)에서 (분모) $\rightarrow 0$ 일 때 극한이 존재하므로 (분자)$\rightarrow 0$이다.

$\therefore \lim _{x \rightarrow 0} \sin (\pi f(x))=\sin (\pi f(0))=0 $

$\therefore f(0)=n$ ($n$은 정수)

$p(x)=\sin (\pi f(x))$라고 하면

$p(0)=\sin (n \pi)=0$이므로 조건 (가)에서

$\lim _{x \rightarrow 0} \frac{p(x)}{x}=p^{\prime}(0)=0$

$p^{\prime}(x)=\cos (\pi f(x)) \times \pi f^{\prime}(x)$에서

$p^{\prime}(0)=\cos (n \pi) \times \pi f^{\prime}(0)=0$이므로

$f^{\prime}(0)=0$이다.

따라서 $f(x)=9 x^{3}-a x^{2}+n$ (단, $a$ 는 상수 $)$ 이라 할 수 있다.

함수 $g(x)$ 가 $x=1$ 에서 연속이므로 $\lim _{x \rightarrow 1-} g(x)=\lim _{x \rightarrow 1+} g(x)$

이때 $\lim _{x \rightarrow 1-} g(x)=\lim _{x \rightarrow 1-} f(x)=f(1)=9-a+n$ 이고

$\lim _{x \rightarrow 1+} g(x) $

$=\lim _{x \rightarrow 1+} g(x-1)$

$=\lim _{t \rightarrow 0+} g(t)$

$=\lim _{t \rightarrow 0+} f(t)$

$=f(0)=n$ 이므로 $9-a+n=n$ 에서 $a=9$ 이다.

$\therefore f(x)=9 x^{3}-9 x^{2}+n$

$f^{\prime}(x)=27 x^{2}-18 x=9 x(3 x-2)$이므로

함수 $f(x)$ 는 $x=0$ 에서 극대, $x=\frac{2}{3}$ 에서 극소이다.

$f(0)=n$이고

$f\left(\frac{2}{3}\right)=9 \times \frac{8}{27}-9 \times \frac{4}{9}+n=n-\frac{4}{3}$이므로

$n\left(n-\frac{4}{3}\right)=5$

정리하면 $(n-3)(3 n+5)=0$ 이고 $n$ 은 정수이므로 $n=3$ 이다.

$\therefore f(x)=9 x^{3}-9 x^{2}+3$

정수 $m$ 에 대하여

$\int_{m}^{m+1} x g(x) d x$

$=\int_{0}^{1}(x+m) g(x+m) d x$

$=\int_{0}^{1}(x+m) g(x) d x$

$=\int_{0}^{1}(x+m) f(x) d x$

$=\int_{0}^{1} x f(x) d x+m \int_{0}^{1} f(x) d x$

$\int_{0}^{1} x f(x) d x=\int_{0}^{1}\left(9 x^{4}-9 x^{3}+3 x\right) d x$

$=\left[\frac{9}{5} x^{5}-\frac{9}{4} x^{4}+\frac{3}{2} x^{2}\right]_{0}^{1}$

$=\frac{9}{5}-\frac{9}{4}+\frac{3}{2}=\frac{36-45+30}{20}=\frac{21}{20}$

$\int_{0}^{1} f(x) d x=\int_{0}^{1}\left(9 x^{3}-9 x^{2}+3\right) d x$

$=\left[\frac{9}{4} x^{4}-3 x^{3}+3 x\right]_{0}^{1}=\frac{9}{4}$

이므로 $\int_{m}^{m+1} x g(x) d x=\frac{21}{20}+\frac{9}{4} m$ 이다.

따라서 $\int_{0}^{5} x g(x) d x $
$=\sum_{m=0}^{4}\left(\frac{21}{20}+\frac{9}{4} m\right)$

$=\frac{21}{20} \times 5+\frac{9}{4} \times \frac{4 \times 5}{2}$

$=\frac{21}{4}+\frac{45}{2}$

$=\frac{111}{4}$

$\therefore p+q=4+111=115$


<<<P>>>30. 최고차항의 계수가 9 인 삼차함수 $f(x)$ 가 다음 조건을 만족시킨다.

(가) $\lim _{x \rightarrow 0} \frac{\sin (\pi \times f(x))}{x}=0$

(나) $f(x)$ 의 극댓값과 극솟값의 곱은 5 이다.

함수 $g(x)$ 는 $0 \leq x<1$ 일 때 $g(x)=f(x)$ 이고 모든 실수 $x$ 에 대하여 $g(x+1)=g(x)$ 이다.
$g(x)$ 가 실수 전체의 집합에서 연속일 때, $\int_{0}^{5} x g(x) d x=\frac{q}{p}$ 이다. $p+q$ 의 값을 구하시오. (단, $p$ 와 $q$ 는 서로소인 자연수이다.) [4점]



<<<A>>>⑤

<<<S>>>



점 $\mathrm{A}$ 를 $x y$ 평면에 대하여 대칭이동하면 $z$ 좌표의 부호만 반대로 하면 되므로 $\mathrm{B}(3,0,2), \mathrm{C}(0,4,2)$

$\therefore \overline{\mathrm{BC}}=\sqrt{3^{2}+4^{2}}=5$


<<<P>>>23. 좌표공간의 점 $\mathrm{A}(3,0,-2)$ 를 $x y$ 평면에 대하여 대칭이동한 점을 $\mathrm{B}$ 라 하자. 점 $\mathrm{C}(0,4,2)$ 에 대하여 선분 $\mathrm{BC}$ 의 길이는? [2점]

① $1$
② $2$
③ $3$
④ $4$
⑤ $5$


<<<A>>>④

<<<S>>>



(점근선의 기울기 )$=\pm \frac{4}{a}$ 이므로 $\frac{4}{a}=3$

$\therefore a=\frac{4}{3}$


<<<P>>>24. 쌍곡선 $\frac{x^{2}}{a^{2}}-\frac{y^{2}}{16}=1$ 의 점근선 중 하나의 기울기가 3 일 때, 양수 $a$ 의 값은? [3점]

① $\frac{1}{3}$
② $\frac{2}{3}$
③ $1$
④ $\frac{4}{3}$
⑤ $\frac{5}{3}$


<<<A>>>②

<<<S>>>



$ \overrightarrow{p}=(x, y)$ 라 하고 원점을 $\mathrm{O}$ 라 하자.

$\overrightarrow{p} \cdot \overrightarrow{a}=\overrightarrow{a} \cdot \overrightarrow{b}$에서 $(x, y) \cdot(3,0)=(3,0) \cdot(1,2)$

$3 x=3 \quad \therefore x=1$

따라서 $\overrightarrow{p}=\overrightarrow{\mathrm{OP}}$ 에서 점 $\mathrm{P}$ 는 직선 $x=1$ 위의 점이다.

$\overrightarrow{q}=\overrightarrow{\mathrm{OQ}}$ 라 하면 $|\overrightarrow{q}-\overrightarrow{c}|=1$ 에서 점 $\mathrm{Q}$ 는 점 $(4,2)$ 를 중심으로 하고 반지름의 길이가 1 인 원 위의 점이다.

$|\overrightarrow{p}-\overrightarrow{q}|=|\overrightarrow{\mathrm{QP}}|$ 이므로 두 점 $\mathrm{P}, \mathrm{Q}$ 사이의 거리의 최솟값은 $4-1-1=2$


<<<P>>>25. 좌표평면에서 세 벡터 $\overrightarrow{a}=(3,0), \quad \overrightarrow{b}=(1,2), \quad \overrightarrow{c}=(4,2)$에 대하여 두 벡터 $\overrightarrow{p}, \overrightarrow{q}$ 가
$\overrightarrow{p} \cdot \overrightarrow{a}=\overrightarrow{a} \cdot \overrightarrow{b}, \quad|\overrightarrow{q}-\overrightarrow{c}|=1$을 만족시킬 때, $|\overrightarrow{p}-\overrightarrow{q}|$ 의 최솟값은? [3점]

① $1$
② $2$
③ $3$
④ $4$
⑤ $5$



<<<A>>>③

<<<S>>>



문제의 조건에 따라 $\overline{\mathrm{AB}}=\overline{\mathrm{BF}}$ 이고 포물선의 정 의에서 $\overline{\mathrm{AB}}=\overline{\mathrm{AF}}$ 이므로 $\triangle \mathrm{ABF}$ 는 정삼각형이고 $\angle \mathrm{BFO}=60^{\circ}$ 이다.
또, 점 $\mathrm{C}$ 에서 준선에 내린 수선의 발을 $\mathrm{H}$ 라 하면 포물선의 정의에서 $\overline{\mathrm{CF}}=\overline{\mathrm{CH}}=k$ 로 놓자.
또, $\angle \mathrm{BCH}=\angle \mathrm{CFO}=60^{\circ}$ 이므로

$\overline{\mathrm{CH}}=k$이면 $\overline{\mathrm{BC}}=2 k$이다.

따라서 $\overline{\mathrm{BC}}+3 \overline{\mathrm{CF}}=6$에서

$2 k+3 k=6$, 즉 $k=\frac{6}{5}$이고

점 $\mathrm{F}$ 에서 준선에 내린 수선의 발을 $\mathrm{H}^{\prime}$ 이라 하면

$\overline{\mathrm{FH}^{\prime}}=2 p=k+\frac{k}{2}=\frac{9}{5}$

$p=\frac{9}{10}$


<<<P>>>26. 초점이 $\mathrm{F}$ 인 포물선 $y^{2}=4 p x$ 위의 한 점 $\mathrm{A}$ 에서 포물선의 준선에 내린 수선의 발을 $\mathrm{B}$ 라 하고, 선분 $\mathrm{BF}$ 와 포물선이 만나는 점을 $\mathrm{C}$ 라 하자. $\overline{\mathrm{AB}}=\overline{\mathrm{BF}}$ 이고 $\overline{\mathrm{BC}}+3 \overline{\mathrm{CF}}=6$ 일 때, 양수 $p$ 의 값은? [3점]

① $\frac{7}{8}$
② $\frac{8}{9}$
③ $\frac{9}{10}$
④ $\frac{10}{11}$
⑤ $\frac{11}{12}$


[IMG]


<<<A>>>①

<<<S>>>



$ \overline{\mathrm{DP}}$ 는 점 $\mathrm{P}$ 가 점 $\mathrm{D}$ 에서 선분 $\mathrm{BC}$ 에 내린 수선 의 발 $\mathrm{H}$ 일 때 최소이고, $\overline{\mathrm{DB}} \times \overline{\mathrm{DC}}=\overline{\mathrm{BC}} \times \overline{\mathrm{DH}}$ 에서 $(\overline{\mathrm{DP}}$ 의 최솟값 $)=\overline{\mathrm{DH}}=\sqrt{3}$
또, $\overline{\mathrm{AD}}$ 와 $\overline{\mathrm{DP}}$ 는 서로 수직이므로 $\overline{\mathrm{AP}}^{2}=\overline{\mathrm{AD}}^{2}+\overline{\mathrm{DP}}^{2}=3^{2}+\overline{\mathrm{DP}}^{2}$ 에서 $\overline{\mathrm{AP}}$ 는 $\overline{\mathrm{DP}}$ 가 최소일 때 최소이다. 따라서 $\overline{\mathrm{AP}}+\overline{\mathrm{DP}}$ 의 최솟값은 $\sqrt{3^{2}+(\sqrt{3})^{2}}+\sqrt{3}=3 \sqrt{3}$


<<<P>>>27. 그림과 같이 $\overline{\mathrm{AD}}=3, \overline{\mathrm{DB}}=2, \overline{\mathrm{DC}}=2 \sqrt{3}$ 이고 $\angle \mathrm{ADB}=\angle \mathrm{ADC}=\angle \mathrm{BDC}=\frac{\pi}{2}$ 인 사면체 $\mathrm{ABCD}$ 가 있다. 선분 $\mathrm{BC}$ 위를 움직이는 점 $\mathrm{P}$ 에 대하여 $\overline{\mathrm{AP}}+\overline{\mathrm{DP}}$ 의 최솟값은? [3점]


[IMG]

① $3 \sqrt{3}$
② $\frac{10 \sqrt{3}}{3}$
③ $\frac{11 \sqrt{3}}{3}$
④ $4 \sqrt{3}$
⑤ $\frac{13 \sqrt{3}}{3}$



<<<A>>>①

<<<S>>>



$ \mathrm{P}(2,3)$ 에서 접선의 방정식은 $\frac{2 x}{16}+\frac{3 y}{12}=1$ 에서 $y=-\frac{1}{2} x+4$ 이고 $x$ 축과의 교점 $\mathrm{S}$ 의 좌표는 $(8,0)$ 이다.
또, $\mathrm{F}(c, 0)$ 이라 하면

$c^{2}=16-12=4$

$c=2$이므로 $\mathrm{F}^{\prime}(-2,0), \mathrm{F}(2,0)$이다.

이때 두 삼각형 $\mathrm{QF}^{\prime} \mathrm{F}$ 와 $\mathrm{RF}^{\prime} \mathrm{S}$ 는 $\angle \mathrm{QF}^{\prime} \mathrm{F}$ 는 공통이고 $\angle \mathrm{QFF}^{\prime}=\angle \mathrm{RSF}^{\prime}$ 이므로 닮음이다.
또, 그 닮음비는 $\overline{\mathrm{F}^{\prime} \mathrm{F}}: \overline{\mathrm{F}^{\prime} \mathrm{S}}=4: 10=2: 5$ 이다.
따라서 ($\triangle \mathrm{SRF}^{\prime}$의 둘레의 길이)

$=\frac{5}{2} \times$ ( $\triangle \mathrm{FQF}^{\prime}$의 둘레의 길이)

$=\frac{5}{2} \times(8+4)=30$


<<<P>>>28. 그림과 같이 두 점 $\mathrm{F}(c, 0), \mathrm{F}^{\prime}(-c, 0)(c>0)$ 을 초점으로 하는 타원 $\frac{x^{2}}{16}+\frac{y^{2}}{12}=1$ 위의 점 $\mathrm{P}(2,3)$ 에서 타원에 접하는 직선을 $l$ 이라 하자. 점 $\mathrm{F}$ 를 지나고 $l$ 과 평행한 직선이 타원과 만나는 점 중 제 2 사분면 위에 있는 점을 $\mathrm{Q}$ 라 하자. 두 직선 $\mathrm{F}^{\prime} \mathrm{Q}$ 와 $l$ 이 만나는 점을 $\mathrm{R}, l$ 과 $x$ 축이 만나는 점을 $\mathrm{S}$ 라 할 때, 삼각형 $\mathrm{SRF}^{\prime}$ 의 둘레의 길이는? [4점]


[IMG]

① $30$
② $31$
③ $32$
④ $33$
⑤ $34$



<<<A>>>$40$

<<<S>>>



점 $\mathrm{G}$ 에서 $\overline{\mathrm{AB}}$ 에 내린 수선의 발을 $\mathrm{H}_{1}$ 이라 하면 삼수선의 정리에 의해 점 $\mathrm{P}$ 에서 $\overline{\mathrm{AB}}$ 에 내린 수선 의 발과 일치한다.

마찬가지로 점 $\mathrm{H}$ 에서 $\overline{\mathrm{CD}}$ 에 내린 수선의 발을 $\mathrm{H}_{2}$ 라 하면 점 $\mathrm{Q}$ 에서 $\overline{\mathrm{CD}}$ 에 내 린 수선의 발과 일치한다.

$\overline{\mathrm{PH}}_{1}=2 \sqrt{3}, \overline{\mathrm{QH}}_{2}=4$ 이므로
$\overline{\mathrm{GH}_{1}}=\sqrt{\overline{\mathrm{PH}_{1}}^{2}-\overline{\mathrm{P} G}^{2}}=\sqrt{12-3}=3$,
$\overline{\mathrm{HH}}_{2}=\sqrt{\overline{\mathrm{QH}_{2}}^{2}-\overline{\mathrm{QH}}^{2}}=2$

[IMG]

위 그림에서 $\overline{\mathrm{GH}}=\sqrt{13}$

$\overline{\mathrm{CQ}}=\sqrt{\overline{\mathrm{QH}}^{2}+\overline{\mathrm{CH}}^{2}}=\sqrt{32}=4 \sqrt{2}$

$\overline{\mathrm{CP}}=\sqrt{\overline{\mathrm{CG}}^{2}+\overline{\mathrm{GP}}^{2}}=\sqrt{32}=4 \sqrt{2}$

$\overline{\mathrm{PQ}}=\sqrt{\overline{\mathrm{GH}}^{2}+(\overline{\mathrm{QH}}-\overline{\mathrm{P} G})^{2}}=4$

$\triangle \mathrm{P} \mathrm{QC}$ 는 세 변의 길이가 $4 \sqrt{2}, 4 \sqrt{2}, 4$ 이므로

( $\triangle \mathrm{PQC}$ 의 넓이 )$=4 \sqrt{7}$ 이고 점 $\mathrm{G}$ 에서 $\overline{\mathrm{CD}}$ 에 내린 수선의 발을 $\mathrm{H}_{3}$ 이라 하면

$\triangle \mathrm{GHC}$

$=\triangle \mathrm{GH}_{3} \mathrm{C}+\square \mathrm{GHH}_{2} \mathrm{H}_{3}-\triangle \mathrm{HH}_{2} \mathrm{C}$

$=5+7-4=8$

$\cos \theta=\frac{\triangle \mathrm{GHC}}{\triangle \mathrm{P} \mathrm{QC}}=\frac{8}{4 \sqrt{7}}=\frac{2}{\sqrt{7}}$

$\therefore 70 \cos ^{2} \theta=70 \times \frac{4}{7}=40$


<<<P>>>29. 그림과 같이 한 변의 길이가 8 인 정사각형 $\mathrm{ABCD}$ 에 두 선분 $\mathrm{AB}, \mathrm{CD}$ 를 각각 지름으로 하는 두 반원이 붙어 있는 모양의 종이가 있다. 반원의 호 $\mathrm{AB}$ 의 삼등분점 중 점 $\mathrm{B}$ 에 가까운 점을 $\mathrm{P}$ 라 하고, 반원의 호 $\mathrm{CD}$ 를 이등분하는 점을 $\mathrm{Q}$ 라 하자. 이 종이에서 두 선분 $\mathrm{AB}$ 와 $\mathrm{CD}$ 를 접는 선으로 하여 두 반원을 접어 올렀을 때 두 점 $\mathrm{P}, \mathrm{Q}$ 에서 평면 $\mathrm{ABCD}$ 에 내린 수선의 발을 각각 $\mathrm{G}, \mathrm{H}$ 라 하면 두 점 $\mathrm{G}, \mathrm{H}$ 는 정사각형 $\mathrm{ABCD}$ 의 내부에 놓여 있고, $\overline{\mathrm{PG}}=\sqrt{3}, \overline{\mathrm{QH}}=2 \sqrt{3}$ 이다. 두 평면 $\mathrm{PCQ}$ 와 $\mathrm{ABCD}$ 가 이루는 각의 크기가 $\theta$ 일 때, $70 \times \cos ^{2} \theta$ 의 값을 구하시오. (단, 종이의 두께는 고려하지 않는다.) [4점]


[IMG]


<<<A>>>$45$

<<<S>>>



$\overrightarrow{\mathrm{AP}} \cdot \overrightarrow{\mathrm{OC}} \geq \frac{\sqrt{2}}{2}$ 에서 $\overrightarrow{\mathrm{AP}}$ 와 $\overrightarrow{\mathrm{OC}}$ 가 이루는
각을 $\alpha$ 라 하면

$1 \times 1 \times \cos \alpha \geq \frac{\sqrt{2}}{2} $

$\cos \alpha \geq \frac{\sqrt{2}}{2}$이므로 $0 \leq \alpha \leq \frac{\pi}{4}$ 이다. 또

$\overrightarrow{\mathrm{AP}} \cdot \overrightarrow{\mathrm{AQ}}=\overrightarrow{\mathrm{AP}} \cdot(\overrightarrow{\mathrm{AB}}+\overrightarrow{\mathrm{BQ}})$

$=\overrightarrow{\mathrm{AP}} \cdot \overrightarrow{\mathrm{AB}}+\overrightarrow{\mathrm{AP}} \cdot \overrightarrow{\mathrm{BQ}}$

에서 $\overrightarrow{\mathrm{AP}} \cdot \overrightarrow{\mathrm{AB}}$ 는 $\overrightarrow{\mathrm{AP}}$ 와 $\overrightarrow{\mathrm{AB}}$ 가 이루는 각이 최대일 때 최소이고

그 때의 점 $\mathrm{P}_{0}$ 에 대해 $\overrightarrow{\mathrm{AP}_{0}}$ 과 $\overrightarrow{\mathrm{BQ}}$ 가 반대 방향이면 최소이다.

따라서 점 $\mathrm{P}_{0}$ 과 점 $\mathrm{Q}_{0}$ 은 다음 그림과 같다.

[IMG]

또, $\overrightarrow{\mathrm{BX}} \cdot \overrightarrow{\mathrm{BQ}}_{0} \geq 1$ 에서 $\overrightarrow{\mathrm{AP}_{0}}$ 위의 점 $\mathrm{X}$ 에서 $\overrightarrow{\mathrm{BQ}_{0}}$ 에 내린 수선의 발의 위치는 점 $\mathrm{B}$ 에서 점 $\mathrm{Q}_{0}$ 방향으로 $\frac{1}{2}$ 이상 떨어져야 하므로
$\overrightarrow{\mathrm{BX}} \cdot \overrightarrow{\mathrm{BQ}_{0}}=1$ 이 되는 점 $\mathrm{X}$ 를 점 $\mathrm{X}_{0}$ 이라 하면 $|\overrightarrow{\mathrm{QX}}|^{2}$ 의 최댓값은 $\left|\overrightarrow{\mathrm{Q} X_{0}}\right|^{2}$ 이다.

$\therefore\left|\overrightarrow{\mathrm{Q} \mathrm{X}_{0}}\right|^{2}=\left(\frac{3}{2}\right)^{2}+(2 \sqrt{2})^{2}=\frac{41}{4}$


<<<P>>>2022학년도 대학수학능력시험 9월 모의평가 끝

30. 좌표평면에서 세 점 $\mathrm{A}(-3,1), \mathrm{B}(0,2), \mathrm{C}(1,0)$ 에 대하여 두 점 $\mathrm{P}, \mathrm{Q}$ 가

$|\overrightarrow{\mathrm{AP}}|=1, \quad|\overrightarrow{\mathrm{BQ}}|=2, \quad \overrightarrow{\mathrm{AP}} \cdot \overrightarrow{\mathrm{OC}} \geq \frac{\sqrt{2}}{2}$

를 만족시킬 때, $\overrightarrow{\mathrm{AP}} \cdot \overrightarrow{\mathrm{AQ}}$ 의 값이 최소가 되도록 하는 두 점 $\mathrm{P}, \mathrm{Q}$ 를 각각 $\mathrm{P}_{0}, \mathrm{Q}_{0}$ 이라 하자.

선분 $\mathrm{AP}_{0}$ 위의 점 $\mathrm{X}$ 에 대하여 $\overrightarrow{\mathrm{BX}} \cdot \overrightarrow{\mathrm{BQ}_{0}} \geq 1$ 일 때, $\left|\overrightarrow{\mathrm{Q}_{0} \mathrm{X}}\right|^{2}$ 의 최댓값은 $\frac{q}{p}$ 이다. $p+q$ 의 값을 구하시오. (단, $\mathrm{O}$ 는 원점이고, $p$ 와 $q$ 는 서로소인 자연수이다.) [4점]


\end{document}
