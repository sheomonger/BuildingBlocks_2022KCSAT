
%!TEX program=xelatex
\documentclass{oblivoir}
%\usepackage{amsmath}
\usepackage{ikps}
\setlength\parindent{0pt}
\linespread{1.5} 

\begin{document}


<<<A>>>②

<<<S>>>출제의도 : 지수법칙을 이용하여 식의 값을 구할 수 있는가?

[[지수법칙을 적용하기 위해 밑을 일치시킵니다.]]

$\left(2^{\sqrt{3}} \times 4\right)^{\sqrt{3}-2}$

$=\left(2^{\sqrt{3}} \times 2^{2}\right)^{\sqrt{3}-2}$

[[지수법칙을 이용하여 지수를 계산합니다.]]

$=\left(2^{\sqrt{3}+2}\right)^{\sqrt{3}-2}$

$=2^{(\sqrt{3}+2)(\sqrt{3}-2)}$

$=2^{3-4}$

$=2^{-1}$

$=\dfrac{1}{2}$

<<<P>>>32111 2022학년도 대학수학능력시험 시작

1. $\left(2^{\sqrt{3}} \times 4\right)^{\sqrt{3}-2}$ 의 값은? [2점]

<<<C>>>① $\dfrac{1}{4}$
② $\dfrac{1}{2}$
③ $1$
④ $2$
⑤ $4$


<<<A>>>⑤

<<<S>>>출제의도 : 다항함수의 미분계수를 구할 수 있는가?

[[주어진 함수를 미분하여 도함수를 구합니다.]]

$f^{\prime}(x)=3 x^{2}+6 x+1$이므로

[[[도함수에 $x=1$을 대입하여 미분계수를 구합니다.]]

$f^{\prime}(1)=3+6+1=10$

<<<P>>>2. 함수 $f(x)=x^{3}+3 x^{2}+x-1$ 에 대하여 $f^{\prime}(1)$ 의 값은? [2점]

<<<C>>>① $6$
② $7$
③ $8$
④ $9$
⑤ $10$

<<<A>>>⑤

<<<S>>>출제의도 : 등차수열의 일반항을 이해하고 있는가?

[[첫째항과 공차를 미지수로 도입합니다.]]

등차수열 $\left\{a_{n}\right\}$ 의 공차를 $d$ 라 하면

[[주어진 조건으로부터 첫째항과 공차에 대한 연립방정식을 세웁니다.]]

$a_{2}=a_{1}+d=6 \quad \cdots \cdots$ ㉠

$a_{4}+a_{6}=36$ 에서
$\left(a_{1}+3 d\right)+\left(a_{1}+5 d\right)=36$

$2 a_{1}+8 d=36$

$a_{1}+4 d=18 \quad \cdots \cdots$ ㉡

[[[앞에서 구한 연립방정식을 풉니다.]]

㉠, ㉡에서
$a_{1}=2, d=4$

[[[연립방정식의 결과를 이용하여 문제에서 요구하는 것을 구합니다.]]

따라서
$a_{10}=2+9 \times 4=38$

<<<P>>>3. 등차수열 $\left\{a_{n}\right\}$ 에 대하여
$$
a_{2}=6, \quad a_{4}+a_{6}=36
$$
일 때, $a_{10}$ 의 값은? [3점]

<<<C>>>① $30$
② $32$
③ $34$
④ $36$
⑤ $38$

<<<A>>>④

<<<S>>>출제의도 : 함수의 극한값을 구할 수 있는가?

[[주어진 점에서 좌극한값을 구합니다.]]

$x \rightarrow-1-$일 때, $f(x) \rightarrow 3$ 이므로 $\displaystyle\lim _{x \rightarrow-1-} f(x)=3$

[[주어진 점에서 극한값을 구합니다.]]

또, $x \rightarrow 2$ 일 때, $f(x) \rightarrow 1$ 이므로 $\displaystyle\lim _{x \rightarrow 2} f(x)=1$

[[[앞에서 구한 극한값의 합을 구합니다.]]

따라서,
$\displaystyle\lim _{x \rightarrow-1-} f(x)+\displaystyle\lim _{x \rightarrow 2} f(x)=3+1=4$

<<<P>>>4. 함수 $y=f(x)$ 의 그래프가 그림과 같다.

[IMG]

$\displaystyle\lim _{x \rightarrow-1-} f(x)+\displaystyle\lim _{x \rightarrow 2} f(x)$ 의 값은? [3점]

<<<C>>>① $1$
② $2$
③ $3$
④ $4$
⑤ $5$

<<<A>>>①

<<<S>>>출제의도 : 귀납적으로 정의된 수열의 합을 구할 수 있는가?


[[주어진 수열의 관계식을 이용하여 첫째항부터 차례대로 항들을 구합니다.]]

$a_{1}=1$ 이므로 $a_{2}=2$

$a_{2}=2$ 이므로 $a_{3}=4$

$a_{3}=4$ 이므로 $a_{4}=8$

$a_{4}=8$ 이므로 $a_{5}=1$

$a_{5}=1$ 이므로 $a_{6}=2$

$a_{6}=2$ 이므로 $a_{7}=4$

$a_{7}=4$ 이므로 $a_{8}=8$

[[차례대로 구한 항으로부터 수열의 규칙성을 조사합니다.]]

[[앞에서 구한 항들을 이용하여 문제에서 요구하는 합을 구합니다.]]

따라서
$\displaystyle\sum_{k=1}^{8} a_{k}=2 \times(1+2+4+8)$

$=2 \times 15$

$=30$

<<<P>>>5. 첫째항이 $1$인 수열 $\left\{a_{n}\right\}$ 이 모든 자연수 $n$ 에 대하여
$$
a_{n+1}=\begin{cases}
2 a_{n} & \left(a_{n}< 7\right) \\
a_{n}-7 & \left(a_{n} \geq 7\right)
\end{cases}
$$
일 때, $\displaystyle\sum_{k=1}^{8} a_{k}$ 의 값은? [3점]

<<<C>>>① $30$
② $32$
③ $34$
④ $36$
⑤ $38$

<<<A>>>③

<<<S>>>출제의도 : 미분을 활용하여 방정식의 실근의 개수를 구할 수 있는가?


[[주어진 방정식을 (다항함수)=$k$의 꼴로 변형합니다.]]

방정식 $2 x^{3}-3 x^{2}-12 x+k=0$,

즉 $2 x^{3}-3 x^{2}-12 x=-k \quad \cdots \cdots$ ㉠

에서 $f(x)=2 x^{3}-3 x^{2}-12 x$ 라 하자.

[[도함수의 부호를 조사하여 함수의 증가, 감소를 표로 나타냅니다.]]

$f^{\prime}(x)=6 x^{2}-6 x-12$

$=6(x+1)(x-2)$

$f^{\prime}(x)=0$ 에서 $x=-1$ 또는 $x=2$

함수 $f(x)$ 의 증가와 감소를 표로 나타내면 다음과 같다.

[IMG]

함수 $f(x)$ 는 $x=-1$ 에서 극댓값 $7$ 을 갖고, $x=2$ 에서 극솟값 $-20$ 을 갖는다.

[[좌변과 우변의 함수의 그래프를 같은 좌표평면에 그립니다.]]

[IMG]

[[문제에서 요구하는 교점의 개수를 갖도록 두 그래프의 조건을 구합니다.]]

방정식 ㉠이 서로 다른 세 실근을 가지려면 함수 $y=f(x)$ 의 그래프와 직선 $y=-k$ 가 서로 다른 세 점에서 만나야 하므로 $-20< -k< 7$

즉, $-7< k< 20$ 이다.

[[[구해진 범위에 속하는 정수의 개수를 구합니다.]]

따라서 정수 $k$ 의 값은 $-6,-5,-4, \cdots, 19$이고, 그 개수는 $26$ 이다.

<<<P>>>6. 방정식 $2 x^{3}-3 x^{2}-12 x+k=0$ 이 서로 다른 세 실근을 갖도록 하는 정수 $k$ 의 개수는? [3점]

<<<C>>>① $20$
② $23$
③ $26$
④ $29$
⑤ $32$

<<<A>>>①

<<<S>>>출제의도 : 삼각함수의 정의와 관계를 이용하여 삼각함수의 값을 구할 수 있는가?

[[양변에 $\tan \theta$를 곱한 다음 우변이 $0$이 되도록 정리합니다.]]

$\tan \theta-\dfrac{6}{\tan \theta}=1$ 이므로

양변에 $\tan \theta$ 를 곱하면
$\tan ^{2} \theta-6=\tan \theta$

$\tan ^{2} \theta-\tan \theta-6=0$

[[좌변을 인수분해하여 $\tan \theta$의 값을 구합니다.]]

$(\tan \theta+2)(\tan \theta-3)=0$

$\tan \theta=-2$ 또는 $\tan \theta=3$

이때, $\pi< \theta< \dfrac{3}{2} \pi$ 이므로
$\tan \theta=3$

[[앞의 결과를 이용하여 $\sin \theta$와 $\cos \theta$의 값을 구합니다.]]

[IMG]

$\theta$가 제$3$사분면의 각이고 $\tan \theta=3 = \dfrac{-3}{-1}$에서

점 $\mathrm{P}$의 좌표를 $\mathrm{P}(-1,-3)$이라 하고, 원점을 $\mathrm{O}$라 하면

$\theta$는 동경 $\mathrm{OP}$가 나타내는 각이므로 삼각함수의 정의에 의해서 

$\cos \theta=-\dfrac{1}{\sqrt{10}} \quad \cdots \cdot$ ㉠

$\sin \theta=-\dfrac{3}{\sqrt{10}} \quad \cdots \cdot$ ㉡

따라서, ㉠과 ㉡에서 $\sin \theta+\cos \theta$

$=\left(-\dfrac{3}{\sqrt{10}}\right)+\left(-\dfrac{1}{\sqrt{10}}\right)$

$=-\dfrac{4}{\sqrt{10}}$

$=-\dfrac{2 \sqrt{10}}{5}$

[다른 풀이]

[[양변에 $\tan \theta$를 곱한 다음 우변이 $0$이 되도록 정리합니다.]]

$\tan \theta-\dfrac{6}{\tan \theta}=1$ 이므로

양변에 $\tan \theta$ 를 곱하면
$\tan ^{2} \theta-6=\tan \theta$

$\tan ^{2} \theta-\tan \theta-6=0$

[[좌변을 인수분해하여 $\tan \theta$의 값을 구합니다.]]

$(\tan \theta+2)(\tan \theta-3)=0$

$\tan \theta=-2$ 또는 $\tan \theta=3$

이때, $\pi< \theta< \dfrac{3}{2} \pi$ 이므로
$\tan \theta=3$

[[앞의 결과를 이용하여 $\sin \theta$와 $\cos \theta$의 값을 구합니다.]]

이때,
$\dfrac{\sin \theta}{\cos \theta}=3$,
$\sin \theta=3 \cos \theta$
이므로

$\sin ^{2} \theta+\cos ^{2} \theta=1$ 에 대입하면

$9 \cos ^{2} \theta+\cos ^{2} \theta=1$

$ \cos ^{2} \theta=\dfrac{1}{10}$이고 $ \sin ^{2} \theta=\dfrac{9}{10}$

이때, $\pi< \theta< \dfrac{3}{2} \pi$ 이므로

$\cos \theta=-\dfrac{1}{\sqrt{10}} \quad \cdots \cdot$ ㉠

$\sin \theta=-\dfrac{3}{\sqrt{10}} \quad \cdots \cdot$ ㉡

따라서, ㉠과 ㉡에서 $\sin \theta+\cos \theta$

$=\left(-\dfrac{3}{\sqrt{10}}\right)+\left(-\dfrac{1}{\sqrt{10}}\right)$

$=-\dfrac{4}{\sqrt{10}}$

$=-\dfrac{2 \sqrt{10}}{5}$



<<<P>>>7. $\pi< \theta< \dfrac{3}{2} \pi$ 인 $\theta$ 에 대하여 $\tan \theta-\dfrac{6}{\tan \theta}=1$ 일 때, $\sin \theta+\cos \theta$ 의 값은? [3점]

<<<C>>>① $-\dfrac{2 \sqrt{10}}{5}$
② $-\dfrac{\sqrt{10}}{5}$
③ $0$
④ $\dfrac{\sqrt{10}}{5}$
⑤ $\dfrac{2 \sqrt{10}}{5}$


<<<A>>>①

<<<S>>>출제의도 : 정적분을 이용하여 곡선과 직선으로 둘러싸인 부분의 넓이를 구할 수 있는가?

[[곡선과 직선이 만나는 점의 $x$좌표를 구합니다.]]

곡선과 직선이 만나는 점의 $x$좌표를 구하면 

$x^{2}-5 x=x$ 에서 $x(x-6)=0$

$x=0$ 또는 $x=6$

곡선 $y=x^{2}-5 x$ 와 직선 $y=x$ 가 만나는 점은 원점과 $(6,6)$ 이다.

[[곡선과 직선의 그래프를 그립니다.]]

[IMG]

[[곡선과 직선으로 둘러싸인 전체 넓이를 정적분으로 구합니다.]]

곡선 $y=x^{2}-5 x$ 와 직선 $y=x$ 로 둘러싸인 부분의 넓이는

$\displaystyle\int_{0}^{6}\left\{x-\left(x^{2}-5 x\right)\right\} d x$

$=\displaystyle\int_{0}^{6}\left(6 x-x^{2}\right) d x$

$=\left[3 x^{2}-\dfrac{1}{3} x^{3}\right]_{0}^{6}$

$=36$

[[전체 넓이를 이등분하는 조건에 맞게 정적분의 관계식을 세웁니다.]]

한편, $\displaystyle\int_{0}^{k}\left\{x-\left(x^{2}-5 x\right)\right\} d x$

$=\displaystyle\int_{0}^{k}\left(6 x-x^{2}\right) d x$

$=\left[3 x^{2}-\dfrac{1}{3} x^{3}\right]_{0}^{k}$

$=3 k^{2}-\dfrac{1}{3} k^{3}$

따라서 직선 $x=k$ 가 넓이를 이등분하므로

$3 k^{2}-\dfrac{1}{3} k^{3} =18$

[[[정리한 다음 인수분해하여 $k$의 값을 구합니다.]]

정리하면 $k^{3}-9 k^{2}+54=0$

$(k-3)\left(k^{2}-6 k-18\right)=0$

따라서 $0< k< 6$ 이므로 $k=3$

<<<P>>>8. 곡선 $y=x^{2}-5 x$ 와 직선 $y=x$ 로 둘러싸인 부분의 넓이를 직선 $x=k$ 가 이등분할 때, 상수 $k$ 의 값은? [3점]

<<<C>>>① $3$
② $\dfrac{13}{4}$
③ $\dfrac{7}{2}$
④ $\dfrac{15}{4}$
⑤ $4$

<<<A>>>④

<<<S>>>출제의도 : 지수방정식의 해를 구할 수 있는가?

[[직선의 기울기를 이용하여 교점의 $x$좌표를 미지수로 도입합니다.]]

두 점 $\mathrm{P},\:\mathrm{Q}$에서 각각 $x$축, $y$축에 평행한 직선을 그어서 두 직선이 만나는 점을 $\mathrm{R}$이라 하자.

직선 $\mathrm{PQ}$의 기울기가 $2$이고, $\overline{\mathrm{PQ}}=\sqrt{5}$이므로 그림에서 $\overline{\mathrm{PR}}=1$, $\overline{\mathrm{QR}}=2$이다.

[IMG]

점 $\mathrm{P}$의 $x$좌표를 $t$라 하면 점 $\mathrm{Q}$의 좌표는 $t+1$이다.

[[교점의 $y$좌표를 이용하여 미지수에 대한 방정식을 세웁니다.]]

$\overline{\mathrm{QR}}=2$에서 

$\overline{\mathrm{QR}}=\left(\dfrac{2}{3}\right)^{(t+1)+1}+\dfrac{8}{3}-\left(\dfrac{2}{3}\right)^{t+3}-1=2$

정리하면 $\left(\dfrac{2}{3}\right)^{t+2}\times\dfrac{1}{3}=\dfrac{1}{3}$

따라서 $t=-2$

이 때, 점 $\mathrm{P}$의 좌표는 $\left(-2,\:\dfrac{5}{3}\right)$이므로 $\dfrac{5}{3}= 2(-2)+k$에서 $k=\dfrac{17}{3}$

[다른 풀이]

[IMG]

[[교점의 $x$좌표를 미지수로 도입하고, 교점을 미지수로 나타냅니다.]]

두 점 $\mathrm{P}, \mathrm{Q}$ 의 $x$ 좌표를 각각 $p, q(p< q)$ 라 하면

두 점 $\mathrm{P}, \mathrm{Q}$ 는 직선 $y=2 x+k$ 위의 점이므로

$\mathrm{P}(p, 2 p+k), \mathrm{Q}(q, 2 q+k)$ 로 놓을 수 있다.

이때, $\overline{\mathrm{PQ}}=\sqrt{5}$,

즉 $\overline{\mathrm{PQ}}^{2}=5$ 이므로

[[교점 사이의 거리를 식으로 나타내고, 그것을 정리하여 미지수 관계식을 구합니다.]]

$(q-p)^{2}+(2 q-2 p)^{2}=5$

$(q-p)^{2}=1$

$q-p>0$ 이므로 $q-p=1$

즉, $q=p+1$ 이다.

[[각 교점의 $x$좌표를 주어진 함수에 대입하여 연립방정식을 세웁니다.]]

한편, 점 $\mathrm{P}$ 는 함수 $y=\left(\dfrac{2}{3}\right)^{x+3}+1$ 의 그래프 위의 점이므로

$\left(\dfrac{2}{3}\right)^{p+3}+1=2 p+k \quad \ldots \ldots$ ㉠

점 $\mathrm{Q}$ 는 함수 $y=\left(\dfrac{2}{3}\right)^{x+1}+\dfrac{8}{3}$ 의 그래프
위의 점이므로

$\left(\dfrac{2}{3}\right)^{p+2}+\dfrac{8}{3}=2 p+k+2 \quad \cdots \cdots$ ㉡

[[구해진 연립방정식을 풉니다.]]

㉠, ㉡에서

$\left(\dfrac{2}{3}\right)^{p+2}+\dfrac{8}{3}=\left(\dfrac{2}{3}\right)^{p+3}+3$

$\left(\dfrac{2}{3}\right)^{p+2}=1$

$p+2=0$ , 즉 $p=-2$

[[앞의 결과를 이용하여 문제에서 요구하는 것을 구합니다.]]

$p=-2$를 ㉠에 대입하면

$\left(\dfrac{2}{3}\right)^{-2+3}+1=2 \times(-2)+k$

따라서 $k=\dfrac{17}{3}$


<<<P>>>9. 직선 $y=2 x+k$ 가 두 함수
$$
y=\left(\dfrac{2}{3}\right)^{x+3}+1, \quad y=\left(\dfrac{2}{3}\right)^{x+1}+\dfrac{8}{3}
$$
의 그래프와 만나는 점을 각각 $\mathrm{P}, \mathrm{Q}$ 라 하자. $\overline{\mathrm{PQ}}=\sqrt{5}$ 일 때, 상수 $k$ 의 값은? [4점]

[IMG]

<<<C>>>① $\dfrac{31}{6}$
② $\dfrac{16}{3}$
③ $\dfrac{11}{2}$
④ $\dfrac{17}{3}$
⑤ $\dfrac{35}{6}$


<<<A>>>⑤

<<<S>>>출제의도 : 다항함수의 도함수와 접선의 방정식을 구할 수 있는가?

[[구하려는 삼차함수의 식을 미정계수를 포함하여 도입합니다.]]

$f(x)=a x^{3}+b x^{2}+c x+d$ 라 하자. $f^{\prime}(x)=3 a x^{2}+2 b x+c$ 이다.

[[주어진 곡선 위의 점을 각 곡선의 식에 대입하여 방정식을 만듭니다.]]

점 $(0,0)$는 삼차함수 $y=f(x)$ 의 그래프 위의 점이고,

점 $(1,2)$는 $y=x f(x)$ 의 그래프 위의 점이므로 

$f(0)=0$에서  $d=0 \cdots \cdots $㉠

$1 \times f(1)=2$에서  $a+b+c+d=2$ $\cdots \cdots$ ㉡

[[주어진 곡선 위의 점에서 접선의 방정식을 구합니다.]]

한편, 점 $(0,0)$ 에서 $y=f(x)$의 접선의 방정식은

$y=f^{\prime}(0) x$이고,

$y=x f(x)$의 도함수는 $y^{\prime}=f(x)+x f^{\prime}(x)$이므로

점 $(1,2)$에서 $y=x f(x)$의 접선의 방정식은

$y=\left\{f(1)+f^{\prime}(1)\right\}(x-1)+2$에서

$y=\left\{f^{\prime}(1)+2\right\} x-f^{\prime}(1) $ 이다.

[[두 접선의 방정식이 일치하도록 계수를 비교하여 연립방정식을 세웁니다.]]

위의 두 접선이 일치하므로

$f^{\prime}(0)=f^{\prime}(1)+2, f^{\prime}(1)=0$이다.

따라서 $f^{\prime}(0)=2, f^{\prime}(1)=0$에서

$c=2$  $\cdots \cdots$ ㉢

$3 a+2 b+c=0$  $\cdots \cdots$ ㉣

[[[구해진 연립방정식을 풉니다.]]

㉠, ㉡, ㉢, ㉣을 연립하여 $a,b,c,d$를 구하면

$a=-2, b=2, c=2, d=0$이다.

[[[앞에서 구한 결과를 이용해서 문제에서 요구하는 것을 구합니다.]]

따라서 $f(x)=-2 x^{3}+2 x^{2}+2 x$이고, $f^{\prime}(x)=-6 x^{2}+4 x+2$이므로

$f^{\prime}(2)=-14$


<<<P>>>10. 삼차함수 $f(x)$ 에 대하여 곡선 $y=f(x)$ 위의 점 $(0,0)$ 에서의 접선과 곡선 $y=x f(x)$ 위의 점 $(1,2)$ 에서의 접선이 일치할 때, $f^{\prime}(2)$ 의 값은? [4점]

<<<C>>>① $-18$
② $-17$
③ $-16$
④ $-15$
⑤ $-14$


<<<A>>>③

<<<S>>>출제의도 : 삼각함수의 그래프의 성질을 이용하여 조건을 만족하는 삼각형의 넓이를 구할 수 있는가?

[[주어진 함수의 주기를 구합니다.]]

$\dfrac{\pi}{\dfrac{\pi}{a}}=a$ 이므로
함수 $f(x)$ 의 주기는 $a$ 이다.

[[점 $\mathrm{B}$의 좌표를 미지수로 도입하여 각 점의 좌표를 미지수로 표시합니다.]]

직선 $\mathrm{AB}$ 는 원점을 지나고 기울기가 $\sqrt{3}$인 직선이므로

양수 $t$ 에 대하여
$\mathrm{B}(t, \sqrt{3} t)$ 로 놓으면

$\mathrm{A}(-t,-\sqrt{3} t)$ 이고,

$\overline{\mathrm{AB}}=4 t$ 이다.

[[앞의 결과들을 이용하여 미지수의 값을 구합니다.]]

이때, 함수 $f(x)$ 의 주기가 $a$ 이므로
$\overline{\mathrm{AC}}=4 t=a$ 이고,

$\mathrm{C}(-t+a,-\sqrt{3} t)$, 즉 $\mathrm{C}(3 t,-\sqrt{3} t)$ 이다.

점 $\mathrm{C}$ 가 곡선 $y=\tan \dfrac{\pi x}{a}=\tan \dfrac{\pi x}{4 t}$ 위의 점이므로

$-\sqrt{3} t=\tan \dfrac{\pi \times 3 t}{4 t}$에서 $t=\dfrac{1}{\sqrt{3}}$

[[앞의 결과를 이용해 삼각형의 넓이를 구합니다.]]

따라서 삼각형 $\mathrm{ABC}$ 의 넓이는

$
\begin{aligned}
\dfrac{\sqrt{3}}{4} \times(4 t)^{2} &=\dfrac{\sqrt{3}}{4} \times\left(\dfrac{4}{\sqrt{3}}\right)^{2} \\
&=\dfrac{4}{\sqrt{3}} \\
&=\dfrac{4 \sqrt{3}}{3}
\end{aligned}
$


<<<P>>>11. 양수 $a$ 에 대하여 집합 $\left\{x \mid-\dfrac{a}{2}< x \leq a, x \neq \dfrac{a}{2}\right\}$ 에서 정의된 함수
$$
f(x)=\tan \dfrac{\pi x}{a}
$$
가 있다. 그림과 같이 함수 $y=f(x)$ 의 그래프 위의 세 점 $\mathrm{O}, \mathrm{A}, \mathrm{B}$ 를 지나는 직선이 있다. 점 $\mathrm{A}$ 를 지나고 $x$ 축에 평행한 직선이 함수 $y=f(x)$ 의 그래프와 만나는 점 중 $\mathrm{A}$ 가 아닌 점을 $\mathrm{C}$ 라 하자. 삼각형 $\mathrm{ABC}$ 가 정삼각형일 때, 삼각형 $\mathrm{ABC}$ 의 넓이는? (단, $\mathrm{O}$ 는 원점이다.) [4점]

[IMG]

<<<C>>>① $\dfrac{3 \sqrt{3}}{2}$
② $\dfrac{17 \sqrt{3}}{12}$
③ $\dfrac{4 \sqrt{3}}{3}$
④ $\dfrac{5 \sqrt{3}}{4}$
⑤ $\dfrac{7 \sqrt{3}}{6}$


<<<A>>>③

<<<S>>>출제의도 : 함수의 연속의 성질을 이용하여 함수를 구할 수 있는가?

[[주어진 식의 좌변을 인수분해합니다.]]

$\{f(x)\}^{3}-\{f(x)\}^{2}-x^{2} f(x)+x^{2}=0$ 에서

$\{f(x)-1\}\{f(x)+x\}\{f(x)-x\}=0$이므로

$f(x)=1$ 또는 $f(x)=-x$ 또는 $f(x)=x$

[[실수 전체의 집합에서 연속이고 최대와 최소의 조건에 맞는 함수를 구합니다.]]

이때, $f(0)=1$ 또는 $f(0)=0$ 이다.

(i) $f(0)=1$ 일 때,

함수 $f(x)$ 가 실수 전체의 집합에서 연속이고, 최댓값이 $1$ 이므로 $f(x)=1$이다.

이때, 함수 $f(x)$ 의 최솟값이 $0$ 이 아니므로 주어진 조건을 만족시키지 못한다.

(ii) $f(0)=0$ 일 때,

함수 $f(x)$ 가 실수 전체의 집합에서 연속이고, 최댓값이 $1$ 이므로

$f(x)=\begin{cases}|x| & (|x| \leq 1) \\ 1 & (|x|>1) \end{cases}$
이다.

(i), (ii)에서 $f(x)=\begin{cases} |x| & (|x| \leq 1) \\ 1 & (|x|>1)\end{cases}$

[IMG]

[[앞에서 구한 결과를 이용하여 문제에서 요구하는 것을 구합니다.]]

따라서 $f\left(-\dfrac{4}{3}\right)=1, f(0)=0, f\left(\dfrac{1}{2}\right)=\dfrac{1}{2}$ 이므로

$f\left(-\dfrac{4}{3}\right)+f(0)+f\left(\dfrac{1}{2}\right)=1+0+\dfrac{1}{2}=\dfrac{3}{2}$

<<<P>>>12. 실수 전체의 집합에서 연속인 함수 $f(x)$ 가 모든 실수 $x$ 에 대하여
$$
\{f(x)\}^{3}-\{f(x)\}^{2}-x^{2} f(x)+x^{2}=0
$$
을 만족시킨다. 함수 $f(x)$ 의 최댓값이 $1$이고 최솟값이 $0$일 때, $f\left(-\dfrac{4}{3}\right)+f(0)+f\left(\dfrac{1}{2}\right)$ 의 값은? [4점]

<<<C>>>① $\dfrac{1}{2}$
② $1$
③ $\dfrac{3}{2}$
④ $2$
⑤ $\dfrac{5}{2}$

<<<A>>>②

<<<S>>>출제의도 : 로그의 정의와 성질을 활용하여 식의 값을 구할 수 있는가?

[[주어진 두 점을 지나는 직선의 방정식을 구합니다.]]

두 점 $\left(a, \log _{2} a\right),\left(b, \log _{2} b\right)$ 를 지나는 직선의 방정식은

$y=\dfrac{\log _{2} b-\log _{2} a}{b-a}(x-a)+\log _{2} a $

두 점 $\left(a, \log _{4} a\right),\left(b, \log _{4} b\right)$ 를 지나는 직선의 방정식은

$y=\dfrac{\log _{4} b-\log _{4} a}{b-a}(x-a)+\log _{4} a $



[[각 직선의 방정식에 $x=0$을 대입하여 $y$ 절편을 구합니다.]]

위의 두 직선의 $y$ 절편은 각각

$-\dfrac{a\left(\log _{2} b-\log _{2} a\right)}{b-a}+\log _{2} a \quad \cdots \cdots$ ㉠

$-\dfrac{a\left(\log _{4} b-\log _{4} a\right)}{b-a}+\log _{4} a$

$=-\dfrac{1}{2} \times \dfrac{a\left(\log _{2} b-\log _{2} a\right)}{b-a}+\dfrac{1}{2} \log _{2} a \cdots \cdots $ ㉡

[[앞의 결과와 주어진 조건을 이용하여 방정식을 세웁니다.]]

㉠과 ㉡이 같으므로

$-\dfrac{a\left(\log _{2} b-\log _{2} a\right)}{b-a}+\log _{2} a$

$=-\dfrac{1}{2} \times \dfrac{a\left(\log _{2} b-\log _{2} a\right)}{b-a}+\dfrac{1}{2} \log _{2} a$

[[앞에서 구한 방정식을 정리합니다.]]

이 식을 정리하면

$\dfrac{1}{2} \times \log _{2} a=\dfrac{1}{2} \times \dfrac{a\left(\log _{2} b-\log _{2} a\right)}{b-a}$

$\log _{2} a=\dfrac{a\left(\log _{2} b-\log _{2} a\right)}{b-a}$

$(b-a) \log _{2} a=a \log _{2} \dfrac{b}{a}$

$\log _{2} a^{b-a}=\log _{2}\left(\dfrac{b}{a}\right)^{a}$

$a^{b-a}=\dfrac{b^{a}}{a^{a}}$

$a^{b}=b^{a} \quad \cdots \cdots$ ㉢

[[주어진 함수와 조건을 이용하여 미지수의 관계식을 세웁니다.]]

한편, $f(x)=a^{b x}+b^{a x}$ 이고
$f(1)=40$ 이므로

$a^{b}+b^{a}=40$

㉢을 대입하면 $a^{b}+a^{b}=40$

$a^{b}=20$

따라서 $b^{a}=20$ 이므로

$\begin{aligned} f(2) &=a^{2 b}+b^{2 a} \\ &=\left(a^{b}\right)^{2}+\left(b^{a}\right)^{2} \\ &=20^{2}+20^{2} \\ &=800 \end{aligned}$


<<<P>>>13. 두 상수 $a, b(1< a< b)$ 에 대하여 좌표평면 위의 두 점 $\left(a, \log _{2} a\right)$, $\left(b, \log _{2} b\right)$ 를 지나는 직선의 $y$ 절편과 두 점 $\left(a, \log _{4} a\right)$, $\left(b, \log _{4} b\right)$ 를 지나는 직선의 $y$ 절편이 같다. 함수 $f(x)=a^{b x}+b^{a x}$ 에 대하여 $f(1)=40$ 일 때, $f(2)$ 의 값은? [4점]

<<<C>>>① $760$
② $800$
③ $840$
④ $880$
⑤ $920$


<<<A>>>③

<<<S>>>출제의도 : 수직선 위를 움직이는 점의 운동에 대하여 주어진 명제의 참, 거짓을 판별할 수 있는가?

[[주어진 위치 관계식과 정적분 관계식에서 의미있는 내용을 찾습니다.]]

$x(0)=0, x(1)=0$ 이므로 점 $\mathrm{P}$ 의 위치는 $t=0$ 일 때 수직선의 원점이고, $t=1$ 일 때도 수직선의 원점이다.

또, $\displaystyle\int_{0}^{1}|v(t)| d t=2$ 이므로 점 $\mathrm{P}$ 가 $t=0$ 에서 $t=1$ 까지 움직인 거리가 $2$ 이다.


ㄱ. 

[[주어진 정적분이 위치의 변화량임을 파악합니다.]]

점 $\mathrm{P}$ 의 $t=0$ 에서 $t=1$ 까지 위치의 변화량이 $0$ 이므로
$\displaystyle\int_{0}^{1} v(t) d t=0$ (참)

ㄴ․

[[시각이 $t=t_{1}$일 때 위치와 움직인 거리 사이의 관계를 조사합니다.]]

$\left|x\left(t_{1}\right)\right|>1$ 이면 점 $\mathrm{P}$ 와 원점 사이의 거리가 $1$ 보다 큰 시각 $t_{1}$ 이 존재하므로 점 $\mathrm{P}$ 가 $t=0$ 에서 $t=1$ 까지 움직인 거리가 $2$ 보다 크다. (거짓)

ㄷ.

[[시각이 $0 \leq t \leq 1$일 때 위치와 움직인 거리의 관계를 조사합니다.]]

$0 \leq t \leq 1$ 인 모든 시각 $t$ 에서 점 $\mathrm{P}$ 와 원점 사이의 거리가 $1$보다 작고, 점 $\mathrm{P}$ 가 $t=0$ 에서 $t=1$ 까지 움직인 거리가 $2$ 이므로 점 $\mathrm{P}$ 는 $0< t< 1$ 에서 적어도 한 번 원점을 지나간다. (참)

이상에서 옳은 것은 ㄱ, ㄷ이다.


<<<P>>>14. 수직선 위를 움직이는 점 $\mathrm{P}$ 의 시각 $t$ 에서의 위치 $x(t)$ 가 두 상수 $a, b$ 에 대하여
$$
x(t)=t(t-1)(a t+b) \quad(a \neq 0)
$$
이다. 점 $\mathrm{P}$ 의 시각 $t$ 에서의 속도 $v(t)$ 가 $\displaystyle\int_{0}^{1}|v(t)| d t=2$ 를 만족시킬 때, < 보기>에서 옳은 것만을 있는 대로 고른 것은?
[4점]

ㄱ. $\displaystyle\int_{0}^{1} v(t) d t=0$

ㄴ. $\left|x\left(t_{1}\right)\right|>1$ 인 $t_{1}$ 이 열린구간 $(0,1)$ 에 존재한다.

ㄷ. $0 \leq t \leq 1$ 인 모든 $t$ 에 대하여 $|x(t)|< 1$ 이면 $x\left(t_{2}\right)=0$ 인 $t_{2}$ 가 열린구간 $(0,1)$ 에 존재한다.

<<<C>>>① ㄱ
② ㄱ, ㄴ
③ ㄱ, ㄷ
④ ㄴ, ㄷ
⑤ ㄱ, ㄴ, ㄷ

<<<A>>>②

<<<S>>>출제의도 : 코사인법칙을 이용하여 선분의 길이의 비를 구할 수 있는가?

[IMG]

[[$\theta_{3}$을 $\theta_{2}$로 나타내고, 그 결과와 주어진 각의 조건을 이용하여 $\angle \mathrm{CO_{1}B}=\theta_{1}$임을 알아냅니다.]]

$\angle \mathrm{CO_{2}O_{1}}+\angle \mathrm{O_{1} O_{2} D}=\pi$ 이므로

$\theta_{3}=\dfrac{\pi}{2}+\dfrac{\theta_{2}}{2}$ 이고

$\theta_{3}=\theta_{1}+\theta_{2}$ 에서 $2 \theta_{1}+\theta_{2}=\pi$ 이므로

$\angle \mathrm{CO_{1}B}=\theta_{1}$ 이다.

[[앞의 결과를 이용하여 삼각형 $\mathrm{O_{1} O_{2} B}$ 와 삼각형 $\mathrm{O_{2} O_{1} D}$ 는 합동임을 알아냅니다.]]

이때, $\angle \mathrm{O_{2} O_{1} B}=\theta_{1}+\theta_{2}=\theta_{3}$ 이므로

삼각형 $\mathrm{O_{1} O_{2} B}$ 와 삼각형 $\mathrm{O_{2} O_{1} D}$ 는 합동이다.

[[선분 $\mathrm{AB}$의 길이를 도입하고, 직각삼각형 $\triangle \mathrm{ABO_{2}}$에서 $\fbox{(가)}$, $\fbox{(나)}$를 알아냅니다.]]

$\overline{\mathrm{AB}}=k$ 라 할 때,

$\overline{\mathrm{BO}_{2}}=\overline{\mathrm{O}_{1} \mathrm{D}}=2 \sqrt{2} k$ 이므로

$\overline{\mathrm{AO}_{2}}=\sqrt{k^{2}+(2 \sqrt{2} k)^{2}}=3 k$ 이고,

$\angle \mathrm{BO}_{2} \mathrm{A}=\dfrac{\theta_{1}}{2}$ 이므로

$\cos \dfrac{\theta_{1}}{2}=\dfrac{2 \sqrt{2} k}{3 k}=\dfrac{2 \sqrt{2}}{3}$ 이다.

[[$\overline{\mathrm{O_{2} C}}=x$라 놓고, 삼각형 $\mathrm{BO_{2} C}$에서 코사인법칙을 적용시켜 이차방정식을 세웁니다.]]

삼각형 $\mathrm{O}_{2} \mathrm{BC}$ 에서

$\overline{\mathrm{BC}}=k, \overline{\mathrm{BO}_{2}}=2 \sqrt{2} k, \angle \mathrm{CO}_{2} \mathrm{B}=\dfrac{\theta_{1}}{2}$ 이므로

삼각형 $\mathrm{BO}_{2} \mathrm{C}$ 에서
$\mathrm{O}_{2} \mathrm{C}=x(0< x< 3 k)$ 라 하면

코사인법칙에 의하여

$k^{2}=x^{2}+(2 \sqrt{2} k)^{2}-2 \times x \times 2 \sqrt{2} k \times \cos \dfrac{\theta_{1}}{2}$

$k^{2}=x^{2}+(2 \sqrt{2} k)^{2}-2 \times x \times 2 \sqrt{2} k \times \dfrac{2 \sqrt{2}}{3}$

$3 x^{2}-16 k x+21 k^{2}=0$

$(3 x-7 k)(x-3 k)=0$

$0< x< 3 k$ 이므로 $x=\dfrac{7}{3} k$

[[앞의 결과를 이용하여 구하려는 식의 값을 계산합니다.]]

즉, $\overline{\mathrm{O}_{2} \mathrm{C}}=\boxed{\dfrac{7}{3} k}$ 이다.

$\overline{\mathrm{CD}}=\overline{\mathrm{O}_{2} \mathrm{D}}+\overline{\mathrm{O}_{2} \mathrm{C}}=\overline{\mathrm{O}_{1} \mathrm{O}_{2}}+\overline{\mathrm{O}_{2} \mathrm{C}}$ 이므로

$\overline{\mathrm{AB}}: \overline{\mathrm{CD}}=k:\left(\dfrac{\boxed{3 k}}{2}+\boxed{\dfrac{7}{3} k}\right)$

이상에서 $f(k)=3 k, g(k)=\dfrac{7}{3} k, p=\dfrac{2 \sqrt{2}}{3}$이므로

$f(p) \times g(p)$

$=\left(3 \times \dfrac{2 \sqrt{2}}{3}\right) \times\left(\dfrac{7}{3} \times \dfrac{2 \sqrt{2}}{3}\right)$

$=\dfrac{56}{9}$



<<<P>>>15. 두 점 $\mathrm{O}_{1}, \mathrm{O}_{2}$ 를 각각 중심으로 하고 반지름의 길이가 $\overline{\mathrm{O}_{1} \mathrm{O}_{2}}$ 인 두 원 $C_{1}, C_{2}$ 가 있다. 그림과 같이 원 $C_{1}$ 위의 서로 다른 세 점 $\mathrm{A}, \mathrm{B}, \mathrm{C}$ 와 원 $C_{2}$ 위의 점 $\mathrm{D}$ 가 주어져 있고, 세 점 $\mathrm{A}, \mathrm{O}_{1}, \mathrm{O}_{2}$ 와 세 점 $\mathrm{C}, \mathrm{O}_{2}, \mathrm{D}$ 가 각각 한 직선 위에 있다.
이때 $\angle \mathrm{BO}_{1} \mathrm{A}=\theta_{1}, \angle \mathrm{O}_{2} \mathrm{O}_{1} \mathrm{C}=\theta_{2}, \angle \mathrm{O}_{1} \mathrm{O}_{2} \mathrm{D}=\theta_{3}$ 이라 하자.

[IMG]

다음은 $\overline{\mathrm{AB}}: \overline{\mathrm{O}_{1} \mathrm{D}}=1: 2 \sqrt{2}$ 이고 $\theta_{3}=\theta_{1}+\theta_{2}$ 일 때, 선분 $\mathrm{AB}$ 와 선분 $\mathrm{CD}$ 의 길이의 비를 구하는 과정이다.

$\angle \mathrm{CO}_{2} \mathrm{O}_{1}+\angle \mathrm{O}_{1} \mathrm{O}_{2} \mathrm{D}=\pi$이므로 $\theta_{3}=\dfrac{\pi}{2}+\dfrac{\theta_{2}}{2}$이고 $\theta_{3}=\theta_{1}+\theta_{2}$ 에서 $2 \theta_{1}+\theta_{2}=\pi$ 이므로 $\angle \mathrm{CO}_{1} \mathrm{B}=\theta_{1}$ 이다.

이때 $\angle \mathrm{O}_{2} \mathrm{O}_{1} \mathrm{B}=\theta_{1}+\theta_{2}=\theta_{3}$ 이므로 삼각형 $\mathrm{O}_{1} \mathrm{O}_{2} \mathrm{B}$ 와 삼각형 $\mathrm{O}_{2} \mathrm{O}_{1} \mathrm{D}$ 는 합동이다.

$\overline{\mathrm{AB}}=k$ 라 할 때

$\overline{\mathrm{BO}_{2}}=\overline{\mathrm{O}_{1} \mathrm{D}}=2 \sqrt{2} k$ 이므로 $\overline{\mathrm{AO}_{2}}=\fbox{(가)}$ 이고,

$\angle \mathrm{BO}_{2} \mathrm{A}=\dfrac{\theta_{1}}{2}$ 이므로 $\cos \dfrac{\theta_{1}}{2}=\fbox{(나)}$ 이다.

삼각형 $\mathrm{O}_{2} \mathrm{BC}$ 에서
$\overline{\mathrm{BC}}=k, \overline{\mathrm{BO}_{2}}=2 \sqrt{2} k, \quad \angle \mathrm{CO}_{2} \mathrm{B}=\dfrac{\theta_{1}}{2}$이므로

코사인법칙에 의하여 $\overline{\mathrm{O}_{2} \mathrm{C}}=\fbox{ (다) }$ 이다.

$\overline{\mathrm{CD}}=\overline{\mathrm{O}_{2} \mathrm{D}}+\overline{\mathrm{O}_{2} \mathrm{C}}=\overline{\mathrm{O}_{1} \mathrm{O}_{2}}+\overline{\mathrm{O}_{2} \mathrm{C}}$이므로

$\overline{\mathrm{AB}}: \overline{\mathrm{CD}}=k:\left(\dfrac{\fbox { (가) }}{2}+\fbox{ (다) }\right)$이다.

위의 (가), (다)에 알맞은 식을 각각 $f(k), g(k)$ 라 하고, (나)에 알맞은 수를 $p$ 라 할 때, $f(p) \times g(p)$ 의 값은? [4점]

<<<C>>>① $\dfrac{169}{27}$
② $\dfrac{56}{9}$
③ $\dfrac{167}{27}$
④ $\dfrac{166}{27}$
⑤ $\dfrac{55}{9}$


<<<A>>>$3$

<<<S>>>출제의도 : 로그의 성질을 이용하여 식의 값을 구할 수 있는가?

[[로그의 밑변환공식을 이용하여 한 종류의 밑으로 일치시킵니다.]]

$\log _{2} 120-\dfrac{1}{\log _{15} 2}$

$=\log _{2} 120-\log _{2} 15$

[[로그의 성질을 이용하여 주어진 식을 계산합니다.]]

$=\log _{2} \dfrac{120}{15}$

$=\log _{2} 8$

$=\log _{2} 2^{3}$

$=3$


<<<P>>>16. $\log _{2} 120-\dfrac{1}{\log _{15} 2}$ 의 값을 구하시오. [3점]


<<<A>>>$4$

<<<S>>>출제의도 : 도함수가 주어진 함수의 함숫값을 구할 수 있는가?

[[도함수를 적분하여 함수 $f(x)$의 식을 구합니다.]]

$f(x) =\displaystyle\int f^{\prime}(x) d x$

$=\displaystyle\int\left(3 x^{2}+2 x\right) d x$

$=x^{3}+x^{2}+C$ (단, $C$는 적분상수)

[[주어진 함숫값을 이용해서 적분상수값을 정합니다.]]

이때, $f(0)=2$ 이므로 $C=2$

[[문제에서 요구하는 함숫값을 구합니다.]]

따라서 $f(x)=x^{3}+x^{2}+2$ 이므로 $f(1)=1+1+2=4$

<<<P>>>17. 함수 $f(x)$ 에 대하여 $f^{\prime}(x)=3 x^{2}+2 x$ 이고 $f(0)=2$ 일 때, $f(1)$ 의 값을 구하시오. [3점]


<<<A>>>$12$

<<<S>>>출제의도 : 합의 기호 $\displaystyle\sum$ 의 성질을 이용하여 수열의 항의 값을 구할 수 있는가?

[[$\displaystyle\sum$의 성질을 이용해서 $\displaystyle\sum$가 포함된 식을 정리합니다.]]

$\displaystyle\sum_{k=1}^{10} a_{k}-\displaystyle\sum_{k=1}^{7} \dfrac{a_{k}}{2}=56 \quad \cdots \cdots $㉠

$\displaystyle\sum_{k=1}^{10} 2 a_{k}-\displaystyle\sum_{k=1}^{8} a_{k}=100$ 에서

$\displaystyle\sum_{k=1}^{10} a_{k}-\displaystyle\sum_{k=1}^{8} \dfrac{a_{k}}{2}=50 \quad \cdots \cdots $㉡

[[$\displaystyle\sum$가 포함된 두 식을 연립합니다.]]

㉠ $-$㉡을 계산하면  $\dfrac{a_{8}}{2}=6$

따라서 $a_{8}=12$

<<<P>>>18. 수열 $\left\{a_{n}\right\}$ 에 대하여
$$
\displaystyle\sum_{k=1}^{10} a_{k}-\displaystyle\sum_{k=1}^{7} \dfrac{a_{k}}{2}=56, \quad \displaystyle\sum_{k=1}^{10} 2 a_{k}-\displaystyle\sum_{k=1}^{8} a_{k}=100
$$
일 때, $a_{8}$ 의 값을 구하시오. [3점]



<<<A>>>$6$

<<<S>>>출제의도 : 미분을 이용하여 함수의 그래프의 개형을 알 수 있는가?

[[함수 $f(x)$가 실수 전체의 집합에서 증가하기 위한 조건을 도함수에 적용합니다.]]

$f(x)=x^{3}+a x^{2}-\left(a^{2}-8 a\right) x+3$에서

$f^{\prime}(x)=3 x^{2}+2 a x-\left(a^{2}-8 a\right)$

이때, 함수 $f(x)$ 가 실수 전체의 집합에서 증가하려면

항상 $f^{\prime}(x) \geq 0$이 성립하면 된다.

[[이차부등식이 항상 성립하는 조건을 적용합니다.]]

이때, 이차방정식 $f^{\prime}(x)=0$ 의 판별식을 $D$ 라 하면 $\dfrac{D}{4} \leq 0$ 이어야 하므로

$\begin{aligned} \dfrac{D}{4} &=a^{2}-3\left(-a^{2}+8 a\right)\\
&=4 a^{2}-24 a\\
&=4 a(a-6) \leq 0 \end{aligned}$

그러므로 $0 \leq a \leq 6$

따라서, $a$ 의 최댓값은 $6$ 이다.



<<<P>>>19. 함수 $f(x)=x^{3}+a x^{2}-\left(a^{2}-8 a\right) x+3$ 이 실수 전체의 집합에서 증가하도록 하는 실수 $a$ 의 최댓값을 구하시오. [3점]


<<<A>>>$110$

<<<S>>>출제의도 : 조건을 만족시키는 함수 $f(x)$ 에 대하여 정적분의 값을 구할 수 있는가?

[[주어진 관계식에 적당한 수를 대입하여 미정계수값을 구합니다.]]

$f(x+1)-x f(x)=a x+b$에 $x=0$을 대입하면

$f(1)=b$

닫힌구간 $[0,1]$ 에서 $f(x)=x$ 이므로 $b=1$

[[$x+1=t$ 로 치환해서 구간 $[1, 2]$에서의 함수 $f(t)$를 구합니다.]]

한편, $f(x+1)-x f(x)=a x+1$ 이므로 $0 \leq x \leq 1$일 때, 

$\begin{aligned} f(x+1) &=x f(x)+a x+1\\
&=x^{2}+a x+1\\
\end{aligned}$

$x+1=t$ 로 치환하면 $1 \leq t \leq 2$이고

$\begin{aligned} f(t) &=(t-1)^{2}+a(t-1)+1\\
&=t^{2}+(a-2) t+2-a \cdots \cdots \text{㉠} \end{aligned} $ 

[[앞에서 구한 함수가 미분가능하도록 미정계수값을 구합니다.]]

$1<t<2$에서 $f^{\prime}(t)=2 t+(a-2)$ 이고, 닫힌구간 $[0,1]$ 에서 $f(x)=x$ 이므로,

함수 $f(x)$가 실수 전체의 집합에서 미분가능하므로 $f^{\prime}(1)=1$에서 $a=1$

따라서 $1 \leq x \leq 2$ 일 때 $f(x)=x^{2}-x+1$ 이다.

[[주어진 정적분을 계산합니다.]]

$ \begin{aligned}\displaystyle\int_{1}^{2} f(x) d x &=\displaystyle\int_{1}^{2}\left(x^{2}-x+1\right) d x\\
&=\left[\dfrac{1}{3} x^{3}-\dfrac{1}{2} x^{2}+x\right]_{1}^{2}\\
&=\dfrac{8}{3}-\dfrac{5}{6}=\dfrac{11}{6}\end{aligned}$

따라서 $60 \times \displaystyle\int_{1}^{2} f(x) d x=60 \times \dfrac{11}{6} =110$


<<<P>>>20. 실수 전체의 집합에서 미분가능한 함수 $f(x)$ 가 다음 조건을 만족시킨다.

(가) 닫힌구간 $[0,1]$ 에서 $f(x)=x$ 이다.

(나) 어떤 상수 $a, b$ 에 대하여 구간 $[0, \infty)$ 에서 $f(x+1)-x f(x)=a x+b$ 이다.

$60 \times \displaystyle\int_{1}^{2} f(x) d x$ 의 값을 구하시오. [4점]


<<<A>>>$678$

<<<S>>>출제의도 : 등비수열의 합을 이용하여 수열의 합을 구할 수 있는가?

[[양수인 항들의 합과 음수인 항들의 합을 미지수로 도입합니다.]]

수열 $\left\{a_{n}\right\}$의 첫째항부터 제$10$항까지의 항 중에서

양수인 항의 합을 $P$, 음수인 항의 합을 $N$이라 하자.

[[조건을 이용하여 앞에서 도입한 미지수의 관계식을 세웁니다.]]

$\sum_{n=1}^{10}a_{n}=P+N$이므로 (다)에서 $P+N=-14\cdots\cdots$㉠

한편, 수열 $\left\{\left | a_{n}\right |\right\}$은 첫째항이 $2$이고 공비가 $2$인 등비수열이므로

$\sum_{n=1}^{10}\left | a_{n}\right | =\dfrac{2(2^{10}-1)}{2-1}=2^{11}-2$에서 $P-N=2^{11}-2\cdots\cdots$㉡

㉠, ㉡에서 $N=-2^{10}-6$

[[음수인 항들의 합을 이용하여 음수인 항을 찾습니다.]]

그런데 $N=-$(서로 다른 $2$의 거듭제곱 몇 개의 합)이고, 

$N=-2^{10}-2^{2}-2$로 고쳐 쓸 수 있으므로

수열 $\left\{a_{n}\right\}$의 제$1$항, 제$2$항, 제$10$항이 음수인 항이다.

따라서 $a_{1}+a_{3}+a_{5}+a_{7}+a_{9}=(-2)+2^{3}+2^{5}+2^{7}+2^{9}=678$

[다른 풀이]

[[조건 (가), (나)를 만족하는 수열의 일반항을 알아냅니다.]]

조건 (가), (나)에서 수열 $\left\{\left|a_{n}\right|\right\}$ 은 첫째항이 2, 공비가 2인 등비수열이므로

$\left|a_{n}\right|=2^{n}$

한편,
$\displaystyle\sum_{k=1}^{9}\left|a_{k}\right|=\displaystyle\sum_{k=1}^{9} 2^{k}=\dfrac{2\left(2^{9}-1\right)}{2-1}=2^{10}-2$,

$\left|a_{10}\right|=2^{10}$

[[조건 (다)의 합과 등비수열 $\left\{\left|a_{n}\right|\right\}$의 합으로부터 등비수열  $\left\{a_{n}\right\}$ 항들의 값을 알아냅니다.]]

조건 (다)에서 $\displaystyle\sum_{k=1}^{10} a_{k}=-14$ 를 만족하기 위해서는
$a_{1}=-2, a_{2}=-4$

$\displaystyle\sum_{k=3}^{9} a_{k}=\displaystyle\sum_{k=3}^{9} 2^{k}=\dfrac{2^{3}\left(2^{7}-1\right)}{2-1}=2^{10}-8$,

$a_{10}=-1024$이어야 한다.

따라서
$a_{1}+a_{3}+a_{5}+a_{7}+a_{9}$

$=(-2)+2^{3}+2^{5}+2^{7}+2^{9}$

$=678$


<<<P>>>21. 수열 $\left\{a_{n}\right\}$ 이 다음 조건을 만족시킨다.

(가) $\left|a_{1}\right|=2$

(나) 모든 자연수 $n$ 에 대하여 $\left|a_{n+1}\right|=2\left|a_{n}\right|$ 이다.

(다) $\displaystyle\sum_{n=1}^{10} a_{n}=-14$

$a_{1}+a_{3}+a_{5}+a_{7}+a_{9}$ 의 값을 구하시오. [4점]


<<<A>>>$9$

<<<S>>>출제의도 : 함수의 극한을 이용하여 도함수 $f^{\prime}(x)$ 의 특징을 찾아 조건을 만족시키는 함수 $f(x)$ 를 구할 수 있는가?

[[조건 (나)에 의해서 이차방정식 $f^{\prime}(x)=0$이 서로 다른 두 실근을 갖는 것을 알아냅니다.]]

조건 (나)에서 $g(f(1))=g(f(4))=2$이므로 $g(t)=2$인 $t$가 존재한다.

따라서 이차방정식 $f^{\prime}(x)=0$이 서로 다른 두 실근을 갖고, 그 서로 다른 두 실근을 $\alpha, \beta(\alpha< \beta)$라 하자.

[[닫힌구간 $[t,t+2]$의 길이가 $2$인 것을 고려하여 $\beta-\alpha$의 값에 따라 경우로 나누어 조건을 만족하는지 확인합니다.]]

(i) [[$\beta-\alpha=2$일 때, 조건 (가)와 (나)를 만족하는지 확인합니다.]]

$\beta=\alpha+2$ 일 때, 함수 $y=g(t)$ 의 그래프는 다음과 같다.

[IMG]

이는 조건 (가)를 만족한다.

(ii) [[$\beta-\alpha>2$일 때, 조건 (가)와 (나)를 만족하는지 확인합니다.]]

$\beta-\alpha>2$일 때, 함수 $y=g(t)$ 의 그래프는 다음과 같다.

[IMG]

이는 조건 (나)에서 $g(t)$ 가 함숫값 $2$ 를 갖는 것에 모순이다.

(iii) [[$\beta-\alpha<2$일 때, 조건 (가)와 (나)를 만족하는지 확인합니다.]]

$\beta-\alpha<2$일 때, 함수 $y=g(t)$ 의 그래프는 다음과 같다.

[IMG]

이때, $\beta-2 \leq a \leq \alpha$ 인 $a$ 에 대하여 조건 (가)를 만족시키지 못한다.

이상에서 조건을 만족시키는 것은 ( i )의 경우이다.

[[$\beta-\alpha=2$를 만족하는 $f'(x)$의 식을 세웁니다.]]

함수 $f(x)$ 의 최고차항의 계수가 $\dfrac{1}{2}$ 이므로 함수 $f^{\prime}(x)$ 의 최고차항의 계수
는 $\dfrac{3}{2}$ 이다.

그러므로
$f^{\prime}(x)=\dfrac{3}{2}(x-\alpha)\{x-(\alpha+2)\}$

$=\dfrac{3}{2}\left\{x^{2}-(2 \alpha+2) x+\alpha^{2}+2 \alpha\right\}$로 놓을 수 있다.

이때,
$f(x)=\dfrac{1}{2} x^{3}-\dfrac{3}{2}(\alpha+1) x^{2}+\dfrac{3}{2}\left(\alpha^{2}+2 \alpha\right) x+C \cdots $㉠

( 단, $C$ 는 적분상수 )

[[조건 (나)를 이용해서 $\alpha$의 값을 구합니다.]]

한편, 조건 (나)에서 $g(f(1))=g(f(4))=2$이고 $g(t)=2$인 $t$의 값의 개수는 $1$ 이므로 $f(1)=f(4)$

㉠에서
$\dfrac{1}{2}-\dfrac{3}{2}(\alpha+1)+\dfrac{3}{2}\left(\alpha^{2}+2 \alpha\right)+C$

$=32-24(\alpha+1)+6\left(\alpha^{2}+2 \alpha\right)+C$

위의 식을 정리하면 $\alpha^{2}-3 \alpha+2=0$이므로

$\alpha=1$ 또는 $\alpha=2$

[[$g(f(0))=1$을 만족하는 $\alpha$의 값을 알아냅니다.]]

(a) $\alpha=1$ 일 때,

$f(x)=\dfrac{1}{2} x^{3}-3 x^{2}+\dfrac{9}{2} x+C$

이때, $f(1)=\alpha$ 에서 $f(1)=1$ 이어야 하므로

$\dfrac{1}{2}-3+\dfrac{9}{2}+C=1$

$2+C=1$

$C=-1$

이때, $f(0)=-1$ 이므로 $g(f(0))=g(-1)=1$

그러므로 조건을 만족시킨다.

(b) $\alpha=2$ 일 때,

$f(x)=\dfrac{1}{2} x^{3}-\dfrac{9}{2} x^{2}+12 x+C$

이때, $f(1)=\alpha$ 에서 $f(1)=2$ 이어야 하므로

$\dfrac{1}{2}-\dfrac{9}{2}+12+C=2$

$8+C=2$

$C=-6$

이때, $f(0)=-6$ 이므로 $g(f(0))=g(-6)=0$

그러므로 조건을 만족시키지 못한다.

[[문제에서 요구하는 함숫값을 구합니다.]]

따라서 (a)에서 $f(x) =\dfrac{1}{2} x^{3}-3 x^{2}+\dfrac{9}{2} x-1$이므로

$f(5) =\dfrac{1}{2} \times 5^{3}-3 \times 25+\dfrac{9}{2} \times 5-1 =9$


<<<P>>>22. 최고차항의 계수가 $\dfrac{1}{2}$ 인 삼차함수 $f(x)$ 와 실수 $t$ 에 대하여 방정식 $f^{\prime}(x)=0$ 이 닫힌구간 $[t, t+2]$ 에서 갖는 실근의 개수를 $g(t)$ 라 할 때, 함수 $g(t)$ 는 다음 조건을 만족시킨다.

(가) 모든 실수 $a$ 에 대하여 $\displaystyle\lim _{t \rightarrow a+} g(t)+\displaystyle\lim _{t \rightarrow a-} g(t) \leq 2$ 이다.

(나) $g(f(1))=g(f(4))=2, g(f(0))=1$

$f(5)$ 의 값을 구하시오. [4점]


<<<A>>>④

<<<S>>>출제의도 : 이항정리를 이용하여 전개식의 항의 계수를 구할 수 있는가?

[[x0001]]

$(x+2)^{7}$ 의 전개식의 일반항은

${ }_{7} \mathrm{C}_{r} x^{7-r} \times 2^{r}$ (단, $r=0,1,2, \cdots, 7$ )

[[x0003]]

$x^{5}$ 의 계수는 $r=2$ 일 때

${ }_{7} \mathrm{C}_{2} \times 2^{2}=\dfrac{7 \times 6}{2 \times 1} \times 4=84$


<<<P>>>23. 다항식 $(x+2)^{7}$ 의 전개식에서 $x^{5}$ 의 계수는? [2점]

<<<C>>>① $42$
② $56$
③ $70$
④ $84$
⑤ $98$


<<<A>>>④

<<<S>>>출제의도 : 이항분포를 따르는 확률변수의 분산을 구할 수 있는가?

[[x0004]]

$\mathrm{V}(X)=n \times \dfrac{1}{3} \times \dfrac{2}{3}=\dfrac{2}{9} n$
이므로

[[x0005]]

$V(2 X)=4 V(X)$

$=4 \times \dfrac{2}{9} n$

$=\dfrac{8}{9} n=40$

따라서, $n=45$


<<<P>>>24. 확률변수 $X$ 가 이항분포 $\mathrm{B}\left(n, \dfrac{1}{3}\right)$ 을 따르고 $\mathrm{V}(2 X)=40$ 일 때, $n$ 의 값은? [3점]

<<<C>>>① $30$
② $35$
③ $40$
④ $45$
⑤ $50$


<<<A>>>①

<<<S>>>출제의도 : 조건을 만족시키는 순서 쌍의 개수를 구할 수 있는가?

[[x0006]]

조건 (나)에서
$a^{2}-b^{2}=-5$ 또는 $a^{2}-b^{2}=5$

즉, $(b-a)(b+a)=5$ 또는 $(a-b)(a+b)=5$이고 $a, b$ 는 자연수이므로

$b-a=1, b+a=5$ 또는 $a-b=1, a+b=5$

따라서, $a=2, b=3$ 또는 $a=3, b=2$ 이다.

[[x0007]]

또한, 조건 (가)에서 $a+b+c+d+e=12$ 이므로 $c+d+e=7$

[[x0008]]

$c, d, e$ 는 자연수이므로
$c=c^{\prime}+1, d=d^{\prime}+1, e=e^{\prime}+1$

($c^{\prime}, d^{\prime}, e^{\prime}$은 음이 아닌 정수) 로 놓으면

$\left(c^{\prime}+1\right)+\left(d^{\prime}+1\right)+\left(e^{\prime}+1\right)=7$

$c^{\prime}+d^{\prime}+e^{\prime}=4$

[[x0009]]

이를 만족시키는 모든 순서쌍 $\left(c^{\prime}, d^{\prime}, e^{\prime}\right)$
의 개수는

${ }_{3} \mathrm{H}_{4}={ }_{3+4-1} \mathrm{C}_{4}$

$={ }_{6} \mathrm{C}_{4}$

$={ }_{6} \mathrm{C}_{2}$

$=\dfrac{6 \times 5}{2 \times 1}=15$

[[x0010]]

따라서, 구하는 모든 순서쌍 $(a, b, c, d, e)$의 개수는

$2 \times 15=30$



<<<P>>>25. 다음 조건을 만족시키는 자연수 $a, b, c, d, e$ 의 모든 순서쌍 $(a, b, c, d, e)$ 의 개수는? [3점]

(가) $a+b+c+d+e=12$

(나) $\left|a^{2}-b^{2}\right|=5$

<<<C>>>① $30$
② $32$
③ $34$
④ $36$
⑤ $38$

<<<A>>>③

<<<S>>>출제의도 : 여사건의 확률을 구할 수 있는가?

[[x0011]]

카드에 적혀 있는 세 자연수 중에서 가장 작은 수가 $4$ 이하이거나 $7$ 이상인 사건을 $A$ 라 하면 사건 $A^{C}$ 은 카드에 적혀있는 세 자연수 중에서 가장 작은 수가 $4$ 보다 크고 $7$ 보다 작은 경우이다. 즉, 카드에 적혀 있는 세 자연수 중에서 가장 작은 수가 $5$ 또는 $6$ 이므로

[[x0012]]

$\mathrm{P}\left(A^{C}\right)$

$=\dfrac{{ }_{5} \mathrm{C}_{2}+{ }_{4} \mathrm{C}_{2}}{{ }_{10} \mathrm{C}_{3}}$

$=\dfrac{\dfrac{5 \times 4}{2 \times 1}+\dfrac{4 \times 3}{2 \times 1}}{\dfrac{10 \times 9 \times 8}{3 \times 2 \times 1}}$

$=\dfrac{16}{120}$

$=\dfrac{2}{15}$

따라서
$\mathrm{P}(A)=1-\mathrm{P}\left(A^{C}\right)$

$=1-\dfrac{2}{15}$

$=\dfrac{13}{15}$


<<<P>>>26. $1$부터 $10$까지 자연수가 하나씩 적혀 있는 $10$장의 카드가 들어 있는 주머니가 있다. 이 주머니에서 임의로 카드 $3$장을 동시에 꺼낼 때, 꺼낸 카드에 적혀 있는 세 자연수 중에서 가장 작은 수가 $4$이하이거나 $7$이상일 확률은? [3점]

[IMG]

<<<C>>>① $\dfrac{4}{5}$
② $\dfrac{5}{6}$
③ $\dfrac{13}{15}$
④ $\dfrac{9}{10}$
⑤ $\dfrac{14}{15}$


<<<A>>>②

<<<S>>>출제의도 : 모평균을 추정하여 신뢰 구간을 구할 수 있는가?

[[x0013]]

전기 자동차 $100$ 대를 임의추출하여 얻은 $1$ 회 충전 주행 거리의 표본평균이 $\overline{x_{1}}$ 일 때, 모평균 $m$ 에 대한 신뢰도 $95 \%$ 의 신뢰구간은

$\overline{x_{1}}-1.96 \times \dfrac{\sigma}{\sqrt{100}} \leq m \leq \overline{x_{1}}+1.96 \times \dfrac{\sigma}{\sqrt{100}}$

$\overline{x_{1}}-1.96 \times \dfrac{\sigma}{10} \leq m \leq \overline{x_{1}}+1.96 \times \dfrac{\sigma}{10}$

[[x0013]]

전기 자동차 $400$ 대를 임의추출하여 얻은 $1$ 회 충전 주행 거리의 표본평균이 $\overline{x_{2}}$ 일 때, 모평균 $m$ 에 대한 신뢰도 $99 \%$ 의 신뢰구간은

$\overline{x_{2}}-2.58 \times \dfrac{\sigma}{\sqrt{400}} \leq m \leq \overline{x_{2}}+2.58 \times \dfrac{\sigma}{\sqrt{400}}$

$\overline{x_{2}}-2.58 \times \dfrac{\sigma}{20} \leq m \leq \overline{x_{2}}+2.58 \times \dfrac{\sigma}{20}$

[[x0014]]

이때, $a=c$ 에서

$\overline{x_{1}}-1.96 \times \dfrac{\sigma}{10}=\overline{x_{2}}-2.58 \times \dfrac{\sigma}{20}$

이고 $\overline{x_{1}}-\overline{x_{2}}=1.34$이므로

$\overline{x_{1}}-\overline{x_{2}}=1.96 \times \dfrac{\sigma}{10}-2.58 \times \dfrac{\sigma}{20}$

$\quad=0.67 \times \dfrac{\sigma}{10}=1.34$

$\sigma=\dfrac{1.34 \times 10}{0.67}=20$

따라서,

$b-a=2 \times 1.96 \times \dfrac{\sigma}{10}$

$=2 \times 1.96 \times 2$

$=7.84$


<<<P>>>27. 어느 자동차 회사에서 생산하는 전기 자동차의 $1$ 회 충전 주행 거리는 평균이 $m$ 이고 표준편차가 $\sigma$ 인 정규분포를 따른다고 한다. 이 자동차 회사에서 생산한 전기 자동차 $100$대를 임의추출하여 얻은 $1$회 충전 주행 거리의 표본평균이 $\overline{x_{1}}$ 일 때, 모평균 $m$ 에 대한 신뢰도 $95 \%$ 의 신뢰구간이 $a \leq m \leq b$ 이다. 이 자동차 회사에서 생산한 전기 자동차 $400$대를 임의추출하여 얻은 $1$회 충전 주행 거리의 표본평균이 $\overline{x_{2}}$ 일 때, 모평균 $m$ 에 대한 신뢰도 $99 \%$ 의 신뢰구간이 $c \leq m \leq d$ 이다. $\overline{x_{1}}-\overline{x_{2}}=1.34$ 이고 $a=c$ 일 때, $b-a$ 의 값은? (단, 주행 거리의 단위는 $\mathrm{km}$ 이고, $Z$ 가 표준정규분포를 따르는 확률변수일 때 $\mathrm{P}(|Z| \leq 1.96)=0.95, \mathrm{P}(|Z| \leq 2.58)=0.99$ 로 계산한다.) [3점]

<<<C>>>① $5.88$
② $7.84$
③ $9.80$
④ $11.76$
⑤ $13.72$


<<<A>>>①

<<<S>>>출제의도 : 조건을 만족시키는 함수 $f$ 의 개수를 구할 수 있는가?

조건 (가)에 의하여
$f(1) \geq 1$

$f(2) \geq \sqrt{2}>1$

$f(3) \geq \sqrt{3}>1$

$f(4) \geq \sqrt{4}=2$

$f(5) \geq \sqrt{5}>2$

이고 조건 (나)에 의하여 치역으로 가능한 경우는

$\{1,2,3\},\{1,2,4\},\{1,3,4\},\{2,3,4\}$
이다.

(i) 치역이 $\{1,2,3\}$ 인 경우

$f(1)=1, f(5)=3$ 이므로 $\{2,3,4\}$ 에서 $\{2,3\}$ 으로의 함수 중에서 치역이 $\{3\}$ 인 함수를 제외하면 되므로 조건을 만족시키는 함수의 개수는

${ }_{2} \Pi_{3}-1=2^{3}-1=7$

(ii) 치역이 $\{1,2,4\}$ 인 경우

(i)의 경우와 마찬가지로 조건을 만족시키는 함수의 개수는 $7$ 이다.

(iii) 치역이 $\{1,3,4\}$ 인 경우

$f(1)=1$ 이므로 $\{2,3,4,5\}$ 에서 $\{3,4\}$ 로의 함수 중에서 치역이 $\{3\},\{4\}$ 인 함수를
제외하면 되므로 조건을 만족시키는 함수의 개수는

${ }_{2} \Pi_{4}-2=2^{4}-2=14$

(iv) 치역이 $\{2,3,4\}$ 인 경우

(iv)-① $f(5)=3$ 인 경우

$\{1,2,3,4\}$ 에서 $\{2,3,4\}$ 로의 함수 중에서 치역이 $\{2\},\{3\},\{4\},\{2,3\},\{3,4\}$ 인 함수를 제외하면 되므로 조건을 만족시키는 함수의 개수는

${ }_{3} \Pi_{4}-\left\{3+\left({ }_{2} \Pi_{4}-2\right) \times 2\right\}$

$=3^{4}-\left\{3+\left(2^{4}-2\right) \times 2\right\}$

$=81-31$

$=50$

(iv)-② $f(5)=4$ 인 경우

(iv)-①의 경우와 마찬가지로 조건을 만족시키는 함수의 개수는 $50$ 이다.

(i), (ii), (iii), (iv)에 의하여 구하는 함수 $f$ 의 개수는

$7+7+14+50 \times 2=128$


<<<P>>>28. 두 집합 $X=\{1,2,3,4,5\}, Y=\{1,2,3,4\}$ 에 대하여 다음 조건을 만족시키는 $X$ 에서 $Y$ 로의 함수 $f$ 의 개수는? [4점]

(가) 집합 $X$ 의 모든 원소 $x$ 에 대하여 $f(x) \geq \sqrt{x}$ 이다.

(나) 함수 $f$ 의 치역의 원소의 개수는 $3$이다.

<<<C>>>① $128$
② $138$
③ $148$
④ $158$
⑤ $168$


<<<A>>>$31$

<<<S>>>출제의도 : 확률밀도함수의 그래프를 이용하여 연속확률변수의 확률을 구할 수 있는가?

$0 \leq x \leq 6$ 인 모든 실수 $x$ 에 대하여 $f(x)+g(x)=k$ (k는 상수)이므로

$g(x)=k-f(x)$

이때, $\quad 0 \leq Y \leq 6$ 이고 확률밀도함수의 정의에 의하여 $g(x)=k-f(x) \geq 0$

즉, $k \geq f(x)$ 이므로 그림과 같이 세 직선 $x=0, x=6, y=k$ 및 함수 $y=f(x)$ 의 그래프로 둘러싸인 색칠된 부분의 넓이는 $1$ 이다.

[IMG]

또한, $0 \leq x \leq 6$ 에서 함수 $y=f(x)$ 의 그래프와 $x$ 축으로 둘러싸인 부분의 넓이도 $1$ 이므로
$k \times 6=2$

따라서, $k=\dfrac{1}{3}$

이때, $\mathrm{P}(6 k \leq Y \leq 15 k)=\mathrm{P}(2 \leq Y \leq 5)$

이고 이 값은 세 직선 $x=2, \quad x=5$, $y=\dfrac{1}{3}$ 및 함수 $y=f(x)$ 의 그래프로 둘
러싸인 부분의 넓이와 같고,

$0 \leq x \leq 3$에서 $f(x)=\dfrac{1}{12} x$ 이므로

$\mathrm{P}(6 k \leq Y \leq 15 k)$

$=\mathrm{P}(2 \leq Y \leq 5)$

$=\left[\dfrac{1}{3}-\dfrac{1}{2} \times\{f(2)+f(3)\} \times 1\right]$

$=\left\{\dfrac{1}{3}-\dfrac{1}{2} \times\left(\dfrac{1}{6}+\dfrac{1}{4}\right)\right\}+\left(\dfrac{2}{3}-\dfrac{1}{2}\right)$

$=\dfrac{3}{24}+\dfrac{1}{6}$

$=\dfrac{7}{24}$

따라서, $p=24, q=7$ 이므로 $p+q=31$


<<<P>>>29. 두 연속확률변수 $X$ 와 $Y$ 가 갖는 값의 범위는 $0 \leq X \leq 6$, $0 \leq Y \leq 6$ 이고, $X$ 와 $Y$ 의 확률밀도함수는 각각 $f(x), g(x)$ 이다. 확률변수 $X$ 의 확률밀도함수 $f(x)$ 의 그래프는 그림과 같다.

[IMG]

$0 \leq x \leq 6$ 인 모든 $x$ 에 대하여
$$
f(x)+g(x)=k \quad(k \text { 는 상수 })
$$
를 만족시킬 때, $\mathrm{P}(6 k \leq Y \leq 15 k)=\dfrac{q}{p}$ 이다. $p+q$ 의 값을 구하시오. (단, $p$ 와 $q$ 는 서로소인 자연수이다.) [4점]


<<<A>>>$191$

<<<S>>>출제의도 : 조건부확률을 구할 수 있는가?

$a_{5}+b_{5} \geq 7$ 인 사건을 $A, a_{k}=b_{k}$ 인 자연수 $k(1 \leq k \leq 5)$ 가 존재하는 사건을 $B$ 라 하자.

사건 $A$ 가 일어나는 경우는

$a_{5}+b_{5}=7=2+2+1+1+1$

$a_{5}+b_{5}=8=2+2+2+1+1$

$a_{5}+b_{5}=9=2+2+2+2+1$

$a_{5}+b_{5}=10=2+2+2+2+2$

이고 주사위의 눈의 수가 $5$ 이상일 확률은 $\dfrac{1}{3}, 4$ 이하일 확률은 $\dfrac{2}{3}$ 이므로

(i) $a_{5}+b_{5}=7$ 일 확률은

${ }_{5} \mathrm{C}_{2}\left(\dfrac{1}{3}\right)^{2}\left(\dfrac{2}{3}\right)^{3}=10 \times \dfrac{8}{3^{5}}$

(ii) $a_{5}+b_{5}=8$ 일 확률은

${ }_{5} \mathrm{C}_{3}\left(\dfrac{1}{3}\right)^{3}\left(\dfrac{2}{3}\right)^{2}=10 \times \dfrac{4}{3^{5}}$

(iii) $a_{5}+b_{5}=9$ 일 확률은

${ }_{5} \mathrm{C}_{4}\left(\dfrac{1}{3}\right)^{4}\left(\dfrac{2}{3}\right)^{1}=5 \times \dfrac{2}{3^{5}}$

(iv) $a_{5}+b_{5}=10$ 일 확률은

${ }_{5} \mathrm{C}_{5}\left(\dfrac{1}{3}\right)^{5}=\dfrac{1}{3^{5}}$

(i), (ii), (iii), (iv)에 의하여

$\mathrm{P}(A)=10 \times \dfrac{8}{3^{5}}+10 \times \dfrac{4}{3^{5}}+5 \times \dfrac{2}{3^{5}}+\dfrac{1}{3^{5}}$

또한, 사건 $A \cap B$ 인 경우는 (i), (ii)의 경우 $3$ 번째 시행까지 $5$ 이상의 눈의 수가 $1$ 번, $4$ 이하의 눈의 수가 $2$ 번 일어나야 하고 (iii), (iv)인 경우는 사건 $A \cap B$ 은 일어나지 않는다.

$\mathrm{P}(A \cap B)$

$={ }_{3} \mathrm{C}_{1}\left(\dfrac{1}{3}\right)^{1}\left(\dfrac{2}{3}\right)^{2} \times{ }_{2} \mathrm{C}_{1}\left(\dfrac{1}{3}\right)\left(\dfrac{2}{3}\right) +{ }_{3} \mathrm{C}_{1}\left(\dfrac{1}{3}\right)\left(\dfrac{2}{3}\right)^{2} \times\left(\dfrac{1}{3}\right)^{2} $

$=3 \times \dfrac{16}{3^{5}}+3 \times \dfrac{4}{3^{5}}$

그러므로, 구하는 확률은

$\mathrm{P}(B \mid A)$

$=\dfrac{\mathrm{P}(A \cap B)}{\mathrm{P}(A)}$

$3 \times \dfrac{16}{3^{5}}+3 \times \dfrac{4}{3^{5}}$

$10 \times \dfrac{8}{3^{5}}+10 \times \dfrac{4}{3^{5}}+5 \times \dfrac{2}{3^{5}}+\dfrac{1}{3^{5}}$

$=\dfrac{48+12}{80+40+10+1}$

$=\dfrac{60}{131}$

이므로 $p=131, q=60$

따라서, $p+q=131+60=191$


<<<P>>>30. 흰 공과 검은 공이 각각 $10$개 이상 들어 있는 바구니와 비어 있는 주머니가 있다. 한 개의 주사위를 사용하여 다음 시행을 한다.

주사위를 한 번 던져

나온 눈의 수가 $5$이상이면

바구니에 있는 흰 공 $2$개를 주머니에 넣고,

나온 눈의 수가 $4$이하이면

바구니에 있는 검은 공 $1$개를 주머니에 넣는다.

위의 시행을 $5$번 반복할 때, $n(1 \leq n \leq 5)$ 번째 시행 후 주머니에 들어 있는 흰 공과 검은 공의 개수를 각각 $a_{n}, b_{n}$ 이라 하자. $a_{5}+b_{5} \geq 7$ 일 때, $a_{k}=b_{k}$ 인 자연수 $k(1 \leq k \leq 5)$ 가
존재할 확률은 $\dfrac{q}{p}$ 이다. $p+q$ 의 값을 구하시오. (단, $p$ 와 $q$ 는 서로소인 자연수이다.) [4점]


<<<A>>>⑤

<<<S>>>출제의도 : 수열의 극한값을 구할 수 있는가?

$\displaystyle\lim _{n \rightarrow \infty} \dfrac{\dfrac{5}{n}+\dfrac{3}{n^{2}}}{\dfrac{1}{n}-\dfrac{2}{n^{3}}}$

$=\displaystyle\lim _{n \rightarrow \infty} \dfrac{\left(\dfrac{5}{n}+\dfrac{3}{n^{2}}\right) \times n}{\left(\dfrac{1}{n}-\dfrac{2}{n^{3}}\right) \times n}$

$=\displaystyle\lim _{n \rightarrow \infty} \dfrac{5+\dfrac{3}{n}}{1-\dfrac{2}{n^{2}}}$

$=\dfrac{5+0}{1-0}=5$


<<<P>>>23. $\displaystyle\lim _{n \rightarrow \infty} \dfrac{\dfrac{5}{n}+\dfrac{3}{n^{2}}}{\dfrac{1}{n}-\dfrac{2}{n^{3}}}$ 의 값은? [2점]

<<<C>>>① $1$
② $2$
③ $3$
④ $4$
⑤ $5$


<<<A>>>④

<<<S>>>출제의도 : 합성함수의 미분법을 이용하여 미분계수를 구할 수 있는가?

$f\left(x^{3}+x\right)=e^{x}$ 의 양변을 $x$ 에 대하여 미분하면

$f^{\prime}\left(x^{3}+x\right) \times\left(3 x^{2}+1\right)=e^{x} \cdots$ ㉠

이다.

$x^{3}+x=2$ 에서

$x^{3}+x-2=(x-1)\left(x^{2}+x+2\right)=0$

이므로 $x=1$ 이다.

따라서 ㉠의 양변에 $x=1$ 을 대입하면

$f^{\prime}(1+1) \times(3+1)=e$
이므로

$f^{\prime}(2)=\dfrac{e}{4}$


<<<P>>>24. 실수 전체의 집합에서 미분가능한 함수 $f(x)$ 가 모든 실수 $x$ 에 대하여
$$
f\left(x^{3}+x\right)=e^{x}
$$
을 만족시킬 때, $f^{\prime}(2)$ 의 값은? [3점]

<<<C>>>① $e$
② $\dfrac{e}{2}$
③ $\dfrac{e}{3}$
④ $\dfrac{e}{4}$
⑤ $\dfrac{e}{5}$


<<<A>>>②

<<<S>>>출제의도 : 등비급수의 합을 구할 수 있는가?

등비수열 $\left\{a_{n}\right\}$ 의 첫째항을 $a$, 공비를 $r$라 하면

$a_{n}=a r^{n-1}$

이때
$a_{2 n-1}-a_{2 n}=a r^{2 n-2}-a r^{2 n-1}$

$=a r^{2 n-2}(1-r)$

$=a(1-r)\left(r^{2}\right)^{n-1}$

이므로 수열 $\left\{a_{2 n-1}-a_{2 n}\right\}$ 은 첫째항이
$a(1-r)$ 이고 공비가 $r^{2}$ 인 등비수열이다.

따라서 $\displaystyle\sum_{n=1}^{\infty}\left(a_{2 n-1}-a_{2 n}\right)=3$ 에서

$-1< r< 1$ 이고 $\dfrac{a(1-r)}{1-r^{2}}=3$ 이고 $r \neq 1$ 이므로

$\dfrac{a}{1+r}=3 \cdots$ ㉠

또, $\displaystyle\sum_{n=1}^{\infty} a_{n}^{2}=\displaystyle\sum_{n=1}^{\infty} a^{2} r^{2 n-2}=6$ 이므로

$\dfrac{a^{2}}{1-r^{2}}=\dfrac{a}{1-r} \times \dfrac{a}{1+r}=6$

따라서 ㉠에서 $\dfrac{a}{1-r} \times 3=6$이므로

$\dfrac{a}{1-r}=2$

따라서
$\displaystyle\sum_{n=1}^{\infty} a_{n}=\displaystyle\sum_{n=1}^{\infty} a r^{n-1}$

$=\dfrac{a}{1-r}=2$


<<<P>>>25. 등비수열 $\left\{a_{n}\right\}$ 에 대하여
$$
\displaystyle\sum_{n=1}^{\infty}\left(a_{2 n-1}-a_{2 n}\right)=3, \quad \displaystyle\sum_{n=1}^{\infty} a_{n}^{2}=6
$$
일 때, $\displaystyle\sum_{n=1}^{\infty} a_{n}$ 의 값은? [3점]

<<<C>>>① $1$
② $2$
③ $3$
④ $4$
⑤ $5$


<<<A>>>③

<<<S>>>출제의도 : 급수와 정적분의 관계를 이용하여 값을 구할 수 있는가?

$\displaystyle\lim _{n \rightarrow \infty} \displaystyle\sum_{k=1}^{n} \dfrac{k^{2}+2 k n}{k^{3}+3 k^{2} n+n^{3}}$

$=\displaystyle\lim _{n \rightarrow \infty} \displaystyle\sum_{k=1}^{n}\left\{\dfrac{\left(\dfrac{k}{n}\right)^{2}+2 \times \dfrac{k}{n}}{\left(\dfrac{k}{n}\right)^{3}+3 \times\left(\dfrac{k}{n}\right)^{2}+1} \times \dfrac{1}{n}\right\}$

$=\displaystyle\int_{0}^{1} \dfrac{x^{2}+2 x}{x^{3}+3 x^{2}+1} d x$

$=\left[\dfrac{1}{3} \ln \left(x^{3}+3 x^{2}+1\right)\right]_{0}^{1}$

$=\dfrac{1}{3}(\ln 5-\ln 1)$

$=\dfrac{\ln 5}{3}$


<<<P>>>26. $\displaystyle\lim _{n \rightarrow \infty} \displaystyle\sum_{k=1}^{n} \dfrac{k^{2}+2 k n}{k^{3}+3 k^{2} n+n^{3}}$ 의 값은? [3점]

<<<C>>>① $\ln 5$
② $\dfrac{\ln 5}{2}$
③ $\dfrac{\ln 5}{3}$
④ $\dfrac{\ln 5}{4}$
⑤ $\dfrac{\ln 5}{5}$


<<<A>>>①

<<<S>>>출제의도 : 평면 위의 점이 움직인 거리를 구할 수 있는가?

곡선 $y=x^{2}$ 과 직선 $y=t^{2} x-\dfrac{\ln t}{8}$ 가 만나는 두 점의 $x$ 좌표를 각각 $\alpha, \beta$ 라 하면 두 점의 좌표는

$\left(\alpha, \alpha^{2}\right),\left(\beta, \beta^{2}\right)$

이므로 이 두 점의 중점의 좌표는

$\left(\dfrac{\alpha+\beta}{2}, \dfrac{\alpha^{2}+\beta^{2}}{2}\right) \cdots$ ㉠

이다. 두 식 $y=x^{2}, y=t^{2} x-\dfrac{\ln t}{8}$ 를 연립하면

$x^{2}=t^{2} x-\dfrac{\ln t}{8}$,

$x^{2}-t^{2} x+\dfrac{\ln t}{8}=0$

이 방정식의 두 근이 $\alpha, \beta$ 이므로 근과 계수의 관계에 의하여

$\alpha+\beta=t^{2}$,

$\alpha \beta=\dfrac{\ln t}{8}$

따라서 $\alpha^{2}+\beta^{2}=(\alpha+\beta)^{2}-2 \alpha \beta$ $\quad=t^{4}-\dfrac{\ln t}{4}$ 이므로 

㉠에서 중점의 좌표는 $\left(\dfrac{1}{2} t^{2}, \dfrac{1}{2} t^{4}-\dfrac{\ln t}{8}\right)$ 이다. 

그러므로 점 $\mathrm{P}$ 의 시각 $t$ 에서의 위치는 $x=\dfrac{1}{2} t^{2}, y=\dfrac{1}{2} t^{4}-\dfrac{\ln t}{8}$ 이다.

이때 $\dfrac{d x}{d t}=t, \dfrac{d y}{d t}=2 t^{3}-\dfrac{1}{8 t}$

$\left(\dfrac{1}{2} t^{2}, \dfrac{1}{2} t^{4}-\dfrac{\ln t}{8}\right)$ 이다.

그러므로 점 $\mathrm{P}$ 의 시각 $t$ 에서의 위치는
$x=\dfrac{1}{2} t^{2}, y=\dfrac{1}{2} t^{4}-\dfrac{\ln t}{8}$
이다.

이때
$\dfrac{d x}{d t}=t, \quad \dfrac{d y}{d t}=2 t^{3}-\dfrac{1}{8 t}$

이므로
$\sqrt{\left(\dfrac{d x}{d t}\right)^{2}+\left(\dfrac{d y}{d t}\right)^{2}}$

$=\sqrt{t^{2}+\left(2 t^{3}-\dfrac{1}{8 t}\right)^{2}}$

$=\sqrt{t^{2}+4 t^{6}-\dfrac{1}{2} t^{2}+\dfrac{1}{64 t^{2}}}$

$=\sqrt{4 t^{6}+\dfrac{1}{2} t^{2}+\dfrac{1}{64 t^{2}}}$

$=\sqrt{\left(2 t^{3}+\dfrac{1}{8 t}\right)^{2}}$

$=2 t^{3}+\dfrac{1}{8 t}$

따라서 시각 $t=1$ 에서 $t=e$ 까지 점 $\mathrm{P}$ 가 움직인 거리는

$\displaystyle\int_{1}^{e} \sqrt{\left(\dfrac{d x}{d t}\right)^{2}+\left(\dfrac{d y}{d t}\right)^{2} d t}$

$=\displaystyle\int_{1}^{e}\left(2 t^{3}+\dfrac{1}{8 t}\right) d t$

$=\left[\dfrac{1}{2} t^{4}+\dfrac{1}{8} \ln |t|\right]_{1}^{e}$

$=\dfrac{1}{2} e^{4}+\dfrac{1}{8}-\left(\dfrac{1}{2}+0\right)$

$=\dfrac{e^{4}}{2}-\dfrac{3}{8}$


<<<P>>>27. 좌표평면 위를 움직이는 점 $\mathrm{P}$ 의 시각 $t(t>0)$ 에서의 위치가 곡선 $y=x^{2}$ 과 직선 $y=t^{2} x-\dfrac{\ln t}{8}$ 가 만나는 서로 다른 두 점의 중점일 때, 시각 $t=1$ 에서 $t=e$ 까지 점 $\mathrm{P}$ 가 움직인 거리는? [3점]

<<<C>>>① $\dfrac{e^{4}}{2}-\dfrac{3}{8}$

② $\dfrac{e^{4}}{2}-\dfrac{5}{16}$

③ $\dfrac{e^{4}}{2}-\dfrac{1}{4}$

④ $\dfrac{e^{4}}{2}-\dfrac{3}{16}$

⑤ $\dfrac{e^{4}}{2}-\dfrac{1}{8}$


<<<A>>>②

<<<S>>>출제의도 : 합성함수의 미분법을 이용하여 극소가 되는 $x$ 의 개수를 구할 수 있는가?

$g(x)=3 f(x)+4 \cos f(x)$ 이므로
$g^{\prime}(x)=3 f^{\prime}(x)-4 f^{\prime}(x) \sin f(x)$

$\quad=f^{\prime}(x)\{3-4 \sin f(x)\}$

$\quad=12 \pi(x-1)\left\{3-4 \sin \left(6 \pi(x-1)^{2}\right)\right\}$

이므로 $g^{\prime}(x)=0$ 에서

$x=1$ 또는 $\sin \left(6 \pi(x-1)^{2}\right)=\dfrac{3}{4}$

(i) $x=1$ 일 때

$x=1$ 일 때 $\sin \left(6 \pi(x-1)^{2}\right)=0$ 이므로

$x=1$ 부근에서 $3-4 \sin \left(6 \pi(x-1)^{2}\right)>0$
이다.

이때 $x-1$ 은 $x=1$ 의 좌우에서 음에서 양으로 변하므로

$g^{\prime}(x)=12 \pi(x-1)\left\{3-4 \sin \left(6 \pi(x-1)^{2}\right)\right\}$ 도

$x=1$ 의 좌우에서 음에서 양으로 변한다.

따라서 함수 $g(x)$ 는 $x=1$ 에서 극소이다.

(ii) $1< x< 2$ 일 때

$12 \pi(x-1)>0$ 이고, 함수 $f(x)$ 는 구간 $[1,2]$ 에서 $0$ 에서 $6 \pi$ 까지 증가한다.

즉, $f(x)=t$ 라 하면 $x$ 의 값이 $1$ 에서 $2$ 까지 증가할 때 $t$ 의 값은 $0$ 에서 $6 \pi$ 까
지 증가한다.

이때 함수 $y=3-4 \sin t$ 의 그래프는 다음과 같으므로 $t=\alpha, \beta, \gamma$ 의 좌우에서
$y=3-4 \sin t$ 의 값은 음에서 양으로 변한다.

[IMG]

따라서 $f(x)=\alpha, \beta, \gamma$ 인 $x$ 의 좌우에서 $y=3-4 \sin f(x)$ 의 값은

음에서 양으로 변하고 이러한 $x$ 는 세 수 $\alpha, \beta, \gamma$ 에 대하여 각각 하나씩 존재한다.

따라서 함수 $g(x)$ 가 $1< x< 2$ 에서 극소가 되는 $x$ 의 개수는 $3$ 이다.

(iii) $0< x< 1$ 일 때

함수 $y=f(x)$ 의 그래프는 직선 $x=1$ 에 대하여 대칭이다.

즉, 모든 실수 $x$ 에 대하여 $f(1-x)=f(1+x)$ 가 성립한다.

이때
$g(1-x)=3 f(1-x)+4 \cos f(1-x)$

$=3 f(1+x)+4 \cos f(1+x)$

$=g(1+x)$이므로 함수 $y=g(x)$ 의 그래프도 직선 $x=1$ 에 대하여 대칭이다.

따라서 (iii)와 같이 $0< x< 1$ 에서 함수 $g(x)$ 가 극소가 되는 $x$ 의 개수도 $3$ 이다.

(i), (ii), (iii)에서 구하는 $x$ 의 개수는 $1+3+3=7$이다.

[참고]

$0< x< 2$ 에서 함수 $y=g(x)$ 의 그래프는 다음과 같다.

[IMG]

<<<P>>>28. 함수 $f(x)=6 \pi(x-1)^{2}$ 에 대하여 함수 $g(x)$ 를
$$
g(x)=3 f(x)+4 \cos f(x)
$$
라 하자. $0< x< 2$ 에서 함수 $g(x)$ 가 극소가 되는 $x$ 의 개수는? [4점]

<<<C>>>① $6$
② $7$
③ $8$
④ $9$
⑤ $10$


<<<A>>>$11$

<<<S>>>출제의도 : 도형의 성질을 이용하여 삼각함수의 극한값을 구할 수 있는가?


선분 $\mathrm{AB}$ 의 중점을 $\mathrm{M}$ 이라 하면

$\angle \mathrm{AMQ}=2 \times \angle \mathrm{ABQ}=2 \times 2 \theta=4 \theta$
이므로

(부채꼴 $\mathrm{AMQ}$의 넓이)

$=\dfrac{1}{2} \times 1^{2} \times 4 \theta=2 \theta$,

(삼각형 $\mathrm{MBQ}$ 의 넓이)

$=\dfrac{1}{2} \times 1^{2} \times \sin (\pi-4 \theta)=\dfrac{1}{2} \sin 4 \theta$

삼각형 $\mathrm{RAB}$ 에서 $\angle \mathrm{ARB}=\pi-3 \theta$ 이므로
사인법칙에 의하여

$\dfrac{2}{\sin (\pi-3 \theta)}=\dfrac{\overline{\mathrm{BR}}}{\sin \theta}$,

즉
$\overline{\mathrm{BR}}=\dfrac{2 \sin \theta}{\sin 3 \theta}$
이므로

(삼각헝 $\mathrm{RAB}$ 의 넓이)

$=\dfrac{1}{2} \times \overline{\mathrm{AB}} \times \overline{\mathrm{BR}} \times \sin 2 \theta$

$=\dfrac{1}{2} \times 2 \times \dfrac{2 \sin \theta}{\sin 3 \theta} \times \sin 2 \theta=\dfrac{2 \sin \theta \sin 2 \theta}{\sin 3 \theta}$

그러므로
$f(\theta)=2 \theta+\dfrac{1}{2} \sin 4 \theta-\dfrac{2 \sin \theta \sin 2 \theta}{\sin 3 \theta}$
이므로

$\displaystyle\lim _{\theta \rightarrow 0+} \dfrac{f(\theta)}{\theta}$

$=\displaystyle\lim _{\theta \rightarrow 0+}\left(2+2 \times \dfrac{\sin 4 \theta}{4 \theta}-\dfrac{4 \times \dfrac{\sin \theta}{\theta} \times \dfrac{\sin 2 \theta}{2 \theta}}{3 \times \dfrac{\sin 3 \theta}{3 \theta}}\right)$

$=2+2-\dfrac{4}{3}=\dfrac{8}{3} \cdots$ ㉠

[IMG]

정삼각형 $\mathrm{STU}$ 의 한 변의 길이를 $a$ 라 하면 삼각형 $\mathrm{TSB}$ 에서 사인법칙에 의하여

$\dfrac{a}{\sin 2 \theta}=\dfrac{\overline{\mathrm{BT}}}{\sin \dfrac{\pi}{3}}$,

$\overline{\mathrm{BT}}=\dfrac{\sqrt{3} a}{2 \sin 2 \theta}$

두 삼각형 $\mathrm{RUT}, \mathrm{RAB}$ 가 서로 닮음이므로

$\overline{\mathrm{RT}}: \overline{\mathrm{RB}}=\overline{\mathrm{UT}}: \overline{\mathrm{AB}}$

$\dfrac{2 \sin \theta}{\sin 3 \theta}-\dfrac{\sqrt{3} a}{2 \sin 2 \theta}: \dfrac{2 \sin \theta}{\sin 3 \theta}=a: 2$

$\dfrac{2 \sin \theta}{\sin 3 \theta} a=\dfrac{4 \sin \theta}{\sin 3 \theta}-\dfrac{\sqrt{3}}{\sin 2 \theta} a$

$\left(\dfrac{2 \sin \theta}{\sin 3 \theta}+\dfrac{\sqrt{3}}{\sin 2 \theta}\right) a=\dfrac{4 \sin \theta}{\sin 3 \theta}$

$\dfrac{2 \sin \theta \sin 2 \theta+\sqrt{3} \sin 3 \theta}{\sin 2 \theta \sin 3 \theta} a=\dfrac{4 \sin \theta}{\sin 3 \theta}$

$a=\dfrac{4 \sin \theta}{\sin 3 \theta} \times \dfrac{\sin 2 \theta \sin 3 \theta}{2 \sin \theta \sin 2 \theta+\sqrt{3} \sin 3 \theta}$

이때
$g(\theta)=\dfrac{\sqrt{3}}{4} a^{2}$이고

$\displaystyle\lim _{\theta \rightarrow 0+} \dfrac{a}{\theta}$

$=\displaystyle\lim _{\theta \rightarrow 0+}\left(\dfrac{4 \sin \theta}{\sin 3 \theta} \times \dfrac{\sin 2 \theta \sin 3 \theta}{2 \sin \theta \sin 2 \theta+\sqrt{3} \sin 3 \theta} \times \dfrac{1}{\theta}\right)$

$=\displaystyle\lim _{\theta \rightarrow 0+}\left(\dfrac{4 \sin \theta}{\sin 3 \theta} \times \dfrac{\sin 2 \theta \sin 3 \theta}{2 \sin \theta \sin 2 \theta+\sqrt{3} \sin 3 \theta}\right)$

$=\dfrac{4}{3} \times \dfrac{2 \times 3}{0+3 \sqrt{3}}$

$=\dfrac{8}{3 \sqrt{3}}$이므로

$\displaystyle\lim _{\theta \rightarrow 0+} \dfrac{g(\theta)}{\theta^{2}}=\dfrac{\sqrt{3}}{4} \displaystyle\lim _{\theta \rightarrow 0+}\left(\dfrac{a}{\theta}\right)^{2}$

$=\dfrac{\sqrt{3}}{4} \times\left(\dfrac{8}{3 \sqrt{3}}\right)^{2}$

$=\dfrac{16 \sqrt{3}}{27} \cdots$ ㉢

따라서 ㉠, ㉡에서
$\displaystyle\lim _{\theta \rightarrow 0+} \dfrac{g(\theta)}{\theta \times f(\theta)}$

$=\displaystyle\lim _{\theta \rightarrow 0+} \dfrac{\dfrac{g(\theta)}{f(\theta)}}{\theta}$

$=\dfrac{\displaystyle\lim _{\theta \rightarrow 0+} \dfrac{g(\theta)}{\theta^{2}}}{\displaystyle\lim _{\theta \rightarrow 0+} \dfrac{f(\theta)}{\theta}}$

$=\dfrac{16 \sqrt{3}}{27}$

$=\dfrac{8}{3}$

$\dfrac{2}{9} \sqrt{3}$

이므로 $p+q=9+2=11$


<<<P>>>29. 그림과 같이 길이가 $2$인 선분 $\mathrm{AB}$ 를 지름으로 하는 반원이 있다. 호 $\mathrm{AB}$ 위에 두 점 $\mathrm{P}, \mathrm{Q}$ 를 $\angle \mathrm{PAB}=\theta, \angle \mathrm{QBA}=2 \theta$ 가 되도록 잡고, 두 선분 $\mathrm{AP}, \mathrm{BQ}$ 의 교점을 $\mathrm{R}$ 라 하자.
선분 $\mathrm{AB}$ 위의 점 $\mathrm{S}$, 선분 $\mathrm{BR}$ 위의 점 $\mathrm{T}$, 선분 $\mathrm{AR}$ 위의 점 $\mathrm{U}$ 를 선분 $\mathrm{UT}$ 가 선분 $\mathrm{AB}$ 에 평행하고 삼각형 $\mathrm{STU}$ 가 정삼각형이 되도록 잡는다. 두 선분 $\mathrm{AR}, \mathrm{QR}$ 와 호 $\mathrm{AQ}$ 로 둘러싸인 부분의 넓이를 $f(\theta)$, 삼각형 $\mathrm{STU}$ 의 넓이를 $g(\theta)$ 라 할 때, $\displaystyle\lim _{\theta \rightarrow 0+} \dfrac{g(\theta)}{\theta \times f(\theta)}=\dfrac{q}{p} \sqrt{3}$ 이다. $p+q$ 의 값을 구하시오. (단, $0< \theta< \dfrac{\pi}{6}$ 이고, $p$ 와 $q$ 는 서로소인 자연수이다.) [4점]

[IMG]

<<<A>>>$143$

<<<S>>>출제의도 : 치환적분법과 부분적분법 을 활용하여 정적분의 값을 구할 수 있는가?

조건 (가)에서 $f(1)=1$ 이므로 조건 (나)에 의하여
$g(2)=2 f(1)=2$

따라서 $f(2)=2$ 이므로
$g(4)=2 f(2)=4$

따라서 $f(4)=4$ 이므로
$g(8)=2 f(4)=8$

따라서 $f(8)=8$ 이다.

[IMG]

부분적분법에 의하여

$\displaystyle\int_{1}^{8} x f^{\prime}(x) d x$

$=[x f(x)]_{1}^{8}-\displaystyle\int_{1}^{8} f(x) d x$

$=8 f(8)-f(1)-\displaystyle\int_{1}^{8} f(x) d x$

$=8 \times 8-1-\displaystyle\int_{1}^{8} f(x) d x$

$=63-\displaystyle\int_{1}^{8} f(x) d x \cdots$ 기

$\displaystyle\int_{1}^{8} f(x) d x$

$=\displaystyle\int_{1}^{2} f(x) d x+\displaystyle\int_{2}^{4} f(x) d x+\displaystyle\int_{4}^{8} f(x) d x$

$\cdots$ ㉡
이고,

$\displaystyle\int_{1}^{2} f(x) d x=\dfrac{5}{4} \cdots$ ㉢
이다.

이때 두 함수 $y=f(x), y=g(x)$ 의 그래
프의 대칭성에 의하여
$\displaystyle\int_{2}^{4} f(x) d x=4 \times 4-2 \times 2-\displaystyle\int_{2}^{4} g(y) d y$

$=12-\displaystyle\int_{2}^{4} g(y) d y \cdots$ ㉣
이때 $y=2 t$ 로 놓으면 치환적분법에 의하여

$\displaystyle\int_{2}^{4} g(y) d y=2 \displaystyle\int_{1}^{2} g(2 t) d t$

이므로 조건 (나)에서

$\displaystyle\int_{2}^{4} g(y) d y=2 \displaystyle\int_{1}^{2} g(2 t) d t$

$=2 \displaystyle\int_{1}^{2} 2 f(t) d t$

$=4 \displaystyle\int_{1}^{2} f(x) d x$

$=4 \times \dfrac{5}{4}=5$

㉣에서
$\displaystyle\int_{2}^{4} f(x) d x=12-\displaystyle\int_{2}^{4} g(y) d y$

$=12-5=7 \cdots$ ㉤

또, 두 함수 $y=f(x), y=g(x)$ 의 그래프 의 대칭성에 의하여
$$
\displaystyle\int_{4}^{8} f(x) d x=8 \times 8-4 \times 4-\displaystyle\int_{4}^{8} g(y) d y
$$

$=48-\displaystyle\int_{4}^{8} g(y) d y \cdots$ ㉥

이때 $y=2 t$ 로 놓으면 치환적분법에 의하 여 $\displaystyle\int_{4}^{8} g(y) d y=2 \displaystyle\int_{2}^{4} g(2 t) d t$ 이므로 

조건 (나)에서 $\displaystyle\int_{4}^{8} g(y) d y=2 \displaystyle\int_{2}^{4} g(2 t) d t$

이므로 조건 (나)에서

$\displaystyle\int_{4}^{8} g(y) d y=2 \displaystyle\int_{2}^{4} g(2 t) d t$

$=2 \displaystyle\int_{2}^{4} 2 f(t) d t$

$=4 \displaystyle\int_{2}^{4} f(x) d x$

$=4 \times 7=28$

㉥에서 $\displaystyle\int_{4}^{8} f(x) d x=48-\displaystyle\int_{4}^{8} g(y) d y$ $=48-28=20 \cdots$ ㉦

㉡, ㉢, ㉤, ㉦에서

$\displaystyle\int_{1}^{8} f(x) d x$ $=\displaystyle\int_{1}^{2} f(x) d x+\displaystyle\int_{2}^{4} f(x) d x+\displaystyle\int_{4}^{8} f(x) d x$ $=\dfrac{5}{4}+7+20=\dfrac{113}{4}$

이므로 ㉠에서

$\displaystyle\int_{1}^{8} x f^{\prime}(x) d x$ $=63-\displaystyle\int_{1}^{8} f(x) d x$

$=63-\dfrac{113}{4}=\dfrac{139}{4}$

따라서
$p+q=4+139=143$

[다른 풀이]
$\displaystyle\int_{1}^{8} x f^{\prime}(x) d x$ 에서 $x=g(y)$ 라 하면

$x=1$ 일 때 $y=1, x=8$ 일 때 $y=8$ 이고,

$\dfrac{d x}{d y}=g^{\prime}(y)=\dfrac{1}{f^{\prime}(x)}$

이므로
$\displaystyle\int_{1}^{8} x f^{\prime}(x) d x=\displaystyle\int_{1}^{8} g(y) d y$

$=\displaystyle\int_{1}^{2} g(y) d y+\displaystyle\int_{2}^{4} g(y)+\displaystyle\int_{4}^{8} g(y) d y$

이때
$\displaystyle\int_{1}^{2} g(y) d y=2 \times 2-1 \times 1-\displaystyle\int_{1}^{2} f(x) d x$

$=3-\dfrac{5}{4}=\dfrac{7}{4}$

한편, $\displaystyle\int_{2}^{4} g(y) d y=\displaystyle\int_{2}^{4} 2 f\left(\dfrac{y}{2}\right) d y$ 에서

$\dfrac{y}{2}=t$ 라 하면 $y=2$ 일 때 $t=1, y=4$ 일때 $t=2$ 이고, $\dfrac{1}{2}=\dfrac{d t}{d y}$ 이므로

$\displaystyle\int_{2}^{4} g(y) d y=\displaystyle\int_{2}^{4} 2 f\left(\dfrac{y}{2}\right) d y$

$=\displaystyle\int_{1}^{2} 4 f(t) d t=4 \displaystyle\int_{1}^{2} f(t) d t$

$=4 \times \dfrac{5}{4}=5$,

또, $\displaystyle\int_{4}^{8} g(y) d y=\displaystyle\int_{4}^{8} 2 f\left(\dfrac{y}{2}\right) d y$ 에서

$\dfrac{y}{2}=t$ 라 하면 $y=4$ 일 때 $t=2, y=8$ 일때 $t=4$ 이고, $\dfrac{1}{2}=\dfrac{d t}{d y}$ 이므로

$\displaystyle\int_{4}^{8} g(y) d y=\displaystyle\int_{4}^{8} 2 f\left(\dfrac{y}{2}\right) d y$

$=\displaystyle\int_{2}^{4} 4 f(t) d t=4 \displaystyle\int_{2}^{4} f(t) d t$

$=4 \times\left\{4 \times 4-2 \times 2-\displaystyle\int_{2}^{4} g(y) d y\right\}$

$=4(12-5)=28$

따라서
$\displaystyle\int_{1}^{8} x f^{\prime}(x) d x=\displaystyle\int_{1}^{8} g(y) d y$

$=\displaystyle\int_{1}^{2} g(y) d y+\displaystyle\int_{2}^{4} g(y)+\displaystyle\int_{4}^{8} g(y) d y$

$=\dfrac{7}{4}+5+28=\dfrac{139}{4}$
이므로
$p+q=4+139=143$

< 참고>
조건 (나)의 성질 $g(2 x)=2 f(x)$에서 다음 그림과 같이 각 부분의 넓이가 대각선 방향으로 $4$ 배씩 증가함을 알 수 있다.

[IMG]


<<<P>>>30. 실수 전체의 집합에서 증가하고 미분가능한 함수 $f(x)$ 가 다음 조건을 만족시킨다.

(가) $f(1)=1, \displaystyle\int_{1}^{2} f(x) d x=\dfrac{5}{4}$

(나) 함수 $f(x)$ 의 역함수를 $g(x)$ 라 할 때, $x \geq 1$ 인 모든 실수 $x$ 에 대하여 $g(2 x)=2 f(x)$ 이다.

$\displaystyle\int_{1}^{8} x f^{\prime}(x) d x=\dfrac{q}{p}$ 일 때, $p+q$ 의 값을 구하시오.

(단, $p$ 와 $q$ 는 서로소인 자연수이다.) [4점]

<<<A>>>②

<<<S>>>출제의도 : 좌표공간의 점의 대칭이동을 이해하고 두 점 사이의 거리를 구할 수 있는가?

좌표공간의 점 $\mathrm{A}(2,1,3)$ 을 $x y$ 평면에 대하여 대칭이동시킨 점 $\mathrm{P}$ 의 좌표는
$\mathrm{P}(2,1,-3)$

점 $\mathrm{A}$ 를 $y z$ 평면에 대하여 대칭이동시킨 점 $\mathrm{Q}$ 의 좌표는
$\mathrm{Q}(-2,1,3)$

따라서 구하는 선분 $\mathrm{PQ}$ 의 길이는

$\overline{\mathrm{PQ}}=\sqrt{(2+2)^{2}}+(1-1)^{2}+(-3-3)^{2}$

$=\sqrt{52}$

$=2 \sqrt{13}$


<<<P>>>23. 좌표공간의 점 $\mathrm{A}(2,1,3)$ 을 $x y$ 평면에 대하여 대칭이동한 점을 $\mathrm{P}$ 라 하고, 점 $\mathrm{A}$ 를 $y z$ 평면에 대하여 대칭이동한 점을 $\mathrm{Q}$ 라 할 때, 선분 $\mathrm{PQ}$ 의 길이는? [2점]

<<<C>>>① $5 \sqrt{2}$
② $2 \sqrt{13}$
③ $3 \sqrt{6}$
④ $2 \sqrt{14}$
⑤ $2 \sqrt{15}$


<<<A>>>③

<<<S>>>출제의도 : 초점의 좌표가 주어진 쌍곡선의 방정식을 이해하여 쌍곡선의 주축의 길이를 구할 수 있는가?

쌍곡선 $\dfrac{x^{2}}{a^{2}}-\dfrac{y^{2}}{6}=1$ 의 한 초점의 좌표가
$(3 \sqrt{2}, 0)$ 이므로
$a^{2}+6=18$

$a^{2}=12$

$a>0$ 이므로
$a=2 \sqrt{3}$

따라서 구하는 쌍곡선의 주축의 길이는 $2 a=2 \times 2 \sqrt{3}=4 \sqrt{3}$


<<<P>>>24. 한 초점의 좌표가 $(3 \sqrt{2}, 0)$ 인 쌍곡선 $\dfrac{x^{2}}{a^{2}}-\dfrac{y^{2}}{6}=1$ 의 주축의 길이는? (단, $a$ 는 양수이다.) [3점]

<<<C>>>① $3 \sqrt{3}$
② $\dfrac{7 \sqrt{3}}{2}$
③ $4 \sqrt{3}$
④ $\dfrac{9 \sqrt{3}}{2}$
⑤ $5 \sqrt{3}$


<<<A>>>⑤

<<<S>>>출제의도 : 좌표평면의 두 직선의 방향벡터를 이해하고 이를 이용하여 두 직선이 이루는 예각의 크기에 대한 코사인 값을 구할 수 있는가?

두 직선
$\dfrac{x+1}{2}=y-3, x-2=\dfrac{y-5}{3}$

의 방향벡터를 각각
$\overrightarrow{u_{1}}=(2,1), \overrightarrow{u_{2}}=(1,3)$
이라 하면

$\cos \theta=\dfrac{\left|\overrightarrow{u_{1}} \cdot \overrightarrow{u_{2}}\right|}{\left|\overrightarrow{u_{1}}\right|\left|\overrightarrow{u_{2}}\right|}$

$\quad|2 \times 1+1 \times 3|$

$\sqrt{2^{2}+1^{2}} \times \sqrt{1^{2}+3^{2}}$

$\begin{aligned}=& \dfrac{5}{\sqrt{5} \times \sqrt{10}} \\=& \dfrac{\sqrt{2}}{2} \end{aligned}$



<<<P>>>25. 좌표평면에서 두 직선
$$
\dfrac{x+1}{2}=y-3, \quad x-2=\dfrac{y-5}{3}
$$
가 이루는 예각의 크기를 $\theta$ 라 할 때, $\cos \theta$ 의 값은? [3점]

<<<C>>>① $\dfrac{1}{2}$
② $\dfrac{\sqrt{5}}{4}$
③ $\dfrac{\sqrt{6}}{4}$
④ $\dfrac{\sqrt{7}}{4}$
⑤ $\dfrac{\sqrt{2}}{2}$


<<<A>>>②

<<<S>>>출제의도 : 타원의 정의를 이해하여 조건을 만족시키는 원의 반지름의 길이를 구할 수 있는가?

$\overline{\mathrm{AF}}=p, \overline{\mathrm{AF}^{\prime}}=q$ 라 하면 타원의 정의에
의하여

$p+q=2 \times 8=16$

원 $C$ 가 두 직선 $\mathrm{AF}, \mathrm{AF}^{\prime}$ 과 접하는 두 점을 각각 $\mathrm{P}, \mathrm{Q}$, 원 $C$ 의 반지름의 길이를 $r$ 라 하면

$\overline{\mathrm{BP}}=\overline{\mathrm{BQ}}=r$

[IMG]

사각형 $\mathrm{AFBF}^{\prime}$ 의 넓이를 삼각형 $\mathrm{ABF}$와 삼각헝 $\mathrm{ABF}^{\prime}$ 으로 나누어 생각하면

$\dfrac{1}{2} \times p \times r+\dfrac{1}{2} \times q \times r=72$

따라서

$r=72 \times \dfrac{2}{p+q}$

$=72 \times \dfrac{2}{16}$

$=9$


<<<P>>>26. 두 초점이 $\mathrm{F}, \mathrm{F}^{\prime}$ 인 타원 $\dfrac{x^{2}}{64}+\dfrac{y^{2}}{16}=1$ 위의 점 중 제 $1$사분면에 있는 점 $\mathrm{A}$ 가 있다. 두 직선 $\mathrm{AF}, \mathrm{AF}^{\prime}$ 에 동시에 접하고 중심이 $y$ 축 위에 있는 원 중 중심의 $y$ 좌표가 음수인 것을 $C$ 라 하자. 원 $C$ 의 중심을 $\mathrm{B}$ 라 할 때 사각형 $\mathrm{AFBF}^{\prime}$ 의 넓이가 $72$이다. 원 $C$ 의 반지름의 길이는? [3점]

[IMG]

<<<C>>>① $\dfrac{17}{2}$
② $9$
③ $\dfrac{19}{2}$
④ $10$
⑤ $\dfrac{21}{2}$


<<<A>>>④

<<<S>>>출제의도 : 삼수선의 정리를 이용하여 주어진 삼각형의 넓이를 구할 수 있는가?

그림과 같이 점 $\mathrm{M}$ 에서 선분 $\mathrm{EG}$ 에 내린 수선의 발을 $\mathrm{I}$, 선분 $\mathrm{EH}$ 에 내린 수선의 발을 $\mathrm{J}$ 라 하자.

[IMG]

삼수선의 정리에 의하여

$\overline{\mathrm{JI}} \perp \overline{\mathrm{EG}}$이므로 점 $\mathrm{H}$ 에서 선분 $\mathrm{EG}$ 에 내린 수선의 발을 $\mathrm{K}$ 라 하면 점 $\mathrm{K}$ 는 선분 $\mathrm{EG}$ 의 중점이고
$\overline{\mathrm{IJ}} =\dfrac{1}{2} \times \overline{\mathrm{HK}}$

$=\dfrac{1}{2} \times 2 \sqrt{2}$

$=\sqrt{2}$

한편, 직각삼각헝 $\mathrm{MJI}$ 에서 $\overline{\mathrm{MJ}} =4$이므로

$\overline{\mathrm{MI}} =\sqrt{\overline{\mathrm{MJ}}^{2}+\overline{\mathrm{IJ}}^{2}}$

$=\sqrt{16+2}$

$=3 \sqrt{2}$

따라서 구하는 삼각형 $\mathrm{MEG}$ 의 넓이는

$\dfrac{1}{2} \times \overline{\mathrm{EG}} \times \overline{\mathrm{MI}}$

$=\dfrac{1}{2} \times 4 \sqrt{2} \times 3 \sqrt{2}$

$=12$


<<<P>>>27. 그림과 같이 한 모서리의 길이가 $4$인 정육면체 $\mathrm{ABCD}-\mathrm{EFGH}$ 가 있다. 선분 $\mathrm{AD}$ 의 중점을 $\mathrm{M}$ 이라 할 때, 삼각형 $\mathrm{MEG}$ 의 넓이는? [3점]

[IMG]


<<<C>>>① $\dfrac{21}{2}$
② $11$
③ $\dfrac{23}{2}$
④ $12$
⑤ $\dfrac{25}{2}$


<<<A>>>⑤

<<<S>>>출제의도 : 포물선의 정의와 방정식을 이용하여 문제를 해결할 수 있는가?

두 점 $\mathrm{F}_{1}, \mathrm{F}_{2}$ 의 좌표가 각각

$\mathrm{F}_{1}(p, a), \mathrm{F}_{2}(-1,0)$

이고, $\overline{\mathrm{F}_{1} \mathrm{F}_{2}}=3$ 이므로

$(p+1)^{2}+a^{2}=9$

그림과 같이 점 $\mathrm{P}$ 를 지나고 $x$ 축에 수직인 직선과 점 $\mathrm{Q}$ 를 지나고 $y$ 축에 수직인
직선이 만나는 점을 $\mathrm{R}$ 라 하자.

[IMG]

직선 $\mathrm{PQ}$ 의 기울기는 직선 $\mathrm{F}_{1} \mathrm{F}_{2}$ 의 기울기와 같은 $\dfrac{a}{p+1}$ 이므로 직각삼각형 $\mathrm{PQR}$에서 양수 $t$ 에 대하여

$\overline{\mathrm{PR}}=a t, \overline{\mathrm{QR}}=(p+1) t$
로 놓을 수 있다.

이때 $\overline{\mathrm{PQ}}=1$ 이므로

$a^{2} t^{2}+(p+1)^{2} t^{2}=1$
에서

$t^{2}=\dfrac{1}{a^{2}+(p+1)^{2}}=\dfrac{1}{9}$

즉, $t=\dfrac{1}{3}$

한편, 두 점 $\mathrm{P}, \mathrm{Q}$ 의 $x$ 좌표를 각각 $x_{1}$, $x_{2}$ 라 하면

$\overline{\mathrm{PF}}_{1}=p+x_{1}$,

$\overline{\mathrm{QF}_{2}}=1-x_{2}$,

$\overline{\mathrm{PF}}_{1}+\overline{\mathrm{QF}}_{2}=2$이고

$x_{1}-x_{2}=(p+1) t=\dfrac{1}{3}(p+1)$
이므로

$\left(p+x_{1}\right)+\left(1-x_{2}\right)=2$
에서
$p=1-\left(x_{1}-x_{2}\right)$

$=1-\dfrac{1}{3}(p+1)$

즉, $p=\dfrac{1}{2}$
㉠에서

$\left(\dfrac{1}{2}+1\right)^{2}+a^{2}=9$
이므로

$a^{2}=\dfrac{27}{4}$

따라서
$a^{2}+p^{2}=\dfrac{27}{4}+\dfrac{1}{4}=7$


<<<P>>>28. 두 양수 $a, p$ 에 대하여 포물선 $(y-a)^{2}=4 p x$ 의 초점을 $\mathrm{F}_{1}$ 이라 하고, 포물선 $y^{2}=-4 x$ 의 초점을 $\mathrm{F}_{2}$ 라 하자. 선분 $\mathrm{F}_{1} \mathrm{F}_{2}$ 가 두 포물선과 만나는 점을 각각 $\mathrm{P}, \mathrm{Q}$ 라 할 때, $\overline{\mathrm{F}_{1} \mathrm{F}_{2}}=3, \overline{\mathrm{PQ}}=1$ 이다. $a^{2}+p^{2}$ 의 값은? [4점]

[IMG]


<<<C>>>① $6$
② $\dfrac{25}{4}$
③ $\dfrac{13}{2}$
④ $\dfrac{27}{4}$
⑤ $7$


<<<A>>>$100$

<<<S>>>출제의도 : 평면벡터의 연산과 내적을 이용하여 조건을 만족시키는 벡터의 크기의 최댓값과 최솟값을 구할 수 있는가?

조건 (가)에 의하여 점 $\mathrm{P}$ 는 평행사변형 $\mathrm{OACB}$ 의 둘레 또는 내부에 있는 점이다.

조건 (나)에서
$\overrightarrow{\mathrm{OP}} \cdot \overrightarrow{\mathrm{OB}}+\overrightarrow{\mathrm{BP}} \cdot \overrightarrow{\mathrm{BC}}$

$=\overrightarrow{\mathrm{OP}} \cdot \overrightarrow{\mathrm{OB}}+(\overrightarrow{\mathrm{OP}}-\overrightarrow{\mathrm{OB}}) \cdot \overrightarrow{\mathrm{OA}}$

$=\overrightarrow{\mathrm{OP}} \cdot \overrightarrow{\mathrm{OB}}+\overrightarrow{\mathrm{OP}} \cdot \overrightarrow{\mathrm{OA}}-\overrightarrow{\mathrm{OA}} \cdot \overrightarrow{\mathrm{OB}}$

$=\overrightarrow{\mathrm{OP}} \cdot(\overrightarrow{\mathrm{OA}}+\overrightarrow{\mathrm{OB}})-|\overrightarrow{\mathrm{OA}} \| \overrightarrow{\mathrm{OB}}| \cos (\angle \mathrm{AOB})$

$=\overrightarrow{\mathrm{OP}} \cdot \overrightarrow{\mathrm{OC}}-\sqrt{2} \times 2 \sqrt{2} \times \dfrac{1}{4}$

$=\overrightarrow{\mathrm{OP}} \cdot \overrightarrow{\mathrm{OC}}-1=2$
이므로

$\overrightarrow{\mathrm{OP}} \cdot \overrightarrow{\mathrm{OC}}=3$

(i) 벡터 $3 \overrightarrow{\mathrm{OP}}-\overrightarrow{\mathrm{OX}}$ 의

크기는 $\overrightarrow{\mathrm{OP}}$ 의 크기가 최대이고 $\overrightarrow{\mathrm{OX}}$ 가 $\overrightarrow{\mathrm{OP}}$ 와 반대 방향일 때 최대가 되고,

$\overrightarrow{\mathrm{OP}}$ 의 크기는 점 $\mathrm{P}$ 가 선분 $\mathrm{OA}$ 위에 있을 때 최대가 된다.

다음 그림과 같이 점 $\mathrm{C}$에서 직선 $\mathrm{OA}$ 에 내린 수선의 발을 $\mathrm{H}$ 라 하고 $\angle \mathrm{COA}=\alpha$ 라 하자.

[IMG]

$\cos (\angle \mathrm{CAH})=\cos (\angle \mathrm{AOB})=\dfrac{1}{4}$

$\mathrm{AH} =\overline{\mathrm{AC}} \times \cos (\angle \mathrm{CAH})$

$=\overline{\mathrm{OB}} \times \dfrac{1}{4}$

$=2 \sqrt{2} \times \dfrac{1}{4}$

$=\dfrac{\sqrt{2}}{2} $

한편,

$|\overrightarrow{\mathrm{OC}}|^{2}=|\overrightarrow{\mathrm{OA}}+\overrightarrow{\mathrm{OB}}|^{2}$

$=|\overrightarrow{\mathrm{OA}}|^{2}+|\overrightarrow{\mathrm{OB}}|^{2}+2 \overrightarrow{\mathrm{OA}} \cdot \overrightarrow{\mathrm{OB}}$

$=(\sqrt{2})^{2}+(2 \sqrt{2})^{2}+2 \times \sqrt{2} \times 2 \sqrt{2} \times \dfrac{1}{4}$

$=2+8+2=12$이므로

$|\overrightarrow{\mathrm{OC}}|=2 \sqrt{3}$

그러므로$\cos \alpha=\dfrac{\sqrt{2}+\dfrac{\sqrt{2}}{2}}{2 \sqrt{3}}=\dfrac{\sqrt{6}}{4}$

$\overrightarrow{\mathrm{OP}} \cdot \overrightarrow{\mathrm{OC}}=|\overrightarrow{\mathrm{OP}}||\overrightarrow{\mathrm{OC}}| \cos \alpha$

$ =|\overrightarrow{\mathrm{OP}}| \times 2 \sqrt{3} \times \dfrac{\sqrt{6}}{4} $

$=\dfrac{3 \sqrt{2}}{2}|\overrightarrow{\mathrm{OP}}|=3$ 이므로

$|\overrightarrow{\mathrm{OP}}|=\sqrt{2}$

이때 $\overrightarrow{\mathrm{OX}}$ 가 $\overrightarrow{\mathrm{OP}}$ 와 반대 방향이면

$|3 \overrightarrow{\mathrm{OP}}-\overrightarrow{\mathrm{OX}}|=3|\overrightarrow{\mathrm{OP}}|+|\overrightarrow{\mathrm{OX}}|$

이므로 $M=3 \sqrt{2}+\sqrt{2}=4 \sqrt{2}$

$|\overrightarrow{\mathrm{OP}}|=\sqrt{2}$

이때 $\overrightarrow{\mathrm{OX}}$ 가 $\overrightarrow{\mathrm{OP}}$ 와 반대 방향이면

$|3 \overrightarrow{\mathrm{OP}}-\overrightarrow{\mathrm{OX}}|=3|\overrightarrow{\mathrm{OP}}|+|\overrightarrow{\mathrm{OX}}|$
이므로

$M=3 \sqrt{2}+\sqrt{2}=4 \sqrt{2}$

(ii) 벡터 $3 \overrightarrow{\mathrm{OP}}-\overrightarrow{\mathrm{OX}}$ 의 크기는

$\overrightarrow{\mathrm{OP}}$ 의 크기가 최소이고 $\overrightarrow{\mathrm{OX}}$ 가 $\overrightarrow{\mathrm{OP}}$ 와 같은 방향일 때 최소가 되고,

$\overrightarrow{\mathrm{OP}}$ 의 크기는 점 $\mathrm{P}$ 가 선분 $\mathrm{OC}$ 위에 있을 때 최소가 된다.

이때
$\overrightarrow{\mathrm{OP}} \cdot \overrightarrow{\mathrm{OC}}=|\overrightarrow{\mathrm{OP}}||\overrightarrow{\mathrm{OC}}|$

이므로 $=|\overrightarrow{\mathrm{OP}}| \times 2 \sqrt{3}=3$

$|\overrightarrow{\mathrm{OP}}|=\dfrac{\sqrt{3}}{2}$

이때 $\overrightarrow{\mathrm{OX}}$ 가 $\overrightarrow{\mathrm{OP}}$ 와 같은 방향이면

$|3 \overrightarrow{\mathrm{OP}}-\overrightarrow{\mathrm{OX}}|=3|\overrightarrow{\mathrm{OP}}|-|\overrightarrow{\mathrm{OX}}|$

이므로 $m=3 \times \dfrac{\sqrt{3}}{2}-\sqrt{2}=\dfrac{3 \sqrt{3}}{2}-\sqrt{2}$

(i), (ii)에 의하여

$M \times m =4 \sqrt{2}\left(\dfrac{3 \sqrt{3}}{2}-\sqrt{2}\right)$
$=6 \sqrt{6}-8$

이므로 $a^{2}+b^{2} =6^{2}+(-8)^{2} =100$



<<<P>>>29. 좌표평면에서 $\overline{\mathrm{OA}}=\sqrt{2}, \overline{\mathrm{OB}}=2 \sqrt{2}$ 이고 $\cos (\angle \mathrm{AOB})=\dfrac{1}{4}$ 인 평행사변형 $\mathrm{OACB}$ 에 대하여 점 $\mathrm{P}$ 가 다음 조건을 만족시킨다.

(가) $\overrightarrow{\mathrm{OP}}=s \overrightarrow{\mathrm{OA}}+t \overrightarrow{\mathrm{OB}} \quad(0 \leq s \leq 1,0 \leq t \leq 1)$

(나) $\overrightarrow{\mathrm{OP}} \cdot \overrightarrow{\mathrm{OB}}+\overrightarrow{\mathrm{BP}} \cdot \overrightarrow{\mathrm{BC}}=2$

점 $\mathrm{O}$ 를 중심으로 하고 점 $\mathrm{A}$ 를 지나는 원 위를 움직이는 점 $\mathrm{X}$ 에 대하여 $|3 \overrightarrow{\mathrm{OP}}-\overrightarrow{\mathrm{OX}}|$ 의 최댓값과 최솟값을 각각 $M, m$ 이라 하자. $M \times m=a \sqrt{6}+b$ 일 때, $a^{2}+b^{2}$ 의 값을 구하시오. (단, $a$ 와 $b$ 는 유리수이다.) [4점]

[IMG]

<<<A>>>$23$

<<<S>>>출제의도 : 좌표공간의 구의 방정식 및 도형의 위치관계를 이해하고 정사영의 넓이를 구할 수 있는가?

점 C에서 $x y$ 평면에 내린 수선의 발을 $\mathrm{C}^{\prime}$ 이라 하면 $\overline{\mathrm{CC}^{\prime}}=5$ 이므로 구 $S$ 는 점 $\mathrm{C}^{\prime}$ 에서 $x y$ 평면에 접한다.

[IMG]

평면 $\mathrm{OPC}$ 는 점 $\mathrm{C}^{\prime}$ 을 지나므로 점 $\mathrm{Q}_{1}$ 은 직선 $\mathrm{OC}^{\prime}$ 위에 있다. 이때 선분 $\mathrm{OQ}_{1}$ 의 길이가 최대가 되려면 점 $\mathrm{Q}$ 가 점 $\mathrm{C}$ 를 지나고 직선 $\mathrm{OC}^{\prime}$ 과 평행한 직선이 구 $S$와 만나는 점 중 $x$ 좌표가 양수인 점이어야 한다.

이때
$\overline{\mathrm{OQ}_{1}}=\overline{\mathrm{OC}^{\prime}}+5=3+5=8$

한편, 삼각형 $\mathrm{OQ}_{1} \mathrm{R}_{1}$ 의 넓이가 최대가 되려면 점 $\mathrm{R}$ 가 점 $\mathrm{C}$ 를 지나고 직선 $\mathrm{CQ}$ 에 수직인 직선이 구 $S$ 와 만나는 점이어야 한다.

이때 $\overline{\mathrm{R}_{1} \mathrm{C}^{\prime}} \perp \overline{\mathrm{OC}^{\prime}}$ 이고, 

$\overline{\mathrm{R}_{1} \mathrm{C}^{\prime}}=5$ 이므로 삼각형 $\mathrm{OQ}_{1} \mathrm{R}_{1}$ 의 넓이는

$\dfrac{1}{2} \times 8 \times 5=20$

이제 삼각형 $\mathrm{PQR}$ 의 넓이를 구해 보자.

점 $\mathrm{P}$ 에서 직선 $\mathrm{QQ}_{1}$ 에 내린 수선의 발을 $\mathrm{H}$라 하면

$\overline{\mathrm{PH}}=\overline{\mathrm{OQ}_{1}}=8$,

$\overline{\mathrm{QH}}=\overline{\mathrm{QQ}_{1}}-1=4$

$\overline{\mathrm{PQ}}=\sqrt{64+16}=4 \sqrt{5}$ $\cdots \cdots$㉠

직각삼각형 $\mathrm{CQR}$ 에서

$\overline{\mathrm{QR}}=\sqrt{\overline{\mathrm{CQ}}^{2}+\overline{\mathrm{CR}}^{2}}=\sqrt{25+25}=5 \sqrt{2}$  $\cdots \cdots$ ㉡

직각삼각형 $\mathrm{OC}^{\prime} \mathrm{R}_{1}$ 에서

$\overline{\mathrm{OR}_{1}}=\sqrt{\mathrm{OC}^{\prime} 2}+\overline{\mathrm{R}_{1} \mathrm{C}^{\prime 2}}=\sqrt{9+25}=\sqrt{34}$

이므로 점 $\mathrm{P}$ 에서 직선 $\mathrm{RR}_{1}$ 에 내린 수선의 발을 $\mathrm{G}$ 라 하면

$\overline{\mathrm{PG}}=\overline{\mathrm{OR}_{1}}=\sqrt{34}$

$\overline{\mathrm{RG}}=\overline{\mathrm{RR}_{1}}-1=4$

직각삼각형 $\mathrm{RPG}$ 에서

$\overline{\mathrm{PR}}=\sqrt{\mathrm{PG}^{2}}+\overline{\mathrm{RG}}^{2}=\sqrt{34+16}=5 \sqrt{2}$  $\cdots \cdots$ ㉢,

㉠, ㉡, ㉢에 의하여 삼각형 $\quad \mathrm{PQR}$ 는 $\overline{\mathrm{PR}}=\overline{\mathrm{QR}}$ 인 이등변삼각형이다.

[IMG]

위 그림과 같이 선분 $\mathrm{PQ}$ 의 중점을 $\mathrm{M}$ 이라 하면

$\overline{\mathrm{RM}} =\sqrt{\overline{\mathrm{PR}}^{2}-\overline{\mathrm{PM}}^{2}}$

$=\sqrt{50-20}$

$=\sqrt{30}$

이므로 삼각헝 $\mathrm{PQR}$ 의 넓이는

$\dfrac{1}{2} \times 4 \sqrt{5} \times \sqrt{30}=10 \sqrt{6}$

이때 삼각형 $\mathrm{PQR}$ 의 $x y$ 평면 위로의 정사영이 삼각형 $\mathrm{OQ}_{1} \mathrm{R}_{1}$이므로 

두 평면 $\mathrm{PQR}$ 와 $\mathrm{OQ}_{1} \mathrm{R}_{1}$ 이 이루는 예각의 크기를 $\theta$ 라 하면
$\cos \theta=\dfrac{20}{10 \sqrt{6}}=\dfrac{\sqrt{6}}{3}$

따라서 삼각헝 $\mathrm{OQ}_{1} \mathrm{R}_{1}$ 의 평면 $\mathrm{PQR}$ 위로의 정사영의 넓이는

$20 \times \dfrac{\sqrt{6}}{3}=\dfrac{20}{3} \sqrt{6}$
이므로

$p+q=3+20=23$



<<<P>>>32111 2022학년도 대학수학능력시험 끝

30. 좌표공간에 중심이 $\mathrm{C}(2, \sqrt{5}, 5)$ 이고 점 $\mathrm{P}(0,0,1)$ 을 지나는 구
$$
S:(x-2)^{2}+(y-\sqrt{5})^{2}+(z-5)^{2}=25
$$
가 있다. 구 $S$ 가 평면 $\mathrm{OPC}$ 와 만나서 생기는 원 위를 움직이는 점 $\mathrm{Q}$, 구 $S$ 위를 움직이는 점 $\mathrm{R}$ 에 대하여 두 점 $\mathrm{Q}, \mathrm{R}$ 의 $x y$ 평면 위로의 정사영을 각각 $\mathrm{Q}_{1}, \mathrm{R}_{1}$ 이라 하자.
삼각형 $\mathrm{OQ}_{1} \mathrm{R}_{1}$ 의 넓이가 최대가 되도록 하는 두 점 $\mathrm{Q}, \mathrm{R}$ 에 대하여 삼각형 $\mathrm{OQ}_{1} \mathrm{R}_{1}$ 의 평면 $\mathrm{PQR}$ 위로의 정사영의 넓이는 $\dfrac{q}{p} \sqrt{6}$ 이다. $p+q$ 의 값을 구하시오.
(단, $\mathrm{O}$ 는 원점이고 세 점 $\mathrm{O}, \mathrm{Q}_{1}, \mathrm{R}_{1}$ 은 한 직선 위에 있지 않으며, $p$ 와 $q$ 는 서로소인 자연수이다.) [4점]

[IMG]








<<<A>>>①

<<<S>>>

[[거듭제곱근호를 유리수지수로 바꿉니다.]]

$\frac{1}{\sqrt[4]{3}} \times 3^{-\frac{7}{4}}$

$=3^{-\frac{1}{4}} \times 3^{-\frac{7}{4}}$

[[지수법칙을 이용하여 지수를 계산합니다.]]

$=3^{-\frac{1}{4}-\frac{7}{4}}$

$=3^{-2}=\frac{1}{9}$

<<<P>>>2022학년도 대학수학능력시험 9월 모의평가 시작

1. $\frac{1}{\sqrt[4]{3}} \times 3^{-\frac{7}{4}}$ 의 값은? [2점]

① $\frac{1}{9}$
② $\frac{1}{3}$
③ $1$
④ $3$
⑤ $9$


<<<A>>>⑤

<<<S>>>

[[주어진 함수를 미분하여 도함수를 구합니다.]]

$ f^{\prime}(x)=6 x^{2}+4$에서

[[도함수에 적당한 수를 대입하여 문제에서 요구하는 것을 구합니다.]]

$f^{\prime}(1)=6+4=10$


<<<P>>>2. 함수 $f(x)=2 x^{3}+4 x+5$ 에 대하여 $f^{\prime}(1)$ 의 값은? [2점]

① $6$
② $7$
③ $8$
④ $9$
⑤ $10$

<<<A>>>⑤

<<<S>>>

[[공비를 미지수로 도입합니다.]]

등비수열 $\left\{a_{n}\right\}$ 의 공비를 $r$ 라 하자.

[[주어진 조건으로부터 공비의 관계식을 세웁니다.]]

$a_{2} a_{4}=a_{1} r \times a_{1} r^{3}=a_{1}^{2} r^{4}=4 r^{4}=36$에서 $r^{4}=9$이다.

[[앞에서 구한 결과를 이용해서 문제에서 요구하는 것을 구합니다.]]

따라서 $\frac{a_{7}}{a_{3}}=\frac{a_{1} r^{6}}{a_{1} r^{2}}=r^{4}=9$


<<<P>>>3. 등비수열 $\left\{a_{n}\right\}$ 에 대하여 $a_{1}=2, \quad a_{2} a_{4}=36$ 일 때, $\frac{a_{7}}{a_{3}}$ 의 값은? [3점]

① $1$
② $\sqrt{3}$
③ $3$
④ $3 \sqrt{3}$
⑤ $9$

<<<A>>>④

<<<S>>>

[[구간의 경곗값에서 함숫값, 좌극한값, 우극한값을 구합니다.]]

$ f(-1)=\lim _{x \rightarrow-1-} f(x)=-2+a$
$\lim _{x \rightarrow-1+} f(x)=6-a$

[[연속의 정의를 이용합니다.]]

$f(x)$ 가 $x=1$에서 연속이므로 $-2+a=6-a$
따라서 $a=4$


<<<P>>>4. 함수 $f(x)= \begin{cases}2 x+a & (x \leq-1) \\ x^{2}-5 x-a & (x>-1)\end{cases}$이 실수 전체의 집합에서 연속일 때, 상수 $a$ 의 값은? [3점]

① $1$
② $2$
③ $3$
④ $4$
⑤ $5$


<<<A>>>③

<<<S>>>

[[주어진 함수를 미분하여 도함수를 구합니다.]]

$f(x)=2 x^{3}+3 x^{2}-12 x+1$ 에서

$f^{\prime}(x)=6 x^{2}+6 x-12=6(x+2)(x-1)$

[[도함수의 부호를 조사하여 함수의 증가, 감소를 표로 나타냅니다.]]

$f(x)$의 증가, 감소를 조사하면 다음과 같다.

[IMG]

따라서 $f(x)$는 $x=-2$, $x=1$에서 각각 극대, 극소가 된다.

[[함수의 극댓값과 극솟값을 계산합니다.]]

극댓값은 $M=f(-2)=-16+12+24+1=21$이고
극솟값은 $m=f(1)=2+3-12+1=-6$

따라서 $M+m=21-6=15$


<<<P>>>5. 함수 $f(x)=2 x^{3}+3 x^{2}-12 x+1$ 의 극댓값과 극솟값을 각각 $M, m$ 이라 할 때, $M+m$ 의 값은? [3점]

① $13$
② $14$
③ $15$
④ $16$
⑤ $17$

<<<A>>>①

<<<S>>>

[[주어진 식의 좌변을 통분하여 정리합니다.]]

$ \frac{\sin \theta}{1-\sin \theta}-\frac{\sin \theta}{1+\sin \theta}=\frac{2 \sin ^{2} \theta}{1-\sin ^{2} \theta}=4$

[[$\sin \theta$의 값을 구합니다.]]

$2 \sin ^{2} \theta=4-4 \sin ^{2} \theta$

$\sin ^{2} \theta=\frac{2}{3}$

따라서 $\sin \theta=\frac{\sqrt{6}}{3}\;\left(\because \frac{\pi}{2}<\theta<\pi\right)$

[[앞에서 구한 결과를 이용해서 문제에서 요구하는 것을 구합니다.]]

$\cos \theta<0$ 이므로 $\cos \theta=-\frac{\sqrt{3}}{3}$


<<<P>>>6. $\frac{\pi}{2}<\theta<\pi$ 인 $\theta$ 에 대하여 $\frac{\sin \theta}{1-\sin \theta}-\frac{\sin \theta}{1+\sin \theta}=4$ 일 때, $\cos \theta$ 의 값은? [3점]

① $-\frac{\sqrt{3}}{3}$
② $-\frac{1}{3}$
③ $0$
④ $\frac{1}{3}$
⑤ $\frac{\sqrt{3}}{3}$

<<<A>>>④

<<<S>>>

[[$\displaystyle\sum$ 안의 분수식을 부분분수로 변형합니다.]]

$ \sum_{k=1}^{n} \frac{a_{k+1}-a_{k}}{a_{k} a_{k+1}} =\sum_{k=1}^{n}\left(\frac{1}{a_{k}}-\frac{1}{a_{k+1}}\right)$

[[$\displaystyle\sum$를 풀어 쓴 후 합을 계산합니다.]]

$=\frac{1}{a_{1}}-\frac{1}{a_{n+1}}=\frac{1}{n}$

따라서 $\frac{1}{a_{n+1}}=\frac{1}{a_{1}}-\frac{1}{n}$ 이다.

[[앞에서 구한 결과를 이용해서 문제에서 요구하는 것을 구합니다.]]

$n=12$ 를 대입하면

$\frac{1}{a_{13}}=\frac{1}{a_{1}}-\frac{1}{12}=-\frac{1}{4}-\frac{1}{12}=-\frac{1}{3}$

따라서 $a_{13}=-3$


<<<P>>>7. 수열 $\left\{a_{n}\right\}$ 은 $a_{1}=-4$ 이고, 모든 자연수 $n$ 에 대하여
$$
\sum_{k=1}^{n} \frac{a_{k+1}-a_{k}}{a_{k} a_{k+1}}=\frac{1}{n}
$$
을 만족시킨다. $a_{13}$ 의 값은? [3점]

① $-9$
② $-7$
③ $-5$
④ $-3$
⑤ $-1$

<<<A>>>②

<<<S>>>

[[주어진 극한의 관계식에서 구체적인 조건을 찾아냅니다.]]

$\lim _{x \rightarrow 0} \frac{f(x)}{x}=1$에서 $f(0)=0$, $f^{\prime}(0)=1$

$\lim _{x \rightarrow 1} \frac{f(x)}{x-1}=1$에서 $f(1)=0$, $f^{\prime}(1)=1$

[[앞에서 구한 조건에 맞게 함수의 식을 세웁니다.]]

$f(x)=x(x-1)(a x+b)$(단, $a, b$는 상수)이라 하면

$f^{\prime}(x)=(x-1)(a x+b)+x(a x+b)+a x(x-1)$

[[함수에 $x=1$을 대입해서 미정계수 사이의 관계식을 세웁니다.]]

$f^{\prime}(0)=-b=1$

$f^{\prime}(1)=a+b=1$

$\therefore a=2, b=-1$

[[앞에서 구한 결과를 이용해서 문제에서 요구하는 것을 구합니다.]]

$f(x)=x(x-1)(2 x-1)$
$f(2)=2 \times 1 \times 3=6$


<<<P>>>8. 삼차함수 $f(x)$ 가
$$
\lim _{x \rightarrow 0} \frac{f(x)}{x}=\lim _{x \rightarrow 1} \frac{f(x)}{x-1}=1
$$
을 만족시킬 때, $f(2)$ 의 값은? [3점]

① $4$
② $6$
③ $8$
④ $10$
⑤ $12$

<<<A>>>③

<<<S>>>

[[가속도 조건을 이용하여 미지수의 값을 구합니다.]]

점 $\mathrm{P}$ 의 시각 $t$ 에서의 가속도를 $a(t)$ 라 하자.

$v(t)=-4 t^{3}+12 t^{2}$에서 $a(t)=-12 t^{2}+24 t $ 이다.
$a(k)=-12 k^{2}+24 k=12$에서 $k=1$이다.

$t=3$ 에서 $t=4$ 까지 점 $\mathrm{P}$ 가 움직인 거리를 $l$ 이라 하자.

[[점 $\mathrm{P}$가 움직인 거리를 정적분으로 계산합니다.]]

$l =\int_{3}^{4}\left|-4 t^{3}+12 t^{2}\right| d t$

$=\int_{3}^{4}\left(4 t^{3}-12 t^{2}\right) d t$

$=\left[t^{4}-4 t^{3}\right]_{3}^{4}$

$=(256-256)-(81-108)=27$


<<<P>>>9. 수직선 위를 움직이는 점 $\mathrm{P}$ 의 시각 $t(t>0)$ 에서의 속도 $v(t)$ 가
$$
v(t)=-4 t^{3}+12 t^{2}
$$
이다. 시각 $t=k$ 에서 점 $\mathrm{P}$ 의 가속도가 12 일 때, 시각 $t=3 k$ 에서 $t=4 k$ 까지 점 $\mathrm{P}$ 가 움직인 거리는? (단, $k$ 는 상수이다.) [4점]

① $23$
② $25$
③ $27$
④ $29$
⑤ $31$

<<<A>>>③

<<<S>>>

[[주어진 함수의 주기를 구하고, 이것을 이용하여 $\overline{\mathrm{AC}}$로 나타냅니다.]]

$y=a \sin b \pi x$ 의 주기는 $\frac{2 \pi}{b \pi}=\frac{2}{b}$ 이다.

[[주어진 함수의 주기를 이용하여 문제 해결의 중요한 점의 좌표를 구합니다.]]

따라서 점 $\mathrm{A}\left(\frac{1}{2 b}, a\right)$ 이고, 점 $\mathrm{B}\left(\frac{5}{2 b}, a\right)$ 이다.

[[앞에서 구한 좌표를 이용하여 문제의 조건들을 연립하여 식으로 세웁니다.]]

(삼각형 $\mathrm{OAB}$ 의 넓이) $=\frac{1}{2} \times \frac{4}{2 b} \times a=\frac{a}{b}=5\cdots\cdots$㉠

(직선 $\mathrm{OA}$ 의 기울기 $)=\frac{a}{\frac{1}{2 b}}=2 a b$

(직선 $\mathrm{OB}$ 의 기울기 $)=\frac{a}{\frac{5}{2 b}}=\frac{2 a b}{5}$

$\therefore 2 a b \times \frac{2 a b}{5}=\frac{5}{4}$

[[앞에서 구한 연립방정식의 해를 이용하여 문제에서 요구하는 것을 구합니다.]]

㉠에서 $a=5 b$ 이므로 $\frac{4}{5} \times(5 b)^{2} \times b^{2}=\frac{5}{4}$,

$b^{4}=\frac{1}{16}$

$b=\frac{1}{2}(\because b>0)$

$\therefore a=\frac{5}{2}$

따라서 $a+b=3$ 이다.


<<<P>>>10. 두 양수 $a, b$ 에 대하여 곡선 $y=a \sin b \pi x\left(0 \leq x \leq \frac{3}{b}\right)$ 이 직선 $y=a$ 와 만나는 서로 다른 두 점을 $\mathrm{A}, \mathrm{B}$ 라 하자.

삼각형 $\mathrm{OAB}$ 의 넓이가 5 이고 직선 $\mathrm{OA}$ 의 기울기와 직선 $\mathrm{OB}$ 의 기울기의 곱이 $\frac{5}{4}$ 일 때, $a+b$ 의 값은? (단, $\mathrm{O}$ 는 원점이다.) [4점]

① $1$
② $2$
③ $3$
④ $4$
⑤ $5$

[IMG]

<<<A>>>④

<<<S>>>

[[d0057]]

$x f(x)=2 x^{3}+a x^{2}+3 a+\int_{1}^{x} f(t) d t \cdots \cdots$ ㉠

㉠에 $x=1$ 을 대입하면 $f(1)=4 a+2 \quad \cdots \cdots$ ㉡

[[주어진 정적분 관계식의 양변을 $x$ 에 대하여 미분합니다.]]

㉠의 양변을 $x$ 에 대하여 미분하면

$f(x)+x f^{\prime}(x)=6 x^{2}+2 a x+f(x) $ 이므로

$f^{\prime}(x)=6 x+2 a$

[[도함수를 적분하여 적분상수가 포함된 함수 $f(x)$ 의 꼴을 구합니다.]]

따라서 $f(x)=3 x^{2}+2 a x+C$ (단, $C$ 는 상수 ) $\cdots \cdots$  ㉢

[[앞에서 구한 조건과 문제의 조건을 이용하여 미지수에 대한 식을 세우고 미지수를 구합니다.]]

㉢에 $x=1$ 을 대입하면 $f(1)=3+2 a+C$

㉡에 의해 $3+2 a+C=4 a+2$ 이다.

$C=2 a-1 $ 이므로 $f(x)=3 x^{2}+2 a x+2 a-1$

$\int_{0}^{1} f(t) d t $

$=\int_{0}^{1}\left(3 t^{2}+2 a t+2 a-1\right) d t$

$=\left[t^{3}+a t^{2}+(2 a-1) t\right]_{0}^{1}$

$=1+a+2 a-1=3 a$

$f(1)=\int_{0}^{1} f(t) d t$에서

$3+2 a+2 a-1=3 a$ 이므로 $a=-2$ 이다.

[[앞에서 구한 결과를 이용해서 문제에서 요구하는 것을 구합니다.]]

$f(3)=27+6 a+2 a-1=8 a+26=10$ 이므로 $a+f(3)=-2+10=8$


<<<P>>>11. 다항함수 $f(x)$ 가 모든 실수 $x$ 에 대하여
$$
x f(x)=2 x^{3}+a x^{2}+3 a+\int_{1}^{x} f(t) d t
$$
를 만족시킨다. $f(1)=\int_{0}^{1} f(t) d t$ 일 때, $a+f(3)$ 의 값은? (단, $a$ 는 상수이다.) [4점]

① $5$
② $6$
③ $7$
④ $8$
⑤ $9$

<<<A>>>②

<<<S>>>

[[문제의 삼각형에 외접원의 반지름을 이용하여 사인법칙으로 변의 길이를 구합니다.]]

삼각형 $\mathrm{ABC}$와 $\mathrm{BCD}$에 사인법칙을 적용하면 

$\frac{\overline{\mathrm{BC}}}{\sin A}=2R$, $\frac{\overline{\mathrm{BD}}}{\sin (\angle \mathrm{BCD})} =2R$ 이므로

$\overline{\mathrm{BC}}=2 \times 2 \sqrt{7} \times \sin \frac{\pi}{3}=2 \sqrt{21}$

$\overline{\mathrm{BD}}=2 \times 2 \sqrt{7} \times \frac{2 \sqrt{7}}{7}=8$

[[구하려는 변의 길이를 $x$로 놓고 코사인법칙을 적용하여 식을 세웁니다.]]

한편, $ \angle \mathrm{BDC} = \dfrac{2\pi}{3}$이고

$\overline{\mathrm{CD}}=x$ 로 놓고, 삼각형 $\mathrm{BCD}$ 에서 코사인법칙을 적용하면

$ \left( 2 \sqrt{21} \right)^{2}=x^{2}+8^{2}-2 \times x \times 8 \times \left( -\frac{1}{2} \right)$

$x^{2}+8x-20=0$에서 $x=2$

[[삼차함수 $f(x)$의 적분상수 $f(0)$에 대해 조건(나)를 만족시키는 실근을 알아냅니다.  ]]

$\therefore \overline{\mathrm{BD}}+\overline{\mathrm{CD}}=8+2=10$


<<<P>>>12. 반지름의 길이가 $2 \sqrt{7}$ 인 원에 내접하고 $\angle \mathrm{A}=\frac{\pi}{3}$ 인 삼각형 $\mathrm{ABC}$ 가 있다. 점 $\mathrm{A}$ 를 포함하지 않는 호 $\mathrm{BC}$ 위의 점 $\mathrm{D}$ 에 대하여 $\sin (\angle \mathrm{BCD})=\frac{2 \sqrt{7}}{7}$ 일 때, $\overline{\mathrm{BD}}+\overline{\mathrm{CD}}$ 의 값은? [4점]

① $\frac{19}{2}$
② $10$
③ $\frac{21}{2}$
④ $11$
⑤ $\frac{23}{2}$

[IMG]

<<<A>>>②

<<<S>>>

[[조건 (가)를 정리하여 미지수의 관계식을 세웁니다.]]

$\left|a_{m}\right|=\left|a_{m+3}\right|$ 에서 $a_{m+3}=\pm a_{m}$ 이다.

$d$는 자연수이므로 $a_{m+3} \neq a_{m}$ 이다.

따라서 $a_{m+3}=-a_{m}$ 이므로

$-45+(m+2) d=45-(m-1) d$에서 $(2 m+1) d=90 $ 이다.

[[자연수 조건을 이용하여 미지수를 구합니다.]]

$m, d$ 는 자연수이므로 가능한 $(m, d)$ 의 순서쌍은 $(1,30),(2,18),(4,10),(7,6),(22,2)$ 이다.

[[조건 (나)의 의미를 파악하여 $\sum_{k=1}^{n} a_{k}$의 최소가 되는 경우를 구합니다.]]

모든 자연수 $n$ 에 대하여 $\sum_{k=1}^{n} a_{k}>-100$ 이려면 

$\sum_{k=1}^{n} a_{k}$의 최솟값이 $-100$보다 커야 한다.

앞에서 구한 각각의 $m,d$에 대하여 수열의 항들을 열거하여 $\sum_{k=1}^{n} a_{k}$의 최솟값을 구해보면

조건 (나)를 만족하는 경우는  $(1,30),(2,18)$ 뿐이다.

따라서 구하는 $d$의 값의 합은 $30+18=48$ 이다.

[다른 풀이]

[[조건 (가)를 정리하여 미지수의 관계식을 세웁니다.]]

$\left|a_{m}\right|=\left|a_{m+3}\right|$ 에서 $a_{m+3}=\pm a_{m}$ 이다.

$d$는 자연수이므로 $a_{m+3} \neq a_{m}$ 이다.

따라서 $a_{m+3}=-a_{m}$ 이므로

$-45+(m+2) d=45-(m-1) d$에서 $(2 m+1) d=90 $ 이다.

$m, d$ 는 자연수이므로 가능한 $(m, d)$ 의 순서쌍은 $(1,30),(2,18),(4,10),(7,6),(22,2)$ 이다.

[[조건 (나)의 의미를 파악하여 $\sum_{k=1}^{n} a_{k}$의 최소가 되는 경우를 구합니다.]]

$a_{1}=-45$ 이고 $a_{m}+a_{m+3}=a_{m+1}+a_{m+2}=0$ 이므로 $a_{m+1}<0$ 이고 $a_{m+2}>0$ 이다.

따라서 $\sum_{k=1}^{n} a_{k}$ 가 최소가 되려면 $n=m+1$ 일 때이다.

모든 자연수 $n$ 에 대하여 $\sum_{k=1}^{n} a_{k}>-100$ 이려면

$\sum_{k=1}^{m+1} a_{k}>-100 $이면 된다.

$\sum_{k=1}^{m+1} a_{k}=\frac{(m+1)(-90+m d)}{2}>-100 \cdots \cdots $ ㉠

가능한 $(m, d)$ 의 순서쌍 중에서  ㉠을 만족시키는 것은 $(1,30),(2,18)$ 뿐이다.

따라서 $30+18=48$ 이다.

<<<P>>>13. 첫째항이 $-45$ 이고 공차가 $d$ 인 등차수열 $\left\{a_{n}\right\}$ 이 다음 조건을 만족시키도록 하는 모든 자연수 $d$ 의 값의 합은? [4점]

(가) $\left|a_{m}\right|=\left|a_{m+3}\right|$ 인 자연수 $m$ 이 존재한다.

(나) 모든 자연수 $n$ 에 대하여 $\sum_{k=1}^{n} a_{k}>-100$ 이다.

① $44$
② $48$
③ $52$
④ $56$
⑤ $60$

<<<A>>>⑤

<<<S>>>

[[주어진 조건을 이용하여 $f'(x)$와 $f(x)$의 식을 세웁니다.]]

최고차항의 계수가 1 이고 $f^{\prime}(0)=f^{\prime}(2)=0$이므로 $f^{\prime}(x)=3 x(x-2)$ 이다.

그러므로 $f(x)=x^{3}-3 x^{2}+C$ (단, $C$ 는 상수 $)$

ㄱ.

[[$p=1$ 일 때 $x>0$ 에서 $g'(x)$를 구하여 $x=1$을 대입합니다.]]

$p=1$ 일 때 $x>0$ 에서 $g(x)=f(x+1)-f(1)$ 이므로 $g^{\prime}(1)=f^{\prime}(2)=0$ (참)

ㄴ.

[[$\lim _{x \rightarrow 0-} {g(x)-g(0)} over {x} = \lim _{x \rightarrow 0+} {g(x)-g(0)} over {x}$이 성립하는 $p$의 값을 구합니다.]]

$\lim _{x \rightarrow 0-} g(x)=\lim _{x \rightarrow 0-}\{f(x)-f(0)\}$

$=f(0)-f(0)=0$

$\lim _{x \rightarrow 0+} g(x)=\lim _{x \rightarrow 0+}\{f(x+p)-f(p)\}$

$=f(p)-f(p)=0$이고 $g(0)=0$ 이므로 함수 $g(x)$ 는 실수 전체의 집합에서 연속이다.

$g(x)$ 가 실수 전체의 집합에서 미분가능하려면

$\lim _{x \rightarrow 0-} g^{\prime}(x)=\lim _{x \rightarrow 0+} g^{\prime}(x)$

$\therefore f^{\prime}(0)=f^{\prime}(p)$

따라서 $p=0, p=2$ 이고 양수인 $p$ 는 $p=2$ $1$개 뿐이다. (참)

ㄷ.

[[구간별로 $g(x)$의 식을 구체적으로 구한 후 정적분을 계산합니다.]]

$g(x)= \begin{cases}x^{3}-3 x^{2} & (x \leq 0)\\ (x+p)^{3}-3(x+p)^{2}+C-\left(p^{3}-3 p^{2}+C\right) &(x>0)\end{cases}$

정리하면

$g(x)=\begin{cases}
x^{3}-3 x^{2} & (x \leq 0)\\
x^{3}+(3 p-3) x^{2}+\left(3 p^{2}-6 p\right) x & (x>0)
\end{cases}$

$\int_{-1}^{1} g(x) d x$

$=\int_{-1}^{0}\left(x^{3}-3 x^{2}\right)dx+\int_{0}^{1}\left\{x^{3}+(3 p-3) x^{2}+\left(3 p^{2}-6 p\right) x\right\} d x$

$=-\frac{5}{4}+\left\{\frac{1}{4}+(p-1)+\left(\frac{3}{2} p^{2}-3 p\right)\right\}$

$=\frac{3}{2} p^{2}-2 p-2$

$=\frac{3}{2}\left(p+\frac{2}{3}\right)(p-2)$

$p \geq 2 $ 일 때 $\int_{-1}^{1} g(x) d x \geq 0 $ 이다. (참)

[[지금까지의 각 참/거짓 결과에서 ㄱ, ㄴ, ㄷ 중 옳은 것을 모두 고릅니다.]]

따라서 옳은 것은 ㄱ, ㄴ, ㄷ이다.


<<<P>>>14. 최고차항의 계수가 $1$ 이고 $f^{\prime}(0)=f^{\prime}(2)=0$ 인 삼차함수 $f(x)$ 와 양수 $p$ 에 대하여 함수 $g(x)$ 를
$g(x)= \begin{cases}f(x)-f(0) & (x \leq 0) \\ f(x+p)-f(p) & (x>0)\end{cases}$
이라 하자. <보기>에서 옳은 것만을 있는 대로 고른 것은? [4점]

ㄱ. $p=1$ 일 때, $g^{\prime}(1)=0$ 이다.

ㄴ. $g(x)$ 가 실수 전체의 집합에서 미분가능하도록 하는 양수 $p$ 의 개수는 $1$ 이다.

ㄷ. $p \geq 2$ 일 때, $\int_{-1}^{1} g(x) d x \geq 0$ 이다.

① ㄱ,
② ㄱ, ㄴ
③ ㄱ, ㄷ
④ ㄴ, ㄷ
⑤ ㄱ, ㄴ, ㄷ


<<<A>>>①

<<<S>>>

[[주어진 수열의 관계식이 $a_{n+1}=f\left(a_{n}\right)$이 되는 함수 $f(x)$를 도입합니다.]]

$f(x)=\left\{\begin{array}{cl}-2 x-2 & \left(-1 \leq x<-\frac{1}{2}\right) \\ 2 x & \left(-\frac{1}{2} \leq x \leq \frac{1}{2}\right) \\ -2 x+2 & \left(\frac{1}{2}<x \leq 1\right)\end{array}\right.$

이라 하면 $a_{n+1}=f\left(a_{n}\right)$ 이다.

함수 $y=f(x)$ 의 그래프를 그려보면 다음과 같다.

[IMG]

[[$a_{5}$의 값을 구합니다.]]

$a_{5}+a_{6}=a_{5}+f\left(a_{5}\right)=0$

$a_{5}>0 $ 이면 그래프에서 $ f\left(a_{5}\right) \geq 0 $ 이므로 $a_{5}+f\left(a_{5}\right)>0 $ 이다.

$a_{5}<0 $ 이면 그래프에서 $f\left(a_{5}\right) \leq 0 $ 이므로 $a_{5}+f\left(a_{5}\right)<0 $ 이다.

$a_{5}=0 $ 이면 그래프에서 $f\left(a_{5}\right)=0 $ 이므로 $a_{5}+f\left(a_{5}\right)=0 $ 이다.

따라서 $a_{5}=0$ 이다.

[[$a_{5}=0$에서 출발하여 차례대로 $a_{4},a_{3},\cdots$를 구합니다.]]

$a_{1}<0$ 이면 $a_{n} \leq 0$ 이므로 $\sum_{k=1}^{5} a_{k} \leq 0$ 이 되어 모순이다.

$a_{1} \geq 0$ 이므로 $y=f(x)$ 의 그래프에서 $x \geq 0$ 인 부분만 살펴보면 된다.

$a_{n+1}=0$ 이면 $f\left(a_{n}\right)=0$ 이므로 $a_{n}=0,1$

$a_{n+1}=1$ 이면 $f\left(a_{n}\right)=1$ 이므로 $a_{n}=\frac{1}{2}$

$a_{n+1}=\frac{1}{2}$ 이면 $f\left(a_{n}\right)=\frac{1}{2}$ 이므로 $a_{n}=\frac{1}{4}, \frac{3}{4}$

$a_{n+1}=\frac{1}{4}$ 이면 $f\left(a_{n}\right)=\frac{1}{4} $ 이므로 $a_{n}=\frac{1}{8}, \frac{7}{8}$

$a_{n+1}=\frac{3}{4}$ 이면 $f\left(a_{n}\right)=\frac{3}{4} $ 이므로 $a_{n}=\frac{3}{8}, \frac{5}{8}$

[[여러 가지 가능한 수열을 조사합니다.]]

따라서 가능한 수열 $\left\{a_{n}\right\}$ 을 표로 나타내면 다음과 같다.

[IMG]

따라서 가능한 $a_{1}$ 의 합은 $1+\frac{1}{2}+\frac{1}{4}+\frac{3}{4}+\frac{1}{8}+\frac{7}{8}+\frac{3}{8}+\frac{5}{8}=\frac{9}{2}$


<<<P>>>15. 수열 $\left\{a_{n}\right\}$ 은 $\left|a_{1}\right| \leq 1$ 이고, 모든 자연수 $n$ 에 대하여

$a_{n+1}=\left\{\begin{array}{cl}
-2 a_{n}-2 & \left(-1 \leq a_{n}<-\frac{1}{2}\right)\\
2 a_{n} & \left(-\frac{1}{2} \leq a_{n} \leq \frac{1}{2}\right)\\
-2 a_{n}+2 & \left(\frac{1}{2}<a_{n} \leq 1\right)
\end{array}\right.$

을 만족시킨다. $a_{5}+a_{6}=0$ 이고 $\sum_{k=1}^{5} a_{k}>0$ 이 되도록 하는 모든 $a_{1}$ 의 값의 합은? [4점]

① $\frac{9}{2}$
② $5$
③ $\frac{11}{2}$
④ $6$
⑤ $\frac{13}{2}$

<<<A>>>$2$

<<<S>>>

[[로그의 성질을 이용하여 주어진 식을 계산합니다.]]

$ \log _{2} 100-2 \log _{2} 5$ $=\log _{2} 100-\log _{2} 5^{2}$
$=\log _{2} \frac{100}{25}=\log _{2} 4=2 $


<<<P>>>16. $\log _{2} 100-2 \log _{2} 5$ 의 값을 구하시오. [3점]

<<<A>>>$8$

<<<S>>>

[[도함수를 적분하여 적분상수가 포함된 함수 $f(x)$ 의 꼴을 구합니다.]]

$f^{\prime}(x)=8 x^{3}-12 x^{2}+7$에서
$f(x)=2 x^{4}-4 x^{3}+7 x+C $ (단, $C$ 는 상수)

[[주어진 함숫값을 이용해서 적분상수값을 정해주고 함수 $f(x)$를 결정해 줍니다.]]

$ f(0)=3 $ 이므로 $C=3 $ 이고 $ f(x)=2 x^{4}-4 x^{3}+7 x+3$ 

[[앞에서 구한 결과를 이용해서 문제에서 요구하는 것을 구합니다.]]

따라서 $f(1)=2-4+7+3=8$


<<<P>>>17. 함수 $f(x)$ 에 대하여 $f^{\prime}(x)=8 x^{3}-12 x^{2}+7$ 이고 $f(0)=3$ 일 때, $f(1)$ 의 값을 구하시오. [3점]

<<<A>>>$9$

<<<S>>>

[[주어진 조건식을 변변끼리 빼서 $a_{k}$를 소거합니다.]]

$ \sum_{k=1}^{10}\left(a_{k}+2 b_{k}\right)=45 \quad \cdots \cdots$ ㉠

$\sum_{k=1}^{10}\left(a_{k}-b_{k}\right)=3 \cdots \cdots$ ㉡

㉠, ㉡을 변변끼리 빼서 정리하면

$\sum_{k=1}^{10} 3 b_{k}=42$ 에서 $\sum_{k=1}^{10} b_{k}=14$

[[앞에서 구한 결과를 이용해서 문제에서 요구하는 것을 구합니다.]]

$ \sum_{k=1}^{10}\left(b_{k}-\frac{1}{2}\right) $

$=\sum_{k=1}^{10} b_{k}-10 \times \frac{1}{2}=14-5=9 $

<<<P>>>18. 두 수열 $\left\{a_{n}\right\},\left\{b_{n}\right\}$ 에 대하여 $\sum_{k=1}^{10}\left(a_{k}+2 b_{k}\right)=45, \quad \sum_{k=1}^{10}\left(a_{k}-b_{k}\right)=3$일 때,

$\sum_{k=1}^{10}\left(b_{k}-\frac{1}{2}\right)$ 의 값을 구하시오. [3점]

<<<A>>>$11$

<<<S>>>

[[주어진 구간에서의 평균변화율과 미분계수를 구하여 방정식을 세웁니다.]]

$ \frac{f(4)-f(0)}{4-0}=3 a^{2}-12 a+5$에서

$\frac{64-96+20}{4}=3 a^{2}-12 a+5 $이고 정리하면

$3 a^{2}-12 a+8=0$

[[앞의 방정식에서 범위를 만족하는 모든 근을 구합니다.]]

근의 공식에 대입하면 $a=\frac{6 \pm \sqrt{12}}{3}$ 이므로 두 근은 모두 $0<a<4$ 이다.

따라서 모든 실수 $a$의 값의 곱은 $\frac{8}{3}$이므로 $p+q=11$ 이다.


<<<P>>>19. 함수 $f(x)=x^{3}-6 x^{2}+5 x$ 에서 $x$ 의 값이 0 에서 4 까지 변할 때의 평균변화율과 $f^{\prime}(a)$ 의 값이 같게 되도록 하는 $0<a<4$ 인 모든 실수 $a$ 의 값의 곱은 $\frac{q}{p}$ 이다. $p+q$ 의 값을 구하시오. (단, $p$ 와 $q$ 는 서로소인 자연수이다.) [3점]

<<<A>>>$21$

<<<S>>>

[[주어진 방정식을 $(x\text{의 식})=k$꼴로 변형합니다.]]

주어진 방정식을 정리하면 $f(x)+ \left| f(x)+x \right| -6x =k$이고,

$g(x)=f(x)+ \left| f(x)+x \right| -6x $라 하자.

[[$\left| f(x)+x \right|$에서 구간을 나누어 절댓값이 없는 식으로 변형합니다.]]

$f(x)+x=\frac{1}{2} x^{3}-\frac{9}{2} x^{2}+11 x$

$=\frac{1}{2} x\left(x^{2}-9 x+22\right)$

$x^{2}-9 x+22=\left(x-\frac{9}{2}\right)^{2}+\frac{7}{4}>0 $ 이므로

$|f(x)+x|= \begin{cases}f(x)+x & (x \geq 0) \\
-f(x)-x & (x<0)\end{cases}$

따라서, 
$g(x)=\begin{cases} 2f(x)-5x & (x \geq 0) \\
    -7x & (x<0)\end{cases} = \begin{cases}
x^{3}-9 x^{2}+15 x & (x \geq 0) \\
-7 x & (x<0)
\end{cases}$이고 그래프를 그리면 다음과 같다.

[[좌변과 우변의 함수의 그래프를 같은 좌표평면에 그립니다.]]

[IMG]

[[서로 다른 $4$개의 교점을 갖게 하는 $k$의 값의 범위를 구합니다.]]

$y=g(x)$ 의 그래프와 직선 $y=k$ 가 서로 다른 $4$개의 교점을 갖게 하는 정수 $k$ 의 값의 범위는 $0<k<7$이고,

정수 $k$ 의 합은 $\frac{6 \times 7}{2}=21$ 이다.


<<<P>>>20. 함수 $f(x)=\frac{1}{2} x^{3}-\frac{9}{2} x^{2}+10 x$ 에 대하여 $x$ 에 대한 방정식
$$
f(x)+|f(x)+x|=6 x+k
$$
의 서로 다른 실근의 개수가 $4$가 되도록 하는 모든 정수 $k$의 값의 합을 구하시오. [4점]

<<<A>>>$192$

<<<S>>>

[[로그함수의 역함수를 구하여 주어진 지수함수와 관계를 조사합니다.]]

$y=\log _{a}(x-1)$ 의 역함수를 구해보면 $y=a^{x}+1$

다음 그림과 같이 $y=a^{x}+1$ 과 $y=-x+4$ 의 그래프의 교점을 $\mathrm{P}$ 라 하자.

[IMG]

$y=a^{x}+1$ 의 그래프는 $y=a^{x-1}$ 의 그래프를 $x$축 방향으로 $-1$ 만큼, $y$축 방향으로 1 만큼 평행이동한 그래프이다.

따라서 점 $\mathrm{A}$ 를 $x$ 축의 방향으로 $-1$ 만큼, $y$ 축의 방향으로 1 만큼 평행이동한 점을 $\mathrm{A}^{\prime}$ 이라 하면 점 $\mathrm{A}^{\prime}$ 은 $y=a^{x}+1$ 의 그래프 위에 있다.

또한 $y=-x+4$ 의 기울기가 $-1$ 이므로 점 $\mathrm{A}^{\prime}$ 은 직선 $y=-x+4$ 위에 있다.

따라서 점 $\mathrm{A}^{\prime}$ 은 $y=a^{x}+1$ 과 $y=-x+4$ 의 그래프의 교점인 점 $\mathrm{P}$ 이고 $\overline{\mathrm{AP}}=\sqrt{2}$ 이다.

두 점 $\mathrm{B}, \mathrm{P}$ 는 직선 $y=x$ 에 대하여 대칭이고 $\overline{\mathrm{BP}}$ 의 길이는 $3 \sqrt{3}$ 이므로

점 $\mathrm{P}$ 의 좌표는 $\left(\frac{1}{2}, \frac{7}{2}\right)$ 이다.

따라서 $a^{\frac{1}{2}}+1=\frac{7}{2}$ 에서 $a=\frac{25}{4}$ 이므로 점 $\mathrm{C}$ 의 좌표는 $\left(0, \frac{4}{25}\right)$ 이다.

점 $\mathrm{C}$ 에서 $y=-x+4$ 까지의 거리는
$$
\frac{\left|\frac{4}{25}-4\right|}{\sqrt{1+1}}=\frac{96}{25 \sqrt{2}}
$$

따라서 $\triangle \mathrm{ABC}$ 의 넓이는 $\frac{1}{2} \times 2 \sqrt{2} \times \frac{96}{25 \sqrt{2}}=\frac{96}{25}$ $\therefore 50 S=50 \times \frac{96}{25}=192$


<<<P>>>21. $a>1$ 인 실수 $a$ 에 대하여 직선 $y=-x+4$ 가 두 곡선
$$
y=a^{x-1}, \quad y=\log _{a}(x-1)
$$
과 만나는 점을 각각 $\mathrm{A}, \mathrm{B}$ 라 하고, 곡선 $y=a^{x-1}$ 이 $y$ 축과 만나는 점을 $\mathrm{C}$ 라 하자. $\overline{\mathrm{AB}}=2 \sqrt{2}$ 일 때, 삼각형 $\mathrm{ABC}$ 의 넓이는 $S$ 이다. $50 \times S$ 의 값을 구하시오. [4점]

[IMG]

<<<A>>>$108$

<<<S>>>

[[$f(x)>0,f(x)<0,f(x)=0$일 때로 경우를 나누어 $g(x)$를 정리합니다.]]

$ \lim _{h \rightarrow 0+} \frac{|f(x+h)|-|f(x-h)|}{h}=h(x)$ 라 하자.

$f(x)>0$ 이면

$h(x)=\lim _{h \rightarrow 0+} \frac{f(x+h)-f(x-h)}{h}=2 f^{\prime}(x)$

$f(x)<0 $ 이면

$h(x)=\lim _{h \rightarrow 0+} \frac{-f(x+h)+f(x-h)}{h}=-2 f^{\prime}(x)$

$f(x)=0 $ 일 때, $\left|f^{\prime}(x)\right|=p $ 라 하자.

$\lim _{h \rightarrow 0+} \frac{|f(x+h)|-|f(x-h)|}{h}$

$=\lim _{h \rightarrow 0+}\left\{\frac{|f(x+h)|-|f(x)|}{h} + \frac{|f(x-h)|-|f(x)|}{-h}\right\}$

$=p+(-p)=0$

$\therefore f(x)=0$일 때, $h(x)=0$

따라서 $g(x)=\begin{cases}
2 f(x-3) f^{\prime}(x) & (f(x)>0) \\
-2 f(x-3) f^{\prime}(x) & (f(x)<0) \\
0 & (f(x)=0)
\end{cases}$

[[$f(x)$의 극값의 유무와 $f(x)=0$의 실근의 개수에 따라 $g(x)$의 연속성을 조사합니다.]]

$g(x)$ 가 실수 전체의 집합에서 연속이므로
$f(x)=0$ 일 때, $f(x-3)=0$ 또는 $f^{\prime}(x)=0$

따라서 최고차항의 계수가 1 인 삼차함수 $f(x)$ 의 그래프의 개형은 다음과 같다.

[IMG]

위의 그래프에서 $g(x)=0$의 서로 다른 네 실근은 $\alpha, \alpha+2, \alpha+3, \alpha+6$이므로

$\alpha_{1}+\alpha_{2}+\alpha_{3}+\alpha_{4}=11$에서 $\alpha=-1$이다.

따라서 $f(x)=(x+1)^2 (x-2)$이고, $f(5)=108$이다.

<<<P>>>22. 최고차항의 계수가 1 인 삼차함수 $f(x)$ 에 대하여 함수

$$g(x)=f(x-3) \times \lim _{h \rightarrow 0+} \frac{|f(x+h)|-|f(x-h)|}{h}$$

가 다음 조건을 만족시킬 때, $f(5)$ 의 값을 구하시오. [4점]

(가) 함수 $g(x)$ 는 실수 전체의 집합에서 연속이다.

(나) 방정식 $g(x)=0$ 은 서로 다른 네 실근 $\alpha_{1}, \alpha_{2}, \alpha_{3}, \alpha_{4}$ 를 갖고 $\alpha_{1}+\alpha_{2}+\alpha_{3}+\alpha_{4}=7$ 이다.

<<<A>>>③

<<<S>>>

$\mathrm{E}(X)=60 \times \frac{1}{4}=15$


<<<P>>>23. 확률변수 $X$ 가 이항분포 $\mathrm{B}\left(60, \frac{1}{4}\right)$ 을 따를 때, $\mathrm{E}(X)$ 의 값은? [2점]

① $5$
② $10$
③ $15$
④ $20$
⑤ $25$

<<<A>>>③

<<<S>>>


$(a, b)$ 의 모든 순서쌍의 개수는 $4 \times 4=16$
$a \times b>31$ 인 $(a, b)$ 의 순서쌍은 $(5,8),(7,6)$,
$(7,8)$ 이므로
$a \times b>31$ 일 확률을 $p$ 라 하면
$p=\frac{3}{16}$


<<<P>>>24. 네 개의 수 $1,3,5,7$ 중에서 임의로 선택한 한 개의 수를 $a$ 라 하고, 네 개의 수 $2,4,6,8$ 중에서 임의로 선택한 한 개의 수를 $b$ 라 하자. $a \times b>31$ 일 확률은? [3점]

① $\frac{1}{16}$
② $\frac{1}{8}$
③ $\frac{3}{16}$
④ $\frac{1}{4}$
⑤ $\frac{5}{16}$

<<<A>>>②

<<<S>>>



$\left(x^{2}+\frac{a}{x}\right)^{5}$ 의 전개식의 일반항은


${ }_{5} \mathrm{C}_{r} \times\left(x^{2}\right)^{5-r} \times\left(\frac{a}{x}\right)^{r}$ (단, $r=0,1, \cdots, 5)$
$={ }_{5} \mathrm{C}_{r} \times a^{r} \times x^{10-3 r}$


(i) $\frac{1}{x^{2}}$ 의 계수
$10-3 r=-2$ 에서 $r=4$ 이므로
계수는 ${ }_{5} \mathrm{C}_{4} \times a^{4}=5 a^{4}$ 이다.
(ii) $x$ 의 계수
$10-3 r=1$ 에서 $r=3$ 이므로
계수는 ${ }_{5} \mathrm{C}_{3} \times a^{3}=10 a^{3}$ 이다.
따라서 $5 a^{4}=10 a^{3}$ 이므로 $a=2$ 이다.


<<<P>>>25. $\left(x^{2}+\frac{a}{x}\right)^{5}$ 의 전개식에서 $\frac{1}{x^{2}}$ 의 계수와 $x$ 의 계수가 같을 때, 양수 $a$ 의 값은? [3점]

① $1$
② $2$
③ $3$
④ $4$
⑤ $5$

<<<A>>>①

<<<S>>>



주사위의 눈이 5 이상인 사건을 $X$, 꺼낸 공이 모 두 흰색인 사건을 $Y$ 라 하자.

$\mathrm{P}(X \mid Y) $

$=\frac{\mathrm{P}(X \cap Y)}{\mathrm{P}(Y)}$

$=\frac{\mathrm{P}(X \cap Y)}{\mathrm{P}(X \cap Y)+\mathrm{P}\left(X^{c} \cap Y\right)}$

$=\frac{\frac{2}{6} \times \frac{{ }_{2} \mathrm{C}_{2}}{{ }_{6} \mathrm{C}_{2}}}{\frac{2}{6} \times \frac{{ }_{2} \mathrm{C}_{2}}{{ }_{6} \mathrm{C}_{2}}+\frac{4}{6} \times \frac{{ }_{3} \mathrm{C}_{2}}{{ }_{6} \mathrm{C}_{2}}}=\frac{1}{7}$


<<<P>>>26. 주머니 $\mathrm{A}$ 에는 흰 공 2 개, 검은 공 4 개가 들어 있고, 주머니 B에는 흰 공 3 개, 검은 공 3 개가 들어 있다.
두 주머니 $\mathrm{A}, \mathrm{B}$ 와 한 개의 주사위를 사용하여 다음 시행을 한다.
주사위를 한 번 던져
나온 눈의 수가 5 이상이면
주머니 $\mathrm{A}$ 에서 임의로 2 개의 공을 동시에 꺼내고,
나온 눈의 수가 4 이하이면
주머니 B에서 임의로 2 개의 공을 동시에 꺼낸다.
이 시행을 한 번 하여 주머니에서 꺼낸 2 개의 공이 모두 흰색일 때, 나온 눈의 수가 5 이상일 확률은? [3점]

① $\frac{1}{7}$
② $\frac{3}{14}$
③ $\frac{2}{7}$
④ $\frac{5}{14}$
⑤ $\frac{3}{7}$

[IMG]

<<<A>>>⑤

<<<S>>>



확률변수 $X$ 의 표준편차를 $\sigma$ 라고 하면
확률변수 $X$ 는 정규분포 $\mathrm{N}\left(220, \sigma^{2}\right)$ 을 따르고,
확률변수 $\overline{X}$ 는 정규분포 $\mathrm{N}\left(220, \frac{\sigma^{2}}{n}\right)$ 을 따른다.
표준정규분포를 따르는 확률변수 $Z$ 에 대하여

$Z=\frac{\overline{X}-220}{\frac{\sigma}{\sqrt{n}}}$이므로

$\mathrm{P}(\overline{X} \leq 215) $

$=\mathrm{P}\left(Z \leq \frac{215-220}{\frac{\sigma}{\sqrt{n}}}\right)$

$=\mathrm{P}\left(Z \leq-\frac{5 \sqrt{n}}{\sigma}\right)=0.1587$

$\mathrm{P}(Z \leq-1)=0.5-\mathrm{P}(0 \leq Z \leq 1)$

$= 0.5-0.3413=0.1587$

이므로 $-\frac{5 \sqrt{n}}{\sigma}=-1 \quad \therefore \frac{\sigma}{\sqrt{n}}=5$

확률변수 $Y$ 의 표준편차는 $1.5 \sigma=\frac{3}{2} \sigma$ 이므로

확률변수 $Y$ 는 정규분포 $\mathrm{N}\left(240, \frac{9}{4} \sigma^{2}\right)$ 을 따르고,

확률변수 $\overline{Y}$ 는 정규분포 $\mathrm{N}\left(240, \frac{\sigma^{2}}{4 n}\right)$ 을 따른다.

표준정규분포를 따르는 확률변수 $Z$ 에 대하여

$Z=\frac{\overline{Y}-240}{\frac{\sigma}{2 \sqrt{n}}}$이므로

$\mathrm{P}(\overline{Y} \geq 235) $

$=\mathrm{P}\left(Z \geq \frac{235-240}{\frac{\sigma}{2 \sqrt{n}}}\right)$

$=\mathrm{P}(Z \geq-2)$

$=0.5+(0 \leq Z \leq 2)$

$=0.5+0.4772$

$=0.9772$


<<<P>>>27. 지역 $\mathrm{A}$ 에 살고 있는 성인들의 $1$ 인 하루 물 사용량을 확률변수 $X$, 지역 $\mathrm{B}$ 에 살고 있는 성인들의 $1$ 인 하루 물 사용량을 확률변수 $Y$ 라 하자. 두 확률변수 $X, Y$ 는 정규분포를 따르고 다음 조건을 만족시킨다.

(가) 두 확률변수 $X, Y$ 의 평균은 각각 $220$ 과 $240$ 이다.

(나) 확률변수 $Y$ 의 표준편차는 확률변수 $X$ 의 표준편차의 $1.5$ 배이다.

지역 $\mathrm{A}$ 에 살고 있는 성인 중 임의추출한 $n$ 명의 $1$ 인 하루 물 사용량의 표본평균을 $\overline{X}$, 지역 $\mathrm{B}$ 에 살고 있는 성인 중 임의추출한 $9 n$ 명의 $1$ 인 하루 물 사용량의 표본평균을 $\overline{Y}$ 라 하자. $\mathrm{P}(\overline{X} \leq 215)=0.1587$ 일 때, $\mathrm{P}(\overline{Y} \geq 235)$ 의 값을 오른쪽 표준정규분포표를 이용하여 구한 것은? (단, 물 사용량의 단위는 $\mathrm{L}$이다.) [3점]

[IMG]

① $0.6915$
② $0.7745$
③ $0.8185$
④ $0.8413$
⑤ $0.9772$



<<<A>>>④

<<<S>>>



조건 (가)에서 $f(3)+f(4)=5$ 또는 $10$ 이므로

$ f(3)+f(4)=1+4=2+3=3+2=4+1$ 또는 $f(3)+f(4)=4+6=5+5=6+4 $

두 조건 (나), (다)에서 $f(3) \geq 2, f(4) \leq 5 $이므로

$f(3)+f(4)=2+3=3+2=4+1$ 또는 $f(3)+f(4)=5+5=6+4$

(i) $f(3)=2, f(4)=3$ 인 경우

$f(1), f(2)<2$ 이고 $f(5), f(6)>3$ 이므로

(경우의 수 )$=1^{2} \times 3^{2}=9$

(ii) $f(3)=3, f(4)=2$ 인 경우

$f(1), f(2)<3$이고 $f(5), f(6)>2$ 이므로 $2^{2} \times 4^{2}=64$


(iii) $f(3)=4, f(4)=1$

$f(1), f(2)<4$이고 $f(5), f(6)>1 $이므로 $3^{2} \times 5^{2}=225$

(iv) $f(3)=5, f(4)=5$

$f(1), f(2)<5$이고 $f(5), f(6)>5 $이므로

$4^{2} \times 1^{2}=16$

(v) $f(3)=6, f(4)=4$

$f(1), f(2)<6$이고 $f(5), f(6)>4 $이므로 

$5^{2} \times 2^{2}=100$

(i), (ii), (iii), (iv), (v)에서 $ 9+64+225+16+100=414$

<<<P>>>28. 집합 $X=\{1,2,3,4,5,6\}$ 에 대하여 다음 조건을 만족시키는 함수 $f: X \rightarrow X$ 의 개수는? [4점]

(가) $f(3)+f(4)$ 는 5 의 배수이다.

(나) $f(1)<f(3)$ 이고 $f(2)<f(3)$ 이다.

(다) $f(4)<f(5)$ 이고 $f(4)<f(6)$ 이다.

① $384$
② $394$
③ $404$
④ $414$
⑤ $424$


<<<A>>>$78$

<<<S>>>



$\mathrm{E}(X)=1 \times a+3 \times b+5 \times c+7 \times b+9 \times a$
$=10 a+10 b+5 c$

$\mathrm{E}(Y)=1 \times\left(a+\frac{1}{20}\right)+3 \times b+5 \times\left(c-\frac{1}{10}\right)$
$+7 \times b+9 \times\left(a+\frac{1}{20}\right)$
$=10 a+10 b+5 c+\frac{1}{20}-\frac{10}{20}+\frac{9}{20}$
$=\mathrm{E}(X)$

$\mathrm{E}\left(X^{2}\right)=1^{2} \times a+3^{2} \times b+5^{2} \times c+7^{2} \times b$
$+9^{2} \times a$

$=82 a+58 b+25 c$

$ \mathrm{E}\left(Y^{2}\right)$

$=1^{2} \times\left(a+\frac{1}{20}\right)+3^{2} \times b+5^{2} \times\left(c-\frac{1}{10}\right) +7^{2} \times b+9^{2} \times\left(a+\frac{1}{20}\right)$

$\therefore$ $=82 a+58 b+25 c+\frac{1}{20}-\frac{50}{20}+\frac{81}{20}$

$= \mathrm{E}\left(X^{2}\right)+\frac{8}{5}$

$\therefore$ $\mathrm{V}(Y)$

$=\mathrm{E}\left(Y^{2}\right)-\{\mathrm{E}(Y)\}^{2}$

$=\mathrm{E}\left(X^{2}\right)+\frac{8}{5}-\{\mathrm{E}(X)\}^{2}$

$=\mathrm{V}(X)+\frac{8}{5}$

$=\frac{39}{5}$

$\therefore$ $10 \times \mathrm{V}(Y)=10 \times \frac{39}{5}=78 $


<<<P>>>29. 두 이산확률변수 $X, Y$ 의 확률분포를 표로 나타내면 각각 다음과 같다.

[IMG]

[IMG]

$\mathrm{V}(X)=\frac{31}{5}$ 일 때, $10 \times \mathrm{V}(Y)$ 의 값을 구하시오. [4점]

<<<A>>>$218$

<<<S>>>



$\mathrm{A}, \mathrm{B}, \mathrm{C}, \mathrm{D}$ 학생이 받는 사인펜의 개수를 각각 $a, b, c, d$ 라 하면 $a+b+c+d=14$ 이고
두 조건 (가), (나)에서 $1 \leq a, b, c, d \leq 9$ 이고

$a=a^{\prime}+1, b=b^{\prime}+1, c=c^{\prime}+1$

$d=d^{\prime}+1$이라 놓으면

$0 \leq a^{\prime}, b^{\prime}, c^{\prime}, d^{\prime} \leq 8 $이다.

따라서 $a^{\prime}+b^{\prime}+c^{\prime}+d^{\prime}=10$ 의 음이 아닌 정수해 중 조건 (나)에 따라 $9$ 또는 $10$ 인 해가 있는 경우와 조건 (다)에 따라 $a^{\prime}, b^{\prime}, c^{\prime}, d^{\prime}$ 이 모두 짝수인 경우 를 제외한다.

$a^{\prime}+b^{\prime}+c^{\prime}+d^{\prime}=10$ 의 음이 아닌 정수해의 개수는
${ }_{4} \mathrm{H}_{10}=286 \cdots \cdots $㉠

$a^{\prime}, b^{\prime}, c^{\prime}, d^{\prime}$ 중 9 가 있는 경우의 수
${ }_{4} \mathrm{C}_{1} \times{ }_{3} \mathrm{H}_{1}=4 \times 3=12 \cdots \cdots $㉡

$a^{\prime}, b^{\prime}, c^{\prime}, d^{\prime}$ 중 10 이 있는 경우의 수
${ }_{4} \mathrm{C}_{1} \times{ }_{3} \mathrm{H}_{0}=4 \cdots \cdots $㉢

$a^{\prime}, b^{\prime}, c^{\prime}, d^{\prime}$ 이 모두 짝수인 경우는

$a^{\prime}=2 a^{\prime \prime}, b^{\prime}=2 b^{\prime \prime}, c^{\prime}=2 c^{\prime \prime}, d^{\prime}=2 d^{\prime \prime}$ 이라 하면

$a^{\prime \prime}+b^{\prime \prime}+c^{\prime \prime}+d^{\prime \prime}=5 $ 이므로
${ }_{4} \mathrm{H}_{5}={ }_{8} \mathrm{C}_{5}=56 \cdots \cdots $㉣

이때, $a^{\prime}, b^{\prime}, c^{\prime}, d^{\prime}$ 중 10 이 있는 경우는 $a^{\prime}, b^{\prime}, c^{\prime}, d^{\prime}$ 이 모두 짝수인 경우에 포함되므로 문제의 경우의 수는 ㉠$-$㉡$-$㉢$-$㉣$+$㉢$=218$


<<<P>>>30. 네 명의 학생 $\mathrm{A}, \mathrm{B}, \mathrm{C}, \mathrm{D}$ 에게 같은 종류의 사인펜 14 개를 다음 규칙에 따라 남김없이 나누어 주는 경우의 수를 구하시오. [4점]

(가) 각 학생은 1 개 이상의 사인펜을 받는다.

(나) 각 학생이 받는 사인펜의 개수는 9 이하이다.

(다) 적어도 한 학생은 짝수 개의 사인펜을 받는다.


<<<A>>>③

<<<S>>>





$\lim _{n \rightarrow \infty} \frac{2 \times 3^{n+1}+5}{3^{n}+2^{n+1}}$

$=\lim _{n \rightarrow \infty} \frac{6+\frac{5}{3^{n}}}{1+2\left(\frac{2}{3}\right)^{n}}$

$=\frac{6+0}{1+0}$
$=6$


<<<P>>>23. $\lim _{n \rightarrow \infty} \frac{2 \times 3^{n+1}+5}{3^{n}+2^{n+1}}$ 의 값은? [2점]

① $2$
② $4$
③ $6$
④ $8$
⑤ $10$


<<<A>>>②

<<<S>>>



$ 2 \cos \alpha =3 \sin \alpha$에서 

$\frac{\sin \alpha}{\cos \alpha}=\tan \alpha=\frac{2}{3}$이다.

$\therefore \tan \beta$

$=\tan ((\alpha+\beta)-\alpha)$

$=\frac{\tan (\alpha+\beta)-\tan \alpha}{1+\tan (\alpha+\beta) \tan \alpha}$

$= \frac{1-\frac{2}{3}}{1+1 \times \frac{2}{3}}$

$=\frac{3-2}{3+2}$

$=\frac{1}{5} $


<<<P>>>24. $2 \cos \alpha=3 \sin \alpha$ 이고 $\tan (\alpha+\beta)=1$ 일 때, $\tan \beta$ 의 값은?
[3점]

① $\frac{1}{6}$
② $\frac{1}{5}$
③ $\frac{1}{4}$
④ $\frac{1}{3}$
⑤ $\frac{1}{2}$


<<<A>>>④

<<<S>>>



$ \frac{d x}{d t}=e^{t}+4 e^{-t}, \frac{d y}{d t}=1$ 이므로

$\frac{d y}{d x}=\frac{\frac{d y}{d t}}{\frac{d x}{d t}}=\frac{1}{e^{t}+4 e^{-t}}$이다.

따라서 $t=\ln 2$ 일 때 $\frac{d y}{d x}$ 의 값은

$\frac{1}{e^{\ln 2}+4 e^{-\ln 2}}=\frac{1}{2+4 \times \frac{1}{2}}=\frac{1}{4}
$


<<<P>>>25. 매개변수 $t$ 로 나타내어진 곡선
$ x=e^{t}-4 e^{-t}, \quad y=t+1 $
에서 $t=\ln 2$ 일 때, $\frac{d y}{d x}$ 의 값은? [3점]

① $1$
② $\frac{1}{2}$
③ $\frac{1}{3}$
④ $\frac{1}{4}$
⑤ $\frac{1}{5}$



<<<A>>>②

<<<S>>>



$x=t$ 에서 단면의 넓이는
$\left(\sqrt{\frac{3 t+1}{t^{2}}}\right)^{2}=\frac{3 t+1}{t^{2}}=\frac{3}{t}+\frac{1}{t^{2}}$이므로

부피는

$\int_{1}^{2}\left(\frac{3}{t}+\frac{1}{t^{2}}\right) d t $

$=\left[3 \ln |t|-\frac{1}{t}\right]_{1}^{2}$

$=\left(3 \ln 2-\frac{1}{2}\right)-\left(3 \ln 1-\frac{1}{1}\right)$

$=3 \ln 2-\frac{1}{2}+1$

$=\frac{1}{2}+3 \ln 2$


<<<P>>>26. 그림과 같이 곡선 $y=\sqrt{\frac{3 x+1}{x^{2}}}(x>0)$ 과 $x$ 축 및 두 직선 $x=1, x=2$ 로 둘러싸인 부분을 밑면으로 하고 $x$ 축에 수직인 평면으로 자른 단면이 모두 정사각형인 입체도형의 부피는? [3점]

[IMG]

① $3 \ln 2$
② $\frac{1}{2}+3 \ln 2$
③ $1+3 \ln 2$
④ $\frac{1}{2}+4 \ln 2$
⑤ $1+4 \ln 2$



<<<A>>>③

<<<S>>>



$\overline{\mathrm{C}_{1} \mathrm{E}_{1}}=\sqrt{3}, \overline{\mathrm{C}_{1} \mathrm{F}_{1}}=\frac{\sqrt{3}}{3}$ 이므로

$\overline{\mathrm{E}_{1} \mathrm{F}_{1}}=\frac{2}{3} \sqrt{3}$이고 $\overline{\mathrm{F}_{1} \mathrm{H}_{1}}=\frac{2}{3}$이다.

$\therefore \overline{\mathrm{G}_{1} \mathrm{H}_{1}}=\frac{2}{3} \sqrt{3}-\frac{2}{3}=\frac{2}{3}(\sqrt{3}-1)$

$\therefore \triangle \mathrm{G}_{1} \mathrm{E}_{1} \mathrm{H}_{1}=\frac{1}{2} \times \frac{2}{3}(\sqrt{3}-1) \times \frac{2}{3} \sqrt{3}$

$=\frac{2 \sqrt{3}(\sqrt{3}-1)}{9}$

$\triangle \mathrm{H}_{1} \mathrm{F}_{1} \mathrm{D}_{1}=\frac{1}{2} \times \frac{2}{3} \times \frac{\sqrt{3}}{3}=\frac{\sqrt{3}}{9}$

$\therefore S_{1}=\frac{2 \sqrt{3}(\sqrt{3}-1)}{9}+\frac{\sqrt{3}}{9}=\frac{6-\sqrt{3}}{9}$

$\overline{\mathrm{AB}_{2}}=a$라고 하면 $\overline{\mathrm{B}_{2} \mathrm{C}_{2}}=2 a$이다.


점 $\mathrm{E}_{1}$ 에서 $\overline{\mathrm{B}_{2} \mathrm{C}_{2}}$ 에 내린 수선의 발을 $\mathrm{T}$ 라 하자.

$\overline{\mathrm{B}_{1} \mathrm{B}_{2}}=\overline{\mathrm{E}_{1} \mathrm{T}}=1-a$ 이고, $\angle \mathrm{E}_{1} \mathrm{C}_{2} \mathrm{T}=\frac{\pi}{4}$ 이므로 $\overline{\mathrm{C}_{2} \mathrm{T}}=1-a$ 이다. 또, $\overline{\mathrm{B}_{1} \mathrm{E}_{1}}=\overline{\mathrm{B}_{2} \mathrm{T}}=2-\sqrt{3}$ 이므로 $\overline{\mathrm{B}_{2} \mathrm{T}}+\overline{\mathrm{C}_{2} \mathrm{T}}=\overline{\mathrm{B}_{2} \mathrm{C}_{2}}$ 에서 $2-\sqrt{3}+1-a=2 a, a=\frac{3-\sqrt{3}}{3}=\frac{\sqrt{3}-1}{\sqrt{3}}$ 즉, 두 사각형 $\mathrm{AB}_{1} \mathrm{C}_{1} \mathrm{D}_{1}$ 과 $\mathrm{AB}_{2} \mathrm{C}_{2} \mathrm{D}_{2}$ 의 닮음비가 $\frac{\sqrt{3}-1}{\sqrt{3}}$ 이므로 $\lim _{n \rightarrow \infty} S_{n}$ 은 첫째항이 $\frac{6-\sqrt{3}}{9}$ 이고 공비가 $\left(\frac{\sqrt{3}-1}{\sqrt{3}}\right)^{2}=\frac{4-2 \sqrt{3}}{3}$ 인 등비급수이다.

$\lim _{n \rightarrow \infty} S_{n}= \frac{\frac{6-\sqrt{3}}{9}}{1-\frac{4-2 \sqrt{3}}{3}} $

$=\frac{\frac{1}{3}(6-\sqrt{3})}{3-4+2 \sqrt{3}}$

$=\frac{\frac{\sqrt{3}}{3}(2 \sqrt{3}-1)}{2 \sqrt{3}-1}=\frac{\sqrt{3}}{3}$

<<<P>>>27. 그림과 같이 $\overline{\mathrm{AB}_{1}}=1, \overline{\mathrm{B}_{1} \mathrm{C}_{1}}=2$ 인 직사각형 $\mathrm{AB}_{1} \mathrm{C}_{1} \mathrm{D}_{1}$ 이 있다.

$\angle \mathrm{AD}_{1} \mathrm{C}_{1}$ 을 삼등분하는 두 직선이 선분 $\mathrm{B}_{1} \mathrm{C}_{1}$ 과 만나는 점 중 점 $\mathrm{B}_{1}$ 에 가까운 점을 $\mathrm{E}_{1}$,
점 $\mathrm{C}_{1}$ 에 가까운 점을 $\mathrm{F}_{1}$ 이라 하자.

$\overline{\mathrm{E}_{1} \mathrm{F}_{1}}=\overline{\mathrm{F}_{1} \mathrm{G}_{1}}, \angle \mathrm{E}_{1} \mathrm{F}_{1} \mathrm{G}_{1}=\frac{\pi}{2}$ 이고
선분 $\mathrm{AD}_{1}$ 과 선분 $\mathrm{F}_{1} \mathrm{G}_{1}$ 이 만나도록 점 $\mathrm{G}_{1}$ 을 잡아 삼각형 $\mathrm{E}_{1} \mathrm{F}_{1} \mathrm{G}_{1}$ 을 그린다.

선분 $\mathrm{E}_{1} \mathrm{D}_{1}$ 과 선분 $\mathrm{F}_{1} \mathrm{G}_{1}$ 이 만나는 점을 $\mathrm{H}_{1}$ 이라 할 때,

두 삼각형 $\mathrm{G}_{1} \mathrm{E}_{1} \mathrm{H}_{1}, \mathrm{H}_{1} \mathrm{F}_{1} \mathrm{D}_{1}$ 로 만들어진 [IMG] 모양의 도형에 색칠하여 얻은 그림을 $R_{1}$ 이라 하자.

그림 $R_{1}$ 에 선분 $\mathrm{AB}_{1}$ 위의 점 $\mathrm{B}_{2}$, 선분 $\mathrm{E}_{1} \mathrm{G}_{1}$ 위의 점 $\mathrm{C}_{2}$, 선분 $\mathrm{AD}_{1}$ 위의 점 $\mathrm{D}_{2}$ 와 점 $\mathrm{A}$ 를 꼭짓점으로 하고 

$\overline{\mathrm{AB}_{2}}: \overline{\mathrm{B}_{2} \mathrm{C}_{2}}=1: 2$ 인 직사각형 $\mathrm{AB}_{2} \mathrm{C}_{2} \mathrm{D}_{2}$ 를 그린다.

직사각형 $\mathrm{AB}_{2} \mathrm{C}_{2} \mathrm{D}_{2}$ 에 그림 $R_{1}$ 을 얻은 것과 같은 방법으로 [IMG] 모양의 도형을 그리고 색칠하여 얻은 그림을 $R_{2}$ 라 하자.

이와 같은 과정을 계속하여 $n$ 번째 얻은 그림 $R_{n}$ 에 색칠되어 있는 부분의 넓이를 $S_{n}$ 이라 할 때, $\lim _{n \rightarrow \infty} S_{n}$ 의 값은? [3점]


[IMG]

① $\frac{2 \sqrt{3}}{9}$
② $\frac{5 \sqrt{3}}{18}$
③ $\frac{\sqrt{3}}{3}$
④ $\frac{7 \sqrt{3}}{18}$
⑤ $\frac{4 \sqrt{3}}{9}$


<<<A>>>①

<<<S>>>



$\mathrm{A}^{\prime}(0,2)$ 라고 하자. $\frac{\pi}{6} \leq \theta \leq \frac{\pi}{3}$ 일 때 $1 \leq 2 \cos \theta \leq \sqrt{3}$ 이므로 점 Q는 선분 $\mathrm{OA}^{\prime}$ 위에 있고, 점 $\mathrm{R}$ 는 선분 $\mathrm{PB}$ 위 에 있다.

$\overline{\mathrm{A}^{\prime} \mathrm{B}}$ 는 지름이므로 $\angle \mathrm{A}^{\prime} \mathrm{PB}=\frac{\pi}{2}$ 이고 원주각 성 질에 의해 $\angle \mathrm{PA}^{\prime} \mathrm{B}=\angle \mathrm{P} \mathrm{AB}=\theta$ 이다.
점 $\mathrm{Q}$ 에서 $\overline{\mathrm{A}^{\prime} \mathrm{P}}$ 에 내린 수선의 발을 $\mathrm{H}$ 라고 하면 사각형 $\mathrm{QHPR}$ 는 직사각형이므로 $\overline{\mathrm{PR}}=\overline{\mathrm{QH}}$ 이다.

$\overline{\mathrm{A}^{\prime} \mathrm{Q}}=2-2 \cos \theta$ 이므로
직각삼각형 $\mathrm{A}^{\prime} \mathrm{QH}$ 에서
$\overline{\mathrm{QH}}=(2-2 \cos \theta) \sin \theta$ 이다.
$\therefore f(\theta)=(2-2 \cos \theta) \sin \theta$

$\therefore \quad \int_{\frac{\pi}{6}}^{\frac{\pi}{3}} f(\theta) d \theta$

$=\int_{\frac{\pi}{6}}^{\frac{\pi}{3}}(2-2 \cos \theta) \sin \theta d \theta$

$=\int_{\frac{\sqrt{3}}{2}}^{\frac{1}{2}}(2-2 t)(-d t)(\cos \theta=t$ 로 치환 $)$

$=\int_{\frac{1}{2}}^{\frac{\sqrt{3}}{2}}(2-2 t) d t$

$=\left[2 t-t^{2}\right]_{\frac{1}{2}}^{\frac{\sqrt{3}}{2}}=\left(\sqrt{3}-\frac{3}{4}\right)-\left(1-\frac{1}{4}\right)$

$=\sqrt{3}-\frac{3}{2}=\frac{2 \sqrt{3}-3}{2}$


<<<P>>>28. 좌표평면에서 원점을 중심으로 하고 반지름의 길이가 2 인 원 $C$ 와 두 점 $\mathrm{A}(2,0), \mathrm{B}(0,-2)$ 가 있다. 원 $C$ 위에 있고 $x$ 좌표가 음수인 점 $\mathrm{P}$ 에 대하여 $\angle \mathrm{PAB}=\theta$ 라 하자.
점 $\mathrm{Q}(0,2 \cos \theta)$ 에서 직선 $\mathrm{BP}$ 에 내린 수선의 발을 $\mathrm{R}$ 라 하고, 두 점 $\mathrm{P}$ 와 $\mathrm{R}$ 사이의 거리를 $f(\theta)$ 라 할 때, $\int_{\frac{\pi}{6}}^{\frac{\pi}{3}} f(\theta) d \theta$ 의 값은? [4점]

① $\frac{2 \sqrt{3}-3}{2}$
② $\sqrt{3}-1$
③ $\frac{3 \sqrt{3}-3}{2}$
④ $\frac{2 \sqrt{3}-1}{2}$
⑤ $\frac{4 \sqrt{3}-3}{2}$


[IMG]


<<<A>>>$24$

<<<S>>>



$f(x)$ 의 최고차항의 계수가 양수이면 $x \rightarrow \pm \infty$ 일 때 $f(x) \rightarrow \infty$ 이므로 함수 $g(x)$ 는 최댓값을 갖지 않는다.
따라서 $f(x)$ 의 최고차항의 계수는 음수이다. $x \rightarrow \pm \infty$ 일 때 $f(x) \rightarrow-\infty$ 이므로 $g(x)<0$ 이며 $g(x) \rightarrow 0$ 이다.
또, 이차함수의 그래프는 대칭축에 대하여 대칭이므 로 $y=g(x)$ 의 그래프도 $y=f(x)$ 의 그래프의 대 칭축에 대하여 대칭이다.
따라서 $y=g(x)$ 의 그래프의 개형은 다음과 같다.

[IMG]

이때 $b$와 $b+6$이 $a$에 대하여 대칭이므로 $a=b+3$ 

즉, 방정식 $g^{\prime}(x)=0$ 의 서로 다른 세 실근이 $b, b+3, b+6$ 이다.

$g^{\prime}(x)=f^{\prime}(x) \times e^{f(x)}+\{f(x)+2\} e^{f(x)} \times f^{\prime}(x)$

$=e^{f(x)} \times f^{\prime}(x) \times\{f(x)+3\}=0$ 에서 방정식 $f(x)+3=0$ 의 서로 다른 두 실근이 $b, b+6$ 이고

방정식 $f^{\prime}(x)=0$ 의 실근이 $b+3$ 이다.

음수 $k$ 에 대하여 $f(x)+3=k(x-b)(x-b-6)$ 이라 하자.

$x=a=b+3$ 을 대입하면 $6+3=k \times 3 \times(-3)$ 에서 $k=-1$ 이다.

$\therefore f(x)=-(x-b)(x-b-6)-3$

방정식 $f(x)=0$ 의 서로 다른 두 실근이 $\alpha, \beta$ 이므로 방정식 $f(x+b)=0$ 의 서로 다른 두 실근은 $\alpha-b, \beta-b$ 이다.

이때 $\alpha-\beta=(\alpha-b)-(\beta-b)$ 이므로 $(\alpha-\beta)^{2}$ 의 값은 방정식 $f(x+b)=0$ 에서 구할 수 있다.

$f(x+b)=-x(x-6)-3$이므로 $x^{2}-6 x+3=0$에서 $x=3 \pm \sqrt{6}$이다.

$\therefore(\alpha-\beta)^{2}=(2 \sqrt{6})^{2}=24$



<<<P>>>29. 이차함수 $f(x)$ 에 대하여 함수 $g(x)=\{f(x)+2\} e^{f(x)}$ 이 다음 조건을 만족시킨다.

(가) $f(a)=6$ 인 $a$ 에 대하여 $g(x)$ 는 $x=a$ 에서 최댓값을 갖는다.

(나) $g(x)$ 는 $x=b, x=b+6$ 에서 최솟값을 갖는다.

방정식 $f(x)=0$ 의 서로 다른 두 실근을 $\alpha, \beta$ 라 할 때, $(\alpha-\beta)^{2}$ 의 값을 구하시오. (단, $a, b$ 는 실수이다.) [4점]



<<<A>>>$115$

<<<S>>>



조건 (가)에서 (분모) $\rightarrow 0$ 일 때 극한이 존재하므로 (분자)$\rightarrow 0$이다.

$\therefore \lim _{x \rightarrow 0} \sin (\pi f(x))=\sin (\pi f(0))=0 $

$\therefore f(0)=n$ ($n$은 정수)

$p(x)=\sin (\pi f(x))$라고 하면

$p(0)=\sin (n \pi)=0$이므로 조건 (가)에서

$\lim _{x \rightarrow 0} \frac{p(x)}{x}=p^{\prime}(0)=0$

$p^{\prime}(x)=\cos (\pi f(x)) \times \pi f^{\prime}(x)$에서

$p^{\prime}(0)=\cos (n \pi) \times \pi f^{\prime}(0)=0$이므로

$f^{\prime}(0)=0$이다.

따라서 $f(x)=9 x^{3}-a x^{2}+n$ (단, $a$ 는 상수 $)$ 이라 할 수 있다.

함수 $g(x)$ 가 $x=1$ 에서 연속이므로 $\lim _{x \rightarrow 1-} g(x)=\lim _{x \rightarrow 1+} g(x)$

이때 $\lim _{x \rightarrow 1-} g(x)=\lim _{x \rightarrow 1-} f(x)=f(1)=9-a+n$ 이고

$\lim _{x \rightarrow 1+} g(x) $

$=\lim _{x \rightarrow 1+} g(x-1)$

$=\lim _{t \rightarrow 0+} g(t)$

$=\lim _{t \rightarrow 0+} f(t)$

$=f(0)=n$ 이므로 $9-a+n=n$ 에서 $a=9$ 이다.

$\therefore f(x)=9 x^{3}-9 x^{2}+n$

$f^{\prime}(x)=27 x^{2}-18 x=9 x(3 x-2)$이므로

함수 $f(x)$ 는 $x=0$ 에서 극대, $x=\frac{2}{3}$ 에서 극소이다.

$f(0)=n$이고

$f\left(\frac{2}{3}\right)=9 \times \frac{8}{27}-9 \times \frac{4}{9}+n=n-\frac{4}{3}$이므로

$n\left(n-\frac{4}{3}\right)=5$

정리하면 $(n-3)(3 n+5)=0$ 이고 $n$ 은 정수이므로 $n=3$ 이다.

$\therefore f(x)=9 x^{3}-9 x^{2}+3$

정수 $m$ 에 대하여

$\int_{m}^{m+1} x g(x) d x$

$=\int_{0}^{1}(x+m) g(x+m) d x$

$=\int_{0}^{1}(x+m) g(x) d x$

$=\int_{0}^{1}(x+m) f(x) d x$

$=\int_{0}^{1} x f(x) d x+m \int_{0}^{1} f(x) d x$

$\int_{0}^{1} x f(x) d x=\int_{0}^{1}\left(9 x^{4}-9 x^{3}+3 x\right) d x$

$=\left[\frac{9}{5} x^{5}-\frac{9}{4} x^{4}+\frac{3}{2} x^{2}\right]_{0}^{1}$

$=\frac{9}{5}-\frac{9}{4}+\frac{3}{2}=\frac{36-45+30}{20}=\frac{21}{20}$

$\int_{0}^{1} f(x) d x=\int_{0}^{1}\left(9 x^{3}-9 x^{2}+3\right) d x$

$=\left[\frac{9}{4} x^{4}-3 x^{3}+3 x\right]_{0}^{1}=\frac{9}{4}$

이므로 $\int_{m}^{m+1} x g(x) d x=\frac{21}{20}+\frac{9}{4} m$ 이다.

따라서 $\int_{0}^{5} x g(x) d x $
$=\sum_{m=0}^{4}\left(\frac{21}{20}+\frac{9}{4} m\right)$

$=\frac{21}{20} \times 5+\frac{9}{4} \times \frac{4 \times 5}{2}$

$=\frac{21}{4}+\frac{45}{2}$

$=\frac{111}{4}$

$\therefore p+q=4+111=115$


<<<P>>>30. 최고차항의 계수가 9 인 삼차함수 $f(x)$ 가 다음 조건을 만족시킨다.

(가) $\lim _{x \rightarrow 0} \frac{\sin (\pi \times f(x))}{x}=0$

(나) $f(x)$ 의 극댓값과 극솟값의 곱은 5 이다.

함수 $g(x)$ 는 $0 \leq x<1$ 일 때 $g(x)=f(x)$ 이고 모든 실수 $x$ 에 대하여 $g(x+1)=g(x)$ 이다.
$g(x)$ 가 실수 전체의 집합에서 연속일 때, $\int_{0}^{5} x g(x) d x=\frac{q}{p}$ 이다. $p+q$ 의 값을 구하시오. (단, $p$ 와 $q$ 는 서로소인 자연수이다.) [4점]



<<<A>>>⑤

<<<S>>>



점 $\mathrm{A}$ 를 $x y$ 평면에 대하여 대칭이동하면 $z$ 좌표의 부호만 반대로 하면 되므로 $\mathrm{B}(3,0,2), \mathrm{C}(0,4,2)$

$\therefore \overline{\mathrm{BC}}=\sqrt{3^{2}+4^{2}}=5$


<<<P>>>23. 좌표공간의 점 $\mathrm{A}(3,0,-2)$ 를 $x y$ 평면에 대하여 대칭이동한 점을 $\mathrm{B}$ 라 하자. 점 $\mathrm{C}(0,4,2)$ 에 대하여 선분 $\mathrm{BC}$ 의 길이는? [2점]

① $1$
② $2$
③ $3$
④ $4$
⑤ $5$


<<<A>>>④

<<<S>>>



(점근선의 기울기 )$=\pm \frac{4}{a}$ 이므로 $\frac{4}{a}=3$

$\therefore a=\frac{4}{3}$


<<<P>>>24. 쌍곡선 $\frac{x^{2}}{a^{2}}-\frac{y^{2}}{16}=1$ 의 점근선 중 하나의 기울기가 3 일 때, 양수 $a$ 의 값은? [3점]

① $\frac{1}{3}$
② $\frac{2}{3}$
③ $1$
④ $\frac{4}{3}$
⑤ $\frac{5}{3}$


<<<A>>>②

<<<S>>>



$ \overrightarrow{p}=(x, y)$ 라 하고 원점을 $\mathrm{O}$ 라 하자.

$\overrightarrow{p} \cdot \overrightarrow{a}=\overrightarrow{a} \cdot \overrightarrow{b}$에서 $(x, y) \cdot(3,0)=(3,0) \cdot(1,2)$

$3 x=3 \quad \therefore x=1$

따라서 $\overrightarrow{p}=\overrightarrow{\mathrm{OP}}$ 에서 점 $\mathrm{P}$ 는 직선 $x=1$ 위의 점이다.

$\overrightarrow{q}=\overrightarrow{\mathrm{OQ}}$ 라 하면 $|\overrightarrow{q}-\overrightarrow{c}|=1$ 에서 점 $\mathrm{Q}$ 는 점 $(4,2)$ 를 중심으로 하고 반지름의 길이가 1 인 원 위의 점이다.

$|\overrightarrow{p}-\overrightarrow{q}|=|\overrightarrow{\mathrm{QP}}|$ 이므로 두 점 $\mathrm{P}, \mathrm{Q}$ 사이의 거리의 최솟값은 $4-1-1=2$


<<<P>>>25. 좌표평면에서 세 벡터 $\overrightarrow{a}=(3,0), \quad \overrightarrow{b}=(1,2), \quad \overrightarrow{c}=(4,2)$에 대하여 두 벡터 $\overrightarrow{p}, \overrightarrow{q}$ 가
$\overrightarrow{p} \cdot \overrightarrow{a}=\overrightarrow{a} \cdot \overrightarrow{b}, \quad|\overrightarrow{q}-\overrightarrow{c}|=1$을 만족시킬 때, $|\overrightarrow{p}-\overrightarrow{q}|$ 의 최솟값은? [3점]

① $1$
② $2$
③ $3$
④ $4$
⑤ $5$



<<<A>>>③

<<<S>>>



문제의 조건에 따라 $\overline{\mathrm{AB}}=\overline{\mathrm{BF}}$ 이고 포물선의 정 의에서 $\overline{\mathrm{AB}}=\overline{\mathrm{AF}}$ 이므로 $\triangle \mathrm{ABF}$ 는 정삼각형이고 $\angle \mathrm{BFO}=60^{\circ}$ 이다.
또, 점 $\mathrm{C}$ 에서 준선에 내린 수선의 발을 $\mathrm{H}$ 라 하면 포물선의 정의에서 $\overline{\mathrm{CF}}=\overline{\mathrm{CH}}=k$ 로 놓자.
또, $\angle \mathrm{BCH}=\angle \mathrm{CFO}=60^{\circ}$ 이므로

$\overline{\mathrm{CH}}=k$이면 $\overline{\mathrm{BC}}=2 k$이다.

따라서 $\overline{\mathrm{BC}}+3 \overline{\mathrm{CF}}=6$에서

$2 k+3 k=6$, 즉 $k=\frac{6}{5}$이고

점 $\mathrm{F}$ 에서 준선에 내린 수선의 발을 $\mathrm{H}^{\prime}$ 이라 하면

$\overline{\mathrm{FH}^{\prime}}=2 p=k+\frac{k}{2}=\frac{9}{5}$

$p=\frac{9}{10}$


<<<P>>>26. 초점이 $\mathrm{F}$ 인 포물선 $y^{2}=4 p x$ 위의 한 점 $\mathrm{A}$ 에서 포물선의 준선에 내린 수선의 발을 $\mathrm{B}$ 라 하고, 선분 $\mathrm{BF}$ 와 포물선이 만나는 점을 $\mathrm{C}$ 라 하자. $\overline{\mathrm{AB}}=\overline{\mathrm{BF}}$ 이고 $\overline{\mathrm{BC}}+3 \overline{\mathrm{CF}}=6$ 일 때, 양수 $p$ 의 값은? [3점]

① $\frac{7}{8}$
② $\frac{8}{9}$
③ $\frac{9}{10}$
④ $\frac{10}{11}$
⑤ $\frac{11}{12}$


[IMG]


<<<A>>>①

<<<S>>>



$ \overline{\mathrm{DP}}$ 는 점 $\mathrm{P}$ 가 점 $\mathrm{D}$ 에서 선분 $\mathrm{BC}$ 에 내린 수선 의 발 $\mathrm{H}$ 일 때 최소이고, $\overline{\mathrm{DB}} \times \overline{\mathrm{DC}}=\overline{\mathrm{BC}} \times \overline{\mathrm{DH}}$ 에서 $(\overline{\mathrm{DP}}$ 의 최솟값 $)=\overline{\mathrm{DH}}=\sqrt{3}$
또, $\overline{\mathrm{AD}}$ 와 $\overline{\mathrm{DP}}$ 는 서로 수직이므로 $\overline{\mathrm{AP}}^{2}=\overline{\mathrm{AD}}^{2}+\overline{\mathrm{DP}}^{2}=3^{2}+\overline{\mathrm{DP}}^{2}$ 에서 $\overline{\mathrm{AP}}$ 는 $\overline{\mathrm{DP}}$ 가 최소일 때 최소이다. 따라서 $\overline{\mathrm{AP}}+\overline{\mathrm{DP}}$ 의 최솟값은 $\sqrt{3^{2}+(\sqrt{3})^{2}}+\sqrt{3}=3 \sqrt{3}$


<<<P>>>27. 그림과 같이 $\overline{\mathrm{AD}}=3, \overline{\mathrm{DB}}=2, \overline{\mathrm{DC}}=2 \sqrt{3}$ 이고 $\angle \mathrm{ADB}=\angle \mathrm{ADC}=\angle \mathrm{BDC}=\frac{\pi}{2}$ 인 사면체 $\mathrm{ABCD}$ 가 있다. 선분 $\mathrm{BC}$ 위를 움직이는 점 $\mathrm{P}$ 에 대하여 $\overline{\mathrm{AP}}+\overline{\mathrm{DP}}$ 의 최솟값은? [3점]


[IMG]

① $3 \sqrt{3}$
② $\frac{10 \sqrt{3}}{3}$
③ $\frac{11 \sqrt{3}}{3}$
④ $4 \sqrt{3}$
⑤ $\frac{13 \sqrt{3}}{3}$



<<<A>>>①

<<<S>>>



$ \mathrm{P}(2,3)$ 에서 접선의 방정식은 $\frac{2 x}{16}+\frac{3 y}{12}=1$ 에서 $y=-\frac{1}{2} x+4$ 이고 $x$ 축과의 교점 $\mathrm{S}$ 의 좌표는 $(8,0)$ 이다.
또, $\mathrm{F}(c, 0)$ 이라 하면

$c^{2}=16-12=4$

$c=2$이므로 $\mathrm{F}^{\prime}(-2,0), \mathrm{F}(2,0)$이다.

이때 두 삼각형 $\mathrm{QF}^{\prime} \mathrm{F}$ 와 $\mathrm{RF}^{\prime} \mathrm{S}$ 는 $\angle \mathrm{QF}^{\prime} \mathrm{F}$ 는 공통이고 $\angle \mathrm{QFF}^{\prime}=\angle \mathrm{RSF}^{\prime}$ 이므로 닮음이다.
또, 그 닮음비는 $\overline{\mathrm{F}^{\prime} \mathrm{F}}: \overline{\mathrm{F}^{\prime} \mathrm{S}}=4: 10=2: 5$ 이다.
따라서 ($\triangle \mathrm{SRF}^{\prime}$의 둘레의 길이)

$=\frac{5}{2} \times$ ( $\triangle \mathrm{FQF}^{\prime}$의 둘레의 길이)

$=\frac{5}{2} \times(8+4)=30$


<<<P>>>28. 그림과 같이 두 점 $\mathrm{F}(c, 0), \mathrm{F}^{\prime}(-c, 0)(c>0)$ 을 초점으로 하는 타원 $\frac{x^{2}}{16}+\frac{y^{2}}{12}=1$ 위의 점 $\mathrm{P}(2,3)$ 에서 타원에 접하는 직선을 $l$ 이라 하자. 점 $\mathrm{F}$ 를 지나고 $l$ 과 평행한 직선이 타원과 만나는 점 중 제 2 사분면 위에 있는 점을 $\mathrm{Q}$ 라 하자. 두 직선 $\mathrm{F}^{\prime} \mathrm{Q}$ 와 $l$ 이 만나는 점을 $\mathrm{R}, l$ 과 $x$ 축이 만나는 점을 $\mathrm{S}$ 라 할 때, 삼각형 $\mathrm{SRF}^{\prime}$ 의 둘레의 길이는? [4점]


[IMG]

① $30$
② $31$
③ $32$
④ $33$
⑤ $34$



<<<A>>>$40$

<<<S>>>



점 $\mathrm{G}$ 에서 $\overline{\mathrm{AB}}$ 에 내린 수선의 발을 $\mathrm{H}_{1}$ 이라 하면 삼수선의 정리에 의해 점 $\mathrm{P}$ 에서 $\overline{\mathrm{AB}}$ 에 내린 수선 의 발과 일치한다.

마찬가지로 점 $\mathrm{H}$ 에서 $\overline{\mathrm{CD}}$ 에 내린 수선의 발을 $\mathrm{H}_{2}$ 라 하면 점 $\mathrm{Q}$ 에서 $\overline{\mathrm{CD}}$ 에 내 린 수선의 발과 일치한다.

$\overline{\mathrm{PH}}_{1}=2 \sqrt{3}, \overline{\mathrm{QH}}_{2}=4$ 이므로
$\overline{\mathrm{GH}_{1}}=\sqrt{\overline{\mathrm{PH}_{1}}^{2}-\overline{\mathrm{P} G}^{2}}=\sqrt{12-3}=3$,
$\overline{\mathrm{HH}}_{2}=\sqrt{\overline{\mathrm{QH}_{2}}^{2}-\overline{\mathrm{QH}}^{2}}=2$

[IMG]

위 그림에서 $\overline{\mathrm{GH}}=\sqrt{13}$

$\overline{\mathrm{CQ}}=\sqrt{\overline{\mathrm{QH}}^{2}+\overline{\mathrm{CH}}^{2}}=\sqrt{32}=4 \sqrt{2}$

$\overline{\mathrm{CP}}=\sqrt{\overline{\mathrm{CG}}^{2}+\overline{\mathrm{GP}}^{2}}=\sqrt{32}=4 \sqrt{2}$

$\overline{\mathrm{PQ}}=\sqrt{\overline{\mathrm{GH}}^{2}+(\overline{\mathrm{QH}}-\overline{\mathrm{P} G})^{2}}=4$

$\triangle \mathrm{P} \mathrm{QC}$ 는 세 변의 길이가 $4 \sqrt{2}, 4 \sqrt{2}, 4$ 이므로

( $\triangle \mathrm{PQC}$ 의 넓이 )$=4 \sqrt{7}$ 이고 점 $\mathrm{G}$ 에서 $\overline{\mathrm{CD}}$ 에 내린 수선의 발을 $\mathrm{H}_{3}$ 이라 하면

$\triangle \mathrm{GHC}$

$=\triangle \mathrm{GH}_{3} \mathrm{C}+\square \mathrm{GHH}_{2} \mathrm{H}_{3}-\triangle \mathrm{HH}_{2} \mathrm{C}$

$=5+7-4=8$

$\cos \theta=\frac{\triangle \mathrm{GHC}}{\triangle \mathrm{P} \mathrm{QC}}=\frac{8}{4 \sqrt{7}}=\frac{2}{\sqrt{7}}$

$\therefore 70 \cos ^{2} \theta=70 \times \frac{4}{7}=40$


<<<P>>>29. 그림과 같이 한 변의 길이가 8 인 정사각형 $\mathrm{ABCD}$ 에 두 선분 $\mathrm{AB}, \mathrm{CD}$ 를 각각 지름으로 하는 두 반원이 붙어 있는 모양의 종이가 있다. 반원의 호 $\mathrm{AB}$ 의 삼등분점 중 점 $\mathrm{B}$ 에 가까운 점을 $\mathrm{P}$ 라 하고, 반원의 호 $\mathrm{CD}$ 를 이등분하는 점을 $\mathrm{Q}$ 라 하자. 이 종이에서 두 선분 $\mathrm{AB}$ 와 $\mathrm{CD}$ 를 접는 선으로 하여 두 반원을 접어 올렀을 때 두 점 $\mathrm{P}, \mathrm{Q}$ 에서 평면 $\mathrm{ABCD}$ 에 내린 수선의 발을 각각 $\mathrm{G}, \mathrm{H}$ 라 하면 두 점 $\mathrm{G}, \mathrm{H}$ 는 정사각형 $\mathrm{ABCD}$ 의 내부에 놓여 있고, $\overline{\mathrm{PG}}=\sqrt{3}, \overline{\mathrm{QH}}=2 \sqrt{3}$ 이다. 두 평면 $\mathrm{PCQ}$ 와 $\mathrm{ABCD}$ 가 이루는 각의 크기가 $\theta$ 일 때, $70 \times \cos ^{2} \theta$ 의 값을 구하시오. (단, 종이의 두께는 고려하지 않는다.) [4점]


[IMG]


<<<A>>>$45$

<<<S>>>



$\overrightarrow{\mathrm{AP}} \cdot \overrightarrow{\mathrm{OC}} \geq \frac{\sqrt{2}}{2}$ 에서 $\overrightarrow{\mathrm{AP}}$ 와 $\overrightarrow{\mathrm{OC}}$ 가 이루는
각을 $\alpha$ 라 하면

$1 \times 1 \times \cos \alpha \geq \frac{\sqrt{2}}{2} $

$\cos \alpha \geq \frac{\sqrt{2}}{2}$이므로 $0 \leq \alpha \leq \frac{\pi}{4}$ 이다. 또

$\overrightarrow{\mathrm{AP}} \cdot \overrightarrow{\mathrm{AQ}}=\overrightarrow{\mathrm{AP}} \cdot(\overrightarrow{\mathrm{AB}}+\overrightarrow{\mathrm{BQ}})$

$=\overrightarrow{\mathrm{AP}} \cdot \overrightarrow{\mathrm{AB}}+\overrightarrow{\mathrm{AP}} \cdot \overrightarrow{\mathrm{BQ}}$

에서 $\overrightarrow{\mathrm{AP}} \cdot \overrightarrow{\mathrm{AB}}$ 는 $\overrightarrow{\mathrm{AP}}$ 와 $\overrightarrow{\mathrm{AB}}$ 가 이루는 각이 최대일 때 최소이고

그 때의 점 $\mathrm{P}_{0}$ 에 대해 $\overrightarrow{\mathrm{AP}_{0}}$ 과 $\overrightarrow{\mathrm{BQ}}$ 가 반대 방향이면 최소이다.

따라서 점 $\mathrm{P}_{0}$ 과 점 $\mathrm{Q}_{0}$ 은 다음 그림과 같다.

[IMG]

또, $\overrightarrow{\mathrm{BX}} \cdot \overrightarrow{\mathrm{BQ}}_{0} \geq 1$ 에서 $\overrightarrow{\mathrm{AP}_{0}}$ 위의 점 $\mathrm{X}$ 에서 $\overrightarrow{\mathrm{BQ}_{0}}$ 에 내린 수선의 발의 위치는 점 $\mathrm{B}$ 에서 점 $\mathrm{Q}_{0}$ 방향으로 $\frac{1}{2}$ 이상 떨어져야 하므로
$\overrightarrow{\mathrm{BX}} \cdot \overrightarrow{\mathrm{BQ}_{0}}=1$ 이 되는 점 $\mathrm{X}$ 를 점 $\mathrm{X}_{0}$ 이라 하면 $|\overrightarrow{\mathrm{QX}}|^{2}$ 의 최댓값은 $\left|\overrightarrow{\mathrm{Q} X_{0}}\right|^{2}$ 이다.

$\therefore\left|\overrightarrow{\mathrm{Q} \mathrm{X}_{0}}\right|^{2}=\left(\frac{3}{2}\right)^{2}+(2 \sqrt{2})^{2}=\frac{41}{4}$


<<<P>>>2022학년도 대학수학능력시험 9월 모의평가 끝

30. 좌표평면에서 세 점 $\mathrm{A}(-3,1), \mathrm{B}(0,2), \mathrm{C}(1,0)$ 에 대하여 두 점 $\mathrm{P}, \mathrm{Q}$ 가

$|\overrightarrow{\mathrm{AP}}|=1, \quad|\overrightarrow{\mathrm{BQ}}|=2, \quad \overrightarrow{\mathrm{AP}} \cdot \overrightarrow{\mathrm{OC}} \geq \frac{\sqrt{2}}{2}$

를 만족시킬 때, $\overrightarrow{\mathrm{AP}} \cdot \overrightarrow{\mathrm{AQ}}$ 의 값이 최소가 되도록 하는 두 점 $\mathrm{P}, \mathrm{Q}$ 를 각각 $\mathrm{P}_{0}, \mathrm{Q}_{0}$ 이라 하자.

선분 $\mathrm{AP}_{0}$ 위의 점 $\mathrm{X}$ 에 대하여 $\overrightarrow{\mathrm{BX}} \cdot \overrightarrow{\mathrm{BQ}_{0}} \geq 1$ 일 때, $\left|\overrightarrow{\mathrm{Q}_{0} \mathrm{X}}\right|^{2}$ 의 최댓값은 $\frac{q}{p}$ 이다. $p+q$ 의 값을 구하시오. (단, $\mathrm{O}$ 는 원점이고, $p$ 와 $q$ 는 서로소인 자연수이다.) [4점]




<<<A>>>④

<<<S>>>

[[주어진 식에 지수법칙을 적용합니다.]]

$2^{\sqrt{3}}\times 2^{2-\sqrt{3}}$$=2^{\sqrt{3}+(2-\sqrt{3})}$$=2^{2}$$=4$

<<<P>>>2022학년도 대학수학능력시험 6월 모의평가 시작

$2^{\sqrt{3}}\times 2^{2-\sqrt{3}}$의 값은? [2점]

<<<C>>>① $\sqrt{2}$ ② $2$ ③ $2\sqrt{2}$ ④ $4$ ⑤ $4\sqrt{2}$

<<<A>>>⑤

<<<S>>>

[[$f'(x)$를 적분하여 $f(x)$의 식을 세웁니다.]]

$f(x)$$=\displaystyle\int(3x^{2}-2x)dx$$=x^{3}-x^{2}+C$ ($C$는 적분상수)

[[주어진 조건을 이용하여 적분상수를 구합니다.]]

$f(1)=1^{3}-1^{2}+C=1$에서 $C=1$

따라서 $f(x)= x^{3}-x^{2}+1$이므로

$f(2)= 2^{3}-2^{2}+1=5$

<<<P>>>

함수 $f(x)$가

$f'(x)=3x^{2}-2x$, $f(1)=1$

을 만족시킬 때, $f(2)$의 값은? [2점]

<<<C>>>① $1$ ② $2$ ③ $3$ ④ $4$ ⑤ $5$

<<<A>>>①

<<<S>>>

[[동경 $\mathrm{OP}$가 나타내는 각이 $\theta$가 되도록 제$3$사분면에 적당한 점 $\mathrm{P}$의 좌표를 도입합니다.]]

$\tan\theta =\dfrac{12}{5}$이고 $\pi < \theta < \dfrac{3}{2}\pi$이므로

원점을 $\mathrm{O}$, $\mathrm{P}(-5,\:-12)$라 하면 동경 $\mathrm{OP}$가 나타내는 각이 $\theta$이다.

[[삼각함수의 정의에 따라 $\sin\theta, \cos\theta$의 값을 구합니다.]]

따라서 $\overline{\mathrm{OP}}=\sqrt{(-5)^{2}+(-12)^{2}}=13$이므로

$\sin\theta +\cos\theta =-\dfrac{12}{13}-\dfrac{5}{13}= -\dfrac{17}{13}$

<<<P>>>

$\pi < \theta < \dfrac{3}{2}\pi$인 $\theta$에 대하여 $\tan\theta =\dfrac{12}{5}$일 때, $\sin\theta +\cos\theta$의 값은? [3점]

<<<C>>>① $-\dfrac{17}{13}$ ② $-\dfrac{7}{13}$ ③ $0$ ④ $\dfrac{7}{13}$ ⑤ $\dfrac{17}{13}$

<<<A>>>①

<<<S>>>

[[그래프를 참고하여 주어진 각각의 점에서 좌극한, 우극한을 구합니다.]]

$\displaystyle\lim_{x\to 0-}f(x)= -2$

$\displaystyle\lim_{x\to 2+}f(x)=0$

따라서

$\displaystyle\lim_{x\to 0-}f(x)+\displaystyle\lim_{x\to 2+}f(x)= -2+0= -2$

<<<P>>>

함수 $y=f(x)$의 그래프가 그림과 같다.

[IMG]

$\displaystyle\lim_{x\to 0-}f(x)+\displaystyle\lim_{x\to 2+}f(x)$의 값은? [3점]

<<<C>>>① $-2$ ② $-1$ ③ $0$ ④ $1$ ⑤ $2$

<<<A>>>③

<<<S>>>

[[$g(x)$에 곱의 미분을 적용하여 $g'(x)$를 구합니다.]]

$g(x)=(x^{2}+3)f(x)$에서

$g'(x)= 2xf(x)+(x^{2}+3)f'(x)$

따라서

$g'(1)$ $=2f(1)+4f'(1)$ $= 2\times 2 + 4\times 1$ $=8$

<<<P>>>

다항함수 $f(x)$에 대하여 함수 $g(x)$를

$g(x)=(x^{2}+3)f(x)$

라 하자. $f(1)=2$, $f'(1)=1$일 때, $g'(1)$의 값은? [3점]

<<<C>>>① $6$ ② $7$ ③ $8$ ④ $9$ ⑤ $10$

<<<A>>>④

<<<S>>>

[[주어진 곡선과 직선이 만나는 점의 $x$표를 구합니다.]]

곡선 $y=3x^{2}-x$와 직선 $y=5x$의 교점의 $x$좌표는 $3x^{2}-x=5x$에서 $x=0$ 또는 $x=2$

[[넓이를 구하는 정적분의 식을 세우고 계산합니다.]]

구간 $[0,\:2]$에서 직선 $y=5x$가 곡선 $y=3x^{2}-x$보다 위쪽에 있거나 만나므로 구하는 넓이는

$S=\displaystyle\int_{0}^{2}\left\{5x-(3x^{2}-x)\right\}dx$

$=\displaystyle\int_{0}^{2}(6x-3x^{2})dx$

$=\left[3x^{2}-x^{3}\right]_{0}^{2}$

$=3(4-0)-(8-0)$

$=4$

<<<P>>>

곡선 $y=3x^{2}-x$와 직선 $y=5x$로 둘러싸인 부분의 넓이는? [3점]

<<<C>>>① $1$ ② $2$ ③ $3$ ④ $4$ ⑤ $5$

<<<A>>>②

<<<S>>>

[[문제의 조건식을 정리합니다.]]

$S_{3}-S_{2}=a_{3}$이므로 $a_{6}= 2a_{3}$

[[공차를 미지수로 도입합니다.]]

등차수열 $\left\{a_{n}\right\}$의 공차를 $d$라 하면 $2+5d=2(2+2d)$

$2+5d=4+4d$에서 $d=2$

[[등차수열의 합을 구합니다.]]

따라서 $a_{10}= 2+ 9\times 2 =20$이므로

$S_{10}$$=\dfrac{10(a_{1}+a_{10})}{2}$$=\dfrac{10\times(2+20)}{2}$$=110$

<<<P>>>

첫째항이 $2$인 등차수열 $\left\{a_{n}\right\}$의 첫째항부터 제$n$항까지의 합을 $S_{n}$이라 하자.

$a_{6}= 2(S_{3}-S_{2})$

일 때, $S_{10}$의 값은? [3점]

<<<C>>>① $100$ ② $110$ ③ $120$ ④ $130$ ⑤ $140$

<<<A>>>④

<<<S>>>

[[실수 전체의 집합에서 연속이기 위해 구간의 경계점에서 연속일 조건을 확인합니다.]]

함수 $f(x)$가 $x=a$를 제외한 실수 전체의 집합에서 연속이므로

함수 $\{f(x)\}^{2}$이 $x=a$에서 연속이면 함수 $\{f(x)\}^{2}$은 실수 전체의 집합에서 연속이다.

함수 $\{f(x)\}^{2}$이 $x=a$에서 연속이려면 

$\displaystyle\lim_{x\to a+}\{f(x)\}^{2}=\displaystyle\lim_{x\to a-}\{f(x)\}^{2}=\{f(a)\}^{2}$

이어야 한다.

[[구간의 경곗값에서 좌극한, 우극한, 함숫값을 조사하여 연속의 정의에 적용합니다.]]

$\{f(x)\}^{2}=\begin{cases}
    (2x-a)^{2}&(x\ge a)\\
    (-2x+6)^{2}&(x< a)
    \end{cases}$이므로

$\displaystyle\lim_{x\to a+}\{f(x)\}^{2}=\displaystyle\lim_{x\to a+}(2x-a)^{2}=a^{2}$

$\displaystyle\lim_{x\to a-}\{f(x)\}^{2}=\displaystyle\lim_{x\to a-}(-2x+6)^{2}=(-2a+6)^{2}$

$\{f(a)\}^{2}=(2a-a)^{2}=a^{2}$

$a^{2}=(-2a+6)^{2}$에서

$3(a-2)(a-6)=0$

$a=2$ 또는 $a=6$

따라서 모든 상수 $a$의 값의 합은 $2+6=8$

<<<P>>>

함수 

$f(x)=\begin{cases}
-2x+6&(x< a)\\
2x-a&(x\ge a)
\end{cases}$

에 대하여 함수 $\{f(x)\}^{2}$이 실수 전체의 집합에서 연속이 되도록 하는 모든 상수 $a$의 값의 합은? [3점]

<<<C>>>① $2$ ② $4$ ③ $6$ ④ $8$ ⑤ $10$

<<<A>>>⑤

<<<S>>>

[[$a_{1}=k$라 놓고 차례대로 항을 구해서 규칙성을 찾습니다.]]

$a_{1}=k$라고 하면 $a_{2}=\dfrac{1}{k}$, $a_{3}=\dfrac{8}{k}$, $a_{4}=\dfrac{k}{8}$이고 $a_{5}=k$이다.

$a_{1}=a_{5}$이고 $1$과 $5$는 모두 홀수이므로 수열 $\left\{a_{n}\right\}$은 $k$, $\dfrac{1}{k}$, $\dfrac{8}{k}$, $\dfrac{k}{8}$이 계속 순서대로 반복되는 수열이다.

[[주어진 조건을 이용하여 $k$의 값을 구합니다.]]

$a_{12}=a_{4}=\dfrac{k}{8}$에서 $\dfrac{k}{8}=\dfrac{1}{2}$이고 $k=4$이다.

따라서 $a_{1}+a_{4}=k +\dfrac{k}{8}=4+\dfrac{1}{2}=\dfrac{9}{2}$

[다른 풀이]

[[$a_{12}$에서 시작하여 차례대로 $a_{11},a_{10},a_{9},\cdots$을 구해서 규칙을 찾습니다. ]]

$a_{12}=\dfrac{1}{2}$이고 $a_{12}=\dfrac{1}{a_{11}}$이므로 $a_{11}=2$

$a_{11}=8a_{10}$이므로 $a_{10}=\dfrac{1}{4}$

$a_{10}=\dfrac{1}{a_{9}}$이므로 $a_{9}=4$

$a_{9}= 8a_{8}$이므로 $a_{8}=\dfrac{1}{2}$

$a_{8}=\dfrac{1}{a_{7}}$이므로 $a_{7}=2$

$a_{7}= 8a_{6}$이므로 $a_{6}=\dfrac{1}{4}$

$a_{6}=\dfrac{1}{a_{5}}$이므로 $a_{5}=4$

$a_{5}= 8a_{4}$이므로 $a_{4}=\dfrac{1}{2}$

$a_{4}=\dfrac{1}{a_{3}}$이므로 $a_{3}=2$

$a_{3}= 8a_{2}$이므로 $a_{2}=\dfrac{1}{4}$

$a_{2}=\dfrac{1}{a_{1}}$이므로 $a_{1}=4$

따라서 $a_{1}+a_{4}= 4+\dfrac{1}{2}=\dfrac{9}{2}$

<<<P>>>

수열 $\left\{a_{n}\right\}$이 모든 자연수 $n$에 대하여

$a_{n+1}=\begin{cases}
\dfrac{1}{a_{n}}&(n \text{이 홀수인 경우})\\
8a_{n}&(n \text{이 짝수인 경우})
\end{cases}$

이고 $a_{12}=\dfrac{1}{2}$일 때, $a_{1}+a_{4}$의 값은? [4점]

<<<C>>>① $\dfrac{3}{4}$ ② $\dfrac{9}{4}$ ③ $\dfrac{5}{2}$ ④ $\dfrac{17}{4}$ ⑤ $\dfrac{9}{2}$

<<<A>>>②

<<<S>>>

[[두 곡선의 그래프를 그립니다.]]

[IMG]

$y=\log_{n}x$는 증가하고 $y=-\log_{n}(x+3)+1$은 감소하므로

[[교점이 존재하도록 $x=1,x=2$에서 $y$좌표를 비교합니다.]]

$x=1$일 때 $\log_{n}1 < -\log_{n}(1+3)+1$$\cdots\cdots$㉠

$x=2$일 때 $\log_{n}2>-\log_{n}(2+3)+1$$\cdots\cdots$㉡

이 동시에 성립하면 된다.

[[앞에서 구한 부등식을 풉니다.]]

㉠을 정리하면 $\log_{n}4 < 1$이 되고 만족하는 $n$의 범위는 $n> 4$

㉡을 정리하면 $\log_{n}10 > 1$이 되고 만족하는 $n$의 범위는 $n< 10$

따라서 ㉠, ㉡를 동시에 만족하는 $n$의 범위는 $4< n < 10$이므로 구하는 $n$의 합은 $5+6+7+8+9=35$이다.

[다른 풀이]

[[교점의 $x$좌표를 구하는 방정식을 세우고 정리합니다.]]

두 곡선이 만나는 점의 $x$좌표를 구하는 방정식 $\log_{n}x=-\log_{n}(x+3)+1$을 정리하면

$\log_{n}x(x+3)=1$에서 $x(x+3)=n$이고, 이 방정식이 $1< x < 2$에서 실근을 가지면 된다.

[[정리된 방정식이 $1< x < 2$에서 실근을 가질 조건을 조사합니다.]]

$f(x)=x(x+3)$이라 하면 $y=f(x)$와 $y=n$이 만나는 점이 $1< x < 2$에 존재하면 된다.

$1< x < 2$에서 $f(x)$는 증가하므로 $f(1)< n < f(2)$이 성립하면 된다.

따라서 $n$의 범위는 $4< n < 10$이므로 구하는 $n$의 합은 $5+6+7+8+9=35$이다.

[다른 풀이]

[[교점의 $x$좌표를 구하는 방정식을 세우고 정리합니다.]]

진수 조건에서 $x>0$

$-\log_{n}(x+3)+1=\log_{n}\dfrac{n}{x+3}$이므로

$\log_{n}x=\log_{n}\dfrac{n}{x+3}$에서 $x=\dfrac{n}{x+3}$

$x^{2}+3x-n=0$

[[구해진 방정식의 좌변의 그래프를 그려서 $1< x < 2$에서 실근을 가질 조건을 조사합니다.]]

$f(x)= x^{2}+3x-n$이라 하면

$f(1)< 0$, $f(2)>0$이어야 한다.

$f(1)=4-n< 0$에서 $n >4$

$f(2)=10-n>0$에서 $n< 10$

따라서 $4< n< 10$이므로

$n$의 값은 $5,\:6,\:7,\:8,\:9$이고, 그 합은 $5+6+7+8+9=35$

<<<P>>>

$n\ge 2$인 자연수 $n$에 대하여 두 곡선

$y=\log_{n}x$, $y= -\log_{n}(x+3)+1$

이 만나는 점의 $x$좌표가 $1$보다 크고 $2$보다 작도록 하는 모든 $n$의 값의 합은? [4점]

<<<C>>>① $30$ ② $35$ ③ $40$ ④ $45$ ⑤ $50$

<<<A>>>②

<<<S>>>

[[$[0,\:1]$에서 주어진 함수의 그래프 개형을 그립니다.]]

닫힌구간 $[0,\:1]$에서 연속이고 $f(0)=0$, $f(1)=1$, $\displaystyle\int_{0}^{1}f(x)dx =\dfrac{1}{6}$를 만족하는 함수 $f(x)$의 그래프가 예를 들어 오른쪽 그림과 같다고 하자.

[IMG]

[[$x$축에 대한 대칭이동과 평행이동을 이용하여 $[-1,\:0]$에서 함수의 그래프 개형을 그립니다.]]

$y=-f(x+1)+1$ ($-1\le x\le 0$)의 그래프는 $y=f(x)$ ($0\le x\le 1$)의 그래프를 $x$축에 대하여 대칭이동시킨 후, 그 그래프를 $x$축 방향으로 $-1$만큼, $y$축 방향으로 $1$만큼 평행이동한 그래프이다. 

[IMG]

[[주어진 정적분 조건을 이용하여 $[-1,\:1]$에서 $y=g(x)$의 적분값을 계산합니다.]]

따라서 $\displaystyle\int_{-1}^{1}g(x)dx=\dfrac{5}{6}+\dfrac{1}{6}=1$이다.

[[함수 $y=g(x)$의 주기를 이용합니다.]]

한편, (나)에 의해서 $y=g(x)$는 주기가 $2$인 주기함수이므로, 닫힌구간 $[-3,\: 2]$에서 $y=g(x)$의 그래프는 아래 그림과 같다.

[IMG]

따라서, $\displaystyle\int_{-3}^{2}g(x)dx =\displaystyle\int_{-3}^{-1}g(x)dx +\displaystyle\int_{-1}^{1}g(x)dx +\displaystyle\int_{1}^{2}g(x)dx=1+1+\dfrac{5}{6}=\dfrac{17}{6}$

[다른 풀이]

[[평행이동을 이용하여 $\displaystyle\int_{-1}^{0}g(x)dx$을 계산합니다.]]

$\displaystyle\int_{0}^{1}f(x)dx =\dfrac{1}{6}$이므로 

$\displaystyle\int_{-1}^{0}g(x)dx$$=\displaystyle\int_{-1}^{0}\{-f(x+1)+1\}dx$

$=-\displaystyle\int_{-1}^{0}f(x+1)dx +[x]_{-1}^{0}$

$=-\displaystyle\int_{0}^{1}f(x)dx +1$

$=\dfrac{5}{6}$

[[주어진 정적분 조건을 이용하여 $[-1,\:1]$에서 $y=g(x)$의 적분값을 계산합니다.]]

$\displaystyle\int_{0}^{1}g(x)$$=\displaystyle\int_{0}^{1}f(x)dx$$=\dfrac{1}{6}$

따라서 $\displaystyle\int_{-1}^{1}g(x)dx$$=\displaystyle\int_{-1}^{0}g(x)dx+\displaystyle\int_{0}^{1}g(x)dx$$=\dfrac{5}{6}+\dfrac{1}{6}=1$

[[함수 $y=g(x)$의 주기를 이용합니다.]]

조건 (나)에서 $g(x+2)=g(x)$이므로

$\displaystyle\int_{-3}^{-1}g(x)dx=\displaystyle\int_{-1}^{1}g(x)dx=1$이고

$\displaystyle\int_{1}^{2}g(x)dx=\displaystyle\int_{-1}^{0}g(x)dx =\dfrac{5}{6}$이다.

따라서

$\displaystyle\int_{-3}^{2}g(x)dx$

$=\displaystyle\int_{-3}^{-1}g(x)dx+\displaystyle\int_{-1}^{1}g(x)dx+\displaystyle\int_{1}^{2}g(x)dx$

$=2\displaystyle\int_{-1}^{1}g(x)dx+\displaystyle\int_{-1}^{0}g(x)dx$

$=2\times 1+\dfrac{5}{6}$

$=\dfrac{17}{6}$

<<<P>>>

닫힌구간 $[0,\:1]$에서 연속인 함수 $f(x)$가

$f(0)=0$,  $f(1)=1$,  $\displaystyle\int_{0}^{1}f(x)dx =\dfrac{1}{6}$

을 만족시킨다. 실수 전체의 집합에서 정의된 함수 $g(x)$가 다음 조건을 만족시킬 때, $\displaystyle\int_{-3}^{2}g(x)dx$의 값은? [4점]

(가) $g(x)=\begin{cases}
-f(x+1)+1&(-1< x< 0)\\
f(x)&(0\le x\le 1)
\end{cases}$

(나) 모든 실수 $x$에 대하여 $g(x+2)=g(x)$이다.

<<<C>>>① $\dfrac{5}{2}$ ② $\dfrac{17}{6}$ ③ $\dfrac{19}{6}$ ④ $\dfrac{7}{2}$ ⑤ $\dfrac{23}{6}$

<<<A>>>③

<<<S>>>

[[삼각형 $\mathrm{ABC}$에서 코사인법칙을 적용하여 $\overline{\mathrm{BC}}$를 구합니다.]]

삼각형 $\mathrm{ABD}$는 이등변삼각형이고 $\overline{\mathrm{AB}}=4$이므로 $\overline{\mathrm{BD}}=4$

$\angle\mathrm{BAC}=\angle\mathrm{BDA}=\angle\mathrm{BED}=\theta$라 하자.

삼각형 $\mathrm{ABC}$에서 코사인법칙을 적용하면

$\overline{\mathrm{BC}}^{2}$$=\overline{\mathrm{AB}}^{2}+\overline{\mathrm{AC}}^{2}-2\times\overline{\mathrm{AB}}\times\overline{\mathrm{AC}}\times\cos\theta$

$=4^{2}+5^{2}-2\times 4\times 5\times\dfrac{1}{8}$$=36$이므로

$\overline{\mathrm{BC}}=6$

[[$\overline{\mathrm{AD}}$를 구합니다.]]

$\overline{\mathrm{AD}}=2\overline{\mathrm{AB}}\cos\theta =2\times 4\times\dfrac{1}{8}= 1$이므로 삼각형 $\mathrm{BCD}$는 $\overline{\mathrm{DB}}=\overline{\mathrm{DC}}=4$인 이등변삼각형이다.

[[삼각형 $\mathrm{BED}$에서 사인법칙을 적용합니다.]]

이등변삼각형 $\mathrm{BCD}$의 높이는 $\sqrt{4^{2}-3^{2}}=\sqrt{7}$이므로 $\sin(\angle\mathrm{DBC})=\dfrac{\sqrt{7}}{4}$

$\cos\theta =\dfrac{1}{8}$에서 $\sin\theta =\sqrt{1-\left(\dfrac{1}{8}\right)^{2}}=\dfrac{3\sqrt{7}}{8}$

삼각형 $\mathrm{BED}$에서 사인법칙을 적용하면 $\dfrac{\overline{\mathrm{DE}}}{\sin(\angle\mathrm{DBC})}=\dfrac{\overline{\mathrm{BD}}}{\sin\theta}$

따라서 $\overline{\mathrm{DE}}=\dfrac{\sqrt{7}}{4}\times\dfrac{4}{\dfrac{3\sqrt{7}}{8}}=\dfrac{8}{3}$

[다른 풀이]

[[$\overline{\mathrm{AD}}$를 구합니다.]]

삼각형 $\mathrm{ABD}$에서

$\angle\mathrm{BAC}=\angle\mathrm{BDA}$이고 $\overline{\mathrm{AB}}=4$이므로 $\overline{\mathrm{BD}}=4$

[IMG]

이때, 점 $\mathrm{B}$에서 선분 $\mathrm{AD}$에 내린 수선의 발을 $\mathrm{H}$라 하면

$\overline{\mathrm{AH}}=\overline{\mathrm{AB}}\cos(\angle\mathrm{BAC})=4\times\dfrac{1}{8}=\dfrac{1}{2}$

그러므로 $\overline{\mathrm{AD}}=1$

[[삼각형 $\mathrm{ABC}$에서 코사인법칙을 적용하여 $\overline{\mathrm{BC}}$를 구합니다.]]

삼각형 $\mathrm{ABC}$에서

$\overline{\mathrm{BC}}^{2}$$=\overline{\mathrm{AB}}^{2}+\overline{\mathrm{AC}}^{2}-2\times\overline{\mathrm{AB}}\times\overline{\mathrm{AC}}\times\cos(\angle\mathrm{BAC})$

$=4^{2}+5^{2}-2\times 4\times 5\times\dfrac{1}{8}$

$=36$

이므로 $\overline{\mathrm{BC}}=6$

[[점 $\mathrm{D}$에서 변 $\mathrm{BC}$에 수선을 내립니다.]]

한편, 삼각형 $\mathrm{BCD}$는 $\overline{\mathrm{DB}}=\overline{\mathrm{DC}}=4$인 이등변삼각형이다.

[IMG]

점 $\mathrm{D}$에서 변 $\mathrm{BC}$에 내린 수선의 발을 $\mathrm{H}'$이라 하면

$\overline{\mathrm{BH'}}=\dfrac{1}{2}\overline{\mathrm{BC}}=3$

$\overline{\mathrm{DH'}}=\sqrt{4^{2}-3^{2}}=\sqrt{7}$

$\sin (\angle \mathrm{DEH'})=\sqrt{1-\left( \dfrac{1}{8} \right)^{2}}=\dfrac{3\sqrt{7}}{8}$

[[직각삼각형 $\mathrm{DBH}'$에서  $\overline{\mathrm{DE}}$를 구합니다.]]

직각삼각형 $\mathrm{DBH}'$에서

$\sin (\angle \mathrm{DEH'})=\dfrac{\sqrt{7}}{\overline{\mathrm{DE}}}$

따라서 $\overline{\mathrm{DE}}=\dfrac{8}{3}$

<<<P>>>

그림과 같이 $\overline{\mathrm{AB}} = 4$, $\overline{\mathrm{AC}} =5$이고 $\cos(\angle\mathrm{BAC})=\dfrac{1}{8}$인 삼각형 $\mathrm{ABC}$가 있다.

선분 $\mathrm{AC}$ 위의 점 $\mathrm{D}$와 선분 $\mathrm{BC}$ 위의 점 $\mathrm{E}$에 대하여 $\angle\mathrm{BAC}=\angle\mathrm{BDA}=\angle\mathrm{BED}$일 때, 선분 $\mathrm{DE}$의 길이는? [4점]

[IMG]

<<<C>>>① $\dfrac{7}{3}$ ② $\dfrac{5}{2}$ ③ $\dfrac{8}{3}$ ④ $\dfrac{17}{6}$ ⑤ $3$

<<<A>>>⑤

<<<S>>>

[[$k=1,\:4,\:9,\:16$일 때와 $k \neq 1,\:4,\:9,\:16$일 때로 경우를 나누어 $f(\sqrt{k})$를 계산한다.]]

(a) $k=1,\:4,\:9,\:16$일 때

$f(1)=1$이고 $f(x+1)=f(x)$이므로

$f(1)=f(2)=f(3)=f(4)=1$에서 $f(\sqrt{k})=1$

(b) $k\ne 1,\:4,\:9,\:16$일 때

$f(\sqrt{k})=3$

즉, $1\le k\le 20$인 정수 $k$에 대하여 $f(\sqrt{k})=\begin{cases}
1&(k=1,\:4,\: 9,\: 16)\\
3&(k\ne 1,\:4,\: 9,\: 16)
\end{cases}$이다.

[[$k=1,\:4,\:9,\:16$일 때와 $k \neq 1,\:4,\:9,\:16$일 때로 나누어 합을 계산한다.]]

따라서

$\displaystyle\sum_{k=1}^{20}k=210$이고, $1+4+9+16=30$이므로

$\displaystyle\sum_{k=1}^{20}\dfrac{k\times f(\sqrt{k})}{3}$

$=\left( 1+4+9+16 \right) \times \dfrac{1}{3}+\left(\displaystyle\sum_{k=1}^{20} k-(1+4+9+16) \right) \times \dfrac{3}{3}$ 

$=10 +(210-30)$

$=190$

<<<P>>>

실수 전체의 집합에서 정의된 함수 $f(x)$가 구간 $(0,\:1]$에서

$f(x)=\begin{cases}
3&(0< x< 1)\\
1&(x=1)
\end{cases}$

이고, 모든 실수 $x$에 대하여 $f(x+1)=f(x)$를 만족시킨다.

$\displaystyle\sum_{k=1}^{20}\dfrac{k\times f(\sqrt{k})}{3}$의 값은? [4점]

<<<C>>>① $150$ ② $160$ ③ $170$ ④ $180$ ⑤ $190$

<<<A>>>③

<<<S>>>

[[함수 $f(x)$의 증가와 감소 그리고 극값을 조사한다.]]

$f'(x)$$=3x^{2}-6x-9$$=3(x+1)(x-3)$

$f'(x)=0$에서 $x=-1$ 또는 $x=3$

함수 $f(x)$의 증가와 감소를 표로 나타내면 다음과 같다.

[IMG]

함수 $f(x)$는 $x=-1$에서 극댓값 $f(-1)=-7$을 갖고, $x=3$에서 극솟값 $f(3)=-39$를 갖는다.

[[$x>0$일 때와 $x<0$일 때로 경우를 나누어 $g(x)$를 조사한다.]]

조건 (가)에서 $xg(x)= | xf(x-p)+qx |$이므로

$g(x)=\begin{cases}
| f(x-p)+q |&(x>0)\\
g(0)&(x=0)\\
- | f(x-p)+q |&(x< 0)
\end{cases}$

[[함수 $g(x)$가 $x=0$에서 연속인 조건에서 $p,q$의 관계식을 구한다.]]

함수 $g(x)$가 $x=0$에서 연속이므로 $| f(-p)+q | =- | f(-p)+q |$

즉, $| f(-p)+q | =0$이어야 한다.

[[조건 (나)를 만족하는  $p,q$의 값을 구한다.]]

한편, 함수 $y= | f(x-p)+q |$의 그래프는 함수 $y=f(x)$의 그래프를 $x$축의 방향으로 $p$만큼, $y$축의 방향으로 $q$만큼 평행이동시킨 후, $y< 0$인 부분에 그려진 부분을 $x$축에 대하여 대칭이동시킨 것이다.

이때, $p,\:q$가 모두 양수이고 조건 (나)에서 함수 $g(x)$가 $x=a$에서 미분가능하지 않은 실수 $a$의 개수가 $1$이므로 $p=1,\:q=7$이어야 한다.

따라서 $p+q=1+7=8$

[IMG]

<<<P>>>

두 양수 $p,\: q$와 함수 $f(x)=x^{3}-3x^{2}-9x-12$에 대하여 실수 전체의 집합에서 연속인 함수 $g(x)$가 다음 조건을 만족시킬 때, $p+q$의 값은? [4점]

(가) 모든 실수 $x$에 대하여 $xg(x)= | xf(x-p)+qx |$이다.

(나) 함수 $g(x)$가 $x=a$에서 미분가능하지 않는 실수 $a$의 개수는 $1$이다.

<<<C>>>① $6$ ② $7$ ③ $8$ ④ $9$ ⑤ $10$

<<<A>>>②

<<<S>>>

방정식 $\left(\sin\dfrac{\pi x}{2}-t\right)\left(\cos\dfrac{\pi x}{2}-t\right)=0$에서
$\sin\dfrac{\pi x}{2}=t$ 또는 $\cos\dfrac{\pi x}{2}=t$

이 방정식의 실근은 두 함수 $y=\sin\dfrac{\pi x}{2},\:y=\cos\dfrac{\pi x}{2}$의 그래프와 $y=t$와의 교점의 $x$좌표이다.

한편, 두 함수 $y=\sin\dfrac{\pi x}{2},\:y=\cos\dfrac{\pi x}{2}$의 주기가 모두 $4$이므로 다음과 같다.

[IMG]

ㄱ. [[두 함수 $y=\sin\dfrac{\pi x}{2},\:y=\cos\dfrac{\pi x}{2}$의 대칭성을 이용합니다.]]

$-1\le t< 0$일 때, $\alpha(t),\:\beta(t)$는 다음 그림과 같이 정해진다.

[IMG]

두 함수 $y=\sin\dfrac{\pi x}{2},\:y=\cos\dfrac{\pi x}{2}$의 그래프는 직선 $x=\dfrac{5}{2}$에 대하여 대칭이므로 $\dfrac{\alpha(t)+\beta(t)}{2}=\dfrac{5}{2}$

따라서 $\alpha(t)+\beta(t)=5$ (참)

ㄴ. [[$t$의 값이 변함에 따라 $\beta(t)-\alpha(t)$의 값을 조사한다.]]

실근 $\alpha(t),\:\beta(t)$는 집합 $\{x\vert  0\le x< 4\}$의 원소이므로 $\beta(0)=3,\:\alpha(0)=0$

그러므로 주어진 식은 $\{t\vert \beta(t)-\alpha(t)=\beta(0)-\alpha(0)\}=\{t\vert \beta(t)-\alpha(t)=3\}$

(a) $-1\le t< 0$일 때,

$1 \leq \beta(t)-\alpha(t)< 3$

(b) $0\le t\le\dfrac{\sqrt{2}}{2}$일 때,

$\beta(t)-\alpha(t)=3$

(c) $\dfrac{\sqrt{2}}{2}< t< 1$일 때,

$3< \beta(t)-\alpha(t) <4 $

(d) $t=1$일 때,

$\alpha(1)=0,\:\beta(1)=1$이므로 $\beta(1)-\alpha(1)=1$

따라서 (a), (b), (c), (d)에서

$\left\{t \left| \beta(t)-\alpha(t)=3 \right.\right\}=\left\{t \left| 0\le t\le\dfrac{\sqrt{2}}{2} \right.\right\}$ (참)

ㄷ. [[ $\alpha\left(t_{1}\right)=\alpha\left(t_{2}\right)=\theta$라 놓고 $t_{1}, t_{2}$를 삼각함수로 나타낸다.]]

$t_{1} < t_{2}$일 때 $\alpha\left(t_{1}\right)=\alpha\left(t_{2}\right)$이기 위해서는 $0< t_{1}< \dfrac{\sqrt{2}}{2}< t_{2}$

이때, $\alpha\left(t_{1}\right)=\alpha\left(t_{2}\right)=\theta$라 하면 $t_{1}=\sin\theta ,\:t_{2}=\cos \theta$

이때, $\cos\theta -\sin\theta =\dfrac{1}{2}$

양변을 제곱하면 $\cos^{2}\theta +\sin^{2}\theta -2\sin\theta \cos\theta =\dfrac{1}{4}$

정리하면 $\sin\theta\cos\theta =\dfrac{3}{8}$

따라서 $t_{1}\times t_{2}=\sin\theta\cos\theta =\dfrac{3}{8}$  (거짓)




[다른 풀이]

방정식 $\left(\sin\dfrac{\pi x}{2}-t\right)\left(\cos\dfrac{\pi x}{2}-t\right)=0$에서 $\sin\dfrac{\pi x}{2}=t$ 또는 $\cos\dfrac{\pi x}{2}=t$

이 방정식의 실근은 두 함수 $y=\sin\dfrac{\pi x}{2},\:y=\cos\dfrac{\pi x}{2}$의 그래프와 $y=t$와의 교점의 $x$좌표이다.

한편, 두 함수 $y=\sin\dfrac{\pi x}{2},\:y=\cos\dfrac{\pi x}{2}$의 주기가 모두 $4$이므로 다음과 같다.

[IMG]

ㄱ.  [[두 함수 $y=\sin\dfrac{\pi x}{2},\:y=\cos\dfrac{\pi x}{2}$의 대칭성을 이용합니다.]]

$-1\le t< 0$이면 직선 $y=t$와 $\alpha(t),\:\beta(t)$는 다음 그림과 같다.

[IMG]

이때, 함수 $y=\cos\dfrac{\pi x}{2}$의 그래프는 함수 $y=\sin\dfrac{\pi x}{2}$의 그래프를 평행이동시키면 겹쳐질 수 있고 함수 $y=\sin\dfrac{\pi x}{2}$의 그래프는 직선 $x=1,\:x=3$에 대하여 대칭이고 점 $(2,\:0)$에 대하여 대칭이다.

그러므로 $\alpha(t)=1+k(0< k\le 1)$로 놓으면 $\beta(t)=4-k$

따라서 $\alpha(t)+\beta(t)=5$ (참)

ㄴ. [[$t$의 값이 변함에 따라 $\beta(t)-\alpha(t)$의 값을 조사한다.]]

실근 $\alpha(t),\:\beta(t)$는 집합 $\{x\vert  0\le x< 4\}$의 원소이므로 $\beta(0)=3,\:\alpha(0)=0$

그러므로 주어진 식은 $\{t\vert \beta(t)-\alpha(t)=\beta(0)-\alpha(0)\}=\{t\vert \beta(t)-\alpha(t)=3\}$

(a) $0\le t\le\dfrac{\sqrt{2}}{2}$일 때,

$t=0$이면 $\beta(0)-\alpha(0)=3-0=3$

$t\ne 0$이면 다음 그림과 같다.

[IMG]

이때, $\alpha(t)=k\left(0< k\le\dfrac{1}{2}\right)$이라 하면 $\beta(t)=3+k$

그러므로 $\beta(t)-\alpha(t)=3$

(b) $\dfrac{\sqrt{2}}{2}< t< 1$일 때,

[IMG]

이때, $\alpha(t)=k\left(0< k< \dfrac{1}{2}\right)$이라 하면 $\beta(t)=4-k$

그러므로 $\beta(t)-\alpha(t)=4-2k(0< 2k< 1)$

(c) $t=1$일 때,

$\alpha(1)=0,\:\beta(1)=1$이므로 $\beta(1)-\alpha(1)=1$

(d) $-1\le t< 0$일 때,

[IMG]

$1< \alpha(t)\le 2,\:3\le\beta(t)< 4$이므로 $\beta(t)-\alpha(t)< 3$

따라서 (a), (b), (c), (d)에서

$\{t \left| \beta(t)-\alpha(t)=3 \right. \}=\left\{t \left| 0\le t\le\dfrac{\sqrt{2}}{2} \right. \right\}$ (참)

ㄷ. [[ $\alpha\left(t_{1}\right)=\alpha\left(t_{2}\right)=\theta$라 놓고 $t_{1}, t_{2}$를 삼각함수로 나타낸다.]]

$\alpha\left(t_{1}\right)=\alpha\left(t_{2}\right)$이기 위해서는 $0< t_{1}< \dfrac{\sqrt{2}}{2}< t_{2}$

이때, $\alpha\left(t_{1}\right)=\alpha\left(t_{2}\right)=\theta$라 하면 $t_{1}=\sin\theta ,\:t_{2}=\cos\theta$

이때, $\cos\theta -\sin\theta =\dfrac{1}{2}$

양변을 제곱하면 $\cos^{2}\theta +\sin^{2}\theta -2\sin\theta\cos\theta =\dfrac{1}{4}$

정리하면 $\sin\theta\cos\theta =\dfrac{3}{8}$

따라서 $t_{1}\times t_{2}=\sin\theta\cos\theta =\dfrac{3}{8}$  (거짓)

<<<P>>>

$-1\le t\le 1$인 실수 $t$에 대하여 $x$에 대한 방정식

$\left(\sin\dfrac{\pi x}{2}-t\right)\left(\cos\dfrac{\pi x}{2}-t\right)=0$

의 실근 중에서 집합 $\{x | 0\le x < 4\}$에 속하는 가장 작은 값을 $\alpha(t)$, 가장 큰 값을 $\beta(t)$라 하자. < 보기>에서 옳은 것만을 있는 대로 고른 것은? [4점]

< 보 기>

ㄱ. $-1\le t < 0$인 모든 실수 $t$에 대하여 $\alpha(t)+\beta(t)=5$이다.

ㄴ. $\{t |\beta(t)-\alpha(t)=\beta(0)-\alpha(0)\}=\left\{ t \left | 0\le t \le \dfrac{\sqrt{2}}{2} \right. \right\}$

ㄷ. $\alpha\left(t_{1}\right)=\alpha\left(t_{2}\right)$인 두 실수 $t_{1}$, $t_{2}$에 대하여
$t_{2}-t_{1}=\dfrac{1}{2}$이면 $t_{1}\times t_{2}=\dfrac{1}{3}$이다.

<<<C>>>① ㄱ           ② ㄱ, ㄴ        ③ ㄱ, ㄷ ④ ㄴ, ㄷ ⑤ ㄱ, ㄴ, ㄷ

<<<A>>>$2$

<<<S>>>

[[로그의 성질을 이용하여 계산합니다.]]

$\log_{4}\dfrac{2}{3}+\log_{4}24$

$=\log_{4}\left(\dfrac{2}{3}\times 24\right)$

$=\log_{4}16$

$=\log_{4}4^{2}=2$

<<<P>>>

$\log_{4}\dfrac{2}{3}+\log_{4}24$의 값을 구하시오. [3점]

<<<A>>>$11$

<<<S>>>

[[$f'(x)$의 부호를 조사하여 $f(x)$의 증가와 감소를 표로 나타냅니다.]]

$f'(x)=3x^{2}-3=3(x+1)(x-1)$

$f'(x)=0$에서 $x=-1$ 또는 $x=1$

함수 $f(x)$의 증가와 감소를 표로 나타내면 다음과 같다.

[IMG]

[[증감표에서 극소가 되는 $x$값을 알아냅니다.]]

함수 $f(x)$는 $x=1$에서 극소이므로 $a=1$

$f(a)=f(1)=1^{3}-3\times 1+12=10$이므로

$a+f(a)=1+f(1)=1+10=11$

<<<P>>>

함수 $f(x)=x^{3}-3x+12$가 $x=a$에서 극소일 때, $a+f(a)$의 값을 구하시오. (단, $a$는 상수이다.) [3점]

<<<A>>>$4$

<<<S>>>

[[공비를 미지수로 도입합니다.]]

등비수열 $\left\{a_{n}\right\}$의 공비를 $r$이라고 하면 

$a_{2}=36$이고 $a_{7}=\dfrac{1}{3}a_{5}$에서 $36r^{5}=\dfrac{1}{3}\times 36r^{3}$이다. 

그러므로 $r^{2}=\dfrac{1}{3}$

$a_{6}=a_{2}r^{4}=36\times\dfrac{1}{9}=4$

[다른 풀이]

[[첫째항과 공비를 미지수로 도입합니다.]]

등비수열 $\left\{a_{n}\right\}$의 공비를 $r$, $a_{1}=a$라 하자.

[[주어진 조건으로 연립방정식을 세웁니다.]]

$a_{2}=36$에서 $ar=36\cdots\cdots$㉠

또, $a_{7}=\dfrac{1}{3}a_{5}$에서 $ar^{6}=\dfrac{1}{3}ar^{4}$

$r^{2}=\dfrac{1}{3}\cdots\cdots$㉡

[[앞에서 나온 결과로부터 문제에서 요구하는 것을 구합니다.]]

따라서 ㉠과 ㉡에서

$a_{6}$$=ar^{5}$

$=ar\times r^{4}$

$=36\times\left(\dfrac{1}{3}\right)^{2}$

$=4$

<<<P>>>

모든 항이 양수인 등비수열 $\left\{a_{n}\right\}$에 대하여
$a_{2}= 36 ,\: a_{7}=\dfrac{1}{3}a_{5}$
일 때, $a_{6}$의 값을 구하시오. [3점]

<<<A>>>$6$

<<<S>>>

[[주어진 조건을 이용하여 위치를 구합니다.]]

시각 $t$에서 점 $\mathrm{P}$의 위치를 $x(t)$라 하면 시각 $t=0$에서 점 $\mathrm{P}$의 위치가 $0$이므로

$v(t)=3t^{2}-4t+k$에서 $x(t)=t^{3}-2t^{2}+kt$

이때 $x(1)= -3$에서 $-1+k = -3$, $k= -2$

따라서 $x(t)=t^{3}-2t^{2}-2t$이고, $x(3)=27-18-6=3$이다.

[[앞에서 나온 결과로부터 문제에서 요구하는 것을 구합니다.]]

그러므로 시각 $t=1$에서 $t=3$까지 점 $\mathrm{P}$의 위치의 변화량은

$x(3)-x(1)=3-(-3)=6$

[다른 풀이]

[[주어진 조건을 이용하여 위치의 변화량을 정적분으로 나타냅니다.]]

$x(1)=x(0)+\displaystyle\int_{0}^{1}v(t)dt$에서 $-3=0+\displaystyle\int_{0}^{1}(3t^{2}-4t+k)dt$

정리하면 $-3=\left[t^{3}-2t^{2}+k\right]_{0}^{1}=1-2+k$

따라서 $k=-2$

[[위치의 변화량을 구합니다.]]

$t=1$에서 $t=3$까지 점 $\mathrm{P}$의 위치의 변화량은

$x(3)-x(1)=\displaystyle\int_{1}^{3}v(t)dt=\displaystyle\int_{1}^{3}(3t^{2}-4t-2)dt=\left[t^{3}-2t^{2}-2t\right]_{1}^{3}=6$

<<<P>>>

수직선 위를 움직이는 점 $\mathrm{P}$의 시각 $t(t\ge 0)$에서의 속도 $v(t)$가

$v(t)=3t^{2}-4t+k$

이다. 시각 $t=0$에서 점 $\mathrm{P}$의 위치는 $0$이고, 시각 $t=1$에서 점 $\mathrm{P}$의 위치는 $-3$이다. 시각 $t=1$에서 $t=3$까지 점 $\mathrm{P}$의 위치의 변화량을 구하시오. (단, $k$는 상수이다.) [3점]

<<<A>>>$8$

<<<S>>>

[[정적분함수를 미분합니다.]]

$f'(x)$ $=3x^{2}-24x+45$ $=3(x-3)(x-5)$이고

$g(x)$ $=\displaystyle\int_{a}^{x}\{f(x)-f(t)\}\times\{f(t)\}^{4}dt$
$=f(x)\displaystyle\int_{a}^{x}\{f(t)\}^{4}dt -\displaystyle\int_{a}^{x}\{f(t)\}^{5}dt$

$g'(x)$ $=f'(x)\displaystyle\int_{a}^{x}\{f(t)\}^{4}dt +\{f(x)\}^{5}-\{f(x)\}^{5}$
$=f'(x)\displaystyle\int_{a}^{x}\{f(t)\}^{4}dt$

$=3(x-3)(x-5)\displaystyle\int_{a}^{x}\{f(t)\}^{4}dt$

[[주어진 함수 $g(x)$가 오직 하나의 극값을 가질 조건을 조사합니다.]]

$h(x)=\displaystyle\int_{a}^{x}\{f(t)\}^{4}dt$라 하면 $h(a)=0$이고 $h'(x)=\{f(x)\}^{4}\ge 0$이다.

즉, $h(x)$는 증가함수이므로 방정식 $h(x)=0$의 오직 한 개의 실근은 $x=a$를 갖는다.

함수 $g(x)$가 오직 하나의 극값을 가지려면 방정식 $g'(x)=0$은 중근이 아닌 근이 하나만 존재해야 한다.

그러므로 $a=3$ 또는 $a=5$이어야 한다.

따라서 모든 $a$의 값의 합은 $3+5 =8$

[다른 풀이]

[[정적분함수를 미분합니다.]]

$f(x)=x^{3}-12x^{2}+45x+3$에서

$f'(x)$ $=3x^{2}-24x+45$$=3(x-3)(x-5)$

$g(x)$ $=\displaystyle\int_{a}^{x}\{f(x)-f(t)\}\times\{f(t)\}^{4}dt$

$=f(x)\displaystyle\int_{a}^{x}\{f(t)\}^{4}dt -\displaystyle\int_{a}^{x}\{f(t)\}^{5}dt$

$g'(x)$ $=f'(x)\displaystyle\int_{a}^{x}\{f(t)\}^{4}dt +\{f(x)\}^{5}-\{f(x)\}^{5}$

$=f'(x)\displaystyle\int_{a}^{x}\{f(t)\}^{4}dt$

$g'(x)=0$에서 $f'(x)=0$ 또는 $x=a$

[[주어진 함수 $g(x)$가 오직 하나의 극값을 가질 조건을 조사합니다.]]

(a) $a\ne 3$, $a\ne 5$일 때, 

$g'(x)=0$에서 $x=3$ 또는 $x=5$ 또는 $x=a$

함수 $g(x)$는 $x=3$, $x=5$, $x=a$에서 모두 극값을 갖는다.

(b) $a=3$일 때

$g'(x)=0$에서 $x=3$ 또는 $x=5$

함수 $g(x)$의 증가와 감소를 표로 나타내면 다음과 같다.

[IMG]

함수 $g(x)$는 $x=5$에서만 극값을 갖는다.

(c) $a=5$일 때

$g'(x)=0$에서 $x=3$ 또는 $x=5$

함수 $g(x)$의 증가와 감소를 표로 나타내면 다음과 같다.

[IMG]

함수 $g(x)$는 $x=3$에서만 극값을 갖는다.

(a), (b), (c)에서 함수 $g(x)$가 오직 하나의 극값을 갖도록 하는 $a$의 값은 $3$ 또는 $5$이다.

따라서 모든 $a$의 값의 합은 $3+5=8$

<<<P>>>

실수 $a$와 함수 $f(x)=x^{3}-12x^{2}+45x+3$에 대하여 함수

$g(x)=\displaystyle\int_{a}^{x}\{f(x)-f(t)\}\times\{f(t)\}^{4}dt$

가 오직 하나의 극값을 갖도록 하는 모든 $a$의 값의 합을 구하시오. [4점]

<<<A>>>$24$

<<<S>>>

[[조건 (가)를 이용하여 $f(x)$의 식을 세웁니다.]]

방정식 $x^{n}-64=0$은 자연수 $n$이 홀수이면 한 개의 실근을, 짝수이면 두 개의 실근을 갖는다.

조건 (가)로부터 자연수 $n$은 짝수이고, 방정식 $f(x)=0$의 서로 다른 두 실근은 방정식 $x^{n}-64=0$의 두 실근 $\pm 2^{\frac{6}{n}}$과 각각 같아야 한다.

즉, $f(x)=\left(x-2^{\frac{6}{n}}\right)\left(x+2^{\frac{6}{n}}\right)=x^{2}-2^{\frac{12}{n}}$

[[조건 (나)를 이용하여 구하려는 값을 구합니다.]]

조건 (나)에서 $f(x)$의 최솟값 $-2^{\frac{12}{n}}$이 음의 정수이므로 짝수인 자연수 $n$은 $12$의 약수이어야 한다. 따라서 $n=2$, $4$, $6$, $12$이고, 그 합은 $2+4+6+12 = 24$이다.

[다른 풀이]

[[조건 (가)를 이용하여 $f(x)$의 식을 세웁니다.]]

함수 $f(x)$는 최고차항의 계수가 $1$이고 최솟값이 음수이므로 방정식 $f(x)=0$은 서로 다른 두 실근을 갖는다.

(a) $n$이 홀수일 때,

방정식 $x^{n}=64$의 실근의 개수는 $1$이다.

그러므로 방정식 $(x^{n}-64)f(x)=0$의 근이 모두 중근일 수 없다.

(b) $n$이 짝수일 때,

방정식 $x^{n}=64$의 실근은 $x=\sqrt[n]{64}$ 또는 $x= -\sqrt[n]{64}$

즉, $x=2^{\frac{6}{n}}$ 또는 $x= -2^{\frac{6}{n}}$

이때, 조건 (가)를 만족하기 위해서는
$f(x)=\left(x-2^{\frac{6}{n}}\right)\left(x+2^{\frac{6}{n}}\right)$ $\cdots\cdots$㉠

[[조건 (나)를 이용하여 구하려는 값을 구합니다.]]

한편, 조건 (나)에서 함수 $f(x)$의 최솟값은 음의 정수이다.

㉠에서 함수 $f(x)$는 $x=0$에서 최솟값을 갖고 그 값은 $-2^{\frac{6}{n}}\times 2^{\frac{6}{n}}= -2^{\frac{12}{n}}$

이 값이 음의 정수이기 위해서는 $n$의 값은 $2$, $4$, $6$, $12$  

따라서 모든 $n$의 값의 합은 $2+4+6+12 = 24$이다.

<<<P>>>

다음 조건을 만족시키는 최고차항의 계수가 $1$인 이차함수 $f(x)$가 존재하도록 하는 모든 자연수 $n$의 값의 합을 구하시오. [4점]

(가) $x$에 대한 방정식 $(x^{n}-64)f(x)=0$은 

서로 다른 두 실근을 갖고, 각각의 실근은 중근이다. 

(나) 함수 $f(x)$의 최솟값은 음의 정수이다.

<<<A>>>$61$

<<<S>>>

[[조건 (가)를 만족하는 함수를 식으로 나타냅니다.]]

조건 (가)에서 방정식 $f(x)=0$의 서로 다른 두 실근을 $\alpha$, $\beta$라 하면

$f(x)=a(x-\alpha)^{2}(x-\beta)$로 놓을 수 있다.

[[앞에서 정해진 함수가 조건 (나)를 만족하는 경우를 조사합니다.]]

조건 (나)에서 $x-f(x)=\alpha$ 또는 $x-f(x)=\beta$를 만족시키는 서로 다른 $x$의 값의 개수가 $3$이어야 한다.

즉, $f(x)=x-\alpha$ 또는 $f(x)=x-\beta$에서 곡선 $y=f(x)$와 두 직선 $y=x-\alpha$, $y=x-\beta$가 만나는 서로 다른 점의 개수가 $3$이어야 한다.

그런데 $a>0$인 경우는 다음 그림과 같이 만나는 서로 다른 점의 개수가 $5$개 이상이므로 $a<0$이어야 한다.

[IMG] [IMG] [IMG] [IMG]

$a<0$인 경우는 다음 두 가지 가능성이 있다.

[IMG] [IMG]

두 번째 그림의 경우는 점 $(1,4)$가 $x< \alpha$ 범위에 있고, 이 때 $f'(1)<0$이므로 성립하지 않는다.

따라서 구하는 경우는 첫 번째 그림이다.

[[곡선 $y=f(x)$와 직선 $y=x-\alpha$의 접점의 $x$좌표를 구합니다.]]

[IMG]

첫번째 그림에서 접선의 기울기가 $1$인 점이 두 개 있는데,

만약 그림에서 점 $\mathrm{P}$의 $x$좌표가 $1$이라 하면 $f'(0)>1$에 모순이므로

곡선 $y=f(x)$와 직선 $y=x-\alpha$의 접점의 $x$좌표가 $1$이다.

[[앞의 결과를 이용하여 미지수의 값을 구합니다.]]

따라서  $f(x)-(x-\alpha)=a(x-1)^2 (x-\alpha)$로 나타낼 수 있으므로

$f(x)=a(x-1)^2 (x-\alpha)+(x-\alpha)$이다.

$f(1)=1-\alpha =4$에서 $\alpha=-3$

$f(x)=a(x-1)^2 (x+3)+(x+3)$이고, $f'(x)=a(x-1)^{2}+2a(x+3)\times (x-1)+1$

이때, $f'(-3)=0$이므로 $a\times 16 +1=0$에서 $a= -\dfrac{1}{16}$

이때, $f(x)= -\dfrac{1}{16}(x+3)(x-1)^{2}+x+3$이므로 $f(0)= -\dfrac{1}{16}\times 3\times 1 +3 =\dfrac{45}{16}$

따라서 구하는 $p+q$의 값은 $p +q =16 +45 =61$

<<<P>>>

삼차함수 $f(x)$가 다음 조건을 만족시킨다.

(가) 방정식 $f(x)=0$의 서로 다른 실근의 개수는 $2$이다. 

(나) 방정식 $f(x-f(x))=0$의 서로 다른 실근의 개수는 $3$이다.

$f(1)=4$, $f'(1)=1$, $f'(0)>1$일 때, $f(0)=\dfrac{q}{p}$이다. $p+q$의 값을 구하시오. (단, $p$와 $q$는 서로소인 자연수이다.) [4점]

확률과 통계

<<<A>>>④

<<<S>>>

$(2x+1)^{5}$의 전개식의 일반항은

${}_{5}\mathrm{C}_{r}(2x)^{5-r}(1)^{r}={}_{5}\mathrm{C}_{r}\times 2^{5-r}\times x^{5-r}$ $(r= 0,\:1,\:2,\:\cdots ,\:5)$

$x^{3}$항은 $5-r = 3$, 즉 $r=2$일 때이므로 $x^{3}$의 계수는

${}_{5}\mathrm{C}_{2}\times 2^{5-2}= 10\times 8 = 80$

<<<P>>>

다항식 $(2x+1)^{5}$의 전개식에서 $x^{3}$의 계수는? [2점]

<<<C>>>① $20$ ② $40$ ③ $60$ ④ $80$ ⑤ $100$

<<<A>>>②

<<<S>>>

이 조사에 참여한 학생 $20$명 중에서 임의로 선택한 한 명이 진로활동 $\mathrm{B}$를 선택한 학생이라는 것은, 진로활동 $\mathrm{B}$를 선택한 학생 중에서 임의로 한 명을 선택하는 것과 같다.

진로활동 $\mathrm{B}$를 선택한 학생 $9$명 중에 $1$학년 학생이 $5$명 있으므로 구하는 확률은 $\dfrac{5}{9}$이다.

[다른 풀이]

이 조사에 참여한 학생 중에서 한 명을 선택하는 경우의 수는 $20$

이 조사에 참여한 학생 중에서 임의로 선택한 한 명이 진로활동 $\mathrm{B}$를 선택한 학생인 사건을 $B$, $1$학년 학생인 사건을 $E$라 하면 구하는 확률은 $\mathrm{P}(E | B)$이다.

이때 $\mathrm{P}(B)=\dfrac{9}{20}$이고, 사건 $E\cap B$는 진로활동 $\mathrm{B}$를 선택한 $1$학년 학생을 선택하는 사건이므로

$\mathrm{P}(E\cap B)=\dfrac{5}{20}$

따라서 구하는 확률은

$\mathrm{P}(E | B)$$=\dfrac{\mathrm{P}(E\cap B)}{\mathrm{P}(B)}$

$=\dfrac{\dfrac{5}{20}}{\dfrac{9}{20}}=\dfrac{5}{9}$

<<<P>>>

어느 동아리의 학생 $20$명을 대상으로 진로활동 $\mathrm{A}$와 진로활동 $\mathrm{B}$에 대한 선호도를 조사하였다. 이 조사에 참여한 학생은 진로활동 $\mathrm{A}$와 진로활동 $\mathrm{B}$ 중 하나를 선택하였고, 각각의 진로활동을 선택한 학생 수는 다음과 같다.

[IMG]

이 조사에 참여한 학생 $20$명 중에서 임의로 선택한 한 명이 진로활동 $\mathrm{B}$를 선택한 학생일 때, 이 학생이 $1$학년일 확률은? [3점]

<<<C>>>① $\dfrac{1}{2}$ ② $\dfrac{5}{9}$ ③ $\dfrac{3}{5}$ ④ $\dfrac{7}{11}$ ⑤ $\dfrac{2}{3}$

<<<A>>>③

<<<S>>>

숫자 $1,\:2,\:3,\:4,\:5$ 중에서 중복을 허락하여 $4$개를 택해 일렬로 나열하여 만들 수 있는 모든 네 자리의 자연수의 개수는 ${}_{5}\Pi_{4}= 5^{4}$

이 중에서 $3500$보다 큰 경우는 다음과 같다.

(ⅰ) 천의 자리가 숫자가 $3$, 백의 자리의 숫자가 $5$인 경우
십의 자리의 숫자와 일의 자리의 숫자를 택하는 경우의 수는 ${}_{5}\Pi_{2}= 5^{2}$

(ⅱ) 천의 자리의 숫자가 $4$ 또는 $5$인 경우
천의 자리의 숫자를 택하는 경우의 수는 $2$
이 각각에 대하여 나머지 세 자리의 숫자를 택하는 경우의 수는 ${}_{5}\Pi_{3}= 5^{3}$이므로 이 경우의 수는 $2\times 5^{3}$

(ⅰ), (ⅱ)에 의하여 $3500$보다 큰 자연수의 개수는 $5^{2}+ 2\times 5^{3}$

따라서 구하는 확률은 $\dfrac{5^{2}+2\times 5^{3}}{5^{4}}=\dfrac{11}{25}$

<<<P>>>

숫자 $1,\:2,\:3,\:4,\:5$ 중에서 중복을 허락하여 $4$개를 택해 일렬로 나열하여 만들 수 있는 모든 네 자리의 자연수 중에서 임의로 하나의 수를 선택할 때, 선택한 수가 $3500$보다 클 확률은? [3점]

<<<C>>>① $\dfrac{9}{25}$ ② $\dfrac{2}{5}$ ③ $\dfrac{11}{25}$ ④ $\dfrac{12}{25}$ ⑤ $\dfrac{13}{25}$

<<<A>>>③

<<<S>>>

(ⅰ) $3$가지 색의 카드를 각각 한 장 이상 받는 학생에게는 노란색 카드 $1$장을 반드시 주어야 한다.

노란색 카드 $1$장을 받을 학생을 선택하는 경우의 수는 ${}_{3}\mathrm{C}_{1}= 3$

(ⅱ) (ⅰ)의 각 경우에 대하여 노란색 카드 $1$장을 받은 학생에게 파란색 카드 $1$장을 먼저 준 후 나머지 파란색 카드 $1$장을 줄 학생을 선택하는 경우의 수는 ${}_{3}\mathrm{C}_{1}=3$

(ⅲ) (ⅰ), (ⅱ)의 각 경우에 대하여 노란색 카드 $1$장을 받은 학생에게 빨간색 카드 $1$장도 먼저 준 후 나머지 빨간색 카드 $3$장을 나누어 줄 학생을 선택하는 경우의 수는 

${}_{3}\mathrm{H}_{3}={}_{3+3-1}\mathrm{C}_{3}={}_{5}\mathrm{C}_{3}={}_{5}\mathrm{C}_{2}=10$

따라서 구하는 경우의 수는 $3\times 3\times 10 =90$

<<<P>>>

빨간색 카드 $4$장, 파란색 카드 $2$장, 노란색 카드 $1$장이 있다. 이 $7$장의 카드를 세 명의 학생에게 남김없이 나누어 줄 때, $3$가지 색의 카드를 각각 한 장 이상 받는 학생이 있도록 나누어 주는 경우의 수는? (단, 같은 색 카드끼리는 서로 구별하지 않고, 카드를 받지 못하는 학생이 있을 수 있다.) [3점]

<<<C>>>① $78$ ② $84$ ③ $90$ ④ $96$ ⑤ $102$

<<<A>>>①

<<<S>>>

주사위 $2$개와 동전 $4$개를 동시에 던질 때 나오는 모든 경우의 수는

$6^{2}\times 2^{4}$

(ⅰ) 앞면이 나온 동전의 개수가 $1$인 경우의 수는 ${}_{4}\mathrm{C}_{1}=4$

이때 두 주사위에서 나온 눈의 수가 $(1,\:1)$이어야 하므로 이 경우의 수는 $4\times 1=4$

(ⅱ) 앞면이 나온 동전의 개수가 $2$인 경우의 수는 ${}_{4}\mathrm{C}_{2}=\dfrac{4\times 3}{2}=6$

이때 두 주사위에서 나온 눈의 수가 $(1,\:2)$ 또는 $(2,\:1)$이어야 하므로 이 경우의 수는 $6\times 2=12$

(ⅲ) 앞면이 나온 동전의 개수가 $3$인 경우의 수는 ${}_{4}\mathrm{C}_{3}={}_{4}\mathrm{C}_{1}=4$

이때 두 주사위에서 나온 눈의 수가 $(1,\:3)$ 또는 $(3,\:1)$이어야 하므로 이 경우의 수는 $4\times 2=8$

(ⅳ) 앞면이 나온 동전의 개수가 $4$인 경우의 수는 ${}_{4}\mathrm{C}_{4}=1$

이때 두 주사위에서 나온 눈의 수가 $(1,\:4)$ 또는 $(2,\:2)$ 또는 $(4,\:1)$이어야 하므로 이 경우의 수는 $1\times 3=3$

(ⅰ)~(ⅳ)에 의하여 조건을 만족시키는 경우의 수는 $4+12+8+3=27$

따라서 구하는 확률은

$\dfrac{27}{6^{2}\times 2^{4}}=\dfrac{3}{64}$

<<<P>>>

주사위 $2$개와 동전 $4$개를 동시에 던질 때, 나오는 주사위의 눈의 수의 곱과 앞면이 나오는 동전의 개수가 같을 확률은? [3점]

<<<C>>>① $\dfrac{3}{64}$ ② $\dfrac{5}{96}$ ③ $\dfrac{11}{192}$ ④ $\dfrac{1}{16}$ ⑤ $\dfrac{13}{192}$

<<<A>>>⑤

<<<S>>>

이 주사위를 네 번 던질 떄 나온 눈의 수가 $4$이상인 경우의 수에 따라 다음과 같이 나누어 생각할 수 있다.

(ⅰ) 나온 눈의 수가 $4$ 이상인 경우의 수가 $0$인 경우

$1$의 눈만 네 번 나와야 하므로 이 경우의 수는 $1$

(ⅱ) 나온 눈의 수가 $4$ 이상인 경우의 수가 $1$인 경우

$1$의 눈이 두 번, $2$의 눈이 한 번 나와야 하므로 점수 $0,\: 1,\: 1,\: 2$를 일렬로 나열하는 경우의 수는 $\dfrac{4!}{2!}=12$

이 각각에 대하여 $4$ 이상의 눈이 한 번 나오는 경우의 수는 $3$이므로 이 경우의 수는 $12\times 3=36$

(ⅲ) 나온 눈의 수가 $4$ 이상인 경우의 수가 $2$인 경우

㉠ $1$의 눈이 한 번, $3$의 눈이 한번 나올 때, 점수 $0,\: 0,\: 1,\: 3$을 일렬로 나열하는 경우의 수는 $\dfrac{4!}{2!}=12$

㉡ $2$의 눈이 두 번 나올 때, 점수 $0,\: 0,\: 2,\: 2$를 일렬로 나열하는 경우의 수는 $\dfrac{4!}{2!2!}=6$

㉠,㉡ 각각에 대하여 $4$ 이상의 눈이 두 번 나오는 경우의 수는 $3\times 3=9$이므로 이 경우의 수는

$(12+6)\times 9=162$

(ⅰ)~(ⅲ)에 의하여 구하는 경우의 수는 $1+36+162=199$

<<<P>>>

한 개의 주사위를 한 번 던져 나온 눈의 수가 $3$ 이하이면 나온 눈의 수를 점수로 얻고, 나온 눈의 수가 $4$ 이상이면 $0$점을 얻는다. 이 주사위를 네 번 던져 나온 눈의 수를 차례로 $a,\: b,\: c,\: d$라 할 때, 얻는 네 점수의 합이 $4$가 되는 모든 순서쌍 $(a,\: b,\: c,\: d)$의 개수는? [4점]

<<<C>>>① $187$ ② $190$ ③ $193$ ④ $196$ ⑤ $199$

<<<A>>>$48$

<<<S>>>

$6$개의 의자를 원형으로 배열하는 경우의 수는 $(6-1)! =5! = 120$

이때 서로 이웃한 $2$개의 의자에 적혀 있는 수의 곱이 $12$가 되는 경우가 있도록 배열하는 경우는 다음과 같이 생각할 수 있다.

(ⅰ) $2,\: 6$이 각각 적힌 두 의자가 이웃하게 배열되는 경우

$2,\: 6$이 각각 적힌 두 의자를 $1$개로 생각하여 의자 $5$개를 원형으로 배열하는 경우의 수는 $(5-1)! =4! =24$

이 각각에 대하여 $2,\: 6$이 각각 적힌 두 의자의 자리를 서로 바꾸는 경우의 수는 $2! =2$ 

그러므로 $2,\: 6$이 각각 적힌 두 의자가 이웃하게 배열되는 경우의 수는 $24\times 2=48$

(ⅱ) $3,\: 4$가 각각 적힌 두 의자가 이웃하게 배열되는 경우

$3,\: 4$가 각각 적힌 두 의자를 $1$개로 생각하여 의자$5$개를 원형으로 배열하는 경우의 수는 $(5-1)! =4! =24$

이 각각에 대하여 $3,\: 4$가 각각 적힌 두 의자의 자리를 서로 바꾸는 경우의 수는 $2! =2$ 

그러므로 $3,\: 4$가 각각 적힌 두 의자가 이웃하게 배열되는 경우의 수는 $24\times 2=48$

(ⅲ) $2,\: 6$가 각각 적힌 두 의자가 이웃하게 배열되는 경우

$2,\: 6$이 각각 적힌 두 의자를 $1$개로 생각하고 $3,\: 4$가 각각 적흰 두 의자를 $1$개로 생각하여 의자 $4$개를 원형으로 배열하는 경우의 수는 $(4-1)! =3! =6$

이 각각에 대하여 $2,\: 6$이 각각 적힌 두 의자의 자리를 서로 바꾸고 $3,\: 4$가 각각 적힌 두 의자의 자리를 서로 바꾸는 경우의 수는 $2!\times 2! =4$

그러므로 $2,\: 6$가 각각 적힌 두 의자가 이웃하게 배열되는 경우의 수는 $6\times 4=24$

(ⅰ)~(ⅲ)에 의하여 서로 이웃한 $2$개의 의자에 적혀 있는 수의 곱이 $12$가 되는 경우가 있도록 배열하는 경우의 수는 $48+48-24=72$

따라서 구하는 경우의 수는 $120-72=48$

<<<P>>>

$1$부터 $6$까지의 자연수가 하나씩 적혀 있는 $6$개의 의자가 있다. 이 $6$개의 의자를 일정한 간격을 두고 원형으로 배열할 때, 서로 이웃한 $2$개의 의자에 적혀 있는 수의 곱이 $12$가 되지 않도록 배열하는 경우의 수를 구하시오. (단, 회전하여 일치하는 것은 같은 것으로 본다) [4점]

[IMG]

<<<A>>>$47$

<<<S>>>

$3$개의 공이 들어 있는 주머니에서 임의로 한 개의 공을 꺼내어 공에 적혀 있는 수를 확인한 후 다시 넣는 시행을 $5$번 반복할 때 나오는 모든 경우의 수는 $3^{5}$

이때 확인한 $5$개의 수의 곱이 $6$의 배수가 아닌 경우는 다음과 같다.

(ⅰ) 한 개의 숫자만 나오는 경우

이 경우의 수는 $3$

(ⅱ) 두 개의 숫자가 나오는 경우

$1,\: 2$가 적혀 있는 공이 나오는 경우의 수는

$2^{5}-2=30$

$1,\: 3$이 적혀 있는 공이 나오는 경우의 수는

$2^{5}-2=30$

그러므로 이 경우의 수는

$30+30=60$

(ⅰ),(ⅱ) 에 의하여 확인한 $5$개의 수의 곱이 $6$의 배수가 아닌 경우의 수는 $3+60=63$

따라서 구하는 확률은

$1-\dfrac{63}{3^{5}}=1-\dfrac{7}{27}=\dfrac{20}{27}$

이므로

$p+q=27+20=47$

[다른 풀이]

$5$개의 수의 곱이 $6$의 배수이려면 $2$와 $3$이 각각 적어도 한 번 나와야 한다.

전체 경우의 수는 $3^{5}=243$

$2$가 나오지 않는 경우의 수는 $2^{5}=32$

$3$가 나오지 않는 경우의 수는 $2^{5}=32$

$2$와 $3$이 나오지 않는 경우의 수는 $1^{5}=1$

그러므로 $5$번 중 $2$와 $3$이 각각 적어도 한 번이상 나오는 경우의 수는 

$243-32-32+1=180$

구하는 확률은 $\dfrac{180}{243}=\dfrac{20}{27}$이다.

따라서 $p+q=27+20=47$

<<<P>>>

숫자 $1,\: 2,\: 3$이 하나씩 적혀 있는 $3$개의 공이 들어 있는 주머니가 있다. 이 주머니에서 임의로 한 개의 공을 꺼내어 공에 적혀 있는 수를 확인한 후 다시 넣는 시행을 한다.

이 시행을 $5$번 반복하여 확인한 $5$개의 수의 곱이 $6$의 배수일 확률이 $\dfrac{q}{p}$일 때, $p+q$의 값을 구하시오. (단, $p$와 $q$는 서로소인 자연수이다)[4점]

[IMG]

미적분

<<<A>>>②

<<<S>>>

$\displaystyle\lim_{n\to \infty}\dfrac{1}{\sqrt{n^{2}+n+1}-n}$$=\displaystyle\lim_{n\to \infty}\dfrac{\sqrt{n^{2}+n+1}+n}{n+1}$

$=\displaystyle\lim_{n\to \infty}\dfrac{\sqrt{1 +\dfrac{1}{n}+\dfrac{1}{n^{2}}}+1}{1+\dfrac{1}{n}}$

$=\dfrac{1+1}{1}=2$

<<<P>>>

$\displaystyle\lim_{n\to \infty}$$\dfrac{1}{\sqrt{n^{2}+n-1}-n}$의 값은? [2점]

<<<C>>>① $1$ ② $2$ ③ $3$ ④ $4$ ⑤ $5$

<<<A>>>②

<<<S>>>

$\dfrac{dx}{dt}= e^{t}-\sin t ,\:\dfrac{dy}{dt}=\cos t$이므로 $\dfrac{dy}{dx}=\dfrac{\dfrac{dy}{dt}}{\dfrac{dx}{dt}}=\dfrac{\cos t}{e^{t}-\sin t}$

따라서 $t=0$일 때 $\dfrac{dy}{dx}$의 값은 $\dfrac{1}{1-0}=1$

<<<P>>>

매개변수 $t$로 나타내어진 곡선 

$x= e^{t}+\cos t,\:y=\sin t$

에서 $t=0$ 일 때, $\dfrac{dy}{dx}$의 값은? [3점]

<<<C>>>① $\dfrac{1}{2}$ ② $1$ ③ $\dfrac{3}{2}$ ④ $2$ ⑤ $\dfrac{5}{2}$

<<<A>>>④

<<<S>>>

곡선 $y=e^{| x |}$는 $y$축에 대하여 대칭이다. 

$x\ge 0$일 때 $y=e^{x}$이고 접점을 $(t,\: e^{t})$이라 하면 $y'=e^{x}$이므로 접선의 방정식은

$y-e^{t}=e^{t}(x-t)$

이 접선이 원점을 지나므로

$-e^{t}=e^{t}(-t),\: t=1$

따라서 접선의 기울기는 $e$이고 이 접선과 $y$축에 대하여 대칭인 접선의 기울기는 $-e$이다.

$\tan\theta =\dfrac{-e-e}{1+(-e)\times e}=\dfrac{-2e}{1-e^{2}}=\dfrac{2e}{e^{2}-1}$

[다른 풀이]

곡선 $y=e^{| x |}$는 $y$축에 대하여 대칭이고, 원점도 $y$축 위의 점이므로 두 접선도 $y$축에 대하여 대칭이다. $x\ge 0$일 때 $y=e^{x}$이고 접점을 $(t,\: e^{t})$이라 하면 $\dfrac{e^{t}-0}{t-0}=e^{t}$이므로 $t=1$이다.

두 접선의 기울기는 $e$와 $-e$이므로 $\tan\theta =\left |\dfrac{e-(-e)}{1+e\times(-e)}\right | =\dfrac{2e}{e^{2}-1}$이다.

<<<P>>>

원점에서 곡선 $y=e^{| x |}$에 그은 두 접선이 이루는 예각의 크기를 $\theta$라 할 때, $\tan\theta$의 값은? [3점]

<<<C>>>① $\dfrac{e}{e^{2}+1}$ ② $\dfrac{e}{e^{2}-1}$ ③ $\dfrac{2e}{e^{2}+1}$ ④ $\dfrac{2e}{e^{2}-1}$ ⑤ $1$

<<<A>>>③

<<<S>>>

$S_{1}=\dfrac{1}{2}\times 1^{2}\times\dfrac{\pi}{4}=\dfrac{\pi}{8}$

$\angle\mathrm{O}_{1}\mathrm{A}_{2}\mathrm{O}_{2}=\dfrac{\pi}{4}$이므로 삼각형 $\mathrm{O}_{1}\mathrm{A}_{2}\mathrm{O}_{2}$에서 사인법칙에 의하여

$\overline{\dfrac{\mathrm{O}_{2} \mathrm{A}_{2}}{\sin\dfrac{\pi}{6}}}=\overline{\dfrac{\mathrm{O}_{1} \mathrm{O}_{2}}{\sin\dfrac{\pi}{4}}}$에서 $\overline{\dfrac{\mathrm{O}_{2} \mathrm{A}_{2}}{\dfrac{1}{2}}}=\dfrac{1}{\sqrt{\dfrac{2}{2}}}$이고, 정리하면 $\overline{\mathrm{O_{2}A_{2}}}=\dfrac{1}{\sqrt{2}}$

따라서 닮음비는 $1 :\dfrac{1}{\sqrt{2}}$이므로 넓이의 비는 $1 :\dfrac{1}{2}$이다.

즉, 구하는 극한값은 첫째항이 $\dfrac{\pi}{8}$이고, 공비가 $\dfrac{1}{2}$인 등비급수의 합이므로

$\displaystyle\lim_{n\to \infty}S_{n}=\dfrac{\dfrac{\pi}{8}}{1-\dfrac{1}{2}}=\dfrac{\pi}{4}$

<<<P>>>

그림과 같이 중심이 $\mathrm{O}_{1}$, 반지름의 길이가 $1$이고 중심각의 크기가 $\dfrac{5\pi}{12}$인 부채꼴 $\mathrm{O}_{1}\mathrm{A}_{1}\mathrm{O}_{2}$가 있다. 호 $\mathrm{A}_{1}\mathrm{O}_{2}$ 위에 점 $\mathrm{B}_{1}$을 $\angle\mathrm{A}_{1}\mathrm{O}_{1}\mathrm{B}_{1}=\dfrac{\pi}{4}$가 되도록 잡고, 부채꼴 $\mathrm{O}_{1}\mathrm{A}_{1}\mathrm{B}_{1}$에 색칠하여 얻은 그림을 $R_{1}$이라 하자.

그림 $R_{1}$에서 점 $\mathrm{O}_{2}$를 지나고 선분 $\mathrm{O}_{1}\mathrm{A}_{1}$에 평행한 직선이 직선 $\mathrm{O}_{1}\mathrm{B}_{1}$과 만나는 점을 $\mathrm{A}_{2}$라 하자. 중심이 $\mathrm{O}_{2}$이고 중심각의 크기가 $\dfrac{5\pi}{12}$인 부채꼴 $\mathrm{O}_{2}\mathrm{A}_{2}\mathrm{O}_{3}$을 부채꼴 $\mathrm{O}_{1}\mathrm{A}_{1}\mathrm{B}_{1}$과 겹치지 않도록 그린다. 호 $\mathrm{A}_{2}\mathrm{O}_{3}$ 위에 점 $\mathrm{B}_{2}$를 $\angle\mathrm{A}_{2}\mathrm{O}_{2}\mathrm{B}_{2}=\dfrac{\pi}{4}$가 되도록 잡고, 부채꼴 $\mathrm{O}_{2}\mathrm{A}_{2}\mathrm{B}_{2}$에 색칠하여 얻은 그림을 $R_{2}$라 하자.

이와 같은 과정을 계속하여 $n$번째 얻은 그림 $R_{n}$에 색칠되어 있는 부분의 넓이를 $S_{n}$이라 할 때, $\displaystyle\lim_{n\to \infty}S_{n}$의 값은? [3점]

[IMG]

<<<C>>>① $\dfrac{3\pi}{16}$ ② $\dfrac{7\pi}{32}$ ③ $\dfrac{\pi}{4}$ ④ $\dfrac{9\pi}{32}$ ⑤ $\dfrac{5\pi}{16}$

<<<A>>>④

<<<S>>>

방정식 $f(x)=g(x)$의 서로 다른 양의 실근의 개수가 $3$이려면 아래 그림처럼 $y=e^{x}$와 $y=k\sin x$가 열린구간 $(2\pi ,\: 3\pi)$에서 접하면 된다.

[IMG]

두 함수의 도함수는 $y'=e^{x}$, $y'= k\cos x$이고

접점의 $x$좌표를 $t$라 하면 $x=t$에서

두 함수의 $y$값이 같으므로 $e^{t}=k\sin t$ $\cdots\cdots$㉠

두 함수의 접선의 기울기가 같으므로 $e^{t}=k\cos t$ $\cdots\cdots$㉡

㉠, ㉡에서 열린구간 $(2\pi ,\: 3\pi)$에서 $\tan t=1$을 만족하는 $t =\dfrac{9}{4}\pi$이다.

이 때, $k$의 값을 구하면 $k=\dfrac{e^{t}}{\sin t}=\sqrt{2}e^{\dfrac{9}{4}\pi}$

[다른 풀이]

방정식 $f(x)=g(x)$의 서로 다른 양의 실근의 개수는

$e^{x}= k\sin x$에서 $\dfrac{1}{k}=e^{-x}\sin x$$\cdots\cdots$ ㉠의 양의 실근의 개수와 같다.  

$h(x)=e^{-x}\sin x$라 하면

$h'(x)=-e^{-x}\sin x +e^{-x}\cos x =e^{-x}(\cos x -\sin x)$

따라서 $x>0$에서 $h'(x)=0$을 만족시키는 $x$의 값은

$x=\dfrac{\pi}{4},\:\dfrac{5}{4}\pi ,\:\dfrac{9}{4}\pi ,\:\cdots$

이므로 함수 $y=h(x)$의 증가와 감소를 표로 나타내면 다음과 같다.

[IMG]

[IMG]

이때 ㉠의 서로 다른 양의 실근의 개수가 $3$이기 위해서는 그림과 같이 직선 $y=\dfrac{1}{k}$이 $x=\dfrac{9}{4}\pi$에서 곡선 $y=h(x)$ 와 접해야 하고 $h\left(\dfrac{9}{4}\pi\right)=\dfrac{1}{\sqrt{2}e^{\dfrac{9}{4}\pi}}$이므로

$\dfrac{1}{k}=\dfrac{1}{\sqrt{2}e^{\dfrac{9}{4}\pi}}$이다.

따라서 $k=\sqrt{2}e^{\dfrac{9}{4}\pi}$

<<<P>>>

두 함수

$f(x)=e^{x},\:g(x)=k\sin x$

에 대하여 방정식 $f(x)=g(x)$의 서로 다른 양의 실근의 개수가 $3$일 때, 양수 $k$의 값은? [3점]

<<<C>>>① $\sqrt{2}e^{\dfrac{3\pi}{2}}$ ② $\sqrt{2}e^{\dfrac{7\pi}{4}}$ ③ $\sqrt{2}e^{2\pi}$ ④ $\sqrt{2}e^{\dfrac{9\pi}{4}}$ ⑤ $\sqrt{2}e^{\dfrac{5\pi}{2}}$

<<<A>>>①

<<<S>>>

$f(\theta)=\dfrac{1}{2}\times 1\times 1\times\sin(\pi -2\theta)=\dfrac{\sin 2\theta}{2}$

또한, $\angle\mathrm{APO}=\angle\mathrm{QPR}=\theta$이므로

점$\mathrm{P}$에서 두 선분 $\mathrm{AB}$, $\mathrm{BQ}$에 내린 수선의 발을 각각 $\mathrm{S},\:\mathrm{T}$라 하면 $\angle\mathrm{QPT}=2\theta$

즉, 점 $\mathrm{R}$는 삼각형 $\mathrm{PTQ}$의 내심이다.

이때, $\overline{\mathrm{OS}}=\cos 2\theta ,\:\overline{\mathrm{PS}}=\sin 2\theta ,\:\overline{\mathrm{BQ}}=\tan 2\theta$이므로

$\overline{\mathrm{PT}}=1-\cos 2\theta$

$\overline{\mathrm{QT}}=\tan 2\theta -\sin 2\theta =\tan 2\theta(1 -\cos 2\theta)$이고 

$\overline{\mathrm{PQ}}=\dfrac{1}{\cos 2\theta}-1=\dfrac{1-\cos 2\theta}{\cos 2\theta}$

따라서 삼각형 $\mathrm{PTQ}$의 내접원의 반지름의 길이를 $r$이라 하면 

$\dfrac{1}{2}\times(1-\cos 2\theta)\times\tan 2\theta(1-\cos 2\theta)$

$=\dfrac{1}{2}\times r\times\left\{\dfrac{1-\cos 2\theta}{\cos 2\theta}+1-\cos 2\theta +\tan 2\theta(1-\cos 2\theta)\right\}$에서

$r =\dfrac{(1-\cos 2\theta)\sin 2\theta}{1+\sin 2\theta +\cos 2\theta}$이다. 

그러므로

$g(\theta)=\dfrac{1}{2}\times\dfrac{1-\cos 2\theta}{\cos 2\theta}\times\dfrac{(1-\cos 2\theta)\sin 2\theta}{1+\sin 2\theta +\cos 2\theta}$

$=\dfrac{1}{2}\times\dfrac{(1-\cos 2\theta)^{2}\sin 2\theta}{\cos 2\theta(1+\sin 2\theta +\cos 2\theta)}$

$=\dfrac{1}{2}\times\dfrac{\sin^{4}2\theta\times\sin 2\theta}{\cos 2\theta(1+\sin 2\theta +\cos 2\theta)(1+\cos 2\theta)^{2}}$

따라서

$\displaystyle\lim_{\theta\rightarrow 0+}\dfrac{g(\theta)}{\theta^{4}\times f(\theta)}$

$ =\displaystyle\lim_{\theta\rightarrow 0+}\left\{\left(\dfrac{\sin 2\theta}{2\theta}\right)^{4}\times 16\times\dfrac{1}{\cos 2\theta(1+\sin 2\theta +\cos 2\theta)(1+\cos 2\theta)^{2}}\right\}$

$=1^{4}\times 16\times\dfrac{1}{8}=2$

<<<P>>>

그림과 같이 길이가 $2$인 선분 $\mathrm{AB}$를 지름으로 하는 반원의 호 $\mathrm{AB}$ 위에 점$\mathrm{P}$가 있다. 선분 $\mathrm{AB}$ 의 중점을 $\mathrm{O}$라 할 때, 점 $\mathrm{B}$를 지나고 선분 $\mathrm{AB}$에 수직인 직선이 직선 $\mathrm{OP}$와 만나는 점을 $\mathrm{Q}$라 하고, $\angle\mathrm{OQB}$의 이등분선이 직선 $\mathrm{AP}$와 만나는 점을 $\mathrm{R}$라 하자. $\angle\mathrm{OAP}=\theta$ 일 때, 삼각형 $\mathrm{OAP}$의 넓이를 $f(\theta)$,  삼각형 $\mathrm{PQR}$의 넓이를 $g(\theta)$라 하자. $\displaystyle\lim_{\theta\rightarrow 0+}\dfrac{g(\theta)}{\theta^{4}\times f(\theta)}$의 값은? (단, $0< \theta < \dfrac{\pi}{4}$) [4점]

[IMG]

<<<C>>>① $2$ ② $\dfrac{5}{2}$ ③ $3$  ④ $\dfrac{7}{2}$ ⑤ $4$

<<<A>>>$17$

<<<S>>>

$f'(x)=\dfrac{2t\ln x}{x}-2x=\dfrac{2t\ln x-2x^{2}}{x}$이고 

$f(x)$는 $x=k$에서 극대이므로 $2t\ln k-2k^{2}=0,\:t\ln k=k^{2}$

이때 실수 $k$의 값을 $g(t)$라 했으므로 

$t\ln g(t)=\{g(t)\}^{2}$   ………㉠

그런데  $g(\alpha)=e^{2}$이므로 ㉠에 $t=\alpha$를 대입하면 $\alpha\ln g(\alpha)=\{g(\alpha)\}^{2}$

$2\alpha =e^{4},\:\alpha =\dfrac{e^{4}}{2}$

또한, ㉠의 양변을 $t$에 대하여 미분하면

$\ln g(t)+t\times\dfrac{g'(t)}{g(t)}=2 g(t)\times g'(t)$

이 식에 $t=\alpha$를 대입하면

$\ln g(\alpha)+\alpha\times\dfrac{g'(\alpha)}{g(\alpha)}=2 g(\alpha)\times g'(\alpha)$

$2+\dfrac{e^{4}}{2}\times\dfrac{g'(\alpha)}{e^{2}}=2 e^{2}\times g'(\alpha)$

$\dfrac{3}{2}e^{2}\times g'(\alpha)=2$

$g'(\alpha)=\dfrac{4}{3e^{2}}$

$\alpha\{g'(\alpha)\}^{2}=\dfrac{e^{4}}{2}\times\dfrac{16}{9e^{4}}=\dfrac{8}{9}$

따라서 $p=9,\:q=8$ 이므로 $p+q=17$

<<<P>>>

$t >2e$인 실수 $t$에 대하여 함수 $f(x)=t(\ln x)^{2}-x^{2}$ 이 $x=k$에서 극대일 때, 실수 $k$의 값을 $g(t)$라 하면 $g(t)$는 미분가능한 함수이다. $g(\alpha)=e^{2}$ 인 실수 $\alpha$에 대하여  $\alpha\times\{g'(\alpha)\}^{2}=\dfrac{q}{p}$일 때, $p+q$의 값을 구하시오. (단, $p$와 $q$는 서로소인 자연수이다.) [4점]

<<<A>>>$11$

<<<S>>>

곡선 $y=\ln(1+e^{2x}-e^{-2t})$과 직선 $y=x+t$가 만나는 두 점을 

$\mathrm{P}(\alpha ,\:\alpha +\mathrm{t}),\:\mathrm{Q}(\beta ,\:\beta +\mathrm{t})(\alpha < \beta)$로 놓으면

$f(t)=\sqrt{(\beta -\alpha)^{2}+(\beta -\alpha)^{2}}=\sqrt{2}(\beta -\alpha)$

이때, $\alpha ,\:\beta$는 방정식 $\ln(1+e^{2x}-e^{-2t})=x+t$의 서로 다른 두 실근이므로

$1+e^{2x}-e^{-2t}=e^{x+t}$로 놓으면 $e^{2x}-e^{t}\times e^{x}+1-e^{-2t}=0$

$e^{x}=k(k>0)$ 로 놓으면 $k^{2}-e^{t}k +1-e^{-2t}=0$

$k^{2}-e^{t}k+e^{-t}\left(e^{t}-e^{-t}\right)=0$

$\left(k-e^{^{-t}}\right)\left(k-e^{t}+e^{-t}\right)=0$

이므로 $k=e^{-t}$또는 $k =e^{t}-e^{-t}$ 이고 $t >\dfrac{1}{2}\ln 2$이므로 $e^{-t}< e^{t}-e^{-t}$

즉, $\alpha =\ln e^{-t}=-t$이고 $\beta =\ln\left(e^{t}-e^{-t}\right)$이다.

$\beta -\alpha =\ln\left(e^{t}-e^{-t}\right)+t =\ln\left(e^{2t}-1\right)$

이므로

$f(t)=\sqrt{2}\ln\left(e^{2t}-1\right)$이고 $f'(t)=\sqrt{2}\dfrac{2e^{2t}}{e^{2t}-1}$

$f'(\ln 2)=\sqrt{2}\times\dfrac{2\times 4}{4-1}=\dfrac{8}{3}\sqrt{2}$이므로 $p=3q=8$

따라서 $p+q=11$

<<<P>>>

$t >\dfrac{1}{2}\ln 2$인 실수 $t$에 대하여 곡선 $y=\ln(1+e^{2x}-e^{-2t})$과 직선 $y=x+t$가 만나는 서로 다른 두 점 사이의 거리를 $f(t)$라 할 때, $f'(\ln 2)=\dfrac{q}{p}\sqrt{2}$이다. $p+q$의 값을 구하시오. (단, $p$와 $q$는 서로소인 자연수이다.) [4점]

기하

<<<A>>>②

<<<S>>>

두 벡터 $\overrightarrow{a}=(k+3,\: 3k-1)$과 $\overrightarrow{b}=(1,\: 1)$이 서로 평행하므로 적당한 실수 $m$에 대하여 $\overrightarrow{a}=m\overrightarrow{b}$가 성립한다.

$(k+3,\: 3k-1)=m(1,\: 1)$에서 $k+3=m$, $3k-1=m$

따라서 $k=2$, $m=5$

<<<P>>>

두 벡터 $\overrightarrow{a}=(k+3,\: 3k-1)$과 $\overrightarrow{b}=(1,\: 1)$이 서로 평행할 때, 실수 $k$의 값은? [2점]

<<<C>>>① $1$ ② $2$ ③ $3$ ④ $4$ ⑤ $5$

<<<A>>>⑤

<<<S>>>

타원 $\dfrac{x^{2}}{8}+\dfrac{y^{2}}{4}=1$위의 점 $(2,\:\sqrt{2})$에서의 접선의 방정식은 $\dfrac{2x}{8}+\dfrac{\sqrt{2}y}{4}=1$

이므로 이 직선의 $x$절편은 $4$이다.

<<<P>>>

타원  $\dfrac{x^{2}}{8}+\dfrac{y^{2}}{4}=1$위의 점 $(2,\:\sqrt{2})$에서의 접선의 $x$절편은? [3점]

<<<C>>>① $3$ ② $\dfrac{13}{4}$ ③ $\dfrac{7}{2}$ ④ $\dfrac{15}{4}$ ⑤ $4$

<<<A>>>①

<<<S>>>

$|\overrightarrow{\mathrm{OP}}-\overrightarrow{\mathrm{OA}}| = |\overrightarrow{\mathrm{AB}}|$에서 $|\overrightarrow{\mathrm{AP}}| = |\overrightarrow{\mathrm{AB}}|$

이때 $\overline{\mathrm{AB}}=\sqrt{(-3-1)^{2}+(5-2)^{2}}=5$이므로

$|\overrightarrow{\mathrm{AP}}| =5$

따라서 점 $\mathrm{P}$가 나타내는 도형은 점 $\mathrm{A}$를 중심으로 하고 반지름의 길이가 $5$인 원이므로 그 길이는 $10\pi$이다.

<<<P>>>

좌표평면 위의 두 점 $\mathrm{A}(1,\: 2)$, $\mathrm{B}(-3,\: 5)$에 대하여 

$|\overrightarrow{\mathrm{OP}}-\overrightarrow{\mathrm{OA}}| = |\overrightarrow{\mathrm{AB}}|$

를 만족시키는 점 $\mathrm{P}$가 나타내는 도형의 길이는? (단, $\mathrm{O}$는 원점이다.) [3점]

<<<C>>>① $10\pi$ ② $12\pi$ ③ $14\pi$ ④ $16\pi$ ⑤ $18\pi$

<<<A>>>②

<<<S>>>

정육각형에서 직선 $\mathrm{EF}$와 직선 $\mathrm{CD}$의 교점을 $\mathrm{G}$라고 하자.

$\overrightarrow{\mathrm{BC}}=\overrightarrow{\mathrm{EG}}$이므로 $|\overrightarrow{\mathrm{AE}}+\overrightarrow{\mathrm{BC}}| = |\overrightarrow{\mathrm{AE}}+\overrightarrow{\mathrm{EG}}| = |\overrightarrow{\mathrm{AG}}|$

[IMG]

직각삼각형 $\mathrm{ACG}$에서 $\overline{\mathrm{AC}}= 2\times\dfrac{\sqrt{3}}{2}=\sqrt{3}$, $\overline{\mathrm{CG}}=1+1=2$이므로

$|\overrightarrow{\mathrm{AE}}+\overrightarrow{\mathrm{BC}}| = |\overrightarrow{\mathrm{AG}}| =\overline{\mathrm{AG}}=\sqrt{(\sqrt{3})^{2}+2^{2}}=\sqrt{7}$

[다른 풀이]

두 선분  $\mathrm{AD}$와 $\mathrm{BE}$의 교점을 $\mathrm{O}$라 하고 선분 $\mathrm{OE}$의 중점을 $\mathrm{M}$이라 하면 $\overrightarrow{\mathrm{BC}}=\overrightarrow{\mathrm{AO}}$이므로

[IMG]

$\overrightarrow{\mathrm{AE}}+\overrightarrow{\mathrm{BC}}=\overrightarrow{\mathrm{AE}}+\overrightarrow{\mathrm{AO}}=2\overrightarrow{\mathrm{AM}}$ $\cdots$㉠

삼각형 $\mathrm{AOM}$에서 코사인법칙에 의하여 

$\overline{\mathrm{AM}}^{2}$$=\overline{\mathrm{AO}}^{2}+\overline{\mathrm{OM}}^{2}-2\times\overline{\mathrm{AO}}\times\overline{\mathrm{OM}}\times\cos 120^{\circ}$

$=1^{2}+\left(\dfrac{1}{2}\right)^{2}-2\times 1\times\dfrac{1}{2}\times\left(-\dfrac{1}{2}\right)$

$=\dfrac{7}{4}$

$\overline{\mathrm{AM}}=\dfrac{\sqrt{7}}{2}$이므로 ㉠에서 $|\overrightarrow{\mathrm{AE}}+\overrightarrow{\mathrm{BC}}| =2\overline{\mathrm{AM}}=2\times\dfrac{\sqrt{7}}{2}=\sqrt{7}$

[다른 풀이]

$\overrightarrow{\mathrm{BC}}=\overrightarrow{\mathrm{FE}}$이므로 $\overrightarrow{\mathrm{AE}}+\overrightarrow{\mathrm{BC}}=\overrightarrow{\mathrm{AE}}+\overrightarrow{\mathrm{FE}}= 2\overrightarrow{\mathrm{ME}}$

[IMG]

$|\overrightarrow{\mathrm{AE}}+\overrightarrow{\mathrm{BC}}| = |\overrightarrow{\mathrm{AE}}+\overrightarrow{\mathrm{FE}}| = 2 |\overrightarrow{\mathrm{ME}}|$이다. 

$|\overrightarrow{\mathrm{ME}}| = k$라고 하면 중선정리에 의해서 

$\overline{\mathrm{EA}}^{2}+\overline{\mathrm{EF}}^{2}=2\left(\overline{\mathrm{ME}}^{2}+\overline{\mathrm{MA}}^{2}\right)$이므로 $(\sqrt{3})^{2}+1^{2}=2\left(k^{2}+\left(\dfrac{1}{2}\right)^{2}\right)$이다.

$k^{2}=\dfrac{7}{4}$ 즉, $|\overrightarrow{\mathrm{ME}}| =\dfrac{\sqrt{7}}{2}$이다.

따라서 $|\overrightarrow{\mathrm{AE}}+\overrightarrow{\mathrm{BC}}| = 2 |\overrightarrow{\mathrm{ME}}| =2\times\dfrac{\sqrt{7}}{2}=\sqrt{7}$

<<<P>>>

그림과 같이 한 변의 길이가 $1$인 정육각형 $\mathrm{ABCDEF}$에서 $|\overrightarrow{\mathrm{AE}}+\overrightarrow{\mathrm{BC}}|$의 값은? [3점]

[IMG]

<<<C>>>① $\sqrt{6}$ ② $\sqrt{7}$ ③ $2\sqrt{2}$ ④ $3$ ⑤ $\sqrt{10}$

<<<A>>>③

<<<S>>>

점  $\mathrm{P}(4,\: k)$는 쌍곡선 $\dfrac{x^{2}}{a^{2}}-\dfrac{y^{2}}{b^{2}}=1$위의 점이므로 $\dfrac{16}{a^{2}}-\dfrac{k^{2}}{b^{2}}=1$ $\cdots$ ㉠

점 $\mathrm{P}$에서 쌍곡선에 그은 접선의 방정식은 $\dfrac{4x}{a^{2}}-\dfrac{ky}{b^{2}}=1$

이므로 두 점 $\mathrm{Q}$와 $\mathrm{R}$의 좌표는 각각 $\left .\mathrm{Q}\left(\dfrac{a^{2}}{4},\: 0\right)\right .$, $\mathrm{R}\left(0,\: -\dfrac{b^{2}}{k}\right)$

따라서 삼각형 $\mathrm{QOR}$의 넓이는 

$A_{1}=\dfrac{1}{2}\times\dfrac{a^{2}}{4}\times\left | -\dfrac{b^{2}}{k}\right | =\dfrac{a^{2}b^{2}}{8k}$

삼각형  $\mathrm{PRS}$의 넓이는 $A_{2}=\dfrac{1}{2}\times\overline{\mathrm{PS}}\times\overline{\mathrm{OS}}=\dfrac{1}{2}\times k\times 4=2k$이므로

$A_{1}:A_{2}=9:4$에서 $\dfrac{a^{2}b^{2}}{8k}:2k=9:4$

$36k^{2}=a^{2}b^{2}$ $\cdots$ ㉡

㉡을 ㉠에 대입하여 정리하면

$\dfrac{16}{a^{2}}-\dfrac{k^{2}}{\dfrac{36k^{2}}{a^{2}}}=1$, 즉 $\dfrac{16}{a^{2}}-\dfrac{a^{2}}{36}=1$

$a^{4}+36a^{2}-16\times 36=0$

$(a^{2}-12)(a^{2}+48)=0$

$a^{2}=12$에서 $a=2\sqrt{3}$이므로 주어진 쌍곡선의 주축의 길이는 $2a=4\sqrt{3}$이다.

<<<P>>>

그림과 같이 쌍곡선 $\dfrac{x^{2}}{a^{2}}-\dfrac{y^{2}}{b^{2}}=1$ 위의 점 $\mathrm{P}(4,\:\mathrm{k})(\mathrm{k}>0)$에서의 접선이 $x$축과 만나는 점을 $\mathrm{Q}$, $y$축과 만나는 점을 $\mathrm{R}$라 하자. 점 $\mathrm{S}(4,\: 0)$에 대하여 삼각형 $\mathrm{QOR}$의 넓이를 $A_{1}$. 삼각형 $\mathrm{PRS}$의 넓이를 $A_{2}$라 하자. $A_{1}:A_{2}=9:4$일 때, 이 쌍곡선의 주축의 길이는? (단, $\mathrm{O}$는 원점이고, $a$와$b$는 상수이다.) [3점]

[IMG]

<<<C>>>① $2\sqrt{10}$ ② $2\sqrt{11}$ ③ $4\sqrt{3}$ ④ $2\sqrt{13}$ ⑤ $2\sqrt{14}$

<<<A>>>③

<<<S>>>

그림과 같이 타원의 중심을 원점으로 하고 장축이 $x$축 위에 놓이도록 좌표축을 설정하자.

[IMG]

이때 타원의 장축의 길이가 $2a$이므로 타원의 방정식을 $\dfrac{x^{2}}{a^{2}}+\dfrac{y^{2}}{b^{2}}=1$$(b>0)$라 하면 두 초점의 좌표는 $\mathrm{F}'(-\sqrt{a^{2}-b^{2}},\: 0)$, $\mathrm{F}(\sqrt{a^{2}-b^{2}},\: 0)$이다.

주어진 타원이 음의 $x$축과 만나는 점을 $\mathrm{A}$, 양의 $y$축과 만나는 점을 $\mathrm{B}$라 하면 두 점 $\mathrm{A}$와 $\mathrm{B}$의 좌표는 각각 $A(-a,\:0)$, $B(0,\: b)$ 이다.

점 $\mathrm{A}$를 중심으로 하고 두 점 $\mathrm{B}$과 $\mathrm{F}$를 지나는 원의 반지름의 길이는  $1$이므로

$\overline{\mathrm{AB}}=1$에서 $\sqrt{a^{2}+b^{2}}=1$

$b^{2}=1-a^{2}$ $\cdots$ ㉠

$\overline{\mathrm{AF}}=1$에서 $\sqrt{a^{2}-b^{2}}+a=1$ $\cdots$ ㉡

㉠, ㉡에서

$\sqrt{a^{2}-(1-a^{2})}=1-a$

이 식의 양변을 제곱하여 정리하면 $2a^{2}-1=1-2a+a^{2}$

$a^{2}+2a-2=0$

따라서 $a=-1+\sqrt{3}$ $(a>0)$

<<<P>>>

두 초점이 $\mathrm{F}$, $\mathrm{F}'$이고 장축의 길이가  $2a$인 타원이 있다. 이 타원의 한 꼭짓점을 중심으로 하고 반지름의 길이가 $1$인 원이 이 타원의 서로 다른 두 꼭짓점과 한 초점을 지날 때, 상수 $a$의 값은? [4점]

[IMG]

<<<C>>>① $\dfrac{\sqrt{2}}{2}$ ② $\dfrac{\sqrt{6}-1}{2}$ ③ $\sqrt{3}-1$ ④ $2\sqrt{2}-2$ ⑤ $\dfrac{\sqrt{3}}{2}$

<<<A>>>$80$

<<<S>>>

직선 $y = 2x - 4$가 포물선 $y^{2}= 8x$와 만나는 점 중 $\mathrm{A}$가 아닌 점을 $\mathrm{E}$, $x$축과 만나는 점을 $\mathrm{F}$라 하고, 점 $\mathrm{E}$에서 직선 $x = -2$에 내린 수선의 발을 $\mathrm{H}$라 하자.

또 두 점 $\mathrm{A}$, $\mathrm{B}$에서 직선 $\mathrm{HE}$에 내린 수선의 발을 각각 $\mathrm{I}$, $\mathrm{J}$라 하자.

[IMG]

점 $\mathrm{F}$의 좌표는 $(2,\:0)$이므로 포물선 $y^{2}= 8x$의 초점과 일치한다. 따라서 $\overline{\mathrm{AF}} = p$, $\overline{\mathrm{EF}} = q$라 하면 포물선의 정의에 의하여 $\overline{\mathrm{AC}} = p$, $\overline{\mathrm{EH}} = q$

이때 포물선의 준선의 방정식이 $x = -2$이므로 두 점 $\mathrm{A}$, $\mathrm{E}$의 $x$좌표는 각각 $p-2$, $q-2$이다. 

선분 $\mathrm{AI}$와 $x$축의 교점을 $\mathrm{P}$, 점 $\mathrm{E}$에서 $x$축에 내린 수선의 발을 $\mathrm{Q}$라 하면 

$\overline{\mathrm{FP}} = p-4$, $\overline{\mathrm{FQ}} = 4- q$

이므로 두 직각삼각형 $\mathrm{AFP}$, $\mathrm{EFQ}$에서 

$p=\sqrt{5}(p-4)$, $q=\sqrt{5}(4-q)$

따라서

$p=\dfrac{4\sqrt{5}}{\sqrt{5}-1}=\sqrt{5}(\sqrt{5+}1)=5+\sqrt{5}$

$q=\dfrac{4\sqrt{5}}{\sqrt{5}+1}=\sqrt{5}(\sqrt{5}-1)=5-\sqrt{5}$이므로

$\overline{\mathrm{EI}} = p- q = 2\sqrt{5}$

한편 포물선 $(y-2a)^{2}= 8(x- a)$는 포물선 $y^{2}= 8x$를 $x$축의 방향으로 $a$만큼 $y$축의 방향으로 $2a$만큼 평행이동한 것이므로 두 점 $\mathrm{A}$, $\mathrm{B}$는 각각 두 점 $\mathrm{E}$, $\mathrm{A}$를 $x$축의 방향으로 $a$만큼 $y$축의 방향으로 $2a$만큼 평행이동한 것이다. 

따라서 $\overline{\mathrm{AB}} =\overline{\mathrm{AE}}$이므로 포물선의 정의에 의하여 

$\overline{\mathrm{AC}}+\overline{\mathrm{BD}}-\overline{\mathrm{AB}}$$=\overline{\mathrm{AC}} +\overline{\mathrm{BD}} -\overline{\mathrm{AE}}$

$=\overline{\mathrm{AC}} +\overline{\mathrm{BD}} -(\overline{\mathrm{AF}} +\overline{\mathrm{EF}})$

$=\overline{\mathrm{AC}} +\overline{\mathrm{BD}} -(\overline{\mathrm{AC}} +\overline{\mathrm{EH}})$

$=\overline{\mathrm{BD}} -\overline{\mathrm{EH}}$

$=\overline{\mathrm{EJ}}$$= 2\times\overline{\mathrm{EI}}$$= 2\times 2\sqrt{5}= 4\sqrt{5}$

따라서 $k = 4\sqrt{5}$이므로 $k^{2}= 80$

[다른 풀이]

직선 $y = 2x - 4$가 포물선 $y^{2}= 8x$와 만나는 점 중 $\mathrm{A}$가 아닌 점을 $\mathrm{E}$, $x$축과 만나는 점을 $\mathrm{F}$라 하고, 점 $\mathrm{E}$에서 직선 $x = -2$에 내린 수선의 발을 $\mathrm{H}$라 하자.

또 두 점 $\mathrm{A}$, $\mathrm{B}$에서 직선 $\mathrm{HE}$에 내린 수선의 발을 각각 $\mathrm{I}$, $\mathrm{J}$라 하자.

점  $\mathrm{F}$의 좌표는 $(2,\:0)$이므로 포물선 $y^{2}= 8x$의 초점과 일치한다. 

이때 연립방정식 $\begin{cases}
y^{2}=8x\\
y=2x-4
\end{cases}$의 해는 $y^{2}= 4(y+ 4)$ 즉, $y^{2}- 4y - 16 = 0$에서 $y=2\pm 2\sqrt{5}$

$y=2+2\sqrt{5}$이면 $x=3+\sqrt{5}$, $y=2-2\sqrt{5}$이면 $x=3-\sqrt{5}$이므로 두 점 $\mathrm{A}$와 $\mathrm{E}$의 좌표는 각각 $\mathrm{A}(3+\sqrt{5},\:2+2\sqrt{5})$, $\mathrm{B}(3-\sqrt{5},\:2-2\sqrt{5})$이다.

포물선 $(y-2a)^{2}= 8(x- a)$은 포물선 $y^{2}= 8x$를 $x$축의 방향으로 $a$만큼 $y$축의 방향으로 $2a$만큼 평행이동한 것이므로 $\overline{\mathrm{AB}} =\overline{\mathrm{AE}}$

따라서 포물선의 정의에 의해 

$\overline{\mathrm{AC}}+\overline{\mathrm{BD}}-\overline{\mathrm{AB}}$$=\overline{\mathrm{AC}} +\overline{\mathrm{BD}} -\overline{\mathrm{AE}}$

$=\overline{\mathrm{AC}} +\overline{\mathrm{BD}} -(\overline{\mathrm{AF}} +\overline{\mathrm{EF}})$

$=\overline{\mathrm{AC}} +\overline{\mathrm{BD}} -(\overline{\mathrm{AC}} +\overline{\mathrm{EH}})$

$=\overline{\mathrm{BD}} -\overline{\mathrm{EH}}$

$=\overline{\mathrm{EJ}}$$= 2\times\overline{\mathrm{EI}}$

$=2\times\{(3+\sqrt{5})-(3-\sqrt{5})\}$$=4\sqrt{5}$

$k=4\sqrt{5}$이므로 $k^{2}= 80$

<<<P>>>

포물선 $y^{2}= 8x$와 직선 $y = 2x - 4$가 만나는 점 중 제 $1$사분면 위에 있는 점을 $\mathrm{A}$라 하자. 양수 $a$에 대하여 포물선 $(y-2a)^{2}= 8(x- a)$가 점 $\mathrm{A}$를 지날 때, 직선 $y = 2x - 4$와 포물선 $(y-2a)^{2}= 8(x-a)$가 만나는 점 중 $\mathrm{A}$가 아닌 점을 $\mathrm{B}$라 하자. 두 점 $\mathrm{A}$, $\mathrm{B}$에서 직선 $x = -2$에 내린 수선의 발을 각각 $\mathrm{C}$, $\mathrm{D}$라 할 때, $\overline{\mathrm{AC}}+\overline{\mathrm{BD}}-\overline{\mathrm{AB}}= k$이다. $k^{2}$의 값을 구하시오. [4점]

[IMG]

<<<A>>>$48$

<<<S>>>

조건 (가)에서 $\overrightarrow{\mathrm{PQ}}\cdot\overrightarrow{\mathrm{AB}} = 0$ 또는 $\overrightarrow{\mathrm{PQ}}\cdot\overrightarrow{\mathrm{AD}} = 0$이므로 다음과 같이 두 가지의 경우로 나누어 생각할 수 있다. 

(ⅰ) $\overrightarrow{\mathrm{PQ}}\cdot\overrightarrow{\mathrm{AB}} = 0$, 즉 $\overline{\mathrm{PQ}}\bot\overline{\mathrm{AB}}$인 경우

두 조건 (나)와 (다)에서 $\overrightarrow{\mathrm{OB}}\cdot\overrightarrow{\mathrm{OP}}\ge 0$, $\overrightarrow{\mathrm{OB}}\cdot\overrightarrow{\mathrm{OQ}}\le 0$이므로

점 $\mathrm{P}$는 선분 $\mathrm{AB}$위의 점이고 점 $\mathrm{Q}$는 선분 $\mathrm{CD}$위의 점이다.

[IMG]

점 $\mathrm{P}$의 좌표를 $\mathrm{P}=(a ,\:  2-a)$ ($0\le a\le 2$)라 하면 점 $\mathrm{Q}$의 좌표는 $\mathrm{Q}(a-2 ,\:  -a)$

$\overrightarrow{\mathrm{OA}}\cdot\overrightarrow{\mathrm{OP}}\ge -2$에서 $(2,\: 0)\cdot(a ,\:  2- a)= 2a\ge -2$이므로 

$a\ge -1$ $\cdots\cdots$ ㉠

$\overrightarrow{\mathrm{OB}}\cdot\overrightarrow{\mathrm{OP}}\ge 0$에서 

$(0,\:2)\cdot(a,\: 2-a)= 2(2-a)\ge 0$이므로 

$a\le 2$ $\cdots\cdots$ ㉡

$\overrightarrow{\mathrm{OA}}\cdot\overrightarrow{\mathrm{OQ}}\ge -2$에서

$(2,\:  0)\cdot(a-2 ,\: - a)= 2(a- 2)\ge - 2$이므로 

$a\ge 1$ $\cdots\cdots$ ㉢

$\overrightarrow{\mathrm{OB}}\cdot\overrightarrow{\mathrm{OQ}}\le 0$에서 

$(0,\:2)\cdot(a-2,\:  - a)= -2a\le 0$이므로

$a\ge 0$ $\cdots\cdots$ ㉣

㉠, ㉡, ㉢, ㉣에서 

$1\le a\le 2$ $\cdots\cdots$ ㉤

한편 점 $\mathrm{R}(4,\: 4)$에 대하여

$\overrightarrow{\mathrm{RP}}$$=\overrightarrow{\mathrm{OP}} -\overrightarrow{\mathrm{OR}}$

$=(a,\:  2-a)-(4,\:  4)$

$=(a-4 ,\:  -a -2)$

$\overrightarrow{\mathrm{RQ}}$$=\overrightarrow{\mathrm{OQ}} -\overrightarrow{\mathrm{OR}}$

$=(a-2 ,\:  -a)-(4,\:  4)$

$=(a-6 ,\:  -a -4)$이므로

$\overrightarrow{\mathrm{RP}}\cdot\overrightarrow{\mathrm{RQ}}$$=(a-4 ,\:  -a -2)\cdot(a-6,\:  -a-4)$

$=(a-4)(a-6)+(a+2)(a+4)$

$= 2a^{2}- 4a + 32$

$= 2(a-1)^{2}+ 30$

㉤에서 

$30\le\overrightarrow{\mathrm{RP}}\cdot\overrightarrow{\mathrm{RQ}}\le 32$

(ⅱ) $\overrightarrow{\mathrm{PQ}}\cdot\overrightarrow{\mathrm{AD}} = 0$, 즉 $\overline{\mathrm{PQ}}\bot\overline{\mathrm{AD}}$인 경우 두 조건 (나)와 (다)에서 

$\overrightarrow{\mathrm{OB}}\cdot\overrightarrow{\mathrm{OP}}\ge 0$, $\overrightarrow{\mathrm{OB}}\cdot\overrightarrow{\mathrm{OQ}}\le 0$

이므로 점 $\mathrm{P}$는 선분 $\mathrm{BC}$위의 점이고 점 $\mathrm{Q}$는 선분 $\mathrm{AD}$위의 점이다.

[IMG]

점 $\mathrm{P}$의 좌표를 $\mathrm{P}(a ,\:  a+2)$ ($-2\le a\le 0$)라 하면 점 $\mathrm{Q}$의 좌표는 $\mathrm{Q}(a+2 ,\:  a)$

$\overrightarrow{\mathrm{OA}}\cdot\overrightarrow{\mathrm{OP}}\ge -2$에서 $(2,\:0)\cdot(a,\:a+2)=2a\ge -2$이므로

$a\ge -1$ $\cdots\cdots$ ㉥

$\overrightarrow{\mathrm{OB}}\cdot\overrightarrow{\mathrm{OP}}\ge 0$에서 

$(0,\:2)\cdot(a,\: a+2)= 2(a+2)\ge 0$이므로 

$a\ge -2$ $\cdots\cdots$ ㉦

$\overrightarrow{\mathrm{OA}}\cdot\overrightarrow{\mathrm{OQ}}\ge -2$에서

$(2,\:  0)\cdot(a+2 ,\:  a)= 2(a+ 2)\ge - 2$이므로 

$a\ge -3$ $\cdots\cdots$ ㉧

$\overrightarrow{\mathrm{OB}}\cdot\overrightarrow{\mathrm{OQ}}\le 0$에서 

$(0,\:2)\cdot(a+2,\:  a)= 2a\le 0$이므로

$a\le 0$ $\cdots\cdots$ ㉨

㉥, ㉦, ㉧, ㉨에서 

$-1\le a\le 0$ $\cdots\cdots$ ㉩

한편 점 $\mathrm{R}(4,\: 4)$에 대하여

$\overrightarrow{\mathrm{RP}}$$=\overrightarrow{\mathrm{OP}} -\overrightarrow{\mathrm{OR}}$

$=(a,\:  a+2)-(4,\:  4)$

$=(a-4 ,\:  a -2)$

$\overrightarrow{\mathrm{RQ}}$$=\overrightarrow{\mathrm{OQ}} -\overrightarrow{\mathrm{OR}}$

$=(a+2 ,\:  a)-(4,\:  4)$

$=(a-2 ,\:  a -4)$이므로

$\overrightarrow{\mathrm{RP}}\cdot\overrightarrow{\mathrm{RQ}}$$=(a-4 ,\:  a -2)\cdot(a-2,\:  a-4)$

$= 2(a-4)(a-2)$

$= 2\left(a^{2}- 6a + 8\right)$

$= 2(a-3)^{2}-2$

㉩에서 

$16\le\overrightarrow{\mathrm{RP}}\cdot\overrightarrow{\mathrm{RQ}}\le 30$

(ⅰ), (ⅱ)에서

$16\le\overrightarrow{\mathrm{RP}}\cdot\overrightarrow{\mathrm{RQ}}\le 32$

따라서 $\overrightarrow{\mathrm{RP}}\cdot\overrightarrow{\mathrm{RQ}}$의 최댓값은 $M = 32$, 최솟값은 $m = 16$이므로 

$M + m = 48$

[다른 풀이]

위의 풀이에서 두 점 $\mathrm{P}$, $\mathrm{Q}$가 지나는 영역은 다음 그림의 굵은 선분 위이다.

[IMG]

[IMG]

선분 $\mathrm{PQ}$의 중점을 $\mathrm{N}$이라 하면 조건 (가)에 의하여

$\overrightarrow{\mathrm{NP}} +\overrightarrow{\mathrm{NQ}} =\overrightarrow{0}$

$\overrightarrow{\mathrm{NP}}\cdot\overrightarrow{\mathrm{NQ}}= |\overrightarrow{\mathrm{NP}}| |\overrightarrow{\mathrm{NQ}}|\cos\pi = -\overline{\mathrm{NP}}^{2}$

이므로

$\overrightarrow{\mathrm{RP}}\cdot\overrightarrow{\mathrm{RQ}}$$=(\overrightarrow{\mathrm{RN}} +\overrightarrow{\mathrm{NP}})\cdot(\overrightarrow{\mathrm{RN}} +\overrightarrow{\mathrm{NQ}})$

$=\overrightarrow{\mathrm{RN}}\cdot\overrightarrow{\mathrm{RN}} +\overrightarrow{\mathrm{RN}}\cdot\overrightarrow{\mathrm{NQ}} +\overrightarrow{\mathrm{NP}}\cdot\overrightarrow{\mathrm{RN}} +\overrightarrow{\mathrm{NP}}\cdot\overrightarrow{\mathrm{NQ}}$

$= |\overrightarrow{\mathrm{RN}} |^{2}+\overrightarrow{\mathrm{RN}}\cdot(\overrightarrow{\mathrm{NQ}} +\overrightarrow{\mathrm{NP}})+\overrightarrow{\mathrm{NP}}\cdot\overrightarrow{\mathrm{NQ}}$

$=\overline{\mathrm{RN}}^{2}+0 -\overline{\mathrm{NP}}^{2}$

이때 항상 $\overline{\mathrm{PQ}} = 2\sqrt{2}$이므로

$\overline{\mathrm{NP}} =\dfrac{1}{2}\overline{\mathrm{PQ}} =\sqrt{2}$

따라서 $\overrightarrow{\mathrm{RP}}\cdot\overrightarrow{\mathrm{RQ}}$$=\overline{\mathrm{RN}}^{2}-2$이므로 

$\overrightarrow{\mathrm{RP}}\cdot\overrightarrow{\mathrm{RQ}}$의 최댓값과 최솟값은 $\overline{\mathrm{RN}}$의 최댓값과 최솟값에 의하여 결정된다.

$\overline{\mathrm{RN}}$이 최대일 때의 두 점 $\mathrm{P}$, $\mathrm{Q}$의 좌표는 각각 $(2,\:  0)$, $(0,\:  -2)$이므로 $\mathrm{N}(1,\: -1)$

따라서 

$\overline{\mathrm{RN}} =\sqrt{(4-1)^{2}+(4+1)^{2}}=\sqrt{34}$

이므로 $\overrightarrow{\mathrm{RP}}\cdot\overrightarrow{\mathrm{RQ}}$의 최댓값은 $M=(\sqrt{34})^{2}- 2 = 32$

$\overline{\mathrm{RN}}$이 최소일 때의 두 점 $\mathrm{P}$, $\mathrm{Q}$의 좌표는 각각 $(0,\:  2)$, $(2,\:  0)$이므로 $\mathrm{N}(1,\: 1)$

따라서 

$\overline{\mathrm{RN}} =\sqrt{(4-1)^{2}+(4-1)^{2}}=\sqrt{18}$

이므로 $\overrightarrow{\mathrm{RP}}\cdot\overrightarrow{\mathrm{RQ}}$의 최솟값은 $m=(\sqrt{18})^{2}- 2 = 16$

이상에서 

$M + m = 32+ 16 = 48$

<<<P>>>2022학년도 대학수학능력시험 6월 모의평가 끝

좌표평면 위의 네 점 $\mathrm{A}(2,\: 0)$, $\mathrm{B}(0,\:  2)$, $\mathrm{C}(-2,\: 0)$, $\mathrm{D}(0,\: -2)$를 꼭짓점으로 하는 정사각형 $\mathrm{ABCD}$의 네 변 위의 두 점 $\mathrm{P}$, $\mathrm{Q}$가 다음 조건을 만족시킨다.

(가) $(\overrightarrow{\mathrm{PQ}}\cdot\overrightarrow{\mathrm{AB}})(\overrightarrow{\mathrm{PQ}}\cdot\overrightarrow{\mathrm{AD}})= 0$

(나) $\overrightarrow{\mathrm{OA}}\cdot\overrightarrow{\mathrm{OP}}\ge -2$이고, $\overrightarrow{\mathrm{OB}}\cdot\overrightarrow{\mathrm{OP}}\ge 0$이다.

(다) $\overrightarrow{\mathrm{OA}}\cdot\overrightarrow{\mathrm{OQ}}\ge -2$이고, $\overrightarrow{\mathrm{OB}}\cdot\overrightarrow{\mathrm{OQ}}\le 0$이다.

점 $\mathrm{R}(4,\: 4)$에 대하여 $\overrightarrow{\mathrm{RP}}\cdot\overrightarrow{\mathrm{RQ}}$의 최댓값을 $M$, 최솟값을 $m$이라 할 때, $M+m$의 값을 구하시오. (단, $\mathrm{O}$는 원점이다.) [4점]




\end{document}
